\documentclass{amsart}


\usepackage[utf8]{inputenc}  % (or none if Lua/XeLaTeX)
\usepackage{lmodern}
\usepackage{geometry}
\geometry{margin=1in}
\usepackage[backend=biber,style=alphabetic]{biblatex}
\addbibresource{collatz.bib}

\usepackage{amsmath,amssymb,mathtools}
\usepackage{mathrsfs} % Ralph Smith’s Formal Script
\usepackage{amsthm}
\usepackage{booktabs}
\usepackage{enumitem}
\usepackage{algorithm}
\usepackage{algpseudocode} % from algorithmicx
%\usepackage{hyperref}
\usepackage[colorlinks=true,linkcolor=blue,citecolor=blue,urlcolor=blue]{hyperref}
\usepackage{listings}          % provides \lstset, \lstlisting, \lstinline
\usepackage[scaled=0.9]{inconsolata} % optional: nicer monospace font
\usepackage{xcolor}           % optional: needed if you add colored syntax
\usepackage{xurl}
\usepackage{needspace}
\usepackage{cleveref}
\usepackage[colorinlistoftodos]{todonotes}
%lmodern or mathptmx
%\usepackage[utf8]{inputenc}  % (or none if Lua/XeLaTeX)
%\usepackage{tabularx}                % ok anywhere after microtype
\usepackage[T1]{fontenc}
%\usepackage{microtype} % <- after fonts
% after font packages and \usepackage{microtype}
%\makeatletter
%\AtBeginDocument{%
%  \microtypesetup{activate=true, expansion=true, protrusion=true}%
%  \begingroup
    % warm up common table sizes/families so microtype sets expansion first
%    \rmfamily\normalsize\selectfont X%
%    \rmfamily\small\selectfont X%
%    \rmfamily\footnotesize\selectfont X%
%    \sffamily\small\selectfont X%
%    \ttfamily\small\selectfont X%
%  \endgroup
%}
%\makeatother

% in preamble
\usepackage{tabularx} % for automatic column wrapping
\newcolumntype{Y}{>{\raggedright\arraybackslash}X} % ragged-right X



\usepackage{etoolbox}
\AtBeginEnvironment{proof}{\par\noindent\ignorespaces}

%\hypersetup{bookmarks=false}

\hypersetup{unicode=true}

\pdfstringdefDisableCommands{%
  % math font flattening
  \def\mathrm#1{#1}%
  \def\mathsf#1{#1}%
  \def\mathbf#1{#1}%
  \def\mathcal#1{#1}%
  \def\mathbb#1{#1}%
  \def\mathtt#1{#1}%
  % common math
  \def\dfrac#1#2{#1/#2}%
  % primes & punctuation
  \def\prime{'}%
  \def\colon{:}%
  % drop math delimiters in bookmarks (no arguments!)
  \def\({}%
  \def\){}%
  \def\[{ }%
  \def\]{ }%
}

% helper for math in moving args
\newcommand{\pdfmath}[1]{\texorpdfstring{$#1$}{#1}}


% ---------- theorem styles ----------
\theoremstyle{definition}
\newtheorem{definition}{Definition}
\newtheorem{example}{Example}

\theoremstyle{plain}
\newtheorem{lemma}{Lemma}
\newtheorem{proposition}{Proposition}
\newtheorem{theorem}{Theorem}
\theoremstyle{plain}
\newtheorem{corollary}[theorem]{Corollary}

\theoremstyle{remark}
\newtheorem*{remark}{Remark}

% ---------- tiny helpers for move symbols ----------
\newcommand{\mvpsi}{\psi}
\newcommand{\mvPsi}{\Psi}
\newcommand{\mvomega}{\omega}
\newcommand{\mvOmega}{\Omega}

% ---------- step-line helper (no nested math in args) ----------
% Args: 1=Step label, 2=x, 3=s (\mathrm{e}/\mathrm{o}), 4=m, 5=j,
%       6=move symb (\psi/\Psi/\omega/\Omega), 7=row tag (e.g. \mathrm{e},0),
%       8=rhs text (e.g. x'=96m+5=5)
\newcommand{\StepLine}[8]{%
  \textit{#1: } $x=#2$; $s={#3}$, $m={#4}$, $j={#5}$\,
  $\overset{#6\ \text{ at }\ (#7)}{\longrightarrow}$\,
  $#8$.\\
}

% =========================
% A lightweight Playbook environment (no extra packages needed)
% =========================
\newenvironment{playbook}[1][]%
{%
  \par\medskip
  \noindent\textbf{Playbook.} \textit{#1}\par
  \vspace{2pt}
  \begingroup
  \leftskip=1em\rightskip=0em
}%
{%
  \par\endgroup\medskip
}


\setlength{\textfloatsep}{10pt plus 4pt minus 2pt}
\setlength{\intextsep}{8pt plus 3pt minus 2pt}
\setcounter{topnumber}{3}
\setcounter{totalnumber}{4}
\renewcommand{\textfraction}{0.1}
\renewcommand{\topfraction}{0.9}
\renewcommand{\floatpagefraction}{0.8}



% In the preamble
\newif\ifobjections
%\objectionsfalse % set \objectionstrue for arXiv / long version
\objectionstrue

\newif\ifidentity
%\identityfalse
\identitytrue

\title{An Inverse Calculus for the Odd Layer of the Collatz Map}
\author{Agola Kisira Odero}
\date{\today}

\raggedbottom

% --- Make math safe in captions/headings ---
%\usepackage{texorpdfstring} % usually loaded by hyperref; safe to load
% Helper: \capmath{...} typesets as inline math in PDF/print; plain text in bookmarks/LoF
\newcommand{\capmath}[1]{\texorpdfstring{\(#1\)}{#1}}

% Optional: a tiny helper to move long equations out of captions cleanly
\newenvironment{belowcaption}{\par\vspace{-0.5em}\noindent\begin{minipage}{0.96\linewidth}\small}{\end{minipage}\par\vspace{0.25em}}


% ==== Compact layout block (non-destructive; safe with existing packages) ====
\usepackage{microtype}
%\usepackage{enumitem}
\usepackage[font=small,labelfont=bf]{caption}
%\usepackage{etoolbox}

% Compact lists (affects itemize/enumerate/description)
\setlist{itemsep=0.25em, topsep=0.35em, parsep=0.2em, partopsep=0.1em}

% Compact math/display spacing (keep readable)
\setlength{\abovedisplayskip}{7pt}
\setlength{\belowdisplayskip}{7pt}
\setlength{\abovedisplayshortskip}{5pt}
\setlength{\belowdisplayshortskip}{5pt}

% Float spacing
\setlength{\floatsep}{8pt}
\setlength{\textfloatsep}{10pt}
\setlength{\intextsep}{8pt}

% Paragraph spacing/indent
\setlength{\parskip}{0.35em}
\setlength{\parindent}{1.2em}

% Compact tables by default
\setlength{\tabcolsep}{5pt}
\AtBeginEnvironment{tabular}{\small}
\AtBeginEnvironment{tabularx}{\small}

% Slightly tighter line spread for body text
\linespread{0.98}

% (Optional) reduce space before/after theorems if amsthm is used

% ==== End compact layout block ====
\makeatletter
\g@addto@macro\thm@space@setup{%
  \thm@preskip=6pt \thm@postskip=6pt
}
\makeatother
\begin{document}

\begin{abstract}
\leavevmode\par\noindent
We develop a finite--state, word--based framework for the accelerated odd Collatz map \(U(y)=\frac{3y+1}{2^{\nu_2(3y+1)}}\). Every admissible token (one of \(\Psi,\psi,\omega,\Omega\)) corresponds to a fixed ``row'' with parameters \((\alpha,\beta,c,\delta)\) such that for inputs \(x=18m+6j+p_6\) the update \(x'=6F(p,m)+\delta\) with
\[
F(p,m)=\frac{(9m\,2^{\alpha}+\beta)\,64^{\,p}+c}{9}
\]
satisfies the forward identity \(3x'+1=2^{\alpha+6p}x\). Hence \(U(x')=x\) at every step, providing a per--step certificate independent of the starting value. We formalize \emph{steering} by same--family padding: short words that (i) strictly increase the \(2\)-adic valuation of the affine slope and (ii) control the intercept modulo \(2\) and modulo \(3\). This yields a deterministic lifting procedure that reduces reachability modulo \(3\cdot 2^{K+1}\) to a linear congruence once modulo \(3\) is aligned; a \(2\)-adic refinement then promotes compatible residue solutions to an exact integer solution for a fixed word. We include a reference implementation that verifies each row identity, the mod--3 steering action, and example witnesses modulo \(24\).

The main contribution is a unified, certified inverse--word calculus on the odd layer together with explicit steering gadgets that turn residue targeting into solvable congruences. Because the resulting program would imply convergence of the odd Collatz dynamics to \(1\), we provide machine--checkable tests and artifacts to facilitate scrutiny.
\end{abstract}



\maketitle

\setcounter{tocdepth}{1} % sections only; raise to 2 to include subsections
\tableofcontents


\section{Related work: inverse trees, 2-adic lifting, and modular routing}

Our approach—finite word semantics on the odd layer, certified one–step inverses, and congruence–based “steering” to lift residues from $M_K=3\cdot 2^K$ to $M_{K+1}$—sits alongside several established techniques.

\paragraph{Mod-$2^k$ analysis and lifting.}
Garner studied the $3n{+}1$ dynamics modulo powers of two, organizing inverse branches by congruence classes and effectively “lifting’’ structure from $2^k$ to $2^{k+1}$ \cite{Garner1981}. Our use of the unified rows with a column–lift parameter $p$ (which multiplies the $2$–adic slope by $2^{6p}$) and the residue steering gadgets plays a similar role: we solve linear congruences for $m$ to pass from $M_K$ to $M_{K+1}$ while preserving certified inverses at each step.

\paragraph{Inverse trees and predecessor sets.}
Wirsching’s monograph develops the inverse (predecessor) tree of the $3n{+}1$ function as a dynamical system, with emphasis on structure, measures, and asymptotics on inverse branches \cite{Wirsching1998LNM}. Conceptually, our move alphabet and per–row affine forms are a finite–state presentation of those inverse branches: each token certifies $U(x')=x$ and the composition yields an affine map in the “index’’ $m$, which we then route by residues $M_K$.

\paragraph{The $2$-adic viewpoint and conjugacies.}
Bernstein and Lagarias constructed a $2$-adic conjugacy map relating the odd–accelerated Collatz dynamics to a Bernoulli-like shift \cite{BernsteinLagarias1996}. Our $p$–lift (multiplying by $2^{6p}$) and the parity/valuation steering reflect this same $2$–adic continuity: column–lifts shift $2$–adic scale, while steering gadgets tune intercept parity to land on prescribed residue classes.

\paragraph{Symmetries and autoconjugacy.}
Monks and Yazinski analyzed autoconjugacies of the $3x{+}1$ function and their implications for orbit structure \cite{MonksYazinski2004}. While our framework is more combinatorial/affine, the way we keep the family pattern fixed (Lemma~\ref{lem:family-pattern}) and exploit same–family padding resonates with their use of structural symmetries.

\paragraph{Surveys and context.}
For broad background and additional modular/density insights, see \cite{Lagarias2010survey,Terras1976,Terras1979}; for $2$–adic heuristics and continuity themes, see \cite{Gouvea1997,Nathanson1996}. These perspectives motivate our use of $2$–adic “padding’’ and linear congruences as lifting mechanisms.

\medskip
\noindent\textit{What is new here.}
Our contribution is a single unified $p{=}0$ inverse table on the odd layer (Table~\ref{tab:unified-F0-straight-xprime}) with a per–step column–lift $p\ge 0$ and explicit steering gadgets that (i) raise $v_2$ of the word’s slope and (ii) toggle the intercept parity, ensuring solvability of the lifting congruences modulo $M_{K+1}$ while keeping each step certified by $U(x')=x$.


\subsection*{Contributions}
\begin{itemize}[leftmargin=1.4em]
\item \textbf{One-table, word-driven inverse calculus on the odd layer.}
  We give a unified \(p{=}0\) row table with closed forms \(x'=6F(0,m)+\delta\) indexed only by \((s,j,m)\).
  Once a token and \((s,j)\) are fixed, the step is fully determined and the forward identity
  \(3x'+1=2^{\alpha}x\) holds by construction (Lemma~\ref{lem:row-correctness}).
\item \textbf{Column-lift \(p\) that preserves routing while scaling the 2-adic slope.}
  The parameter \(p\) multiplies the slope by \(2^{6p}\) without changing the token type or output family, yielding a single mechanism that subsumes whole towers of congruence tables (Lemmas~\ref{lem:row-correctness-p}–\ref{lem:mixedp-routing}).
\item \textbf{CRT tag for transparent indexing.}
  The tag \(t=(x-1)/2\) (equivalently \((3x+1-4)/6\)) makes family detection and indices
  \((s,j,m)\) linear in \(t\) (Corollary~\ref{cor:tag-indices}), simplifying routing proofs.
\item \textbf{Steering gadgets that control \(v_2\), \(B\bmod 2\), and \(B\bmod 3\).}
  Short same-family words provably boost the slope’s \(2\)-adic valuation and toggle the affine intercept \(B\bmod 2\), ensuring solvability of the lifting congruence at each modulus (Lemmas~\ref{lem:steering} and~\ref{lem:mod3-steering}, App.~\ref{app:mod3-steering}.).
\item \textbf{From small witnesses to all moduli and exact integers.}
  Starting at \(M_3{=}24\), we give a deterministic induction \(M_K\to M_{K+1}\) (Lemma~\ref{lem:lifting})
  that reaches every odd residue with certified steps, and then a \(2\)-adic refinement to hit any prescribed odd integer exactly (Theorem~\ref{thm:residues-to-integers}).
\item \textbf{Row-level invariance certificates.}
  We isolate a mod-\(54\) one-step invariance (Lemma~\ref{lem:row-invariance-54}) that explains why fixed tokens reselect the same next row across many starts, aiding certification and automation.
\item \textbf{Executable, per-step certificates.}
  A reference implementation emits step traces and verifies \(U(x')=x\) at each step, making all claims reproducible from the table (App.~C).
\end{itemize}

\subsection*{Relation to prior techniques}
\begin{itemize}[leftmargin=1.4em]
\item \textbf{Versus classical modular inverse-tree analyses (Terras, Lagarias, \emph{etc.}).}
  Prior work develops rich residue classifications and stopping-time bounds; our contribution is a \emph{single} finite-state table with a word calculus and an explicit steering mechanism that turns residue reachability into solvable linear congruences with guaranteed \(2\)-adic headroom.
\item \textbf{Versus \(2\)-adic dynamical viewpoints (Gouvêa, Nathanson).}
  Earlier \(2\)-adic studies illuminate topology, measures, and cycles. We use the \(2\)-adic setting constructively: the slope/offset control plus \(2\)-adic completeness converts an infinite ladder of congruences into an exact integer solution anchored to a concrete word.
\item \textbf{Versus “energy”/almost-everywhere results (Tao 2019 and follow-ups).}
  These show near-monotone behavior for a density-one set via probabilistic/analytic Lyapunov methods.
  Our approach is entirely combinatorial and constructive: for each target residue (and ultimately each odd integer) we produce a finite word and certify every inverse step by \(3x'+1=2^{\alpha+6p}x\).
\end{itemize}

\paragraph{Scope note.}
Standard ingredients (accelerated map \(U\), parity splitting, \(v_2\), and modular routing) are classical; the novelty here is the \emph{unified word/table formalism} with a \emph{routing-preserving \(p\)-lift} and \emph{steering gadgets} (including mod-3 control) that together enable a fully constructive lifting from \(\bmod\,24\) to exact integers with stepwise certificates.


\subsection*{Main claim and method}
Our main claim (Theorem~\ref{thm:odd-layer-convergence}) is that every odd \(x\equiv 1,5\pmod 6\) reaches \(1\) in finitely many accelerated odd Collatz steps. The method is modular:
(i) certify row-level inverses \(U(x')=x\) (Lemma~\ref{lem:row-correctness});
(ii) show any admissible word yields an affine form in \(m\) with controlled terminal family (Lemma~\ref{lem:affine-word} and Lemma~\ref{lem:family-pattern});
(iii) furnish base witnesses modulo \(24\) (Table~\ref{tab:base-witnesses-mod24});
(iv) use same-family \emph{steering gadgets} to raise \(v_2(A)\) and control \(B\bmod 2\) and \(B\bmod 3\) (Lemmas~\ref{lem:steering}, \ref{lem:mod3-steering});
(v) lift residues \(M_K\to M_{K+1}\) (Lemma~\ref{lem:lifting}, Theorem~\ref{thm:reachability});
(vi) pass from residues to exact integers by \(2\)-adic refinement (Theorem~\ref{thm:residues-to-integers}).
\ifobjections
For a discussion addressing common misreadings, see Section~\ref{sec:objections}.
\fi


% =========================================================
\section{Notation, indices, and moves}

To unify all Collatz inverse odd orbits we work with an affine form indexed by row parameters \((\alpha,\beta,c)\) and an orbit–type offset \(\delta\in\{1,5\}\). For any nonnegative integer \(p=0,1,2,\ldots\) and \(m=\lfloor x/18\rfloor\), define
\[
F_{\alpha,\beta,c}(p,m)\;:=\;\frac{(9m\,2^{\alpha}+\beta)\,64^{\,p}+c}{9}
\;=\;2^{\alpha+6p}m+\frac{\beta\,64^{\,p}+c}{9},\qquad
x'\;=\;6\,F_{\alpha,\beta,c}(p,m)+\delta.
\]
Here \(p\) is a \emph{column–lift} that preserves routing/type (and \(\delta\)) while multiplying the \(2\)-adic slope by \(2^{6p}\); integrality of \(F_{\alpha,\beta,c}(p,m)\) follows from \(64\equiv 1\pmod{9}\). The base table is recovered at \(p=0\) (so \(F_{\alpha,\beta,c}(0,m)=2^\alpha m+(\beta+c)/9\)), and in all cases the forward identity
\[
3x'+1\;=\;2^{\alpha+6p}\,x
\]
holds, hence \(U(x')=x\). In the following we use orbit and step interchangeably.\\



Let $x$ be odd with $x\not\equiv 3\pmod 6$ and define
\[
s(x)=\begin{cases}\mathrm e,&x\equiv 1\pmod 6,\\ \mathrm o,&x\equiv 5\pmod 6,\end{cases}
\qquad
r=\Big\lfloor\frac{x}{6}\Big\rfloor,\quad
j=r\bmod 3\in\{0,1,2\},\quad
m=\Big\lfloor\frac{x}{18}\Big\rfloor.
\]
We use the accelerated odd Collatz map \(U(y)=\dfrac{3y+1}{2^{\nu_2(3y+1)}}\), standard in the Collatz literature \cite{Lagarias2010survey}.

The move alphabet is \(\mathcal{A}=\{\Psi,\psi,\omega,\Omega\}\) with type mapping
\[
\Psi\leftrightarrow\texttt{ee},\qquad
\psi\leftrightarrow\texttt{eo},\qquad
\omega\leftrightarrow\texttt{oe},\qquad
\Omega\leftrightarrow\texttt{oo}.
\]
Admissibility by family: if $s(x)=\mathrm e$ we may use $\Psi$ or $\psi$; if $s(x)=\mathrm o$ we may use $\omega$ or $\Omega$.

%CRT = Chinese Remainder Theorem

\subsection*{A CRT tag for odds, and re-indexing by \(t\)}
Define, for odd \(x\),
\[
y:=3x+1,\qquad t:=\frac{y-4}{6}=\frac{x-1}{2}\in\mathbb Z.
\]
\begin{table}[!htbp]
\centering
\caption{Notation used throughout. Families \( \mathrm e,\mathrm o \) are \(1,5\!\!\pmod 6\). Indices \(j,m\) come from \(x=18m+6j+p_6\) with \(p_6\in\{1,5\}\).}
\label{tab:notation}
\small
\renewcommand{\arraystretch}{1.12}
\begin{tabularx}{\textwidth}{@{}l X@{}}
\toprule
\textbf{Symbol} & \textbf{Meaning} \\
\midrule
\(U(y)=\dfrac{3y+1}{2^{\nu_2(3y+1)}}\) & Accelerated odd Collatz map (odd layer). \\[2pt]
\(x\) & Current odd, always \(x\equiv 1,5\pmod 6\) on the odd layer. \\[2pt]
\(s(x)\in\{\mathrm e,\mathrm o\}\) & Family of \(x\): \(\mathrm e\) if \(x\equiv 1\pmod 6\), \(\mathrm o\) if \(x\equiv 5\pmod 6\). \\[2pt]
\(j=\big\lfloor \tfrac{x}{6}\big\rfloor \bmod 3\) & Row index (next-row selector), \(j\in\{0,1,2\}\). \\[2pt]
\(m=\big\lfloor \tfrac{x}{18}\big\rfloor\) & Coarse index used in the closed forms \(x'(m)\). \\[2pt]
\(p\in\mathbb{Z}_{\ge 0}\) & Column-lift parameter; each step multiplies the forward power by \(2^{6p}\). \\[2pt]
\(\alpha,\beta,c,\delta\) & Row parameters; \(\delta\in\{1,5\}\) is the output family offset. \\[2pt]
\(k=\dfrac{\beta+c}{9}\) & One-step constant at \(p{=}0\); integrality since \(\beta+c\equiv 0\pmod 9\). \\[2pt]
\(F(p,m)=\dfrac{(9m\,2^\alpha+\beta)\,64^p+c}{9}\) & Lifted per-row form; integral since \(64\equiv 1\pmod 9\). \\[2pt]
\(x'=6F(p,m)+\delta\) & One-step preimage; satisfies \(3x'+1=2^{\alpha+6p}x\). \\[2pt]
\(\nu_2(n)\) & \(2\)-adic valuation of \(n\). \\[2pt]
\(t=\dfrac{x-1}{2}\) & CRT tag (reindexing); bijection \(x=2t+1\). \\[2pt]
\(\mathcal{A}=\{\Psi,\psi,\omega,\Omega\}\) & Token alphabet; types \texttt{ee}, \texttt{eo}, \texttt{oe}, \texttt{oo} respectively. \\[2pt]
\(W\in\mathcal{A}^\ast\) & A word (sequence of tokens). \\[2pt]
\(x_W(m)=6(A_W m+B_W)+\delta_W\) & Affine form after a word; \(A_W=3\cdot 2^{\alpha(W)}\). \\[2pt]
\(M_K=3\cdot 2^K\) & Working modulus for lifting; odd residues split into \(\mathcal E_K,\mathcal O_K\). \\[2pt]
\(\mathcal E_K,\mathcal O_K\) & \(\mathcal E_K=\{1+6t \bmod M_K\}\), \(\mathcal O_K=\{5+6t \bmod M_K\}\). \\[2pt]
\bottomrule
\end{tabularx}
\end{table}

% === Glossary and Notation (added) ===
\section{Glossary and notation}
\begin{description}
  \item[$U$] Accelerated Collatz map $U(n)=(3n+1)/2^{v_2(3n+1)}$ on odd $n$.
  \item[Family $e/o$] Terminal residue class modulo $6$: $e$ for $\equiv 1\pmod 6$, $o$ for $\equiv 5\pmod 6$.
  \item[Router $j$] $j=\big\lfloor x/6\big\rfloor\bmod 3\in\{0,1,2\}$ chooses the table row.
  \item[Internal index $u$] $u=\big\lfloor x/18\big\rfloor$ feeds the next token.
  \item[$(A,B,\delta)$] Linear surrogate: $x=6(Am+B)+\delta$ at each step.
  \item[$M_K$] Modulus $M_K=3\cdot 2^K$.
  \item[Pinning] A last row with exponent $\alpha_p\ge K$ forces $x\equiv 6k^{(p)}+\delta\pmod{M_K}$ independently of $m$.
\end{description}

% ===== Standing assumptions & conventions =====
\section{Standing assumptions and conventions}

We enumerate the ambient assumptions used throughout. None of these are
Collatz–specific hypotheses; they are standard arithmetic facts and
explicitly verified table properties.

\begin{enumerate}[label=(A\arabic*)]
\item \textbf{Universe and variables.}
All variables are integers unless noted. We work on the \emph{odd layer}:
inputs \(x\) are odd with \(x\ge 1\). The column parameter \(p\in\mathbb{Z}_{\ge 0}\),
and the step indices are
\[
m=\Big\lfloor\frac{x}{18}\Big\rfloor,\qquad
j=\Big\lfloor\frac{x}{6}\Big\rfloor\bmod 3,\qquad
x=18m+6j+p_6,\ \ p_6\in\{1,5\}.
\]

\item \textbf{Accelerated odd map.}
We use \(U(y)=\dfrac{3y+1}{2^{\nu_2(3y+1)}}\).
For odd \(y\), one has \(3y+1\equiv 4\pmod 6\), hence \(U(y)\equiv 1\) or \(5\pmod 6\).

\item \textbf{CRT tag.}
For odd \(x\), the tag \(t=\dfrac{3x+1-4}{6}=\dfrac{x-1}{2}\in\mathbb{Z}\) is used only
as a reindexing device; it is bijective via \(x=2t+1\).

\item \textbf{Row parameter table is integral/consistent.}
For every row \((\alpha,\beta,c,\delta)\) in Table~\ref{tab:parameters-abc}:
\[
k=\frac{\beta+c}{9}\in\mathbb{Z},\qquad
\delta=\begin{cases}1,&\texttt{*e}\\[2pt]5,&\texttt{*o}\end{cases}
\]
so that \(F(0,m)=2^\alpha m+k\) and \(x'=6F(0,m)+\delta\) are integer-valued.

\item \textbf{Column lifts are integral.}
For \(p\ge 0\),
\[
F_{\alpha,\beta,c}(p,m)=\frac{(9m\,2^\alpha+\beta)\,64^p+c}{9}\in\mathbb{Z}
\quad\text{since } 64\equiv 1\pmod 9.
\]

\item \textbf{Per–row odd–forward identity.}
For every admissible row and \(p\ge 0\),
\[
3x'+1=2^{\alpha+6p}\,x,
\]
hence \(U(x')=x\). (Proved in the text; used as a stepwise certificate.)

\item \textbf{Word affinity and routing.}
Composing admissible rows yields an affine form
\(x_W(m)=6(A_W m+B_W)+\delta_W\) with \(A_W=3\cdot 2^{\alpha(W)}\).
Family routing (\(\mathrm e/\mathrm o\)) depends only on the token’s type
(\texttt{ee}, \texttt{eo}, \texttt{oe}, \texttt{oo}), not on \(m\) or \(p\).

\item \textbf{Steering gadgets exist and are explicit.}
There are short same–family composites that (i) raise \(v_2(A_W)\) arbitrarily
(by repetition) and (ii) provide a parity toggle \(B_W\mapsto B_W+1\pmod 2\) at \(p=0\).
(Concrete tokens are listed in Appendix~A; e.g.\ rows \(\omega_1\) and \(\Omega_2\) have odd
\(k=(\beta+c)/9\), enabling the toggle.)

\item \textbf{Lifting over powers of two uses only standard facts.}
We use: solvability of linear congruences \(A m\equiv b\pmod{2^K}\);
nested lifting to \(2^{K+1}\) (choose a solution compatible modulo \(2^{K-1}\));
and completeness of \(\mathbb{Z}_2\) to pass from compatible residues to an integer \(m\).
No heuristic or distributional assumptions are used.

\item \textbf{Base witnesses are explicit (no hidden computation).}
The eight \(\bmod 24\) classes \(\{1,5,7,11,13,17,19,23\}\) are each accompanied by a specific
finite word \(W_r\) (Table~\ref{tab:base-witnesses-mod24}), verified stepwise via
\(U(x')=x\). The proof does not rely on unverifiable large-scale searches.

\item \textbf{Scope relative to the classical map.}
All statements are on the odd layer for \(U\). For the classical Collatz map,
even runs are removed by dividing out powers of two between odd iterates; the conclusions then transfer verbatim.
\end{enumerate}

\noindent\emph{Non-assumptions.} We do not assume (i) the Collatz conjecture itself,
(ii) any stochastic/heuristic model for the orbit, or (iii) density or randomness
properties of residue classes. All steps are constructive and finitely checkable.


\section{CRT-tag calculus and odd-step drift}\label{sec:crt-tag}
We collect here the normalization fact for the odd layer via the Chinese Remainder Theorem (\emph{CRT tag}), 
and the resulting closed-form expressions for the single-step drift on the odd layer. 
The parameters $(\alpha,p)$ below match those used in the row/lift transform $F_{\alpha,\beta,c}(p,m)$.
\begin{lemma}[CRT tag for odd inputs]\label{lem:crt-tag}
For odd $x$ one has \(3x+1\equiv 4\pmod 6\) and the tag \(t=\dfrac{3x+1-4}{6}=\dfrac{x-1}{2}\) is an integer.
Moreover, the map \(x\mapsto t\) is a bijection between odd integers and all integers via \(x=2t+1\).
\end{lemma}
\begin{proposition}[Explicit difference of CRT tags along the odd Collatz map]\label{prop:crt-tag-difference-explicit}
Let $x$ be odd and let $x'=\boldsymbol U(x)=\dfrac{3x+1}{2^k}$ be the odd Collatz image with $2^k\Vert(3x+1)$ and $k\ge 1$. 
For odd $y$, write the CRT tag as $t(y)=\dfrac{3y+1-4}{6}=\dfrac{y-1}{2}$ (Lemma~\ref{lem:crt-tag}), and set
\[
d \;:=\; t(x')-t(x),\qquad r := \Big\lfloor \frac{x}{6}\Big\rfloor .
\]
Then $d$ is given by
\begin{align*}
\textbf{(e)}\quad &\text{If } x\equiv 1 \pmod 6, 
&& d \;=\; r\bigl(2^{\alpha+6p}-3\bigr)\,4^p \;+\; 2\,q_p,\\[2mm]
\textbf{(o)}\quad &\text{If } x\equiv 5 \pmod 6, 
&& d \;=\; r\bigl(2^{\alpha+6p}-3\bigr)\,4^p \;+\; 5\,q_p \;-\; 1,
\end{align*}
where 
\[
p\in\mathbb{Z}_{\ge 0},\qquad 
\alpha\ \text{and}\ p\ \text{are the same parameters as in } F_{\alpha,\beta,c}(p,m)\ (\text{see its definition in this paper}),
\]
and $(q_p)_{p\ge 0}$ is given by $q_0=0$, $q_{p+1}=4q_p+1$, equivalently $q_p=\dfrac{4^p-1}{3}$.
Moreover,
\[
x'-x \;=\; 2\,d .
\]
\end{proposition}

\begin{proof}[Proof sketch]
By Lemma~\ref{lem:crt-tag}, $t(y)=\frac{y-1}{2}$ for odd $y$, hence $d=t(x')-t(x)=\frac{x'-x}{2}$. 
Writing $x=6r+\varepsilon$ with $\varepsilon\in\{1,5\}$ and expanding $x'=\frac{3x+1}{2^k}$ along the odd layer yields the two residue-conditioned expressions. 
The parameters $\alpha$ and $p$ coincide with those in the transform $F_{\alpha,\beta,c}(p,m)$ used in the row/lift framework; $q_p$ captures the cumulative geometric contribution $1+4+\cdots+4^{p-1}$.
\end{proof}


\begin{corollary}[Explicit $x'$--$x$ relation]\label{cor:xp-vs-x}
Let $x$ be odd and let $x'=\boldsymbol U(x)$ denote its odd Collatz image. 
Write $x=6r+\varepsilon$ with $\varepsilon\in\{1,5\}$ and $r=\lfloor x/6\rfloor$.
Let $p\in\Bbb Z_{\ge 0}$ and $\alpha$ be the same parameters as in $F_{\alpha,\beta,c}(p,m)$, and set
\[
K:=\bigl(2^{\alpha+6p}-3\bigr)\,4^p
\quad\text{and}\quad
q_p:=\frac{4^p-1}{3}.
\]
Then the following hold:
\begin{align*}
\textbf{(e)}\; &\text{If } x\equiv 1\pmod 6\ (\varepsilon=1), 
&& x' \;=\; x \;+\; 2r\,K \;+\; 4\,q_p,\\[2mm]
\textbf{(o)}\; &\text{If } x\equiv 5\pmod 6\ (\varepsilon=5), 
&& x' \;=\; x \;+\; 2r\,K \;+\; 10\,q_p \;-\; 2.
\end{align*}
Equivalently, eliminating $r=\frac{x-\varepsilon}{6}$ gives the affine-in-$x$ forms
\begin{align*}
\textbf{(e)}\; &x' \;=\; x \;+\; \frac{x-1}{3}\,K \;+\; 4\,q_p
\;=\;\Bigl(1+\frac{K}{3}\Bigr)x \;+\; \Bigl(4\,q_p-\frac{K}{3}\Bigr),\\[2mm]
\textbf{(o)}\; &x' \;=\; x \;+\; \frac{x-5}{3}\,K \;+\; 10\,q_p -2
\;=\;\Bigl(1+\frac{K}{3}\Bigr)x \;+\; \Bigl(10\,q_p-2-\frac{5K}{3}\Bigr).
\end{align*}
In particular, $x'-x=2\bigl(t(x')-t(x)\bigr)$ and the dependence on the row/lift step is concentrated in $K$ (via $\alpha$ and $p$) and the family offset (the $q_p$ terms).
\end{corollary}


\begin{proof}
\leavevmode\par\noindent
\begin{itemize}[leftmargin=1.6em]
\item Mod $2$: $x\equiv 1\Rightarrow 3x+1\equiv 0$ (even).
\item Mod $3$: $3x+1\equiv 1$.
\item The unique residue modulo $6$ that is $0\bmod 2$ and $1\bmod 3$ is $4$, hence $3x+1\equiv 4\pmod 6$, so $t\in\mathbb Z$.
\item The identities \(t=(x-1)/2\) and \(x=2t+1\) give a bijection odd $\leftrightarrow$ integer.
\end{itemize}
\end{proof}

\begin{example}[After Lemma~\ref{lem:crt-tag}]
With $x=19$ one has $y=58\equiv 4\pmod 6$ and $t=(58-4)/6=9=(19-1)/2$; conversely $x=2t+1=19$.
\end{example}

\begin{corollary}[Family and indices from the tag]\label{cor:tag-indices}
Let $t=\frac{x-1}{2}$ for odd $x$. Then
\[
x\bmod 6 \;=\; 2\,(t\bmod 3)+1,\qquad
m=\Big\lfloor\frac{x}{18}\Big\rfloor=\Big\lfloor\frac{t}{9}\Big\rfloor,\qquad
j=\Big\lfloor\frac{x}{6}\Big\rfloor\bmod 3=\Big\lfloor\frac{t}{3}\Big\rfloor \bmod 3,
\]
provided $t\bmod 3\in\{0,2\}$ (i.e.\ $x\not\equiv 3\pmod 6$).
\end{corollary}

\begin{proof}
\leavevmode\par\noindent
\begin{itemize}[leftmargin=1.6em]
\item Write $t=3q+r$ with $r\in\{0,1,2\}$; then $x=2t+1=6q+2r+1\equiv 2r+1\pmod 6$.
Thus $r=0\Rightarrow x\equiv 1$, $r=2\Rightarrow x\equiv 5$, $r=1\Rightarrow x\equiv 3$.
\item For $m$: $\frac{x}{18}=\frac{2t+1}{18}=\frac{t}{9}+\frac{1}{18}$, so $\lfloor x/18\rfloor=\lfloor t/9\rfloor$.
\item For $j$ with $r\in\{0,2\}$: $\frac{x}{6}=\frac{2t+1}{6}=\frac{t}{3}+\frac{1}{6}$ and
$\lfloor t/3+1/6\rfloor=\lfloor t/3\rfloor$, hence $j=\lfloor t/3\rfloor\bmod 3$.
\end{itemize}
\end{proof}

\begin{example}[After Corollary~\ref{cor:tag-indices}]
If $x=53$, then $t=(53-1)/2=26$. We get $t\bmod 3=2\Rightarrow x\bmod 6=5$ (family $\mathrm o$),
$m=\lfloor 26/9\rfloor=2$, and $j=\lfloor 26/3\rfloor\bmod 3=8\bmod 3=2$, matching the table rows used later.
\end{example}

\begin{example}[Even-$\alpha$ grid check]\label{ex:even-alpha-grid}
Fix the residue class $x\equiv 1\pmod 6$ (case \textbf{(e)}), and let $r=\lfloor x/6\rfloor$.
For $\alpha\in\{2,4,6,8,10,12,14\}$ and $p\in\mathbb{Z}_{\ge 0}$ with $q_p=(4^p-1)/3$, the proposition gives
\[
d \;=\; r\bigl(2^{\alpha+6p}-3\bigr)4^p + 2q_p,
\qquad x'-x \;=\; 2d.
\]
Tabulating $d$ across $r$ and even $\alpha$ produces the grid in Table~\ref{tab:even-alpha-d-grid}.
Evaluating $x'=\boldsymbol U(x)$ and verifying $t(x')-t(x)=d$ confirms the formula numerically over the sample.
\end{example}

\begin{table}[!htbp]
\centering
\caption{Difference $d=t(x')-t(x)$ for $x\equiv 1\pmod 6$ at even $\alpha$ (first 10 rows).}
\label{tab:even-alpha-d-grid-10}
\begin{tabular}{c|ccccccc}
$r$ & $\alpha=$2 & $\alpha=$4 & $\alpha=$6 & $\alpha=$8 & $\alpha=$10 & $\alpha=$12 & $\alpha=$14 \\
\hline
0 & 0 & 2 & 10 & 42 & 170 & 682 & 2730 \\
1 & 1 & 15 & 71 & 295 & 1191 & 4775 & 19111 \\
2 & 2 & 28 & 132 & 548 & 2212 & 8868 & 35492 \\
3 & 3 & 41 & 193 & 801 & 3233 & 12961 & 51873 \\
4 & 4 & 54 & 254 & 1054 & 4254 & 17054 & 68254 \\
5 & 5 & 67 & 315 & 1307 & 5275 & 21147 & 84635 \\
6 & 6 & 80 & 376 & 1560 & 6296 & 25240 & 101016 \\
7 & 7 & 93 & 437 & 1813 & 7317 & 29333 & 117397 \\
8 & 8 & 106 & 498 & 2066 & 8338 & 33426 & 133778 \\
9 & 9 & 119 & 559 & 2319 & 9359 & 37519 & 150159 \\
\end{tabular}
\end{table}

\begin{table}[!htbp]
\centering
\caption{Difference $d=t(x')-t(x)$ for $x\equiv 5\pmod 6$ at odd $\alpha$ (first 10 rows).}
\label{tab:odd-alpha-d-grid-10}
\begin{tabular}{c|ccccccc}
$r$ & $\alpha=1$ & $\alpha=3$ & $\alpha=5$ & $\alpha=7$ & $\alpha=9$ & $\alpha=11$ & $\alpha=13$ \\
\hline
0 & -1 & 4 & 24 & 104 & 424 & 1704 & 6824 \\
1 & -2 & 9 & 53 & 229 & 933 & 3749 & 15013 \\
2 & -3 & 14 & 82 & 354 & 1442 & 5794 & 23202 \\
3 & -4 & 19 & 111 & 479 & 1951 & 7839 & 31391 \\
4 & -5 & 24 & 140 & 604 & 2460 & 9884 & 39580 \\
5 & -6 & 29 & 169 & 729 & 2969 & 11929 & 47769 \\
6 & -7 & 34 & 198 & 854 & 3478 & 13974 & 55958 \\
7 & -8 & 39 & 227 & 979 & 3987 & 16019 & 64147 \\
8 & -9 & 44 & 256 & 1104 & 4496 & 18064 & 72336 \\
9 & -10 & 49 & 285 & 1229 & 5005 & 20109 & 80525 \\
\end{tabular}
\end{table}

\begin{remark}[Design knobs for routing via family choice]\label{rem:routing-knobs}
Write $x=6r+\varepsilon$ with $\varepsilon\in\{1,5\}$ (families \textbf{(e)} and \textbf{(o)} respectively).
By Proposition~\ref{prop:crt-tag-difference-explicit} and Corollary~\ref{cor:xp-vs-x}, the single-step drift decomposes as
\[
x'-x \;=\; 2\Big( r\cdot K + \Delta_\varepsilon\Big),\qquad
K:=(2^{\alpha+6p}-3)4^p,\quad
\Delta_1:=2q_p,\ \ \Delta_5:=5q_p-1.
\]
Hence the \emph{knobs} a designer can turn at a given $x$ are:
\begin{enumerate}
\item \textbf{Family selection ($\varepsilon$)} controls the offset term $\Delta_\varepsilon$:
      $\Delta_1\ge 0$ (nonnegative), while $\Delta_5=-1$ when $p=0$ and $\Delta_5>0$ for $p\ge1$.
\item \textbf{Lift depth ($p$)} controls the geometric factor $4^p$ (and $q_p$), amplifying the drift.
\item \textbf{Row parameter ($\alpha$)} controls the multiplier $2^{\alpha+6p}-3$ within $K$.
\end{enumerate}
Operationally, \textbf{(o)} enables small negative drifts at low $r$ when $p=0$, while \textbf{(e)} gives nonnegative drift with a simpler offset.
Choosing $(\varepsilon,\alpha,p)$ thus steers both the \emph{sign} (near the $r\!=\!0$ frontier) and the \emph{magnitude} (via $K$) of $x'-x$.
\end{remark}

\begin{lemma}[Drift sign and thresholds]\label{lem:drift-thresholds}
With the notation above, the drift $x'-x=2(rK+\Delta_\varepsilon)$ obeys:
\begin{enumerate}
\item \textbf{(e) family} ($\varepsilon=1$): $x'-x\ge 0$ for all $r\ge 0$, with equality iff $r=0$ and $p=0$.
\item \textbf{(o) family} ($\varepsilon=5$):
      \[
      x'-x<0 \iff rK+\,(5q_p-1)<0.
      \]
      In particular, for $p=0$ ($q_0=0$) we have $x'-x=-2$ when $r=0$, and $x'-x>0$ as soon as $r\ge 1$.
      For $p\ge 1$, $x'-x>0$ for all $r\ge 0$.
\end{enumerate}
\end{lemma}

\begin{corollary}[Potential function]\label{cor:potential}
Define the potential $V(x):=t(x)=\frac{x-1}{2}$ on the odd layer. Then
\[
\Delta V\;:=\;V(x')-V(x)\;=\;rK+\Delta_\varepsilon
\]
with $K,\Delta_\varepsilon$ as above. Thus $V$ is a \emph{linear} potential whose one-step drift is (i) nonnegative on \textbf{(e)}, and
(ii) negative only in the \textbf{(o)} frontier case $(p,r)=(0,0)$.
Moreover, the magnitude of the drift grows linearly with $r$ and geometrically with $4^p$ via $K$.
\end{corollary}


\begin{corollary}[Fast evaluation and steering]\label{cor:fast-eval-steer}
Given odd $x=6r+\varepsilon$, a chosen row/lift step $(\alpha,p)$ from $F_{\alpha,\beta,c}(p,m)$, and family $\varepsilon\in\{1,5\}$:
\[
d \leftarrow r\,(2^{\alpha+6p}-3)\,4^p \;+\;
\begin{cases}
2q_p, & \varepsilon=1,\\
5q_p-1, & \varepsilon=5,
\end{cases}
\qquad x' \leftarrow x + 2d,
\]
where $q_p=\frac{4^p-1}{3}$. Then $x'$ is the predicted odd Collatz image. This avoids explicit evaluation of $(3x+1)/2^k$ and exposes the design knobs $(\varepsilon,\alpha,p)$ for routing.
\end{corollary}

\begin{algorithm}
\caption{Single-step routing by drift targeting}\label{alg:routing}
\begin{algorithmic}[1]
  \Require odd $x=6r+\varepsilon$, desired drift sign $\sigma\in\{-1,0,+1\}$ or magnitude target $T\ge 0$
  \Ensure parameters $(\varepsilon,\alpha,p)$ and predicted image $x'$
  \For{$p \gets 0,1,2,\dots$}
    \For{each admissible $\alpha$ (per row family)}
      \For{each $\varepsilon \in \{1,5\}$}
        \State $q_p \gets \dfrac{4^p-1}{3}$ \hfill\Comment{geometric accumulator}
        \State $K \gets \bigl(2^{\alpha+6p}-3\bigr)\,4^p$
        \State $\Delta \gets \begin{cases}
            2q_p, & \varepsilon=1,\\
            5q_p-1, & \varepsilon=5
          \end{cases}$
        \State $d \gets r\cdot K + \Delta$
        \If{$\mathrm{sign}(d)=\sigma$ \textbf{ and } $|2d|\ge T$}
          \State $x' \gets x + 2d$
          \State \Return $(\varepsilon,\alpha,p,\,x')$
        \EndIf
      \EndFor
    \EndFor
  \EndFor
\end{algorithmic}
\end{algorithm}

\begin{remark}[Search scope and guarantees]
For \textbf{(e)} ($\varepsilon=1$), $d\ge 0$ always; to achieve $|x'-x|\ge T$, increase $p$ or $\alpha$ until $2rK\ge T-4q_p$.
For \textbf{(o)} ($\varepsilon=5$), the unique negative-drift entry is $(p,r)=(0,0)$ (giving $x'-x=-2$); otherwise $d>0$ and the same growth logic as (e) applies.
\end{remark}


\begin{proposition}[Two-step composition on the odd layer]\label{prop:two-step}
Write $x=6r+\varepsilon$ with $\varepsilon\in\{1,5\}$ and let the first step use parameters $(\alpha_1,p_1)$ and family $\varepsilon$
and the second step use $(\alpha_2,p_2)$ and family $\varepsilon'$. Define, for $j\in\{1,2\}$,
\[
K_j := \bigl(2^{\alpha_j+6p_j}-3\bigr)4^{p_j},\qquad
q_{p_j}:=\frac{4^{p_j}-1}{3},
\]
and the family offsets
\[
B^{(1)}(K,q):=4q-\tfrac{K}{3},\qquad
B^{(5)}(K,q):=10q-2-\tfrac{5K}{3}.
\]
Then the single-step maps admit the affine-in-$x$ forms (Cor.~\ref{cor:xp-vs-x})
\[
x' \;=\; \Bigl(1+\tfrac{K_1}{3}\Bigr)\,x \;+\; B^{(\varepsilon)}(K_1,q_{p_1}),
\qquad
x'' \;=\; \Bigl(1+\tfrac{K_2}{3}\Bigr)\,x' \;+\; B^{(\varepsilon')}(K_2,q_{p_2}),
\]
and the two-step composition is therefore
\[
x'' \;=\; \underbrace{\Bigl(1+\tfrac{K_2}{3}\Bigr)\Bigl(1+\tfrac{K_1}{3}\Bigr)}_{=:A_{2:1}}\;x
\;+\;
\underbrace{\Bigl(1+\tfrac{K_2}{3}\Bigr)B^{(\varepsilon)}(K_1,q_{p_1}) + B^{(\varepsilon')}(K_2,q_{p_2})}_{=:B_{2:1}}.
\]
\emph{Integrality is guaranteed} because $x\equiv\varepsilon\in\{1,5\}$ and $x'\equiv\varepsilon'\in\{1,5\}$, as in Cor.~\ref{cor:xp-vs-x}.
\end{proposition}

\begin{corollary}[Additivity of tag drift and two-step $x$-increment]\label{cor:two-step-drift}
Let $d_1:=t(x')-t(x)$ and $d_2:=t(x'')-t(x')$.
With $\Delta_{1}:=2q_{p_1}$, $\Delta_{5}:=5q_{p_1}-1$ and analogues $\Delta'_{1},\Delta'_{5}$ for $(\alpha_2,p_2)$, we have
\[
d_1 \;=\; r\,K_1 + \Delta_{\varepsilon},
\qquad
d_2 \;=\; r'\,K_2 + \Delta'_{\varepsilon'},
\quad
r'=\Big\lfloor \frac{x'}{6}\Big\rfloor,
\]
and hence
\[
t(x'')-t(x)=d_1+d_2,\qquad
x''-x\;=\;2(d_1+d_2)\;=\;2\bigl(rK_1+\Delta_{\varepsilon} + r'K_2+\Delta'_{\varepsilon'}\bigr).
\]
Equivalently, using Proposition~\ref{prop:two-step}, $x''=A_{2:1}x+B_{2:1}$.
\end{corollary}
\begin{remark}[Family routing for the second step]\label{rem:routing-congruence}
Since $x'=x+2d_1$, the second-step family $\varepsilon' \in\{1,5\}$ is determined by
\[
\varepsilon' \;\equiv\; x' \pmod 6 \;\equiv\; \varepsilon + 2\,d_1 \pmod 6.
\]
Thus, to \emph{target} a desired $\varepsilon'$ after the first step, choose $(\alpha_1,p_1)$ so that
\[
2\bigl(rK_1+\Delta_{\varepsilon}\bigr)\;\equiv\;\varepsilon'-\varepsilon \pmod 6.
\]
Once $\varepsilon'$ is achieved, the second step is given by Cor.~\ref{cor:two-step-drift} or, in closed form, by Prop.~\ref{prop:two-step}.
\end{remark}

\begin{proposition}[Additive tag drift over $n$ steps]\label{prop:nstep-additive-drift}
Let $x^{(0)}:=x$ be odd and for $k=1,\dots,n$ let $x^{(k)}$ be the odd Collatz image of $x^{(k-1)}$.
Write $x^{(k-1)}=6r_{k-1}+\varepsilon_k$ with $\varepsilon_k\in\{1,5\}$ and let the $k$-th step use parameters $(\alpha_k,p_k)$ from $F_{\alpha,\beta,c}(p,m)$.
Define
\[
K_k:=\bigl(2^{\alpha_k+6p_k}-3\bigr)\,4^{p_k},\qquad q_{p_k}:=\frac{4^{p_k}-1}{3},\qquad
\Delta^{(1)}_k:=2q_{p_k},\ \ \Delta^{(5)}_k:=5q_{p_k}-1.
\]
Then the one-step tag drift is
\[
d_k\ :=\ t\!\bigl(x^{(k)}\bigr)-t\!\bigl(x^{(k-1)}\bigr)\ =\ r_{k-1}\,K_k\ +\ \Delta^{(\varepsilon_k)}_k,
\]
and the $n$-step drift is additive:
\[
t\!\bigl(x^{(n)}\bigr)-t(x)\ =\ \sum_{k=1}^n d_k,\qquad
x^{(n)}-x\ =\ 2\sum_{k=1}^n d_k.
\]
\end{proposition}

\begin{corollary}[Affine $n$-step composition]\label{cor:nstep-affine}
For each step $k$, set
\[
B^{(1)}(K_k,q_{p_k})\ :=\ 4q_{p_k}-\tfrac{K_k}{3},\qquad
B^{(5)}(K_k,q_{p_k})\ :=\ 10q_{p_k}-2-\tfrac{5K_k}{3}.
\]
Then the single-step map has the affine form
\[
x^{(k)}\ =\ \Bigl(1+\tfrac{K_k}{3}\Bigr)\,x^{(k-1)}\ +\ B^{(\varepsilon_k)}(K_k,q_{p_k}),
\]
and the $n$-step composition is
\[
x^{(n)}\ =\ \Bigl(\prod_{k=1}^{n}\bigl(1+\tfrac{K_k}{3}\bigr)\Bigr)\,x
\;+\; \sum_{k=1}^{n}\Bigl(\prod_{i=k+1}^{n}\bigl(1+\tfrac{K_i}{3}\bigr)\Bigr)\,B^{(\varepsilon_k)}(K_k,q_{p_k}),
\]
with the empty product interpreted as $1$. Integrality holds along any path with $\varepsilon_k\in\{1,5\}$.
\end{corollary}


\begin{remark}[When does the drift become “purely additive”?]\label{rem:purely-additive}
If $K_k\equiv K$ and $\varepsilon_k\equiv\varepsilon$ are fixed across steps (i.e., constant $(\alpha,p)$ and fixed family), then
\[
d_k\ =\ r_{k-1}K+\Delta^{(\varepsilon)},\quad K=(2^{\alpha+6p}-3)4^p,\ \
\Delta^{(1)}=2q_p,\ \Delta^{(5)}=5q_p-1,
\]
and $(x^{(k)})_{k\ge 0}$ obeys the constant-coefficient affine recurrence
\[
x^{(k)}\ =\ \Bigl(1+\tfrac{K}{3}\Bigr) x^{(k-1)} + B^{(\varepsilon)}(K,q_p).
\]
\end{remark}

\begin{remark}[Routing the family sequence]\label{rem:family-routing-seq}
The family at step $k$ obeys
\[
\varepsilon_{k+1}\ \equiv\ x^{(k)} \equiv \varepsilon_k + 2d_k \pmod{6}.
\]
Thus one can \emph{aim} a desired sequence $(\varepsilon_1,\dots,\varepsilon_n)$ by solving the congruences
\[
2\bigl(r_{k-1}K_k+\Delta^{(\varepsilon_k)}_k\bigr)\ \equiv\ \varepsilon_{k+1}-\varepsilon_k \pmod{6}
\quad (k=1,\dots,n-1),
\]
and then obtaining $x^{(n)}$ from Corollary~\ref{cor:nstep-affine}.
\end{remark}

\begin{lemma}[Tag decomposition by family]\label{lem:tag-decomp}
For odd $x=6r+\varepsilon$ with $\varepsilon\in\{1,5\}$,
\[
t(x)=\frac{x-1}{2}=
\begin{cases}
3r, & \varepsilon=1,\\
3r+2, & \varepsilon=5.
\end{cases}
\]
Equivalently, $t(x)\equiv 0\pmod 3$ on family \textbf{(e)} and $t(x)\equiv 2\pmod 3$ on family \textbf{(o)}.
\end{lemma}

\begin{lemma}[Carry and family update]\label{lem:carry}
Let $d=t(x')-t(x)$ and $x=6r+\varepsilon$ as above. Then
\[
x'=x+2d,\qquad
r'=\Big\lfloor \frac{x'}{6}\Big\rfloor=r+\Big\lfloor \frac{\varepsilon+2d}{6}\Big\rfloor,
\qquad
\varepsilon' \equiv \varepsilon+2d \pmod 6,\ \ \varepsilon'\in\{1,5\}.
\]
Thus the next family is determined by the \emph{carry} $c:=\big\lfloor(\varepsilon+2d)/6\big\rfloor$ alone.
\end{lemma}

\begin{lemma}[Closed form for $\delta_p$]\label{lem:delta-p-closed}
Let $\delta_p$ be the unique residue with $\delta_p\equiv 0\pmod{2^{p+1}}$ and $\delta_p\equiv 1\pmod 3$, chosen in $\{0,1,\dots,6\cdot 2^{p}-1\}$. Then
\[
\delta_p=
\begin{cases}
2^{p+2}, & p\ \text{even},\\
2^{p+1}, & p\ \text{odd}.
\end{cases}
\]
In particular, $\delta_0=4$, $\delta_1=4$, $\delta_2=16$, $\delta_3=16$, etc.
\end{lemma}

\begin{corollary}[Drift bounds]\label{cor:drift-bounds}
With $K=(2^{\alpha+6p}-3)4^p$ and $q_p=(4^p-1)/3$,
\[
2rK - 2 \ \le\ x'-x\ \le\
\begin{cases}
2rK+4\cdot \frac{4^p-1}{3}, & \varepsilon=1,\\[2pt]
2rK+10\cdot \frac{4^p-1}{3}, & \varepsilon=5,
\end{cases}
\]
with equality on the left precisely at $(\varepsilon,p,r)=(5,0,0)$. In particular, $x'-x= \Theta(r\cdot 2^{\alpha+8p})$ uniformly in fixed $(\alpha,p)$.
\end{corollary}

\begin{lemma}[Residue targeting for the next odd]\label{lem:route-target}
Fix a desired next family $\varepsilon'\in\{1,5\}$. Given $x=6r+\varepsilon$ and a choice of $(\alpha,p)$, the step achieves $\varepsilon'$ iff
\[
2\bigl(rK+\Delta_\varepsilon\bigr)\ \equiv\ \varepsilon'-\varepsilon \pmod 6,
\quad
\Delta_1=2q_p,\ \Delta_5=5q_p-1.
\]
When solvable, $x'$ is then given by Cor.~\ref{cor:xp-vs-x}.
\end{lemma}

\begin{remark}[Linear potential]
The potential $V(x):=t(x)$ makes the odd-step update exactly linear:
\[
V(x')-V(x)=rK+\Delta_\varepsilon,\qquad x'-x=2\bigl(V(x')-V(x)\bigr).
\]
Thus all nonlinearity of the odd layer is carried by the \emph{carry} $c=\big\lfloor(\varepsilon+2d)/6\big\rfloor$ in $r'$, cf.\ Lemma~\ref{lem:carry}.
\end{remark}

\subsection{From CRT primitives to the row/lift assembly}\label{subsec:crt-to-assembly}

The CRT calculus provides two callable primitives with precise pre/postconditions:

\begin{definition}[CRT primitives: \textsc{RouteStep} and \textsc{Compose}]\label{def:crt-primitives-1}
\leavevmode
\begin{enumerate}
\item \textsc{RouteStep}$(x;\alpha,p,\varepsilon)$ (single step, Lemma~\ref{lem:route-target}, Cor.~\ref{cor:xp-vs-x}):\\
\textbf{Input.} Odd $x=6r+\varepsilon$ with $\varepsilon\in\{1,5\}$, parameters $(\alpha,p)$ from $F_{\alpha,\beta,c}(p,m)$.\\
\textbf{Output.} $x' = x + 2\bigl(rK+\Delta_\varepsilon\bigr)$, where $K=(2^{\alpha+6p}-3)4^p$, $\Delta_1=2q_p$, $\Delta_5=5q_p-1$, $q_p=(4^p-1)/3$; family update $\varepsilon' \equiv \varepsilon + 2(rK+\Delta_\varepsilon) \pmod 6$ with $\varepsilon'\in\{1,5\}$.

\item \textsc{Compose}$(x;(\alpha_k,p_k,\varepsilon_k)_{k=1}^n)$ (multi-step affine closure, Cor.~\ref{cor:nstep-affine}):\\
\textbf{Input.} An admissible sequence of single steps.\\
\textbf{Output.}
\[
x^{(n)} \;=\; \Bigl(\prod_{k=1}^{n}\bigl(1+\tfrac{K_k}{3}\bigr)\Bigr)\,x
\;+\; \sum_{k=1}^{n}\Bigl(\prod_{i=k+1}^{n}\bigl(1+\tfrac{K_i}{3}\bigr)\Bigr)\,B^{(\varepsilon_k)}(K_k,q_{p_k}),
\]
with $K_k=(2^{\alpha_k+6p_k}-3)4^{p_k}$ and
$B^{(1)}(K,q)=4q-\tfrac{K}{3}$, $B^{(5)}(K,q)=10q-2-\tfrac{5K}{3}$.
\end{enumerate}
\end{definition}

\begin{definition}[CRT primitives: \textsc{RouteStep} and \textsc{Compose}]\label{def:crt-primitives-2}
\leavevmode
\begin{enumerate}
\item \textsc{RouteStep}\((x;\alpha,p,\varepsilon)\)\,: for odd \(x=6r+\varepsilon\) with \(\varepsilon\in\{1,5\}\) and parameters \((\alpha,p)\),
\[
K:=(2^{\alpha+6p}-3)4^p,\qquad q_p:=\frac{4^p-1}{3},\qquad
\Delta_\varepsilon:=\begin{cases}2q_p,&\varepsilon=1\\[2pt]5q_p-1,&\varepsilon=5\end{cases}.
\]
It returns
\[
x' \;=\; x + 2\bigl(rK+\Delta_\varepsilon\bigr),\qquad
\varepsilon' \equiv \varepsilon + 2\bigl(rK+\Delta_\varepsilon\bigr) \pmod 6,\ \ \varepsilon'\in\{1,5\}.
\]
(Equivalently, \(t(x')-t(x)=rK+\Delta_\varepsilon\).)

\item \textsc{Compose}\(\bigl(x;\,(\alpha_k,p_k,\varepsilon_k)_{k=1}^n\bigr)\)\,: for a sequence of admissible steps, define
\[
K_k:=(2^{\alpha_k+6p_k}-3)4^{p_k},\quad q_{p_k}=\frac{4^{p_k}-1}{3},\quad
B^{(1)}_k=4q_{p_k}-\tfrac{K_k}{3},\ \ B^{(5)}_k=10q_{p_k}-2-\tfrac{5K_k}{3}.
\]
It returns the affine closure
\[
x^{(n)} \;=\; \Bigl(\prod_{k=1}^{n}\bigl(1+\tfrac{K_k}{3}\bigr)\Bigr)\,x
\;+\; \sum_{k=1}^{n}\Bigl(\prod_{i=k+1}^{n}\bigl(1+\tfrac{K_i}{3}\bigr)\Bigr)\,B^{(\varepsilon_k)}_k,
\]
where \(B^{(\varepsilon_k)}_k\) means \(B^{(1)}_k\) if \(\varepsilon_k=1\) and \(B^{(5)}_k\) if \(\varepsilon_k=5\).
\end{enumerate}
\end{definition}


\paragraph{Why this helps.}
In the assembly layer we repeatedly need to (i) \emph{choose} a next family/residue for routing, and (ii) \emph{predict} the resulting odd image without re-deriving valuations.
The primitive \textsc{RouteStep} turns family choice into the single congruence
\[
2\bigl(rK+\Delta_\varepsilon\bigr)\ \equiv\ \varepsilon'-\varepsilon \pmod 6,
\]
while \textsc{Compose} tells us exactly how sequences of such choices combine as an affine map, eliminating parity case explosion.

\begin{lemma}[Assembly invariants surfaced by the CRT primitives]\label{lem:assembly-invariants}
Throughout the row/lift assembly:
\begin{enumerate}
\item \textbf{Family/Tag alignment.} $t(x)\equiv 0\pmod 3$ iff $\varepsilon=1$, and $t(x)\equiv 2\pmod 3$ iff $\varepsilon=5$ (Lemma~\ref{lem:tag-decomp}).
\item \textbf{Potential linearity.} $V(x):=t(x)$ satisfies $V(x')-V(x)=rK+\Delta_\varepsilon$ and $x'-x=2\bigl(V(x')-V(x)\bigr)$.
\item \textbf{Carry control.} $r' = r + \big\lfloor(\varepsilon+2(rK+\Delta_\varepsilon))/6\big\rfloor$ (Lemma~\ref{lem:carry}), so the only nonlinearity is the carry.
\end{enumerate}
\end{lemma}

\paragraph{Design pattern for rows/lifts.}
A typical row or $p$-lift gadget needs to hit (A) a target family $\varepsilon^\star$; (B) a target magnitude band for $x'-x$ or $V(x')-V(x)$; (C) a compatibility constraint with $F_{\alpha,\beta,c}(p,m)$.
The CRT view reduces this to solving the congruence in Lemma~\ref{lem:route-target} together with a simple inequality in $K$:
\[
\text{choose $(\alpha,p)$ with}\quad
2\bigl(rK+\Delta_\varepsilon\bigr)\equiv \varepsilon^\star-\varepsilon \pmod 6,\qquad
|2(rK+\Delta_\varepsilon)|\ \text{in desired band}.
\]

\begin{corollary}[Plug-and-play routing in the assembly]\label{cor:plug-play-routing}
In any place the assembly previously branched on parity cases, replace the branch by:
\begin{enumerate}
\item \emph{Target a family} $\varepsilon^\star\in\{1,5\}$ using Lemma~\ref{lem:route-target}.
\item \emph{Pick gain} via $p$ (geometric factor $4^p$) and refine via $\alpha$ to reach the drift band (Cor.~\ref{cor:drift-bounds}).
\item \emph{Compose} the chosen steps with Cor.~\ref{cor:nstep-affine} to obtain the final odd entry for the next module.
\end{enumerate}
\end{corollary}

\paragraph{Algorithmic skeleton for a row block.}
(Uses the \texttt{algorithm} float with \texttt{algpseudocode}.)

\begin{algorithm}
\caption{Row block as CRT-driven router}\label{alg:row-block}
\begin{algorithmic}[1]
  \Require odd $x=6r+\varepsilon$; target family $\varepsilon^\star$; drift band $[L,U]$
  \Ensure parameters $(\alpha,p)$ and predicted image $x'$ meeting the target
  \For{$p \gets 0,1,2,\dots$}
    \For{each admissible $\alpha$ (per row family)}
      \State $K \gets (2^{\alpha+6p}-3)4^p$, \; $q_p \gets (4^p-1)/3$, \;
             $\Delta \gets \begin{cases}2q_p,&\varepsilon=1\\ 5q_p-1,&\varepsilon=5\end{cases}$
      \If{$2(rK+\Delta)\equiv \varepsilon^\star-\varepsilon \pmod 6$ \textbf{and} $2(rK+\Delta)\in[L,U]$}
        \State \Return $(\alpha,p,\ x' \gets x + 2(rK+\Delta))$
      \EndIf
    \EndFor
  \EndFor
\end{algorithmic}
\end{algorithm}

\begin{example}[Real two-step odd Collatz path: $\varepsilon\!=\!1 \to \varepsilon'\!=\!5 \to \varepsilon''\!=\!5$]\label{ex:micro-assembly-real}
Take $x=7$, so $x=6r+\varepsilon$ with $r=1$ and $\varepsilon=1$ (family \textbf{(e)}).
We perform two \emph{actual} odd steps $x\to x'\to x''$ under $\boldsymbol U(y)=(3y+1)/2^{v_2(3y+1)}$.

\medskip\noindent
\textbf{Step 1 (the true odd image of $x=7$).}
Compute $3x+1=22=2^1\!\cdot 11$, hence $k_1=v_2(3x+1)=1$ and
\[
x'=\boldsymbol U(7)=\frac{3\cdot 7 + 1}{2^{k_1}}=\frac{22}{2}=11.
\]
Tags and drift: $t(7)=\frac{7-1}{2}=3$, $t(11)=\frac{11-1}{2}=5$, so
\[
d_1:=t(x')-t(x)=5-3=2,\qquad x'-x=2d_1=4.
\]
Family update by the congruence rule $\varepsilon'\equiv \varepsilon+2d_1\pmod 6$: since $\varepsilon=1$, we get
\[
\varepsilon'\equiv 1 + 2\cdot 2 \equiv 5 \pmod 6,\quad \text{indeed } 11\equiv 5\ (\bmod 6).
\]

\medskip\noindent
\textbf{Step 2 (the true odd image of $x'=11$).}
Compute $3x'+1=34=2^1\!\cdot 17$, so $k_2=v_2(3x'+1)=1$ and
\[
x''=\boldsymbol U(11)=\frac{3\cdot 11 + 1}{2^{k_2}}=\frac{34}{2}=17.
\]
Tags and drift: $t(11)=5$, $t(17)=\frac{17-1}{2}=8$, thus
\[
d_2:=t(x'')-t(x')=8-5=3,\qquad x''-x'=2d_2=6.
\]
Family update: $\varepsilon''\equiv \varepsilon' + 2d_2 \equiv 5 + 2\cdot 3 \equiv 5 \pmod 6$, and indeed $17\equiv 5\ (\bmod 6)$.

\medskip\noindent
\textbf{Drift/Tag check (additivity).}
Additivity gives $t(x'')-t(x)=d_1+d_2=2+3=5$, hence
\[
x''-x \;=\; 2\bigl(t(x'')-t(x)\bigr)\;=\;2\cdot 5=10,
\]
matching $17-7=10$.

\medskip\noindent
\textbf{Composition check via $\boldsymbol U$.}
Directly composing the odd map with the observed valuations $k_1=k_2=1$:
\[
x''=\boldsymbol U(\boldsymbol U(x))
= \frac{3\left(\frac{3x+1}{2^{k_1}}\right)+1}{2^{k_2}}
= \frac{3(3x+1)+2^{k_1}}{2^{k_1+k_2}}
= \frac{9x+3+2}{2^{2}}
= \frac{9x+5}{4}.
\]
For $x=7$ this yields $x''=\frac{9\cdot 7+5}{4}=\frac{68}{4}=17$, consistent with the step-by-step computation.
\end{example}

\begin{remark}[Two-step odd composition with explicit valuations]\label{rem:two-step-k1k2}
Let $x$ be odd and set $k_1:=v_2(3x+1)$, $x'=\dfrac{3x+1}{2^{k_1}}$.
Then set $k_2:=v_2(3x'+1)$ and $x''=\dfrac{3x'+1}{2^{k_2}}$. A direct calculation gives
\[
3x'+1 \;=\; \frac{3(3x+1)}{2^{k_1}} + 1 \;=\; \frac{9x+3+2^{k_1}}{2^{k_1}},
\qquad\Longrightarrow\qquad
x'' \;=\; \frac{9x+3+2^{k_1}}{2^{\,k_1+k_2}}.
\]
In particular, if $k_1=k_2=1$ (as in Example~\ref{ex:micro-assembly-real}), then
\[
x'' \;=\; \frac{9x+5}{4}.
\]
This identity is compatible with the additive tag law $t(x'')-t(x)=(t(x')-t(x))+(t(x'')-t(x'))$ and simply packages the two actual odd steps into one expression.
\end{remark}

\begin{definition}[Parameter space and affine image]\label{def:param-space}
Let $\mathcal{P}$ be the set of admissible tuples $\Theta=(\alpha,\beta,c,\delta,p,m;\varepsilon)$ with $\varepsilon\in\{1,5\}$.
Define the map
\[
\Phi:\ \mathcal{P}\to \mathrm{Aff}^+(\mathbb{Q}),\qquad
\Theta\ \mapsto\ (A(\Theta),B_\varepsilon(\Theta)),
\]
where $A(\Theta)=1+\tfrac{K}{3}$, $K=(2^{\alpha+6p}-3)4^p$, and
\[
B_1=4q_p-\tfrac{K}{3},\qquad B_5=10q_p-2-\tfrac{5K}{3},\qquad q_p=\tfrac{4^p-1}{3}.
\]
\end{definition}

\begin{remark}[Semigroup structure]\label{rem:semigroup}
Under composition in $\mathrm{Aff}^+(\mathbb{Q})$, the images $\Phi(\Theta)$ form a semigroup.
Two parameter tuples can be declared \emph{affine-equivalent} if they share the same pair $(A,B_\varepsilon)$; this quotient collapses different $(\alpha,\beta,c,\delta,p,m)$ that act identically on $x$.
\end{remark}


\begin{lemma}[Continuity and compactness in $(u,v)$]\label{lem:uv-topology}
Let $\mathcal{U}=\Phi(\mathcal{P})$ and map $(A,B)\mapsto (u,v)=(\log A,\ B/(A-1))$.
Then $\mathcal{U}$ is discrete in the product topology induced by integer parameters, but its $(u,v)$-image is relatively closed in $\mathbb{R}\times\mathbb{R}$ and composition is continuous:
\[
(u,v)\oplus(u',v')=(u+u',\ v' + e^{-u'} v).
\]
Moreover, for fixed $(\alpha,p)$ the set $\{(u,v): \varepsilon\in\{1,5\},\ \beta,c,\delta,m\ \text{admissible}\}$ lies on two vertical lines (same $u$, two $v$-values).
\end{lemma}

\begin{lemma}[Operator metric bounds]\label{lem:operator-bounds}
Let $T(x)=Ax+B$ and $S(x)=A'x+B'$. For $x\in[1,X]$,
\[
\sup_{x\in[1,X]} |T(x)-S(x)| \ \le\ |A-A'|\,X + |B-B'|.
\]
In particular, if two parameter tuples share the same $(\alpha,p)$ then $A=A'$ and the distance is controlled by $|B-B'|$, i.e.\ by family choice $\varepsilon$ and lower-order parameters.
\end{lemma}

\paragraph{In general;}
With \textsc{RouteStep} and \textsc{Compose} treated as black boxes, each assembly lemma reduces to:
(i) a solvable congruence (family targeting),
(ii) a monotone parameter search in $K$ (size targeting), and
(iii) an affine composition (gluing).
This replaces ad-hoc parity trees by a uniform call pattern that is compatible with $F_{\alpha,\beta,c}(p,m)$ and stable under $n$-step composition.


\subsection*{Summary and outlook}
This chapter collected a self-contained calculus for the odd layer based on the CRT tag
\(
t(x)=\frac{3x+1-4}{6}=\frac{x-1}{2}
\)
(Lemma~\ref{lem:crt-tag}). We proved:
\begin{itemize}
  \item An explicit one-step drift law for the tag and the state
        (Corollary~\ref{cor:crt-tag-difference} and Proposition~\ref{prop:crt-tag-difference-explicit}),
        with parameters $(\alpha,p)$ aligned to the row/lift transform $F_{\alpha,\beta,c}(p,m)$.
  \item Closed forms relating $x'$ and $x$ (Corollary~\ref{cor:xp-vs-x}) and the routing congruences that steer the next family $\varepsilon'\in\{1,5\}$.
  \item A linear potential viewpoint $V(x)=t(x)$ (Corollary~\ref{cor:potential}) and tight drift thresholds (Lemma~\ref{lem:drift-thresholds}, Corollary~\ref{cor:drift-bounds}).
  \item Additivity over multiple steps and an affine $n$-step composition (Proposition~\ref{prop:nstep-additive-drift}, Corollary~\ref{cor:nstep-affine}),
        together with a carry-based update of $(r,\varepsilon)$ (Lemma~\ref{lem:carry}) and residue targeting (Lemma~\ref{lem:route-target}).
\end{itemize}

\paragraph{Design knobs.}
The pair $(\varepsilon,p)$ acts as a routing control: \textbf{(e)} gives nonnegative drift; \textbf{(o)} admits the unique negative-drift entry at $(p,r)=(0,0)$ and is otherwise positive; $p$ amplifies drift geometrically via $4^p$, and $\alpha$ tunes the multiplier inside $K=(2^{\alpha+6p}-3)4^p$.

\paragraph{Algorithmic use.}
For fast evaluation or constructive routing, use Corollary~\ref{cor:fast-eval-steer} (with Algorithm~\ref{alg:routing}) to pick $(\varepsilon,\alpha,p)$ meeting a sign or magnitude target without computing $k=v_2(3x+1)$ explicitly.


% =========================================================
\section{Parameter table for the unified rows}
Each row in the table below is specified by integers \((\alpha,\beta,c)\) (underlying \(F_{\alpha,\beta,c}\)) and an output offset \(\delta\in\{1,5\}\) determined by the type’s second letter. For convenience we list them all; \(\texttt{*e}\Rightarrow\delta=1\), \(\texttt{*o}\Rightarrow\delta=5\).

\begin{table}[!htbp]
\centering
\caption{Row parameters \((\alpha,\beta,c,\delta)\). Keys: \(\mathrm{ee}j\leftrightarrow \Psi_j\), \(\mathrm{eo}j\leftrightarrow \psi_j\), \(\mathrm{oe}j\leftrightarrow \omega_j\), \(\mathrm{oo}j\leftrightarrow \Omega_j\).}
\label{tab:parameters-abc}
\begin{tabular}{@{}c c c c c c@{}}
\toprule
Row key & $(s,j)$ & type & $\alpha$ & $\beta$ & $c$ \ ( $\delta$)\\\midrule
$\mathrm{ee}0$ & $(\mathrm e,0)$ & \texttt{ee} & $2$ & $2$   & $-2$ \ (1)\\
$\mathrm{ee}1$ & $(\mathrm e,1)$ & \texttt{ee} & $4$ & $56$  & $-2$ \ (1)\\
$\mathrm{ee}2$ & $(\mathrm e,2)$ & \texttt{ee} & $6$ & $416$ & $-2$ \ (1)\\
$\mathrm{oe}0$ & $(\mathrm o,0)$ & \texttt{oe} & $3$ & $20$  & $-2$ \ (1)\\
$\mathrm{oe}1$ & $(\mathrm o,1)$ & \texttt{oe} & $1$ & $11$  & $-2$ \ (1)\\
$\mathrm{oe}2$ & $(\mathrm o,2)$ & \texttt{oe} & $5$ & $272$ & $-2$ \ (1)\\
\midrule
$\mathrm{eo}0$ & $(\mathrm e,0)$ & \texttt{eo} & $4$ & $8$   & $-8$ \ (5)\\
$\mathrm{eo}1$ & $(\mathrm e,1)$ & \texttt{eo} & $6$ & $224$ & $-8$ \ (5)\\
$\mathrm{eo}2$ & $(\mathrm e,2)$ & \texttt{eo} & $2$ & $26$  & $-8$ \ (5)\\
$\mathrm{oo}0$ & $(\mathrm o,0)$ & \texttt{oo} & $5$ & $80$  & $-8$ \ (5)\\
$\mathrm{oo}1$ & $(\mathrm o,1)$ & \texttt{oo} & $3$ & $44$  & $-8$ \ (5)\\
$\mathrm{oo}2$ & $(\mathrm o,2)$ & \texttt{oo} & $1$ & $17$  & $-8$ \ (5)\\
\bottomrule
\end{tabular}
\end{table}

% =========================================================
\section{Unified \pdfmath{p=0} table (straight substitution)}
We evaluate $m=\lfloor x/18\rfloor$ at each step and use
\[
F(0,m)=\frac{9m\,2^{\alpha}+\beta+c}{9},\qquad x'(m)=6F(0,m)+\delta,
\]
with the rows below (no further reindexing).

\begin{table}[!htbp]
\centering
\caption{Unified $p=0$ forms with $F(0,m)=\dfrac{9 m 2^{\alpha} + \beta + c}{9}$ and $x'(m)=6F(0,m)+\delta$.}
\label{tab:unified-F0-straight-xprime}
\begin{tabular}{@{}ccc l l@{}}
\toprule
$(s,j)$ & type & move & $F(0,m)$ & $x'(m)=6\,F(0,m)+\delta$ \\ \midrule
$(\mathrm{e},0)$ & \texttt{ee} & $\Psi_{0}$   & $4m$            & $24m + 1$ \\
$(\mathrm{e},1)$ & \texttt{ee} & $\Psi_{1}$   & $16m + 6$       & $96m + 37$ \\
$(\mathrm{e},2)$ & \texttt{ee} & $\Psi_{2}$   & $64m + 46$      & $384m + 277$ \\
$(\mathrm{o},0)$ & \texttt{oe} & $\omega_{0}$ & $8m + 2$        & $48m + 13$ \\
$(\mathrm{o},1)$ & \texttt{oe} & $\omega_{1}$ & $2m + 1$        & $12m + 7$ \\
$(\mathrm{o},2)$ & \texttt{oe} & $\omega_{2}$ & $32m + 30$      & $192m + 181$ \\
\midrule
$(\mathrm{e},0)$ & \texttt{eo} & $\psi_{0}$   & $16m$           & $96m + 5$ \\
$(\mathrm{e},1)$ & \texttt{eo} & $\psi_{1}$   & $64m + 24$      & $384m + 149$ \\
$(\mathrm{e},2)$ & \texttt{eo} & $\psi_{2}$   & $4m + 2$        & $24m + 17$ \\
$(\mathrm{o},0)$ & \texttt{oo} & $\Omega_{0}$ & $32m + 8$       & $192m + 53$ \\
$(\mathrm{o},1)$ & \texttt{oo} & $\Omega_{1}$ & $8m + 4$        & $48m + 29$ \\
$(\mathrm{o},2)$ & \texttt{oo} & $\Omega_{2}$ & $2m + 1$        & $12m + 11$ \\
\bottomrule
\end{tabular}
\end{table}

\paragraph{Routing by $M_K=3\cdot 2^K$.}
Odd residues split as
\[
\mathcal{E}_K=\{1+6t\pmod{M_K}\},\qquad \mathcal{O}_K=\{5+6t\pmod{M_K}\}.
\]
If $x\bmod M_K\in\mathcal{E}_K$ use an $e$–move (\(\Psi\) or \(\psi\)); if $x\bmod M_K\in\mathcal{O}_K$ use an $o$–move (\(\omega\) or \(\Omega\)).
The row’s \texttt{type} second letter is the output family of $x'$ and constrains the next symbol.

% =========================================================
\section{Row correctness, family pattern, and word semantics}

\begin{lemma}[Row correctness with $m=\lfloor x/18\rfloor$]
\label{lem:row-correctness}
Fix a row in Table~\ref{tab:unified-F0-straight-xprime} with parameters $(\alpha,\beta,c)$ and offset $\delta\in\{1,5\}$.
Set $k:=(\beta+c)/9\in\mathbb{Z}$, $F(0,m)=2^\alpha m+k$, and $x'(m)=6F(0,m)+\delta$.
For any odd input $x=18m+6j+p$ with $p\in\{1,5\}$ one has
\[
3x'(m)+1=2^{\alpha}x,
\qquad\text{hence}\qquad
U\!\bigl(x'(m)\bigr)=x.
\]
\end{lemma}

\begin{proof}
\leavevmode\par\noindent
\begin{itemize}[leftmargin=1.6em]
\item \textbf{Normal form for $x$.} Write $x=18m+6j+p_6$ with $m=\lfloor x/18\rfloor$, $j=\lfloor x/6\rfloor\bmod 3$, $p_6\in\{1,5\}$.
\item \textbf{One–step map.} With $k=(\beta+c)/9$: \(F(0,m)=2^\alpha m + k\), \(x'=6F(0,m)+\delta\).
\item \textbf{Compute $3x'+1$.} \(3x'+1 = 18\cdot 2^\alpha m + (18k+3\delta+1)\).
\item \textbf{Straight–substitution identity.} By construction, \(18k+3\delta+1=2^\alpha(6j+p_6)\), hence \(3x'+1=2^\alpha x\).
\item \textbf{Forward check.} Since $x$ is odd, \(\nu_2(3x'+1)=\alpha\), so \(U(x')=x\).
\end{itemize}
\end{proof}

\begin{example}[After Lemma~\ref{lem:row-correctness}]
Take $x=1$ (so $s=\mathrm e$, $m=0$, $j=0$) and the row $(\mathrm e,0)$ with token $\psi$.
From the table: $x'=96m+5=5$. Then $3x'+1=16=2^4=2^{\alpha}x$ with $\alpha=4$ for this row.
Thus $U(5)=1$.
\end{example}

\begin{lemma}[Family–pattern invariance under change of start]
\label{lem:family-pattern}
Let $W=\sigma_1\cdots\sigma_t\in\{\Psi,\psi,\omega,\Omega\}^*$ be admissible from some $x_0$ with $s(x_0)=S\in\{\mathrm e,\mathrm o\}$.
Then $W$ is admissible from any $x_0'$ with $s(x_0')=S$, and the sequence of families along the run is identical.
\end{lemma}

\begin{proof}
\leavevmode\par\noindent
\begin{itemize}[leftmargin=1.6em]
\item \textbf{Token-only transitions.}
\(\Psi:\mathrm e\!\to\!\mathrm e\), \(\psi:\mathrm e\!\to\!\mathrm o\),
\(\omega:\mathrm o\!\to\!\mathrm e\), \(\Omega:\mathrm o\!\to\!\mathrm o\).
\item \textbf{Start admissibility.} If the first token is an $e$–move (resp.\ $o$–move), it is admissible from any $\mathrm e$– (resp.\ $\mathrm o$–) start.
\item \textbf{Induction.} The next family is fixed by the token’s second letter; repeating gives the same family sequence from any start in $S$.
\end{itemize}
\end{proof}

\begin{example}[After Lemma~\ref{lem:family-pattern}]
Let $W=\psi\,\Omega$. Starting at $x_0=1$ ($\mathrm e$) gives the family pattern $\mathrm e\to \mathrm o\to \mathrm o$.
Starting at $x_0'=19$ ($\mathrm e$) yields the \emph{same} family pattern.
\end{example}

\begin{lemma}[Affine word form]
\label{lem:affine-word}
Let $W$ be admissible (routing by family, navigation by type).
Then there exist $A_W>0$, $B_W\in\mathbb{Z}$, and $\delta_W\in\{1,5\}$ such that
\[
x_W(m)=6\bigl(A_W\,m+B_W\bigr)+\delta_W,
\]
with $A_W=3\cdot 2^{\alpha(W)}$ (product of step multipliers), and $\delta_W$ the last row’s offset.
\end{lemma}

\begin{proof}
\leavevmode\par\noindent
\begin{itemize}[leftmargin=1.6em]
\item \textbf{One step is affine.} Each row acts as \(x\mapsto 6(2^\alpha m+k)+\delta\), affine in $m$.
\item \textbf{Composition.} Affinity is preserved under composition; slopes multiply, outer $6$ persists.
\item \textbf{Collect exponents.} The slope is \(3\cdot 2^{\alpha(W)}\); the terminal offset is $\delta_W$.
\end{itemize}
\end{proof}

\begin{example}[After Lemma~\ref{lem:affine-word}]
For the one-token word $W=\psi$ (row $(\mathrm e,0)$), $x_W(m)=6(2^4 m + 0)+5=96m+5$, so $A_W=3\cdot 2^4$ and $\delta_W=5$.
\end{example}

\begin{lemma}[Forward monotonicity by row and lift]
For any certified step $3y+1=2^{\alpha+6p}x$ with $p\ge 0$, we have:
\[
U(y)\begin{cases}
>y,& \text{iff }\alpha=1\text{ and }p=0,\\
=y,& \text{iff }\alpha=2,\;p=0,\;y=1,\\
<y,& \text{otherwise.}
\end{cases}
\]
In the unified $p{=}0$ table, $\alpha=1$ occurs exactly for the rows $\omega_1$ and $\Omega_2$; hence these are the only forward-increasing cases.
\end{lemma}


% =========================================================
\section{Worked examples (unified table, straight substitution)}

\paragraph{Rule of use.} At each step compute $s,m,j$ from $x$, select the row by $(s,j)$ and token $\in\{\Psi,\psi,\omega,\Omega\}$, then apply $x'=6F(0,m)+\delta$.

\begin{example}[Word $\psi\,\Omega\,\omega\,\psi$ from $x_0=1$.]
\StepLine{Step 1}{1}{\mathrm{e}}{0}{0}{\psi}{\mathrm{e},0}{x'=96m+5=5}
\StepLine{Step 2}{5}{\mathrm{o}}{0}{0}{\Omega}{\mathrm{o},0}{x'=192m+53=53}
\StepLine{Step 3}{53}{\mathrm{o}}{2}{2}{\omega}{\mathrm{o},2}{x'=192m+181=565}
\StepLine{Step 4}{565}{\mathrm{e}}{31}{1}{\psi}{\mathrm{e},1}{x'=384m+149=12053}

\[
\boxed{\,1 \xrightarrow{\psi} 5 \xrightarrow{\Omega} 53 \xrightarrow{\omega} 565 \xrightarrow{\psi} 12053\, }.
\]
\end{example}

\begin{example}[Word $\psi\,\Omega\,\Omega\,\omega\,\psi$ from $x_0=1$.]
\StepLine{Step 1}{1}{\mathrm{e}}{0}{0}{\psi}{\mathrm{e},0}{x'=96m+5=5}
\StepLine{Step 2}{5}{\mathrm{o}}{0}{0}{\Omega}{\mathrm{o},0}{x'=192m+53=53}
\StepLine{Step 3}{53}{\mathrm{o}}{2}{2}{\Omega}{\mathrm{o},2}{x'=12m+11=35}
\StepLine{Step 4}{35}{\mathrm{o}}{1}{2}{\omega}{\mathrm{o},2}{x'=192m+181=373}
\StepLine{Step 5}{373}{\mathrm{e}}{20}{2}{\psi}{\mathrm{e},2}{x'=24m+17=497}

\[
\boxed{\,1 \xrightarrow{\psi} 5 \xrightarrow{\Omega} 53 \xrightarrow{\Omega} 35
\xrightarrow{\omega} 373 \xrightarrow{\psi} 497\, }.
\]
\end{example}


\section{Algebraic completeness of rows and \texorpdfstring{$p$}{p}–lifts}
\label{sec:algebraic-completeness}

A classical description of all odd preimages of an odd $x$ under the accelerated map $U$ is
\[
y_n \;=\; \frac{2^{n}x-1}{3},\qquad n\ge 1,
\]
with a parity restriction on $n$ determined by $x\bmod 3$. We record this and then show that each such $y_n$ is realized by our unified row with a suitable column–lift $p$.

\begin{lemma}[Odd preimages under $U$]\label{lem:odd-preimages}
Let $x$ be odd with $x\equiv 1,5\pmod 6$, and let $n\ge 1$. Define $y_n=(2^{n}x-1)/3$.
Then $y_n$ is an odd integer and $U(y_n)=x$ if and only if
\[
2^n x \equiv 1 \pmod 3\quad\Longleftrightarrow\quad
\begin{cases}
n\ \text{even}, & \text{if } x\equiv 1\pmod 3\ \ (\text{i.e.\ }x\equiv 1\pmod 6),\\
n\ \text{odd},  & \text{if } x\equiv 2\pmod 3\ \ (\text{i.e.\ }x\equiv 5\pmod 6).
\end{cases}
\]
Moreover $\nu_2(3y_n+1)=n$ and hence $U(y_n)=(3y_n+1)/2^n=x$.
\end{lemma}

\begin{proof}
Since $2$ has order $2$ in $(\mathbb Z/3\mathbb Z)^\times$, one has $2^n\equiv 1$ (resp.\ $2$) mod $3$ exactly when $n$ is even (resp.\ odd). Thus $2^n x\equiv 1\pmod 3$ iff $n$ is even when $x\equiv 1\pmod 3$, and iff $n$ is odd when $x\equiv 2\pmod 3$. This is equivalent to integrality of $y_n$. Then
$3y_n+1=2^n x$, so $\nu_2(3y_n+1)=n$ (as $x$ is odd) and $U(y_n)=x$. Also $3y_n+1\equiv 4\pmod 6$ gives $y_n$ odd.
\end{proof}

Now compare with a unified row (with lift $p\ge 0$):
\[
x'_p \;=\; 6\!\left(2^{\alpha+6p}m + \frac{\beta\,64^{\,p}+c}{9}\right)+\delta,
\qquad \delta\in\{1,5\},
\]
which satisfies the per–step identity
\[
3x'_p+1 \;=\; 2^{\alpha+6p}\,x,\qquad x=18m+6j+p_6,\ \ p_6=\begin{cases}1,&s=\mathrm e,\\[2pt]5,&s=\mathrm o.\end{cases}
\]

\begin{proposition}[Row/lift completeness for one–step preimages]\label{prop:row-lift-completeness}
Fix $x$ odd with $x\equiv 1,5\pmod 6$, and let $n\ge 1$ have the parity prescribed by Lemma~\ref{lem:odd-preimages}. Then there exists a row $(s,j,\alpha,\beta,c,\delta)$ with $s=s(x)$ and a lift $p\ge 0$ such that $n=\alpha+6p$ and, for $m=\lfloor x/18\rfloor$,
\[
y \;=\; 6\!\left(2^{\alpha+6p} m + \frac{\beta\,64^{\,p}+c}{9}\right)+\delta
\quad\text{equals}\quad
\frac{2^{n}x-1}{3}.
\]
In particular, $U(y)=x$ and \emph{every} admissible $y_n$ from Lemma~\ref{lem:odd-preimages} arises from some (row,\ $p$) in the unified scheme.
\end{proposition}

\begin{proof}[Proof idea]
Choose $s=s(x)$ and pick any row in that family; its exponent $\alpha$ has the same parity as $n$ (even for $s=\mathrm e$, odd for $s=\mathrm o$). Set $p=(n-\alpha)/6\in\mathbb Z_{\ge 0}$. The row identity gives $3y+1=2^{\alpha+6p}x=2^{n}x$, so $y=(2^{n}x-1)/3$. Row integrality holds since $64\equiv 1\pmod 9$ and $\beta+c\equiv 0\pmod 9$, making $(\beta\,64^{\,p}+c)/9\in\mathbb Z$.
\end{proof}

\begin{corollary}[Completeness of the lifted inverse calculus]\label{cor:completeness}
For each odd $x\equiv 1,5\pmod 6$, the set of odd preimages under $U$ is
\[
\bigl\{\,y_n=(2^{n}x-1)/3:\ n\ge 1,\ n\equiv 0\ (\mathrm{mod}\ 2)\ \text{if}\ x\equiv 1\pmod 6,\
n\equiv 1\ (\mathrm{mod}\ 2)\ \text{if}\ x\equiv 5\pmod 6\,\bigr\}.
\]
Every such $y_n$ is realized by a unified row at some lift $p$ with $n=\alpha+6p$. Hence the row family together with the $p$–lifts is algebraically complete for one–step odd preimages of $U$.
\end{corollary}



% =========================================================
\section{Row-level invariance and many realizations}

\begin{lemma}[One-step row-level invariance within a $54$-residue class]
\label{lem:row-invariance-54}
Let $x,\tilde x$ be odd with $x\equiv \tilde x \pmod{54}$. Write
\[
x=18m+6j+p,\qquad \tilde x=18\tilde m+6\tilde j+\tilde p,
\]
with $p,\tilde p\in\{1,5\}$, $j,\tilde j\in\{0,1,2\}$. Then
\[
p=\tilde p,\qquad j=\tilde j,\qquad \tilde m\equiv m \pmod{3}.
\]
Fix any admissible row $(s,j)$ and let $(\alpha,k,\delta)$ be its parameters, with the update
\[
x' \;=\; 6\bigl(2^\alpha m + k\bigr)+\delta,\qquad
\tilde x' \;=\; 6\bigl(2^\alpha \tilde m + k\bigr)+\delta.
\]
Then:
\begin{enumerate}[label=(\roman*),itemsep=2pt]
\item The output families coincide: $x'\equiv \tilde x' \equiv \delta \pmod{6}$.
\item The \emph{next} index matches:
\[
j' \;:=\; \Big\lfloor \frac{x'}{6}\Big\rfloor \bmod 3
\;=\;
\Big(2^\alpha m + k\Big)\bmod 3
\;=\;
\Big(2^\alpha \tilde m + k\Big)\bmod 3
\;=:\; \tilde j'.
\]
\end{enumerate}
\end{lemma}

\begin{proof}
\leavevmode\par\noindent
\begin{itemize}[leftmargin=1.6em]
\item $x\equiv \tilde x\pmod{54}$ gives $x\equiv \tilde x\pmod{6}$ and $\lfloor x/6\rfloor\equiv \lfloor \tilde x/6\rfloor\pmod{3}$, hence $p=\tilde p$ and $j=\tilde j$.
\item Also $\tilde m-m=\lfloor \tilde x/18\rfloor-\lfloor x/18\rfloor$ is a multiple of $3$, i.e.\ $\tilde m\equiv m\pmod{3}$.
\item Since $\lfloor x'/6\rfloor=2^\alpha m+k$, we get $(2^\alpha m+k)\equiv(2^\alpha\tilde m+k)\pmod 3$ and therefore $j'=\tilde j'$.
\end{itemize}
\end{proof}

\begin{example}[After Lemma~\ref{lem:row-invariance-54}]
Take $x=1$ and $\tilde x=55$ ($\equiv 1 \pmod{54}$): both have $s=\mathrm e$, $j=0$. Under the first token $\psi$ (row $(\mathrm e,0)$), each maps to an $o$-family number with the same next index $j'=0$, so the next row selection (for a fixed token) agrees.
\end{example}

\begin{remark}[Caution: persistence beyond one step]
Lemma~\ref{lem:row-invariance-54} aligns the \emph{next} $(s,j)$ after one identical row.
At the second step $m'$ and $\tilde m'$ can differ by $2^\alpha$ multiples that may not be $0\pmod 3$, so $j''$ may diverge unless stronger congruences hold (e.g.\ modulo $162$). The family pattern remains identical by Lemma~\ref{lem:family-pattern}.
\end{remark}

\begin{corollary}[Infinite integer realizations of a fixed word]
\label{cor:infinitely-many}
Let $W$ be admissible from some start with family $S$. Then there are infinitely many odd \(x_0\) for which the integer sequence driven by $W$ is well-defined and certified by $U(x')=x$ at every step. Moreover, for any $K\ge 3$ and odd residue $r\bmod M_K$ with terminal family matching $r\bmod 6$, the congruence in $m$ has infinitely many solutions, yielding infinitely many realizations with $x_W(m)\equiv r\pmod{M_K}$.
\end{corollary}

\begin{proof}
\leavevmode\par\noindent
\begin{itemize}[leftmargin=1.6em]
\item Varying $m$ in the first step gives infinitely many outputs.
\item Proceeding by the fixed tokens remains valid via routing/type; each step satisfies Lemma~\ref{lem:row-correctness}.
\item For fixed $K$, the linear congruence modulo $2^{K-1}$ has infinitely many solutions in $m$.
\end{itemize}
\end{proof}

\begin{example}[After Corollary~\ref{cor:infinitely-many}]
For $W=\psi$ and $K=3$, $x_W(m)=96m+5\equiv 5\pmod{24}$ for all $m$. Thus infinitely many $x_0$ realize the same residue class $5\bmod 24$.
\end{example}

\begin{theorem}[Backward–uniqueness criterion at $p=0$]
\label{thm:backward-uniqueness}
Fix the $p=0$ inverse table. Let $x_\mathrm{tar}$ be an odd integer. Suppose there exists a finite chain
\[
x_0,\;x_1,\;\dots,\;x_L=x_\mathrm{tar}
\]
such that for each $k=1,\dots,L$ there is \emph{exactly one} table row $R_k$ and an integer $m_{k-1}=\lfloor x_{k-1}/18\rfloor$ with
\[
x_k \;=\; A_{R_k}\,m_{k-1}+B_{R_k},
\]
and the admissibility constraints (family $e/o$ and router $j=\lfloor x_{k-1}/6\rfloor\bmod 3$) required by $R_k$ hold at $x_{k-1}$. Then:
\begin{enumerate}
\item The word $W:=R_1R_2\cdots R_L$ (read forward) is the \emph{unique} $p=0$ word that maps $x_0$ to $x_\mathrm{tar}$ via the certified inverse calculus.
\item Its length $L$ is minimal: there is no shorter $p=0$ word taking $x_0$ to $x_\mathrm{tar}$.
\end{enumerate}
\end{theorem}

\begin{proof}[Proof sketch]
Uniqueness: by assumption, at each backward step $x_k$ admits a single admissible predecessor $x_{k-1}$ and a single row $R_k$ realizing $x_k=A_{R_k}m_{k-1}+B_{R_k}$ with the required $(e/o,j)$ of $x_{k-1}$. Thus the predecessor is determined, and induction forces a single backward chain and hence a single forward word.

Minimality: if a shorter word existed, tracing it backward would produce a contradiction with the “exactly one admissible predecessor” property at the first index where it diverges from the forced chain.
\end{proof}
\begin{corollary}[Uniqueness of the $p=0$ path $1\to 49$]
\label{cor:unique-1-to-49}
In the $p=0$ table, the unique word sending $x_0=1$ to $x_\mathrm{tar}=49$ is
\[
\boxed{\ \psi,\;\omega,\;\psi,\;\Omega,\;\omega,\;\Psi,\;\Psi\ }\!,
\]
with value path
\[
1 \xrightarrow{\psi_0} 5 \xrightarrow{\omega_0} 13 \xrightarrow{\psi_2} 17
\xrightarrow{\Omega_2} 11 \xrightarrow{\omega_1} 7 \xrightarrow{\Psi_1} 37
\xrightarrow{\Psi_0} 49.
\]
\end{corollary}

\begin{proof}
Work backward and note that at each target $x$ there is exactly one admissible predecessor:

\smallskip
\noindent
\textbf{(i) $49$}: Only $\Psi_0$ fits $24m+1=49\Rightarrow m=2$, and among $m\!=\!2$ $e$-family candidates $\{37,43,49\}$ only $x\!=\!37$ has $j{=}\lfloor x/6\rfloor\bmod 3=0$ needed by $\Psi_0$. So $37\!\xrightarrow{\Psi_0}\!49$.

\noindent
\textbf{(ii) $37$}: Only $\Psi_1$ fits $96m+37=37\Rightarrow m=0$, and among $m\!=\!0$ $e$-family $\{1,7,13\}$ only $x\!=\!7$ has $j=1$ required by $\Psi_1$. So $7\!\xrightarrow{\Psi_1}\!37$.

\noindent
\textbf{(iii) $7$}: Only $\omega_1$ fits $12m+7=7\Rightarrow m=0$, and among $m\!=\!0$ $o$-family $\{5,11,17\}$ only $x\!=\!11$ has $j=1$. So $11\!\xrightarrow{\omega_1}\!7$.

\noindent
\textbf{(iv) $11$}: Only $\Omega_2$ fits $12m+11=11\Rightarrow m=0$, and among $\{5,11,17\}$ only $x\!=\!17$ has $j=2$. So $17\!\xrightarrow{\Omega_2}\!11$.

\noindent
\textbf{(v) $17$}: Only $\psi_2$ fits $24m+17=17\Rightarrow m=0$, and among $m\!=\!0$ $e$-family $\{1,7,13\}$ only $x\!=\!13$ has $j=2$. So $13\!\xrightarrow{\psi_2}\!17$.

\noindent
\textbf{(vi) $13$}: Only $\omega_0$ fits $48m+13=13\Rightarrow m=0$, and among $\{5,11,17\}$ only $x\!=\!5$ has $j=0$. So $5\!\xrightarrow{\omega_0}\!13$.

\noindent
\textbf{(vii) $5$}: Only $\psi_0$ fits $96m+5=5\Rightarrow m=0$, and among $\{1,7,13\}$ only $x\!=\!1$ has $j=0$. So $1\!\xrightarrow{\psi_0}\!5$.

Each step is forced; Theorem~\ref{thm:backward-uniqueness} applies, proving uniqueness and minimality.
\end{proof}


\section{Row design and the forward identity}

We parametrize each unified row by $(\alpha,\beta,c,\delta)$ and use
\[
F(p,m)=\frac{(9m\,2^{\alpha}+\beta)\,64^{\,p}+c}{9},\qquad
x'_p \;=\; 6F(p,m)+\delta,
\]
with input written in normal form \(x=18m+6j+p_6\) where \(j\in\{0,1,2\}\) and \(p_6\in\{1,5\}\).
The case \(p=0\) reduces to \(F(0,m)=2^\alpha m + k\) where \(k:=(\beta+c)/9\in\mathbb Z\).

\begin{lemma}[Row design constraints]\label{lem:row-design}
Suppose a row with fixed \((\alpha,\beta,c,\delta)\) satisfies
\begin{equation}\label{eq:row-constraints}
\beta \;=\; 2^{\alpha-1}(6j+p_6),
\qquad
c \;=\; -\,\frac{3\delta+1}{2},
\qquad
k=\frac{\beta+c}{9}\in\mathbb Z.
\end{equation}
Then for every odd input \(x=18m+6j+p_6\) one has the forward identity
\[
3x'_p+1 \;=\; 2^{\alpha+6p}\,x \qquad\text{for all } p\ge 0,
\]
hence \(U(x'_p)=x\).
\end{lemma}

\begin{proof}
Compute
\[
x'_p
=6\!\left(2^{\alpha+6p}m+\frac{\beta\,64^{\,p}+c}{9}\right)+\delta
\quad\Rightarrow\quad
3x'_p+1
=18\cdot 2^{\alpha+6p}m+\left(2\beta\,64^{\,p}+2c+3\delta+1\right).
\]
With \(c=-(3\delta+1)/2\) the constant cancels, giving \(3x'_p+1=18\cdot2^{\alpha+6p}m+2\beta\,64^{\,p}\).
Using \(\beta=2^{\alpha-1}(6j+p_6)\) and \(64^{\,p}=2^{6p}\),
\[
2\beta\,64^{\,p}=2^{\alpha}(6j+p_6)\,2^{6p}=2^{\alpha+6p}(6j+p_6).
\]
Thus \(3x'_p+1=2^{\alpha+6p}(18m+6j+p_6)=2^{\alpha+6p}x\). Since \(x\) is odd, \(\nu_2(3x'_p+1)=\alpha+6p\) and \(U(x'_p)=x\).
\end{proof}

\begin{remark}[Integrality]
Because \(64\equiv 1\pmod 9\), \(\beta\,64^{\,p}+c\equiv \beta+c\pmod 9\); hence \(F(p,m)\in\mathbb Z\) whenever \(k=(\beta+c)/9\in\mathbb Z\). This is enforced row-by-row by \eqref{eq:row-constraints}.
\end{remark}

\begin{proposition}[Checklist for a table row]
To certify a row, it suffices to exhibit integers \((\alpha,\beta,c,\delta)\) and \((j,p_6)\) with \(j\in\{0,1,2\}\), \(p_6\in\{1,5\}\) so that \eqref{eq:row-constraints} holds. Then Lemma~\ref{lem:row-design} implies \(3x'_p+1=2^{\alpha+6p}x\) for all \(p\ge 0\).
\end{proposition}


% =========================================================
\section{Super-families via \pdfmath{p}–lift and a \pdfmath{p=1} table}

For any $p\ge 0$, each row lifts to
\[
F_p(0,m)=\frac{(9m\,2^{\alpha}+\beta)\,64^{\,p}+c}{9}
=2^{\alpha+6p}m+\frac{\beta\,64^{\,p}+c}{9},\qquad
x'_p=6F_p(0,m)+\delta.
\]

\begin{lemma}[Row correctness with $p$–lift]
\label{lem:row-correctness-p}
For any admissible row $(\alpha,\beta,c,\delta)$ and any $p\ge 0$, if $x=18m+6j+p_6$ ($p_6\in\{1,5\}$), then
\[
3x'_p+1 \;=\; 2^{\alpha+6p}\,x,
\qquad\text{so}\qquad
U(x'_p)=x.
\]
\end{lemma}

\begin{proof}
\leavevmode\par\noindent
\begin{itemize}[leftmargin=1.6em]
\item Expand \(3x'_p+1=18\cdot 2^{\alpha+6p}m + \bigl(18\cdot\tfrac{64^p\beta + c}{9}+3\delta+1\bigr)\).
\item Using $64\equiv 1\pmod 9$ and the $p{=}0$ identity, the bracket equals $2^{\alpha+6p}(6j+p_6)$.
\item Thus $3x'_p+1=2^{\alpha+6p}x$ and $U(x'_p)=x$.
\end{itemize}
\end{proof}

\begin{example}[After Lemma~\ref{lem:row-correctness-p}]
Row $(\mathrm e,0)$ with $\Psi_0$ has $\alpha=2,\beta=2,c=-2,\delta=1$. For $p=1$, $F(1,m)=256m+14$, so with $x=1$ ($m=0$) we get $x'_1=85$. Then $3\cdot 85+1=256=2^8=2^{\alpha+6}\cdot 1$.
\end{example}

\begin{corollary}[Words with $p$–lift]
\label{cor:word-p}
If a word $W$ is admissible at $p{=}0$, then its $p$–lifted version is admissible and has
\[
x_{W,p}(m)=6\bigl(A_W\,2^{6p}m+B_{W,p}\bigr)+\delta_W.
\]
Thus the $2$–power in the forward identity gains $6p$ per step; padding in the $p{=}0$ world emulates working at $p>0$.
\end{corollary}

\begin{proof}
\leavevmode\par\noindent
\begin{itemize}[leftmargin=1.6em]
\item Apply Lemma~\ref{lem:row-correctness-p} stepwise; each step multiplies the $2$–power by $2^6$.
\item The type (\(\mathrm{*e}\) vs \(\mathrm{*o}\)) and $\delta_W$ are unchanged, so routing/navigation stays the same.
\item The affine form accumulates the additional $2^{6p}$ in the slope.
\end{itemize}
\end{proof}

\begin{example}[After Corollary~\ref{cor:word-p}]
For $W=\psi$ (one step, slope factor $2^4$ at $p=0$), the $p=1$ lift has slope $2^{4+6}=2^{10}$, i.e.\ $x_{W,1}(m)=6(2^{10}m+\cdots)+5$.
\end{example}

\begin{table}[!htbp]
\centering
\caption{Unified $p=1$ forms with $F(1,m)=\dfrac{(9 m 2^{\alpha} + \beta)\,64 + c}{9}$ and $x'_1(m)=6F(1,m)+\delta$.}
\label{tab:unified-F1-straight-xprime}
\begin{tabular}{@{}ccc l l@{}}
\toprule
$(s,j)$ & type & move & $F(1,m)$ & $x'_1(m)=6\,F(1,m)+\delta$ \\ \midrule
$(\mathrm{e},0)$ & \texttt{ee} & $\Psi_{0}$   & $256m + 14$    & $1536m + 85$ \\
$(\mathrm{e},1)$ & \texttt{ee} & $\Psi_{1}$   & $1024m + 398$  & $6144m + 2389$ \\
$(\mathrm{e},2)$ & \texttt{ee} & $\Psi_{2}$   & $4096m + 2958$ & $24576m + 17749$ \\
$(\mathrm{o},0)$ & \texttt{oe} & $\omega_{0}$ & $512m + 142$   & $3072m + 853$ \\
$(\mathrm{o},1)$ & \texttt{oe} & $\omega_{1}$ & $128m + 78$    & $768m + 469$ \\
$(\mathrm{o},2)$ & \texttt{oe} & $\omega_{2}$ & $2048m + 1934$ & $12288m + 11605$ \\
\midrule
$(\mathrm{e},0)$ & \texttt{eo} & $\psi_{0}$   & $1024m + 56$   & $6144m + 341$ \\
$(\mathrm{e},1)$ & \texttt{eo} & $\psi_{1}$   & $4096m + 1592$ & $24576m + 9557$ \\
$(\mathrm{e},2)$ & \texttt{eo} & $\psi_{2}$   & $256m + 184$   & $1536m + 1109$ \\
$(\mathrm{o},0)$ & \texttt{oo} & $\Omega_{0}$ & $2048m + 568$  & $12288m + 3413$ \\
$(\mathrm{o},1)$ & \texttt{oo} & $\Omega_{1}$ & $512m + 312$   & $3072m + 1877$ \\
$(\mathrm{o},2)$ & \texttt{oo} & $\Omega_{2}$ & $128m + 120$   & $768m + 725$ \\
\bottomrule
\end{tabular}
\end{table}

% =========================================================
\subsection*{Mixing the column parameter \pdfmath{p} stepwise (``mixed-\pdfmath{p}'' words)}

At any step you may use the $p$–lift of a row (possibly with a different $p$ than in the previous step):
\[
F(p,m)=\frac{(9m\,2^{\alpha}+\beta)\,64^{\,p}+c}{9},\qquad
x' \;=\; 6\,F(p,m)+\delta,
\]
where $(\alpha,\beta,c,\delta)$ are the fixed parameters of that row in the unified table and $p\in\mathbb{Z}_{\ge 0}$ is chosen \emph{for that step only}. This preserves both admissibility and the odd‐forward identity.

\begin{lemma}[Step correctness under mixed-$p$]
\label{lem:mixedp-step}
For any odd input \(x=18m+6j+p_6\) with \(p_6\in\{1,5\}\), any admissible row, and any \(p\ge 0\),
\[
3x'+1 \;=\; 2^{\alpha+6p}\,x \qquad\Rightarrow\qquad U(x')=x.
\]
\begin{proof}[Proof outline]
\leavevmode\par\noindent
\begin{itemize}[leftmargin=1.6em]
\item Expand \(3x'+1=18\cdot 2^{\alpha+6p}m+\bigl(18\cdot\tfrac{64^p\beta+c}{9}+3\delta+1\bigr)\).
\item Because \(64\equiv 1\pmod 9\), \(\tfrac{64^p\beta+c}{9}\in\mathbb{Z}\) and the bracket equals \(2^{\alpha+6p}(6j+p_6)\).
\item Hence \(3x'+1=2^{\alpha+6p}x\), so \(U(x')=x\).
\end{itemize}
\end{proof}
\end{lemma}

\begin{lemma}[Routing and type are $p$–invariant]
\label{lem:mixedp-routing}
For a fixed row, the \texttt{type} (\texttt{ee}, \texttt{eo}, \texttt{oe}, \texttt{oo}) and the offset $\delta\in\{1,5\}$ do not depend on $p$. Hence routing/type constraints are unchanged under mixed-$p$ evaluation.
\end{lemma}

\begin{proposition}[Affine form for mixed-$p$ words]
\label{prop:mixedp-word}
Let \(W=\sigma_1\cdots\sigma_t\) be admissible and choose per‐step lifts \(p_i\ge 0\).
Then
\[
x_{W,\vec p}(m)\;=\;6\bigl(A_{W,\vec p}\,m+B_{W,\vec p}\bigr)+\delta_W,\qquad
A_{W,\vec p}=3\cdot 2^{\sum_{i}^t(\alpha_i+6p_i)},
\]
where \(\alpha_i\) is the row exponent used at step \(i\) and \(\delta_W\) is the last row’s offset.
\begin{proof}[Proof sketch]
\leavevmode\par\noindent
\begin{itemize}[leftmargin=1.6em]
\item Compose the one–step affine maps \(x\mapsto 6(2^{\alpha_i+6p_i}m+k_{p_i})+\delta_i\).
\item Slopes multiply and the outer \(6\) carries through; the last \(\delta\) survives.
\end{itemize}
\end{proof}
\end{proposition}

\begin{remark}[Parity caveat for $p\ge 1$]
The step constant is \(k_p=\dfrac{64^p\beta+c}{9}\). For all rows and \(p\ge 1\), \(k_p \equiv k+\beta\pmod 2\) with \(k=(\beta+c)/9\); in many rows this is even. Single–step parity flips visible at \(p{=}0\) can vanish at \(p\ge 1\). Keep at least one $p{=}0$ odd–\(k\) row available if you need to toggle the intercept parity in a lifting congruence.
\end{remark}

% =========================================================
\section{Same–family padding as \emph{steering} (unified notion)}
\begin{definition}[Same–family padding / steering gadget]
A short admissible word $P$ with overall type $s\!\to\! s$ (i.e.\ \texttt{ee} if $s=\mathrm e$,
\texttt{oo} if $s=\mathrm o$) is a \emph{steering gadget}. Appending $P$ to a word $W$ preserves
the terminal family while giving control over:
\begin{itemize}[leftmargin=1.6em]
\item the $2$–adic slope (raising $v_2(A)$ in the affine form), and
\item the intercept parity \(B_W\bmod 2\).
\end{itemize}
\end{definition}

\begin{lemma}[Steering lemma]\label{lem:steering}
Let $W$ be admissible with affine form \(x_W(m)=6(A_W m+B_W)+\delta_W\), \(A_W=3\cdot 2^{\alpha(W)}\).
There exist short same–family words \(P^{(0)},P^{(1)}\) (type \(s\!\to\! s\)) such that
\begin{itemize}[leftmargin=1.6em]
  \item \textbf{Slope boost.} \(x_{W\cdot P^{(\varepsilon)}}(m)=6(A' m+B'_\varepsilon)+\delta_W\) with
        \(A'=A_W\cdot 2^{d}\) for some \(d\ge 1\) (repeat gadgets to enlarge \(d\)).
  \item \textbf{Parity control.} \(B'_0\equiv B_W\pmod 2\) while \(B'_1\equiv B_W+1\pmod 2\).
\end{itemize}
Consequently, for any \(K\ge 3\) and target \(r\equiv \delta_W\pmod 6\), there is padded \(W^\ast\) and \(m\) with
\(x_{W^\ast}(m)\equiv r \pmod{M_K=3\cdot 2^K}\).
\end{lemma}

\begin{proof}
\leavevmode\par\noindent
\begin{itemize}[leftmargin=1.6em]
\item Appending a same–family row multiplies the slope by \(2^{\alpha_{\text{row}}}\!\ge 2\) and keeps \(\delta\); repeating boosts \(v_2(A)\).
\item Among same–family menus, at least one gadget changes \(B\bmod 2\) (via an odd one–step constant \(k\)); another preserves it.
\item With \(v_2(A)\) large enough and parity chosen, the linear congruence
      \(A m \equiv \frac{r-\delta_W}{6}-B \pmod{2^{K-1}}\) is solvable.
\end{itemize}
\end{proof}

\begin{remark}[Steering intuition]
The family (\(\bmod 6\)) is the lane; steering gadgets keep you in that lane and let you nudge the position in \(\bmod\,3\cdot 2^K\) until it matches the target residue.
\end{remark}

\subsection*{Gadget drills: parity toggle and mod-3 steering}

We illustrate the two steering knobs: (i) a parity flip on the intercept $B\bmod 2$ and (ii) setting $B\bmod 3$ to a prescribed value, while staying in the same terminal family.

\begin{example}[Parity flip in family $\mathrm o$]
Start with any word $W$ whose terminal family is $\mathrm o$ and affine form $x_W(m)=6(A m+B)+5$. Appending the single $\Omega_2$ row (type \texttt{oo}, $(\mathrm o,2)$, $x'=12m+11=6(2m+1)+5$) sends
\[
B\ \longmapsto\ B'\equiv 2B+1\pmod 2 \quad\text{(flip)}\!,
\]
and raises $v_2(A)$ by $+1$. Thus $W\cdot\Omega_2$ keeps terminal family $\mathrm o$, flips $B\bmod 2$, and increases divisibility by $2$ in the slope.
\end{example}

\begin{example}[Setting $B\bmod 3$ in family $\mathrm e$]
In family $\mathrm e$, the \texttt{ee} rows have
\[
\Psi_0:\ x'=24m+1=6(4m+0)+1,\qquad
\Psi_2:\ x'=384m+277=6(64m+46)+1.
\]
Modulo $3$, these update $B\mapsto B$ (for $\Psi_0$) and $B\mapsto B+1$ (for $\Psi_2$). Therefore, in at most two \texttt{ee} steps we can force $B'\equiv r\ (\bmod\ 3)$ for any chosen $r\in\{0,1,2\}$ while staying in family $\mathrm e$ and increasing $v_2(A)$.
\end{example}


% =========================================================
\section{Evolution of the index \(m\) along inverse words}
\label{sec:evolution-of-m}

Fix any admissible inverse word \(W=\sigma_0\sigma_1\cdots\sigma_{n-1}\) evaluated by the unified
table (with optional per–step lifts). At step \(t\) the selected row carries parameters
\((\alpha_t,\beta_t,c_t,\delta_t)\) and (parent) index \(j_t\in\{0,1,2\}\); write
\[
a_t:=2^{\alpha_t}\quad(\text{or }2^{\alpha_t+6p_t}\text{ if a lift }p_t\ge 0\text{ is used}),\qquad
k_{p_t}:=\frac{\beta_t\,64^{\,p_t}+c_t}{9}\in\mathbb Z ,
\]
and set \(b_t:=k_{p_t}\) for brevity (so at \(p_t=0\), \(b_t=(\beta_t+c_t)/9\)).

Let \(x_t\) denote the child at step \(t\) and \(m_t=\lfloor x_t/18\rfloor\) the corresponding
index; the unified update reads
\[
x_{t} \;=\; 6\bigl(a_t m_{t-1}+b_t\bigr)+\delta_t,\qquad
3x_t+1 \;=\; 2^{\alpha_t+6p_t}\,x_{t-1}.
\]
Dividing by \(6\) and then by \(3\) gives the index recurrence
\begin{equation}\label{eq:m-recurrence}
m_t \;=\; \frac{a_t m_{t-1}+b_t - j_t}{3},
\qquad
\text{where } j_t \equiv a_t m_{t-1}+b_t \pmod{3},\ j_t\in\{0,1,2\}.
\end{equation}

% --- drop-in box without extra packages ---
\noindent
\setlength{\fboxsep}{8pt}% padding
\setlength{\fboxrule}{0.6pt}% border thickness
\fbox{%
  \parbox{\dimexpr\linewidth-2\fboxsep-2\fboxrule\relax}{%
\paragraph{Note on deriving the index \(j_t\).}
By construction each step has the form
\[
x_t \;=\; 6\bigl(a_t\,m_{t-1}+b_t\bigr)+\delta_t,
\qquad \delta_t\in\{1,5\}.
\]
Hence
\[
\Big\lfloor \frac{x_t}{6}\Big\rfloor \;=\; a_t\,m_{t-1}+b_t,
\]
and, by definition of the per–step index,
\[
j_t \;:=\; \Big\lfloor \frac{x_t}{6}\Big\rfloor \bmod 3
\;=\; \bigl(a_t\,m_{t-1}+b_t\bigr)\bmod 3.
\]
Therefore the index recurrence follows by removing the residue modulo \(3\):
\[
m_t \;=\; \Big\lfloor \frac{x_t}{18}\Big\rfloor
= \Big\lfloor \frac{1}{3}\Big\lfloor \frac{x_t}{6}\Big\rfloor \Big\rfloor
= \frac{\bigl(a_t\,m_{t-1}+b_t\bigr)-\bigl((a_t\,m_{t-1}+b_t)\bmod 3\bigr)}{3}
= \frac{a_t\,m_{t-1}+b_t - j_t}{3}.
\]
  }%
}



\begin{definition}[Tail products]
For \(0\le t\le n-1\) set
\[
P_{n,t+1} \;:=\; \prod_{u=t+1}^{\,n-1} a_u \quad(\text{empty product }=1),
\qquad
A_n \;:=\; P_{n,0} \;=\; \prod_{u=0}^{\,n-1} a_u \;=\; 2^{\sum_{u=0}^{n-1}(\alpha_u+6p_u)} .
\]
\end{definition}

\begin{proposition}[Closed form for \(m_n\)]
\label{prop:mn-closed-form}
Unrolling \eqref{eq:m-recurrence} yields, for every \(n\ge 1\),
\begin{equation}\label{eq:mn-closed}
\boxed{\quad
m_n
\;=\;
\frac{A_n\,m_0}{3^n}
\;+\;
\sum_{t=0}^{n-1}\frac{P_{n,t+1}}{3^{\,n-t}}\,(b_t - j_t)
%\;=\;
%\frac{A_n\,m_0+\displaystyle\sum_{t=0}^{n-1} 3^{\,t}\,P_{n,t+1}\,(b_t-j_t)}{3^n}.
\quad}
\end{equation}
In particular \(m_n\) is affine in \(m_0\) with slope \(A_n/3^n\); all nonlinearity from table
navigation is isolated in the bounded corrections \(j_t\in\{0,1,2\}\).
\end{proposition}

\begin{proof}[Proof sketch]
From \eqref{eq:m-recurrence},
\(
3m_t=a_t m_{t-1} + (b_t-j_t).
\)
Iterating gives
\(
3^2 m_{t+1} = a_{t+1} a_t m_{t-1} + a_{t+1}(b_t-j_t) + (b_{t+1}-j_{t+1}),
\)
and so on. Induction on \(n\) yields \eqref{eq:mn-closed}.
\end{proof}

\begin{corollary}[Including per–step lifts \(p_t\)]
The formula \eqref{eq:mn-closed} remains valid when a lift \(p_t\ge 0\) is used at step \(t\):
replace \(a_t\) by \(2^{\alpha_t+6p_t}\) and \(b_t\) by \(k_{p_t}=\frac{\beta_t\,64^{p_t}+c_t}{9}\).
\end{corollary}

\paragraph{Special start \(x_0=1\).}
If \(x_0=1\), then \(m_0=\lfloor 1/18\rfloor=0\). Hence the first term in
\eqref{eq:mn-closed} vanishes and
\[
m_n \;=\; \sum_{t=0}^{n-1}\frac{P_{n,t+1}}{3^{\,n-t}}\,(b_t-j_t)
\;=\;
\frac{\displaystyle\sum_{t=0}^{n-1} 3^{\,t}\,P_{n,t+1}\,(b_t-j_t)}{3^n},
\]
so the index evolution is completely determined by the chosen rows (via \(a_t,b_t\)) and the
induced remainders \(j_t\).

\paragraph{Remarks.}
\begin{itemize}[leftmargin=1.4em]
\item The “mean-field” trajectory obtained by pretending \(j_t\equiv 0\) has the same form as
\eqref{eq:mn-closed} with \(b_t\) in place of \(b_t-j_t\); the actual \(m_n\) differs by a bounded
\(3\)-adic correction since each \(j_t\in\{0,1,2\}\).
\item When all steps are chosen from a fixed family and with fixed lift, \(a_t\equiv a\) and the tail products
are geometric: \(P_{n,t+1}=a^{\,n-1-t}\). Then \eqref{eq:mn-closed} simplifies to a single weighted
geometric sum in \(3\) and \(a\).
\end{itemize}

\begin{lemma}[Zero-start independence of the index sequence]
\label{lem:zero-start-independence}
Fix any admissible mixed-\(p\) word \(W=\sigma_1\cdots\sigma_n\), and write for each step
\[
a_u:=\alpha(\sigma_u)+6\,p_u,\qquad
k_u:=\frac{\beta(\sigma_u)\,64^{\,p_u}+c(\sigma_u)}{9}\in\mathbb Z
\qquad (u=1,\dots,n).
\]
Let \(m_{u+1}=\big\lfloor \tfrac{2^{a_u}\,m_u+k_u}{3}\big\rfloor\) be the per-step update of the index \(m\) (so \(m_u=\lfloor x_u/18\rfloor\) and \(x_{u+1}=6(2^{a_u}m_u+k_u)+\delta(\sigma_u)\)). If the start index is
\(m_0=0\) (equivalently \(x_0\in\{1,5\}\)), then the entire sequence \(m_1,\dots,m_n\) is uniquely determined by \(W\) alone. In particular, for \(m_0=0\) the values \(m_u\) are independent of any external parameter or target and depend only on the chosen tokens and lifts.
\end{lemma}

\begin{proof}
By definition,
\(m_{u+1}=\big\lfloor (2^{a_u}m_u+k_u)/3\big\rfloor\).
Given \(m_0=0\), \(m_1=\lfloor k_1/3\rfloor\) is determined by \((a_1,k_1)\), hence by \(\sigma_1\) and \(p_1\).
Inductively, if \(m_u\) is determined by \(\sigma_1,\dots,\sigma_u\) and \(p_1,\dots,p_u\), then so is \(m_{u+1}\).
Thus \(m_1,\dots,m_n\) depend only on \(W\) (the rows and lifts) and not on any additional choice. \qedhere
\end{proof}

\begin{remark}
A convenient explicit (floor-free) surrogate is obtained by ignoring the floors:
\[
\widetilde m_{u+1} \;=\; \frac{2^{a_u}\,\widetilde m_u+k_u}{3},\qquad \widetilde m_0=0,
\]
which solves to
\(
\widetilde m_n
= \sum_{t=1}^{n}\dfrac{\bigl(\prod_{r=t+1}^{n}2^{a_r}\bigr)k_t}{3^{\,n-t+1}}.
\)
Then \(m_n=\big\lfloor \widetilde m_n - \varepsilon_n\big\rfloor\) for some \(0\le \varepsilon_n<1\) coming from the nested floors. Either way, once \(m_0\) is fixed (in particular, \(m_0=0\)), the actual \(m_n\) is a function of \(W\) alone.
\end{remark}
\begin{corollary}
Starting from \(x_0\in\{1,5\}\) (so \(m_0=0\)), the sequence of indices \(m_u\) and hence the entire sequence of updates \(x_{u+1}=6(2^{a_u}m_u+k_u)+\delta(\sigma_u)\) is completely determined by the chosen word \(W\).
\end{corollary}

\subsection{Monotonicity of the index \(m_n\) and its relation to row exponents}

Recall that for a single certified inverse step with row parameters
\((\alpha,\beta,c,\delta)\) and lift \(p\ge 0\), the child is
\[
y \;=\; 6\!\left(2^{\alpha+6p} m + k_p\right)+\delta,
\qquad
k_p \;=\; \frac{\beta\,64^{\,p}+c}{9}\in\mathbb Z,
\]
where the parent’s index is \(m=\lfloor x/18\rfloor\) and \(\delta\in\{1,5\}\).
The next index (for the child) is therefore
\begin{equation}\label{eq:m-next}
m^{+}
\;=\;
\Big\lfloor \frac{y}{18}\Big\rfloor
\;=\;
\Big\lfloor \frac{2^{\alpha+6p}}{3}\,m \;+\; \frac{k_p}{3} \;+\; \frac{\delta}{18}\Big\rfloor.
\end{equation}
Set
\[
c(\alpha,p):=\frac{2^{\alpha+6p}}{3},
\qquad
b_p(\delta):=\frac{k_p}{3}+\frac{\delta}{18}\in\Bigl(\,\mathbb{Z}/3\,\Bigr)+\Bigl\{\tfrac{1}{18},\tfrac{5}{18}\Bigr\}.
\]
Then \eqref{eq:m-next} reads concisely as
\begin{equation}\label{eq:affine-m-update}
m^{+} \;=\; \big\lfloor c(\alpha,p)\,m + b_p(\delta)\big\rfloor.
\end{equation}

\begin{lemma}[Forward \(\boldsymbol{U}\) monotonicity vs.\ inverse \(m\) growth]
\label{lem:forward-vs-m}
For a single step with parameters \((\alpha,p)\):
\begin{enumerate}[leftmargin=1.4em]
\item If \(\alpha=1\) and \(p=0\) (the rows \(\omega_1\) and \(\Omega_2\)), then
\(c(\alpha,p)=\tfrac{2}{3}<1\).
In forward time this is exactly the unique case \(U(y)>y\);
in inverse time the index update \eqref{eq:affine-m-update} is a contraction up to a bounded offset:
\[
m^{+}\;=\;\Big\lfloor \tfrac{2}{3}m + b_0(\delta)\Big\rfloor
\;\le\; \tfrac{2}{3}m + b_0(\delta),
\]
hence \(m^{+}<m\) for all \(m > 3\,b_0(\delta)\).
\item If \(\alpha+6p\ge 2\) (all other rows, or any \(p\ge 1\)), then
\(c(\alpha,p)\ge \tfrac{4}{3}>1\).
In forward time these steps strictly decrease (\(U(y)<y\));
in inverse time the index typically grows linearly:
\[
m^{+}\;\ge\; \Big\lfloor \tfrac{4}{3}m + b_p(\delta)\Big\rfloor
\;\ge\; m+ \Big\lfloor \tfrac{m}{3}+b_p(\delta)\Big\rfloor,
\]
so for all sufficiently large \(m\) one has \(m^{+}\ge m+1\).
\end{enumerate}
\end{lemma}

\begin{proof}
The forward monotonicity trichotomy follows from \(U(y)=(3y+1)/2^{\alpha+6p}\) and elementary inequalities (cf.\ the “Forward monotonicity by row and lift” lemma). The statements about \(m^{+}\) are direct consequences of \eqref{eq:affine-m-update} with \(c(\alpha,p)=2^{\alpha+6p}/3\) and the bounds \(0<\delta/18<1/3\).
\end{proof}

\begin{remark}[Interpretation]
Lemma~\ref{lem:forward-vs-m} shows a clean \emph{duality}:
the only forward-increasing case (\(\alpha=1,p=0\)) is the only case in which the
inverse \(m\)-dynamics is (eventually) contracting; all other rows (or any \(p\ge 1\))
are forward-decreasing and produce an expansive \(m\)-update in inverse time.
Thus along an inverse word, the trend of the index \(m_n\) is governed just by the exponents \(\alpha_u+6p_u\) of the chosen rows.
\end{remark}

\paragraph{Two quick instantiations.}
\begin{itemize}[leftmargin=1.4em]
\item \emph{Row \(\omega_1\) (\(\alpha{=}1,p{=}0\), \(\delta{=}1\)).}
Here \(k=(\beta+c)/9=(11-2)/9=1\), so
\[
m^{+}=\Big\lfloor \tfrac{2}{3}m + \tfrac{1}{3} + \tfrac{1}{18}\Big\rfloor
=\Big\lfloor \tfrac{2}{3}m + \tfrac{7}{18}\Big\rfloor,
\]
hence \(m^{+}<m\) for all \(m\ge 3\).
\item \emph{Row \(\Psi_0\) (\(\alpha{=}2,p{=}0\), \(\delta{=}1\)).}
Here \(k=(2-2)/9=0\), so
\[
m^{+}=\Big\lfloor \tfrac{4}{3}m + \tfrac{1}{18}\Big\rfloor
\ge \Big\lfloor \tfrac{4}{3}m \Big\rfloor,
\]
which grows by at least \(1\) once \(m\ge 1\).
\end{itemize}

\noindent
Over a longer admissible word \(W=(\sigma_1,\ldots,\sigma_t)\) with per-step
exponents \(\alpha_u+6p_u\), iterating \eqref{eq:affine-m-update} yields
\[
m_{u+1}=\big\lfloor c_u\,m_u + b_u\big\rfloor,
\qquad
c_u=\frac{2^{\alpha_u+6p_u}}{3},\quad
b_u=\frac{k_{p_u}}{3}+\frac{\delta_u}{18},
\]
so the sign of \(\log c_u\) (i.e.\ whether \(\alpha_u+6p_u\) equals \(1\) or at least \(2\))
controls local contraction/expansion of \(m\) in inverse time, perfectly mirroring
the forward increase/decrease of \(U\) on that step.



% =========================================================

% =========================
% Deriving x_n from m_n with terminal-index cases
% =========================
\section{Recovering \texorpdfstring{$x_n$}{x\_n} from \texorpdfstring{$m_n$}{m\_n} and terminal index}


Let a legal word \(W\) of length \(n\) be fixed, with per–step parameters
\[
(\alpha_t,p_t,b_t,j_t,\delta_t)\qquad (t=0,1,\dots,n-1),
\]
where \(\alpha_t\in\{1,2,3,4,5,6\}\) (row exponent), \(p_t\in\mathbb{Z}_{\ge 0}\) (column lift), \(b_t=(\beta_t+c_t)/9\in\mathbb Z\) (row constant at \(p{=}0\)), \(j_t=\lfloor x_t/6\rfloor\bmod 3\) is the parent index before step \(t\), and \(\delta_t\in\{1,5\}\) encodes the \emph{child} family after step \(t\).
Define the cumulative two–power multiplier (read “from the future back to \(t\)”)
\[
P_{n,t+1}\;:=\;\prod_{u=t+1}^{n-1} 2^{\,\alpha_u+6p_u},
\qquad\text{with}\qquad P_{n,n}:=1 .
\]

\begin{proposition}[Closed form for \(m_n\)]
\label{prop:mn-closed}
With the notation above, the \(18\)–index after \(n\) steps is
\[
m_n \;=\; \sum_{t=0}^{n-1}\frac{P_{n,t+1}}{3^{\,n-t}}\,(b_t-j_t).
\]
\end{proposition}

Write the terminal family and index as
\[
\delta_n:=\delta_{\,n-1}\in\{1,5\},\qquad
j_n:=\Big\lfloor\frac{x_n}{6}\Big\rfloor \bmod 3\in\{0,1,2\}.
\]
By definition of our normal form,
\[
x_n \;=\; 18\,m_n + 6\,j_n + \delta_n .
\]

\begin{proposition}[Closed form for \(x_n\) and terminal-index cases]
\label{prop:xn-closed}
Substituting Proposition~\ref{prop:mn-closed} gives
\[
\boxed{\;
x_n \;=\; 18\sum_{t=0}^{n-1}\frac{P_{n,t+1}}{3^{\,n-t}}\,(b_t-j_t)\;+\;6j_n\;+\;\delta_n \; }.
\]
Equivalently, grouping constants,
\[
x_n \;=\; 6\left( 3\sum_{t=0}^{n-1}\frac{P_{n,t+1}}{3^{\,n-t}}\,(b_t-j_t)\;+\;j_n \right) + \delta_n .
\]
In particular, \(x_n\equiv \delta_n\pmod 6\) automatically. The additive contribution of the terminal index \(j_n\) is exactly \(6j_n\), which yields the following three cases:
\[
\begin{cases}
\textbf{Case } j_n=0: &
\displaystyle x_n \;=\; 18\sum_{t=0}^{n-1}\frac{P_{n,t+1}}{3^{\,n-t}}\,(b_t-j_t)\;+\;\delta_n ,\\[10pt]
\textbf{Case } j_n=1: &
\displaystyle x_n \;=\; 18\sum_{t=0}^{n-1}\frac{P_{n,t+1}}{3^{\,n-t}}\,(b_t-j_t)\;+\;6\;+\;\delta_n ,\\[10pt]
\textbf{Case } j_n=2: &
\displaystyle x_n \;=\; 18\sum_{t=0}^{n-1}\frac{P_{n,t+1}}{3^{\,n-t}}\,(b_t-j_t)\;+\;12\;+\;\delta_n .
\end{cases}
\]
\end{proposition}

\paragraph{Remarks.}
\begin{itemize}[leftmargin=1.4em]
\item The dependence on the intermediate indices \(j_t\) (\(t<n\)) appears only inside the sum through the differences \(b_t-j_t\).
\item The terminal family \(\delta_n\) (the second letter of the last row’s type) fixes \(x_n\bmod 6\); the terminal index \(j_n\) then selects the representative within that residue class modulo \(18\).
\item In the special seed \(x_0=1\) one has \(m_0=0\); the formulas above remain valid with the same \(P_{n,t+1}\) and \(b_t-j_t\).
\end{itemize}


\section{Geometric-series simplifications for \texorpdfstring{$m_n$}{m\_n}
(and \texorpdfstring{$x_n$}{x\_n})}

\label{sec:geom-series-mn}

Recall the index recurrence
\[
m_t \;=\; \frac{a_t\,m_{t-1}+b_t-j_t}{3},
\qquad a_t:=2^{\alpha_t+6p_t},
\]
which unrolls to
\[
m_n \;=\; \Biggl(\prod_{u=1}^{n}\frac{a_u}{3}\Biggr) m_0 \;+\;
\sum_{t=1}^{n}\Biggl(\prod_{u=t+1}^{n}\frac{a_u}{3}\Biggr)\frac{b_t-j_t}{3}. \tag{$\ast$}
\]

\begin{proposition}[Constant-row case: closed form via a geometric series]
\label{prop:constant-row-geom}
Suppose the same row (and lift \(p\)) is used at every step, i.e.\ \(a_t\equiv a\) and \(b_t-j_t\equiv q\) are constant.
Then
\[
m_n
\;=\;
\Bigl(\tfrac{a}{3}\Bigr)^{\!n} m_0
\;+\;
\frac{q}{3}\sum_{k=0}^{n-1}\Bigl(\tfrac{a}{3}\Bigr)^{\!k}.
\]
Hence, for \(a\neq 3\),
\[
m_n
\;=\;
\Bigl(\tfrac{a}{3}\Bigr)^{\!n} m_0
\;+\;
\frac{q}{3}\cdot
\frac{1-\bigl(\tfrac{a}{3}\bigr)^{n}}{1-\tfrac{a}{3}},
\qquad
\text{and if } a=3:\quad
m_n = m_0 + n\,\frac{q}{3}.
\]
\end{proposition}

\begin{proof}
With \(a_t\equiv a\) and \(b_t-j_t\equiv q\), the recurrence is linear:
\(m_t = \tfrac{a}{3}m_{t-1}+\tfrac{q}{3}\).
Unrolling yields the stated geometric sum.
\end{proof}

\begin{corollary}[From \(m_n\) to \(x_n\)]
\label{cor:xn-from-mn}
At time \(n\), \(x_n\) satisfies \(x_n = 18\,m_n + 6\,j_n + \delta_n\).
In the constant-row case of Prop.~\ref{prop:constant-row-geom}, if moreover
\(j_n\) and \(\delta_n\) are fixed (e.g.\ the row is \texttt{ee} or \texttt{oo} and the
family/index stay constant), then
\[
x_n
\;=\;
18\Biggl[
\Bigl(\tfrac{a}{3}\Bigr)^{\!n} m_0
\;+\;
\frac{q}{3}\cdot
\frac{1-\bigl(\tfrac{a}{3}\bigr)^{n}}{1-\tfrac{a}{3}}
\Biggr]
\;+\;
6j+\delta.
\]
When \(m_0=0\) (e.g.\ starting from \(x_0=1\)), this reduces to a pure geometric expression in \(n\).
\end{corollary}

\begin{remark}[Periodic or bounded coefficients]
If \(a_t\equiv a\) is constant but \(q_t:=b_t-j_t\) is \(L\)-periodic, then
\[
m_n \;=\; \Bigl(\tfrac{a}{3}\Bigr)^{\!n} m_0 \;+\;
\sum_{r=0}^{L-1}\frac{q_r}{3}\sum_{\substack{0\le k\le n-1\\k\equiv r\!\!\!\pmod L}}
\Bigl(\tfrac{a}{3}\Bigr)^{\!k},
\]
a sum of \(L\) geometric progressions. In the fully variable case,
\((\ast)\) still gives bounds by comparing \(\prod_{u=t+1}^n \tfrac{a_u}{3}\) to a dominating geometric sequence.
\end{remark}

\paragraph{Index recurrence with \(q\).}
From \(x_t = 6(a\,m_{t-1}+b)+\delta\) and \(m_t=\big\lfloor x_t/18\big\rfloor\), we obtain
\[
m_t \;=\; \frac{a\,m_{t-1}+b-j_t}{3}\,,
\]
where \(a=2^{\alpha+6p}\), \(b=k_p\), and \(j_t=\big\lfloor x_t/6\big\rfloor \bmod 3\).
To avoid notational conflict with the row parameter \(c\), define
\[
\boxed{\,q \;:=\; b - j_t \;=\; k_p - j_t,\qquad
k_p \;=\; \frac{\beta\,64^{\,p}+c_{\text{row}}}{9}\in\mathbb Z\,}
\]
so the recurrence reads
\[
\boxed{\,m_t \;=\; \frac{a\,m_{t-1}+q}{3}\,,\qquad a=2^{\alpha+6p}\,.}
\]

\paragraph{At \(p{=}0\).}
Here \(k_0=\dfrac{\beta+c_{\text{row}}}{9}\), hence \(q=k_0-j_t\).

% ===== Table: index recurrences for p = 0 =====
\begin{table}[!htbp]
\centering
\small
\caption{Index recurrence \(m_t=\frac{a\,m_{t-1}+q}{3}\) for each row at \(p{=}0\).
Here \(a=2^{\alpha}\), \(k_0=\frac{\beta+c}{9}\), \(q=k_0-j\).}
\label{tab:index-recurrences-p0}
\begin{tabular}{@{}l c c c c c l@{}}
\toprule
Row & $(s,j)$ & $\alpha$ & $a$ & $k_0$ & $q=k_0-j$ & Recurrence \\ \midrule
$\Psi_{0}$   & $(\mathrm e,0)$ & 2 & 4   & 0  & 0   & $m_t=\frac{4\,m_{t-1}+0}{3}$ \\
$\Psi_{1}$   & $(\mathrm e,1)$ & 4 & 16  & 6  & 5   & $m_t=\frac{16\,m_{t-1}+5}{3}$ \\
$\Psi_{2}$   & $(\mathrm e,2)$ & 6 & 64  & 46 & 44  & $m_t=\frac{64\,m_{t-1}+44}{3}$ \\
$\omega_{0}$ & $(\mathrm o,0)$ & 3 & 8   & 2  & 2   & $m_t=\frac{8\,m_{t-1}+2}{3}$ \\
$\omega_{1}$ & $(\mathrm o,1)$ & 1 & 2   & 1  & 0   & $m_t=\frac{2\,m_{t-1}+0}{3}$ \\
$\omega_{2}$ & $(\mathrm o,2)$ & 5 & 32  & 30 & 28  & $m_t=\frac{32\,m_{t-1}+28}{3}$ \\
\midrule
$\psi_{0}$   & $(\mathrm e,0)$ & 4 & 16  & 0  & 0   & $m_t=\frac{16\,m_{t-1}+0}{3}$ \\
$\psi_{1}$   & $(\mathrm e,1)$ & 6 & 64  & 24 & 23  & $m_t=\frac{64\,m_{t-1}+23}{3}$ \\
$\psi_{2}$   & $(\mathrm e,2)$ & 2 & 4   & 2  & 0   & $m_t=\frac{4\,m_{t-1}+0}{3}$ \\
$\Omega_{0}$ & $(\mathrm o,0)$ & 5 & 32  & 8  & 8   & $m_t=\frac{32\,m_{t-1}+8}{3}$ \\
$\Omega_{1}$ & $(\mathrm o,1)$ & 3 & 8   & 4  & 3   & $m_t=\frac{8\,m_{t-1}+3}{3}$ \\
$\Omega_{2}$ & $(\mathrm o,2)$ & 1 & 2   & 1  & $-1$& $m_t=\frac{2\,m_{t-1}-1}{3}$ \\
\bottomrule
\end{tabular}
\end{table}

% ===== Table: index recurrences for p = 1 =====
\begin{table}[!htbp]
\centering
\small
\caption{Index recurrence \(m_t=\frac{a\,m_{t-1}+q}{3}\) for each row at \(p{=}1\).
Here \(a=2^{\alpha+6}=64\cdot 2^{\alpha}\), \(k_1=\frac{\beta\cdot 64 + c}{9}\), \(q=k_1-j\).}
\label{tab:index-recurrences-p1}
\begin{tabular}{@{}l c c c c c l@{}}
\toprule
Row & $(s,j)$ & $\alpha$ & $a$ & $k_1$ & $q=k_1-j$ & Recurrence \\ \midrule
$\Psi_{0}$   & $(\mathrm e,0)$ & 2 & 256  & 14    & 14    & $m_t=\frac{256\,m_{t-1}+14}{3}$ \\
$\Psi_{1}$   & $(\mathrm e,1)$ & 4 & 1024 & 398   & 397   & $m_t=\frac{1024\,m_{t-1}+397}{3}$ \\
$\Psi_{2}$   & $(\mathrm e,2)$ & 6 & 4096 & 2958  & 2956  & $m_t=\frac{4096\,m_{t-1}+2956}{3}$ \\
$\omega_{0}$ & $(\mathrm o,0)$ & 3 & 512  & 142   & 142   & $m_t=\frac{512\,m_{t-1}+142}{3}$ \\
$\omega_{1}$ & $(\mathrm o,1)$ & 1 & 128  & 78    & 77    & $m_t=\frac{128\,m_{t-1}+77}{3}$ \\
$\omega_{2}$ & $(\mathrm o,2)$ & 5 & 2048 & 1934  & 1932  & $m_t=\frac{2048\,m_{t-1}+1932}{3}$ \\
\midrule
$\psi_{0}$   & $(\mathrm e,0)$ & 4 & 1024 & 56    & 56    & $m_t=\frac{1024\,m_{t-1}+56}{3}$ \\
$\psi_{1}$   & $(\mathrm e,1)$ & 6 & 4096 & 1592  & 1591  & $m_t=\frac{4096\,m_{t-1}+1591}{3}$ \\
$\psi_{2}$   & $(\mathrm e,2)$ & 2 & 256  & 184   & 182   & $m_t=\frac{256\,m_{t-1}+182}{3}$ \\
$\Omega_{0}$ & $(\mathrm o,0)$ & 5 & 2048 & 568   & 568   & $m_t=\frac{2048\,m_{t-1}+568}{3}$ \\
$\Omega_{1}$ & $(\mathrm o,1)$ & 3 & 512  & 312   & 311   & $m_t=\frac{512\,m_{t-1}+311}{3}$ \\
$\Omega_{2}$ & $(\mathrm o,2)$ & 1 & 128  & 120   & 118   & $m_t=\frac{128\,m_{t-1}+118}{3}$ \\
\bottomrule
\end{tabular}
\end{table}

\subsubsection*{Examples using the base $p=0$ table}

From the unified scheme, a single $p{=}0$ row with parameter $\alpha$ enforces the forward identity
\[
3x_{k+1}+1 \;=\; 2^{\alpha}\,x_k
\quad\Longleftrightarrow\quad
x_{k+1} \;=\; \frac{3}{2^{\alpha}}\,x_k \;+\; \frac{1}{2^{\alpha}}.
\]
Hence for $n$ consecutive steps that all use the \emph{same} $p{=}0$ row (so the exponent is constant $a_k\equiv \alpha$),
\begin{equation}\label{eq:geom-from-p0}
x_n \;=\; \Big(\frac{3}{2^{\alpha}}\Big)^{\!n}\,x_0
\;+\; \frac{1}{2^{\alpha}}\cdot \frac{1-\big(\frac{3}{2^{\alpha}}\big)^{\!n}}{1-\frac{3}{2^{\alpha}}}.
\end{equation}
If during those $n$ steps the residue form $x_k=18\,m_k+\beta_\mathrm{row}$ stays fixed (same $(s,j)$), then
\begin{equation}\label{eq:mn-from-p0}
m_n \;=\; \frac{x_n-\beta_\mathrm{row}}{18}.
\end{equation}
Below we instantiate \eqref{eq:geom-from-p0}–\eqref{eq:mn-from-p0} with exact entries from the base $p{=}0$ table.

\paragraph{Row \texttt{ee0} (from the $p{=}0$ table: $(s,j)=(\mathrm e,0)$, $\alpha=2$, $\beta=2$, $c=-2$, $\delta=1$).}
Here $x\equiv 1\pmod 6$ with $j=0$, so $x=18m+1$ (i.e., $\beta_\mathrm{row}=1$).
Using \eqref{eq:geom-from-p0} with $\alpha=2$ gives
\[
x_n \;=\; \Big(\frac{3}{4}\Big)^{\!n}x_0 \;+\; \frac{1}{4}\cdot \frac{1-(\frac{3}{4})^{n}}{1-\frac{3}{4}}
\;=\; \Big(\frac{3}{4}\Big)^{\!n}x_0 \;+\; 1-\Big(\frac{3}{4}\Big)^{\!n}.
\]
Thus
\[
m_n \;=\; \frac{x_n-1}{18}
\;=\; \frac{1}{18}\Big(\frac{3}{4}\Big)^{\!n}(x_0-1).
\]

\medskip

\noindent\emph{Notes.}
(i) The constants $(\alpha,\beta,c;\delta)$ above were taken verbatim from the base \(p{=}0\) table.
(ii) Equations \eqref{eq:geom-from-p0}–\eqref{eq:mn-from-p0} apply to any other $p{=}0$ row by substituting that row’s $\alpha$ and the corresponding residue offset $\beta_\mathrm{row}=6j+p_6$ (e.g.\ $\beta_\mathrm{row}=1,7,13$ for $s=\mathrm e$ and $=5,11,17$ for $s=\mathrm o$).
(iii) If the proof segment crosses rows (changes $(s,j)$), break the word at the boundaries and apply these formulas piecewise with the appropriate $\alpha$ and $\beta_\mathrm{row}$ for each block.

% ------------------------------------------------------------
% Section: Residue targeting via a last-row congruence
% (canonically choosing one witness per modulus 3·2^K)
% ------------------------------------------------------------

\section{Residue targeting via a last-row congruence}\label{sec:last-row-targeting}

Let $M_K:=3\cdot 2^{K}$. An admissible last step (taken from some column $p\ge 0$) has the form
\[
  x' \;=\; 6\big(2^{\alpha+6p}\,m \;+\; k^{(p)}\big) \;+\; \delta^{(p)},
\]
where $\alpha\in\mathbb Z_{\ge 0}$ is the base exponent of the row, and $k^{(p)},\delta^{(p)}$ are the column-$p$ constants for that row.%
\footnote{In many normalizations one has $k^{(p)}=k+p$ and $\delta^{(p)}=6p+\delta_0$, but the lemma below does not need these specifics.}
Write
\[
  a^{(p)}:=6\cdot 2^{\alpha+6p},\qquad
  r^{(p)}:=x_{\mathrm{tar}}-\big(6k^{(p)}+\delta^{(p)}\big).
\]

\begin{lemma}[Last-row congruence targeting]\label{lem:last-row-p}
The congruence
\[
  a^{(p)}\,m \;\equiv\; r^{(p)} \pmod{M_K}
\]
is solvable iff $g^{(p)}:=\gcd\!\big(a^{(p)},M_K\big)=3\cdot 2^{\min(\alpha+1+6p,\,K)}$ divides $r^{(p)}$.
When solvable:
\begin{enumerate}[label=\emph{(\roman*)}]
  \item If $K\le \alpha+1+6p$, then $r^{(p)}\equiv 0\pmod{M_K}$ and \emph{every} $m\in\mathbb Z$ works (the step pins a residue independently of $m$).
  \item If $K>\alpha+1+6p$, then
  \[
    m \;\equiv\; \frac{\,r^{(p)}\,}{\,3\cdot 2^{\alpha+1+6p}\,}\ \pmod{\,2^{\,K-(\alpha+1+6p)}\,}.
  \]
\end{enumerate}
Admissibility at the input (correct family and router $j$) must also hold.
\end{lemma}

\begin{corollary}[Pinning regime as a special case]\label{cor:pinning}
In the setting of the last-row congruence lemma, if $\alpha+1+6p\ge K$, then for $M_K=3\cdot 2^K$ one has
\[
x' \equiv 6k^{(p)}+\delta \pmod{M_K}
\]
\emph{independently of $m$}. In particular, once routing admissibility holds, choosing the last token with $\alpha+1+6p\ge K$ pins the final residue at modulus $M_K$.
\end{corollary}


% === Parity/GCD corollary (added) ===
\begin{corollary}[Meeting the gcd condition via parity control]\label{cor:last-row-parity}
In the setting of Lemma~\ref{lem:last-row-p}, using the same-family parity toggle from the padding menu one can choose the final $B\bmod 2$ so that $g^{(p)}\mid r^{(p)}$, hence the congruence is solvable for any target modulus $M_K$.
\end{corollary}



\begin{corollary}[When the last step ignores $m$]\label{cor:pin}
The last step pins $x'\equiv 6k^{(p)}+\delta^{(p)}\pmod{M_K}$ independently of $m$ iff $\alpha+6p\ge K$.
\end{corollary}

\begin{corollary}[Refined last-step mapping across moduli]\label{cor:last-row-map}
Fix a starting modulus $M_K=3\cdot 2^K$ and a starting class $x\equiv r\pmod{M_K}$.
Let the last token $T$ (in some column $p$) have parameters
\[
x' \;=\; 6\bigl(2^{\alpha+6p}\,u + k^{(p)}\bigr) + \delta,\qquad
a^{(p)}:=6\cdot 2^{\alpha+6p}=3\cdot 2^{\alpha+1+6p}.
\]
Assume we have fixed a (possibly empty) prefix so that $T$ is admissible at the input
(router $j$ and family match), and choose a target modulus $M_{K'}=3\cdot 2^{K'}$ with $K'\ge K$.

Then there exists a refinement of the \emph{starting} class to a subclass
\[
x \;\equiv\; r^\sharp \pmod{M_{K+\rho}}
\]
for some $\rho\ge 0$ (large enough to guarantee the planned routers in the prefix and at $T$), such that for every $x$ in this subclass the last step produces an $x'$ with
\[
x' \;\equiv\; r' \pmod{M_{K'}},
\]
where $r'$ is determined as follows:

\begin{enumerate}[label=\emph{(\roman*)}]
\item \textbf{Pinning regime ($K'\le \alpha+1+6p$).} The gcd is $g^{(p)}=M_{K'}$, so
\[
x' \;\equiv\; 6k^{(p)}+\delta \pmod{M_{K'}}\quad\text{(independent of $m$ and hence of the start $x$ in the subclass).}
\]
In words: the last token pins a \emph{single} residue $r' \equiv 6k^{(p)}+\delta \pmod{M_{K'}}$ across the entire subclass.

\item \textbf{Solvable congruence ($K'> \alpha+1+6p$).} Write
\[
r^{(p)} \;:=\; \frac{x_{\mathrm{tar}}-(6k^{(p)}+\delta)}{1}
\quad\text{modulo }M_{K'}.
\]
Then the congruence of Lemma~\ref{lem:last-row-p}
\[
a^{(p)}\,m \;\equiv\; r^{(p)} \pmod{M_{K'}}
\]
is solvable iff $g^{(p)}=3\cdot 2^{\alpha+1+6p}$ divides $r^{(p)}$, and in that case
\[
m \;\equiv\; \frac{r^{(p)}}{\,3\cdot 2^{\alpha+1+6p}\,}
\ \ \pmod{\,2^{\,K'-(\alpha+1+6p)}\,}.
\]
Choosing the refined start class so that the router and $m$-class above hold, the last token then maps the entire subclass to the target residue $r'\equiv x_{\mathrm{tar}}\pmod{M_{K'}}$.
\end{enumerate}

\noindent
In both cases the refinement depth $\rho$ can be taken large enough to ensure routing compatibility of the prefix (no branch flips) and the required $m$-class at the last step.
\end{corollary}
\begin{example}[Explicit $M_3\to M_4$ mapping via a 2-step tail]
Start at modulus $M_3=24$ with $x\equiv 17\pmod{24}$, and target $M_4=48$ with $x'\equiv 41\pmod{48}$.
Use the $p=0$ rows
\[
\omega_1:\ x=12m+7\quad(\texttt{oe},\ j=1),\qquad
\psi_2:\ x=24m+17\quad(\texttt{eo},\ j=2).
\]
Refine the start to the subclass $x\equiv 209\pmod{216}$ (so $n\equiv 8\pmod 9$ in $x=24n+17$).
Then the certified two-step tail
\[
\omega_1\ \to\ \psi_2
\]
is admissible at each step and \emph{maps} this subclass from $M_3$ to
\[
x'\equiv 41\pmod{48}=M_4,
\]
as shown in the detailed worked example earlier (the intermediate value $x_1$ is forced by $\omega_1$).
\end{example}
\begin{proposition}[Last-row pinning vs.\ congruence solution]\label{prop:last-row-pinning}
Let a last token $T$ in column $p$ have parameters
\[
x' \;=\; 6\big(2^{\alpha_p}u + k^{(p)}\big) + \delta_T,\qquad
\alpha_p=\alpha+6p,\qquad u=\Big\lfloor \frac{x}{18}\Big\rfloor,
\]
and write $M_K:=3\cdot 2^K$, $a^{(p)}:=6\cdot 2^{\alpha_p}$, $r^{(p)}:=x_{\mathrm{tar}}-(6k^{(p)}+\delta_T)$.
Then the congruence
\[
a^{(p)}\,m \equiv r^{(p)} \pmod{M_K}
\]
is solvable iff $g^{(p)}:=\gcd(a^{(p)},M_K)=3\cdot 2^{\min(\alpha_p+1,K)}$ divides $r^{(p)}$.
When solvable:
\begin{enumerate}[label=\emph{(\roman*)}]
\item (\emph{Pinning}) If $K\le \alpha_p+1$, then $g^{(p)}=M_K$ and $r^{(p)}\equiv 0\pmod{M_K}$.
      Thus \emph{every} $m\in\mathbb{Z}$ works: the step pins \(x'\equiv 6k^{(p)}+\delta_T\pmod{M_K}\) independently of $m$.
\item (\emph{Linear $m$-class}) If $K>\alpha_p+1$, then the solution set is the single congruence class
\[
m \equiv \frac{r^{(p)}}{\,3\cdot 2^{\alpha_p+1}\,}\ \pmod{\,2^{\,K-(\alpha_p+1)}\,}.
\]
\end{enumerate}
Admissibility (correct family, router) must also hold at the input of $T$.
\end{proposition}

\begin{proof}
$\gcd(a^{(p)},M_K)=3\cdot 2^{\min(\alpha_p+1,K)}$ is immediate. The rest is the standard linear congruence criterion:
solvability iff $g^{(p)}\mid r^{(p)}$, with the stated solution set modulo $M_K/g^{(p)}$; interpret cases by whether $\alpha_p+1\ge K$.
\end{proof}


\paragraph{Canonicalization (avoid multiple witnesses per modulus).}
To keep a single witness per modulus $M_K$, fix one of the following conventions:
\begin{itemize}[leftmargin=1.5em]
  \item \emph{Minimal-$\alpha$ convention:} among admissible last rows achieving $x_{\mathrm{tar}}\pmod{M_K}$, pick the one with the smallest $\alpha+6p$ (fewest added powers of $2$).
  \item \emph{Family-priority convention:} prefer a designated terminal family (e.g.\ $o$); if tied, use minimal-$\alpha$.
\end{itemize}
Either rule yields a unique canonical last step for each $M_K$ and target residue.

\subsection*{Representative examples (explicit use of Lemma~\ref{lem:last-row-p})}

We illustrate the two regimes of Lemma~\ref{lem:last-row-p}: (i) $m$-independent pinning when $\alpha+6p\ge K$, and (ii) congruence targeting for $m$ when $\alpha+6p<K$. Throughout $M_K:=3\cdot 2^K$ and
\[
a^{(p)}:=6\cdot 2^{\alpha+6p},\qquad
r^{(p)}:=x_{\mathrm{tar}}-\big(6k^{(p)}+\delta^{(p)}\big),\qquad
g^{(p)}:=\gcd\!\big(a^{(p)},M_K\big).
\]

\paragraph{Example 1 (pin at $K=5$): $x_{\mathrm{tar}}\equiv 53\pmod{96}$.}
Take the $p{=}0$ row $\Omega_0$ (oo), with unified form $x'=6(2^\alpha m+k)+\delta=192m+53$, so
\[
\alpha=5,\quad k=8,\quad \delta=5.
\]
Here $K=5$ and $\alpha+6p=\alpha=5\ge K$, so Cor.~\ref{cor:pin} applies: the last step \emph{ignores $m$} modulo $M_5=96$ and pins
\[
x' \equiv 6k+\delta = 48+5 \equiv 53 \pmod{96},
\]
ending in the $o$-family.

\paragraph{Example 2 (solve for $m$ at $K=10$): $x_{\mathrm{tar}}\equiv 3071\pmod{3072}$.}
Choose the $p{=}0$ row $\Omega_2$ (oo): $x'=12m+11=6(2^1 m+1)+5$, so
\[
\alpha=1,\quad k=1,\quad \delta=5,\qquad K=10.
\]
Compute the lemma data:
\[
a^{(0)}=6\cdot 2^{1}=12,\quad
r^{(0)}=3071-(6\cdot 1+5)=3060,\quad
g^{(0)}=\gcd(12,3072)=12.
\]
Solvability: $g^{(0)}\mid r^{(0)}$ (since $3060=12\cdot 255$), so solutions exist. Because $\alpha+6p+1=2< K$, Lemma~\ref{lem:last-row-p}(ii) gives
\[
m \equiv \frac{r^{(0)}}{3\cdot 2^{\alpha+1}}=\frac{3060}{3\cdot 2^{2}}=255 \pmod{\,2^{K-(\alpha+1)}\,}= \pmod{256}.
\]
Thus any admissible input with $(o,\,j{=}2)$ and $m\equiv 255\ (\bmod\,256)$ maps to $x'\equiv 3071\ (\bmod\,3072)$ on this last step.

\paragraph{Example 3 (congruence at $K=5$ with explicit predecessor): $x_{\mathrm{tar}}\equiv 49\pmod{96}$.}
Use the $p{=}0$ row $\Psi_0$ (ee): $x'=24m+1=6(2^2 m+0)+1$, so
\[
\alpha=2,\quad k=0,\quad \delta=1,\qquad K=5.
\]
Lemma data:
\[
a^{(0)}=6\cdot 2^{2}=24,\quad
r^{(0)}=49-(6\cdot 0+1)=48,\quad
g^{(0)}=\gcd(24,96)=24.
\]
Solvability: $24\mid 48$, so solutions exist. Since $\alpha+1=3<K$, Lemma~\ref{lem:last-row-p}(ii) yields
\[
m \equiv \frac{48}{3\cdot 2^{3}}=\frac{48}{24}=2 \pmod{\,2^{\,K-(\alpha+1)}\,}=\pmod{4}.
\]
Thus the last step hits $49\ (\bmod\,96)$ whenever the input is $(e,\,j{=}0)$ with $m\equiv 2\ (\bmod\,4)$. A concrete admissible predecessor is $x=37$: here $m=\lfloor 37/18\rfloor=2$ and $j=\lfloor 37/6\rfloor\bmod 3=0$, so indeed
\[
37 \xrightarrow{\Psi_0} 24\cdot 2 + 1 = 49 \equiv 49 \pmod{96}.
\]
(From $x_0=1$ one forced path to $37$ is $\psi_0,\omega_0,\psi_2,\Omega_2,\omega_1,\Psi_1$.)


\noindent
These examples illustrate the two regimes: (i) \emph{$m$-independent pins} when $\alpha+6p\ge K$, and (ii) \emph{congruence targeting} when $\alpha+6p<K$. In both cases, the canonicalization rule selects a single last-step witness per $M_K$.

\subsection{From $M_K$ to $M_{K+n}$: a concrete lifting procedure}

Let $M_K:=3\cdot 2^K$. Suppose we are given a starting residue class
\[
x\equiv r \pmod{M_K}\qquad\text{(with }r\equiv 1 \text{ or }5 \pmod{6}\text{)},
\]
and we want a compatible residue $x'\equiv r'\pmod{M_{K+n}}$ that is realized by a certified tail. We give two mechanical variants.

\paragraph{Variant A (pinning by lifting; always works).}
Pick a last token (row) $T$ in some column $p\ge 0$ with unified form
\[
x' \;=\; 6\bigl(2^{\alpha+6p}\,u + k^{(p)}\bigr)+\delta,\qquad
u=\Big\lfloor\frac{x}{18}\Big\rfloor,\quad \delta\in\{1,5\}.
\]
Choose $p$ so that the \emph{pinning threshold}
\[
K' \;\le\; \alpha+1+6p,\qquad\text{where }K':=K+n,
\]
holds. Then $M_{K'}\mid 6\cdot 2^{\alpha+6p}$ and therefore
\[
x' \equiv 6k^{(p)}+\delta \pmod{M_{K'}}\quad\text{(independent of $u$ and $m$).}
\]
Finally, refine the starting class $x\equiv r\pmod{M_K}$ to a subclass so that the router for $T$ is admissible at the last step (and, if you keep a short prefix, at each step of that prefix). This refinement is a standard “routing-compatibility’’ choice.

\paragraph{Variant B (minimal lift + solving the last congruence).}
If you prefer a smaller column $p$ with $K'>\alpha+1+6p$, let
\[
a^{(p)}:=6\cdot 2^{\alpha+6p}=3\cdot 2^{\alpha+1+6p},\qquad M_{K'}=3\cdot 2^{K'}.
\]
To force $x'\equiv r'\pmod{M_{K'}}$ at the last step you need
\[
a^{(p)}u \;\equiv\; r'-(6k^{(p)}+\delta) \pmod{M_{K'}}.
\]
This is solvable iff $g:=\gcd(a^{(p)},M_{K'})=3\cdot 2^{\min(\alpha+1+6p,K')}$ divides the right-hand side. When solvable,
\[
u \;\equiv\; \frac{\,r'-(6k^{(p)}+\delta)\,}{\,3\cdot 2^{\alpha+1+6p}\,}
\ \ \pmod{\,2^{\,K'-(\alpha+1+6p)}\,}.
\]
As in Variant~A, refine the starting class so all routers are admissible and the internal index $u=\lfloor x/18\rfloor$ lies in the solved class.

\begin{example}[Given $x\equiv 13\pmod{24}$, produce $x'\pmod{96}$]\label{ex:13-to-mod96}
We illustrate Variant~A (pinning) with a single last token.

\emph{Step 1: choose the last token and pinning level.}
From the $p{=}0$ table, take $T=\Psi_1$ (type \texttt{ee}) with parameters
\[
\alpha=4,\qquad k^{(0)}=6,\qquad \delta=1,\qquad
x' \;=\; 6\bigl(2^{4}u + 6\bigr)+1 \;=\; 96u + 37.
\]
For $K'=5$ (modulus $M_5=96$) we have $\alpha+1=5$, so the pinning threshold $K'\le \alpha+1$ holds. Therefore, for \emph{any} admissible input,
\[
x' \equiv 6k^{(0)}+\delta \equiv 36+1 \equiv \boxed{37}\pmod{96}.
\]

\emph{Step 2: ensure admissibility at the last step.}
We must feed $\Psi_1$ from the even family with router $j=1$. Starting from
\[
x\equiv 13 \pmod{24}\quad\Longleftrightarrow\quad x=24n+13,
\]
the router at the last step is
\[
j=\Big\lfloor \frac{x}{6}\Big\rfloor\bmod 3 \;=\; (4n+2)\bmod 3 \;\equiv\; n+2 \pmod 3.
\]
Thus $j=1$ exactly when $n\equiv 2\pmod 3$, i.e.\ for the refined subclass
\[
x \equiv 24(3t+2)+13 = 72t + 61 \ \equiv\ 61 \pmod{72}.
\]
This refinement preserves $x\equiv 13\pmod{24}$ and guarantees the last row $\Psi_1$ is admissible.

\emph{Step 3: (optional) concrete instance and check.}
Take $x_0=61$. Then
\[
u=\Big\lfloor\frac{61}{18}\Big\rfloor=3,\qquad
x' = 96u+37 = 96\cdot 3 + 37 = 325 \equiv \boxed{37}\pmod{96}.
\]
Hence, starting from $x\equiv 13\pmod{24}$, a one-step tail $\Psi_1$ (at $p{=}0$) maps the refined-router subclass to the pinned residue $x'\equiv 37\pmod{96}$.
\end{example}

\noindent\emph{Remarks.}
(i) If you want a different target residue $r'\pmod{96}$, pick a different last token (or a different column $p$) so that $6k^{(p)}+\delta\equiv r'\pmod{96}$ under pinning.
(ii) If you prefer not to refine the starting class for router $j=1$, append a short same-family steering block before $\Psi_1$ to achieve the desired router while staying in the even family; the last step still pins $x'\equiv 37\ (\bmod 96)$.

\subsection{Backward recovery: from $M_{K+n}$ down to $M_K$}

Let $M_{K'}:=3\cdot 2^{K'}$ and suppose we are given a target residue
\[
x' \equiv r' \pmod{M_{K'}} \qquad (r'\equiv 1 \text{ or }5 \bmod 6).
\]
We describe how to recover compatible preimages $x$ modulo a lower modulus $M_K$
that map to $x'$ in \emph{one certified last step}. We proceed token–by–token.

\paragraph{Setup (pick a hypothesized last token).}
Choose a last token $T$ (a table row) in some column $p\ge 0$, with unified form
\[
x' \;=\; 6\bigl(2^{\alpha+6p}\,u + k^{(p)}\bigr) + \delta,
\qquad u=\Big\lfloor \frac{x}{18}\Big\rfloor,
\quad \delta\in\{1,5\}.
\]
Write
\[
a^{(p)}:=6\cdot 2^{\alpha+6p} \;=\; 3\cdot 2^{\alpha+1+6p}.
\]

\paragraph{Variant A (pinning case).}
If $K'\le \alpha+1+6p$ then $M_{K'}\mid a^{(p)}$, so the step \emph{pins} the residue:
\[
x' \equiv 6k^{(p)}+\delta \pmod{M_{K'}}.
\]
Hence, this token $T$ can produce the given $x'$ \emph{iff}
\[
r' \equiv 6k^{(p)}+\delta \pmod{M_{K'}}.
\]
When this holds, $u$ is unconstrained mod $2$, and all preimages $x$ are obtained by:
\begin{itemize}
  \item Choosing any $u\in\mathbb Z$;
  \item Taking $x$ in the interval $[18u,18u+17]$ that matches the required router $j$ of $T$:
        \[
        \begin{array}{c|c}
          \text{router }j & \text{allowed }x \in [18u,18u+17] \\ \hline
          0 & 18u,\,\dots,\,18u+5 \\
          1 & 18u+6,\,\dots,\,18u+11 \\
          2 & 18u+12,\,\dots,\,18u+17
        \end{array}
        \]
  \item Within that 6-block, pick the residue matching the \emph{entry family} of $T$:
        $x\equiv 1\ (\bmod 6)$ for family $e$, or $x\equiv 5\ (\bmod 6)$ for family $o$.
\end{itemize}
With $u$ free, $x$ ranges over a \emph{union of classes} modulo $18$. Projecting to $M_K=3\cdot 2^K$ yields preimages at $K=1$ and higher; a convenient canonical modulus is
\[
x \equiv 18u + r \pmod{18\cdot 2^s}
\quad\Longrightarrow\quad
x \equiv 18u_0 + r \pmod{M_{s+1}} \ \ (K=s+1),
\]
whenever $u\equiv u_0\ (\bmod\ 2^s)$.

\paragraph{Variant B (non-pinning case).}
If $K' > \alpha+1+6p$, form the last-step congruence
\[
a^{(p)}u \;\equiv\; r' - \bigl(6k^{(p)}+\delta\bigr) \pmod{M_{K'}}.
\]
Let $g:=\gcd\!\bigl(a^{(p)},M_{K'}\bigr)=3\cdot 2^{\min(\alpha+1+6p,\,K')}=3\cdot 2^{\alpha+1+6p}$.
Solvability requires $g \mid r'-(6k^{(p)}+\delta)$. When solvable, we obtain the internal index class
\[
u \;\equiv\; \frac{\,r'-(6k^{(p)}+\delta)\,}{\,3\cdot 2^{\alpha+1+6p}\,}
\ \ \pmod{\,2^{\,K'-(\alpha+1+6p)}\,}.
\]
Then $x$ is reconstructed from $u$ exactly as in Variant~A:
pick the correct $j$-block inside $[18u,18u+17]$ and the entry-family residue modulo $6$.
If $u$ is known modulo $2^s$ with $s:=K'-(\alpha+1+6p)$, then $x$ is determined modulo
$18\cdot 2^s$, hence also modulo $M_{s+1}=3\cdot 2^{s+1}$.

\begin{example}[Recovering $x\bmod 24$ from $x'\equiv 37\bmod 96$ via $T=\Psi_1$]
Take the last token $T=\Psi_1$ at $p=0$ (type \texttt{ee}) with
\[
\alpha=4,\qquad k^{(0)}=6,\qquad \delta=1,\qquad
x' = 6\bigl(2^{4}u+6\bigr)+1 = 96u+37.
\]
Here $K'=5$ (since $M_{K'}=96$) and $\alpha+1=5$, so we are in the \emph{pinning} regime.
Then
\[
6k^{(0)}+\delta = 36+1 = 37 \equiv r' \pmod{96},
\]
so $\Psi_1$ is consistent with $x'\equiv 37\ (\bmod 96)$ and imposes \emph{no} restriction on $u$.

To recover $x$, respect the router and entry family for $\Psi_1$:
it requires router $j=1$ and entry family $e$ (i.e.\ $x\equiv 1\ (\bmod 6)$).
Given any $u\in\mathbb Z$, the admissible preimages lie in the slice
\[
x \in \{18u+6,\dots,18u+11\}\ \cap\ \{x\equiv 1\ (\bmod 6)\},
\]
which forces $x=18u+7$ (since $18u+r\equiv r\pmod 6$ and the only $r\in[6,11]$ with $r\equiv 1\pmod 6$ is $r=7$).

Thus
\[
x \equiv 18u + 7 \pmod{18}.
\]
Projecting to $M_2=3\cdot 2^2=12$ or $M_3=24$ picks concrete classes. For instance,
choosing $u\equiv 3\ (\bmod 2)$ gives $x=18\cdot 3 + 7 = 61 \equiv 13 \pmod{24}$,
which is a valid preimage class we used elsewhere. More generally, any $u$ produces
\[
x \equiv 18u + 7 \pmod{18\cdot 2^s}\quad\Rightarrow\quad x \equiv 18u_0+7 \pmod{M_{s+1}},
\]
whenever $u\equiv u_0\ (\bmod\ 2^s)$.
\end{example}

\noindent\emph{Notes.}
\begin{itemize}
  \item The backward preimage is typically \emph{not unique}. Pinning makes $u$ free, yielding a whole family of $x$’s consistent with the same last token.
  \item If several tokens $T$ satisfy the divisibility check (or the pinning identity) for the given $r'$, you get multiple admissible preimage families; routing constraints decide which arise from a fixed prefix.
  \item For non-pinning cases, the recovered modulus for $x$ is naturally $M_{s+1}$ with $s=K'-(\alpha+1+6p)$, reflecting that one last step adds $\alpha+1+6p$ binary factors to the $3\cdot 2^K$ modulus.
\end{itemize}

\begin{example}[Recovering $x\bmod 24$ from $x'\equiv 37\bmod 48$ via $T=\Psi_1$]
Work at $K'=4$ (so $M_{K'}=48$). Take the last token $T=\Psi_1$ at $p=0$ (type \texttt{ee}):
\[
x' \;=\; 6\bigl(2^{4}u + 6\bigr)+1 \;=\; 96u + 37,
\qquad (\alpha=4,\ k^{(0)}=6,\ \delta=1).
\]
Since $K'=4\le \alpha+1=5$, we are in the \emph{pinning} regime and indeed
\[
x' \equiv 6k^{(0)}+\delta \equiv 36+1 \equiv 37 \pmod{48}.
\]
Thus \(\Psi_1\) can produce any \(x'\equiv 37\ (\bmod 48)\) and imposes \emph{no} restriction on \(u\).

To recover preimages \(x\), enforce the router and entry family of \(\Psi_1\):
it requires router \(j=1\) and entry family \(e\) (i.e.\ \(x\equiv 1\pmod 6\)).
Given any \(u\in\mathbb Z\), the admissible preimages lie in the slice
\[
x \in \{18u+6,\dots,18u+11\}\ \cap\ \{x\equiv 1\ (\bmod 6)\},
\]
which forces \(x=18u+7\) (the unique residue \(\equiv 1 \pmod 6\) in that $j=1$ block).

Reduce \(x=18u+7\) modulo \(24\):
\[
18u+7 \equiv -6u+7 \pmod{24}.
\]
As \(u\) ranges over \(\mathbb Z\), the residue classes modulo \(24\) are exactly
\[
\boxed{x \equiv 1,\,7,\,13,\,19 \pmod{24}}.
\]
(Indeed, taking \(u\equiv 0,1,2,3 \pmod{4}\) yields \(7,1,19,13\) respectively.)
All four are in family \(e\) (each \(\equiv 1\pmod 6\)) and sit in router block \(j=1\) at the last step.

\emph{Concrete check.} Pick \(x=13\) (one of the four classes). Then
\[
u=\Big\lfloor \tfrac{13}{18}\Big\rfloor=0,\qquad x'=96u+37=37\equiv 37\pmod{48}.
\]
Similarly, \(x=1,7,19\) also map to \(x'\equiv 37\ (\bmod 48)\) under the last step \(\Psi_1\).
\end{example}


% ------------------------------------------------------------


\section{Same-family steering menu and monotone padding (p=0)}
\label{sec:steering-menu}

Throughout this section we fix $p=0$. Our goal is to package the tail
constructions from Sections~\ref{sec:last-row-congruence} and
\ref{sec:routing} into a small “steering menu” of tail blocks that
(i) raise the 2-adic valuation $v_2(A_W)$,
(ii) preserve the terminal family ($e$ or $o$), and
(iii) optionally toggle the parity of $B_W \bmod 2$.


\begin{proposition}[Steering menu at $p{=}0$]\label{prop:menu}
For any certified word $W$ at $p=0$, the following tail blocks can be
appended on the right. In each case the resulting word $\widetilde W$
has strictly larger $v_2(A_{\widetilde W})$ by the shown amount, and
its terminal family (even $e$ or odd $o$) is as indicated.

\begin{center}
\renewcommand{\arraystretch}{1.15}
\begin{tabular}{lll}
\toprule
\textbf{Block} & \textbf{Family effect} & $\boldsymbol{\Delta v_2(A)}$ \\
\midrule
$\Psi_1$ & $e\to e$ & $+4$ \\
$\Psi_2$ & $e\to e$ & $+6$ \\
$\Omega_1$ & $o\to o$ & $+4$ \\
$\Omega_0$ & $o\to o$ & $+5$ \\
$\psi_2\circ\omega_1$ & $e\to o\to e$ (parity-cycle) & $+3$ \\
$\omega_1\circ\psi_2$ & $o\to e\to o$ (parity-cycle) & $+3$ \\
\bottomrule
\end{tabular}
\end{center}
\end{proposition}

\begin{lemma}[Monotone (suffix-only) padding]\label{lem:monotone}
Given a target $K$ and a choice of terminal family ($e$ or $o$),
there exists a finite concatenation $S$ of blocks from
Prop.~\ref{prop:menu} such that, for any word $W$ with that terminal
family, the padded word $\widetilde W := W\,S$ satisfies
$v_2(A_{\widetilde W}) \ge K$ and has the same terminal family.

If a specific residue $B_{\widetilde W} \bmod 2$ is required, appending
one parity-cycle block (from the bottom two rows of
Prop.~\ref{prop:menu}) flips $B_{\widetilde W}\bmod 2$ while preserving
the terminal family and leaving $v_2(A_{\widetilde W})$ unchanged modulo
the lower bound $K$.
\end{lemma}


\begin{lemma}[Routing compatibility for the stabilized prefix]\label{lem:routing-compat-prefix}% alias: use this label for routing compatibility references
Let $W$ be a fixed prefix and $S$ a tail from Prop.~\ref{prop:menu}.
Let $m$ be a solution of the final congruence for the chosen last row
(as in Section~\ref{sec:last-row-congruence}), and let
$(r_t)$ denote the remainders computed along $W$ at input $m$ in the
router scheme of Section~\ref{sec:routing}.

Then these remainders $r_t$ match the planned router indices of $W$;
equivalently, appending $S$ does not invalidate any of $W$’s row choices.
\end{lemma}


\paragraph{Worked examples.}
\begin{itemize}
\item[A.] (o-tail) Append $\Omega_1,\Omega_0$ to raise $v_2$ by $+9$
  and land at $53\bmod 48$; this residue is independent of $m$.

\item[B.] (e-tail) Append $\Psi_2$ to raise $v_2$ by $+6$
  and pin $37\bmod 48$, again independent of $m$.

\item[C.] Target $3071\bmod 3072$: first lift with $\Omega_1,\Omega_0$
  until $v_2(A)$ is large enough, then apply a final $\Omega_2$ with
  $m\equiv 255\pmod{256}$ to hit the exact target residue.
\end{itemize}



% ------------------------------------------------------------
% Section: Routing-compatibility of a fixed prefix under tail padding
% ------------------------------------------------------------

\section{Routing-compatibility of a fixed prefix under tail padding}
\label{sec:routing-compat}

We write a certified word in the form
\begin{equation*}
  x_W(m) = 6\big(A_W m + B_W\big) + \delta_W,
\end{equation*}
with
\begin{equation*}
  A_W = 2^{\sum \alpha_i}.
\end{equation*}
After freezing a prefix $W$, we show one may append a same-family padding tail and solve a final congruence so that all router choices inside $W$ remain as planned.

\begin{lemma}[2-adic control of prefix indices]\label{lem:prefix-2adic-rc}
Let $W_{\le t}$ denote the length-$t$ prefix of a certified word $W$.
There exist integers $U_t,V_t$ and an integer $s_t\ge 0$ such that
\begin{equation*}
  m_t \equiv U_t\,m + V_t \pmod{2^{s_t}},
\end{equation*}
where $m_t := \left\lfloor x_{W_{\le t}}(m)/18 \right\rfloor$, and one may take
\begin{equation*}
  s_t \le 1+\sum_{i< t}\alpha_i = 1+\log_2 A_t,
  \qquad
  x_{W_{\le t}}(m) = 6\big(A_t m + B_t\big)+\delta_t,\ \ A_t=2^{\sum_{i< t}\alpha_i}.
\end{equation*}
\end{lemma}

\begin{proof}
By induction on $t$. For $t=0$, $(U_0,V_0,s_0)=(1,0,0)$.
Assume the claim holds for some $t\ge 0$.
Write $x_{W_{\le t}}(m)=6(A_t m+B_t)+\delta_t$ with $A_t=2^{\sum_{i< t}\alpha_i}$.
Then
\begin{equation*}
  m_t
  = \left\lfloor \frac{6(A_t m+B_t)+\delta_t}{18} \right\rfloor
  = \left\lfloor \frac{A_t m+B_t}{3}
      + \frac{\delta_t-6 r_t}{18} \right\rfloor,
\end{equation*}
where $r_t\in\{0,1,2\}$ depends only on $(A_t m+B_t)\bmod 3$.
Since $3$ is invertible modulo $2^s$ for all $s$, reducing modulo $2^{\,1+\sum_{i< t}\alpha_i}$ gives
$m_t \equiv U_t m + V_t \ (\mathrm{mod}\ 2^{s_t})$ for some integers $U_t,V_t$ with $s_t\le 1+\sum_{i< t}\alpha_i$.
Applying step $t$ (with parameters $\alpha_t,k_t,\delta_t'$) yields the same form for $m_{t+1}$ with $A_{t+1}=2^{\sum_{i\le t}\alpha_i}$, completing the induction.
\end{proof}

\begin{lemma}[Routing compatibility (no branch flips)]\label{lem:routing-compat-rc}
Let $W$ be a fixed prefix of length $L$.
Append a same-family padding tail so that the final congruence you solve constrains
\begin{equation*}
  m \equiv m^\ast \pmod{2^{S^\ast}},
  \qquad
  S^\ast \ge \max_{0\le t< L} s_t,
\end{equation*}
where $s_t$ are as in Lemma~\ref{lem:prefix-2adic-rc}, and choose the solution class whose mod-$3$ part matches the admissibility of $W$.
Then along $W$ at this $m$, all router remainders
\begin{equation*}
  r_{t+1} \equiv 2^{\alpha_t}\, m_t + k_t \pmod{3}
\end{equation*}
coincide with the predeclared routers $j_{t+1}$ of $W$.
Equivalently, every row in $W$ remains admissible (no branch flips in the prefix).
\end{lemma}

\begin{proof}
By Lemma~\ref{lem:prefix-2adic-rc}, each $m_t$ depends on $m$ only through $m \bmod 2^{s_t}$.
The tail is chosen so that $m\equiv m^\ast\ (\bmod\ 2^{S^\ast})$ with $S^\ast \ge s_t$ for all $t<L$, hence each $m_t$ is fixed.
Therefore each router remainder $r_{t+1}\equiv 2^{\alpha_t} m_t + k_t\ (\bmod 3)$ is fixed and equals the planned $j_{t+1}$ (we also selected the mod-$3$ class compatibly).
Thus all prefix rows remain admissible.
\end{proof}

\begin{remark}[Compatibility with backward uniqueness]
This does not conflict with any backward–uniqueness statement: uniqueness fixes the prefix $W$ by forcing a single backward chain; routing-compatibility asserts that suffix padding and the final congruence can be chosen so the already-fixed prefix routers remain unchanged.
\end{remark}
\begin{lemma}[Global routing compatibility]\label{lem:global-routing-compat}
Let $W=T_0T_1\cdots T_{n-1}$ be a fixed certified prefix, with planned router remainders
$r_{t+1}\in\{0,1,2\}$ at each step $t=0,\dots,n-1$, and let
\[
x_t(m)=6\big(A_t m+B_t\big)+\delta_t,\qquad
m_t=\Big\lfloor \frac{x_t}{18}\Big\rfloor,\qquad
j_{t+1}=\Big\lfloor\frac{x_t}{6}\Big\rfloor \bmod 3.
\]
Then there exists a nonempty congruence class $\mathcal{M}\subset\mathbb{Z}$ of $m$
(of the form $m\equiv m^\ast \pmod{2^{S^\ast}}$ together with a fixed residue mod $3$ determined by the entry family of $W$)
such that for all $m\in\mathcal{M}$ the run of $W$ is admissible and
\[
j_{t+1}=r_{t+1}\quad\text{for every }t=0,\dots,n-1.
\]
Equivalently, along $\mathcal{M}$ we have the divisibility constraints
\[
A_t m + B_t \equiv r_{t+1}\pmod 3
\quad\text{for all }t,
\]
so every division by $3$ in the one-step update is integral and the planned rows are used throughout $W$ (no branch flips).
\end{lemma}

\begin{proof}
Proceed by induction on $t$. For $t=0$, admissibility fixes the entry family mod $6$ (hence mod $3$) and imposes
$A_0 m + B_0 \equiv r_1 \pmod 3$, which is a nonempty congruence class in $m$ modulo $3$.
Assuming the claim up to step $t$, the one-step composition with floor (Lemma~\ref{lem:one-step-floor}) yields
\[
m_t=\frac{A_t m + B_t - r_{t+1}}{3}\in\mathbb{Z}\quad\text{on the inductive class},
\]
and the next router is $j_{t+1}\equiv r_{t+1}$ by construction. The affine update for $(A_{t+1},B_{t+1},\delta_{t+1})$
produces a new linear congruence $A_{t+1}m+B_{t+1}\equiv r_{t+2}\pmod 3$. Intersecting the prior class with this new congruence
still leaves a nonempty arithmetic progression in $m$ (Chinese remainder on $\mathbb{Z}$ with moduli powers of $2$
and the fixed mod $3$). Iterating to $t=n-1$ gives the stated $\mathcal{M}$.
\end{proof}

% ------------------------------------------------------------
% End section
% ------------------------------------------------------------

% ------------------------------------------------------------
% Section: Same-family steering menu and monotone padding (all columns p >= 0)
% ------------------------------------------------------------

\section{Same-family steering menu and monotone padding across columns}\label{sec:menu-all-p}
% ==== Added: Finalized same-family steering menu (p=0) ====
\begin{remark}[Notation convention]\label{rmk:notation-convention}
We use $u:=\lfloor x/18\rfloor$ for the per-step floor input to a single token, while the global affine parameter in
$x_W(m)=6(A_W m+B_W)+\delta_W$ is denoted by $m$. When a last token at column $p$ is applied, its unified form is
$x'=6(2^{\alpha+6p}u+k^{(p)})+\delta$ with router $j=\lfloor x/6\rfloor\bmod 3$.
\end{remark}

\subsection*{Explicit same-family steering blocks at \texorpdfstring{$p=0$}{p=0}}
\label{subsec:menu-p0-explicit}

We collect short tail blocks that (i) preserve terminal family and (ii) allow monotone lift and parity control. For each entry, \(\Delta v_2\) refers to the increase contributed by the block (ignoring the harmless division by \(3\) inside the floor composition).

\begin{table}[H]
\centering
\renewcommand{\arraystretch}{1.08}
\setlength{\tabcolsep}{5pt}
\begin{tabularx}{0.98\linewidth}{@{}l c c c X@{}}
\toprule
Block & Entry$\to$Exit & $\Delta v_2$ & $B\bmod 2$ & Justification (from $p{=}0$ rows) \\
\midrule
$\Psi_1$ & $e\to e$ & $+4$ & even & $x'=96m+37$, so $k^{(0)}=6$ (even), $\alpha=4$. \\
$\Psi_2$ & $e\to e$ & $+6$ & even & $x'=6(2^6 m + k)+1$, table $\alpha=6$, parity even. \\
$\Omega_1$ & $o\to o$ & $+3$ & even & $x'=6(2^3 m + k)+5$, table $\alpha=3$, parity even. \\
$\Omega_0$ & $o\to o$ & $+5$ & even & $x'=6(2^5 m + k)+5$, table $\alpha=5$, parity even. \\
$\omega_1\circ\psi_2$ & $o\to e\to o$ & $+1+2=+3$ & toggles & $k^{(0)}(\omega_1)=1$ (odd), $k^{(0)}(\psi_2)=2$ (even) so one odd in the 2-token cycle toggles parity; returns to $o$. \\
$\psi_2\circ\omega_1$ & $e\to o\to e$ & $+2+1=+3$ & toggles & Same reason; returns to $e$. \\
\bottomrule
\end{tabularx}
\caption{A minimal explicit menu at \(p=0\). The \(p\ge 1\) menus are obtained by lifting exponents \(\alpha\mapsto \alpha+6p\), keeping the same entry/exit families and parity effects.}
\end{table}

\noindent
These blocks suffice to (a) raise \(v_2(A)\) to any target \(K\) and (b) realize either parity for \(B\bmod 2\) \emph{without} changing the terminal family, as required by the padding lemmas.


% ------------------------------------------------------------
% Preamble: Composition with floor input u = floor(x/18)
% ------------------------------------------------------------

\subsection*{Preamble: composition with the floor input \texorpdfstring{$u=\lfloor x/18\rfloor$}{u=floor(x/18)}}

\paragraph{Setup and invariant form.}
At any step $t$ of a certified word we maintain the linear surrogate
\begin{equation*}
  x_t(m) \;=\; 6\big(A_t\,m + B_t\big) + \delta_t,
\end{equation*}
where $A_t=2^{S_t}$ is a power of two, $B_t\in\mathbb{Z}$, and $\delta_t\in\{1,5\}$.
The input to the \emph{next} token is the \emph{internal index}
\begin{equation*}
  m_t \;:=\; \Big\lfloor \frac{x_t}{18}\Big\rfloor.
\end{equation*}

\paragraph{Router remainder and the exact formula for $m_t$.}
Write $q_t:=A_t m + B_t$, and decompose $q_t=3\,t_t + r_{t+1}$ with $r_{t+1}\in\{0,1,2\}$.
Then
\begin{equation*}
  \frac{x_t}{18} \;=\; \frac{6q_t+\delta_t}{18}
  \;=\; \frac{q_t}{3} + \frac{\delta_t}{18}
  \;=\; t_t + \frac{r_{t+1}}{3} + \frac{\delta_t}{18}.
\end{equation*}
Since $\frac{r_{t+1}}{3}+\frac{\delta_t}{18}<1$, we have the exact identity
\begin{equation*}
  m_t \;=\; \Big\lfloor \frac{x_t}{18}\Big\rfloor \;=\; t_t \;=\; \frac{A_t m + B_t - r_{t+1}}{3}.
\end{equation*}
Here $r_{t+1}$ is the \emph{router remainder} that determines the next row, and for an admissible execution it equals the planned router $j_{t+1}$.

\begin{lemma}[One-step composition with floor]\label{lem:one-step-floor}
Fix a next token $T$ in column $p$ with unified parameters
\begin{equation*}
  x' \;=\; 6\big(2^{\alpha_p}u + k^{(p)}\big) + \delta_T,
  \qquad \alpha_p:=\alpha+6p,\quad
  k^{(p)}:=\frac{\beta\,64^{\,p}+c}{9}\in\mathbb{Z}.
\end{equation*}
Feeding $T$ with $u=m_t=\big(A_t m + B_t - r_{t+1}\big)/3$ yields
\begin{equation*}
  x_{t+1}(m) \;=\; 6\big(A_{t+1}m + B_{t+1}\big)+\delta_{t+1},
\end{equation*}
with the exact update
\begin{equation*}
  A_{t+1} \;=\; \frac{2^{\alpha_p}}{3}\,A_t,\qquad
  B_{t+1} \;=\; \frac{2^{\alpha_p}}{3}\,\big(B_t - r_{t+1}\big) \;+\; k^{(p)},\qquad
  \delta_{t+1} \;=\; \delta_T.
\end{equation*}
\end{lemma}
\begin{lemma}[Integrality of the floor-driven update]\label{lem:integrality-floor-update}
In the one-step update of Lemma~\ref{lem:one-step-floor},
\[
A'=\frac{2^{\alpha_p}}{3}A,\qquad
B'=\frac{2^{\alpha_p}}{3}(B-r)+k^{(p)},\qquad
\delta'=\delta_T,
\]
all quantities are integers on the certified $m$-class.
In particular, $(B-r)/3\in\mathbb{Z}$ and hence $2^{\alpha_p}(B-r)/3\in\mathbb{Z}$.
\end{lemma}

\begin{proof}
Admissibility enforces $A m + B \equiv r \pmod{3}$ along the chosen class of $m$, so
$B-r\equiv -A m \equiv 0\pmod{3}$. Therefore $(B-r)/3\in\mathbb{Z}$ and the stated expressions are integral.
\end{proof}

\noindent\emph{Integrality.} Although the factor $2^{\alpha_p}/3$ appears, integrality is guaranteed on the certified $m$-class: admissibility enforces $A_t m + B_t \equiv r_{t+1}\ (\mathrm{mod}\ 3)$, so $(B_t-r_{t+1})/3$ is an integer linear form in $m$; equivalently $m_t\in\mathbb{Z}$ and $2^{\alpha_p}m_t\in\mathbb{Z}$.

\begin{lemma}[Final $B$-parity equals the last token’s $k^{(p)}\bmod 2$]\label{lem:B-parity}
With the notation of Lemma~\ref{lem:one-step-floor}, one has
\begin{equation*}
  B_{t+1} \;\equiv\; k^{(p)} \pmod{2}.
\end{equation*}
\end{lemma}

\begin{proof}
From Lemma~\ref{lem:one-step-floor},
\(
B_{t+1}=\frac{2^{\alpha_p}}{3}(B_t-r_{t+1})+k^{(p)}.
\)
The first term is an \emph{even} integer multiple of $2^{\alpha_p}$ (with $\alpha_p\ge 1$ for certified rows), hence vanishes modulo $2$.
\end{proof}

\paragraph{Closed form for a whole word.}
Let $W=T_1\cdots T_n$ with $T_i$ drawn from columns $p_i$ and parameters $(\alpha_{p_i},k_i,\delta_i)$ where
\begin{equation*}
  \alpha_{p_i}=\alpha_i+6p_i,\qquad
  k_i=\frac{\beta_i\,64^{\,p_i}+c_i}{9},\qquad
  \delta_i\in\{1,5\}.
\end{equation*}
Iterating Lemma~\ref{lem:one-step-floor} gives
\begin{equation*}
  x_W(m)=6\big(A_W m + B_W\big)+\delta_W,
\end{equation*}
with
\begin{equation*}
  A_W \;=\; \prod_{i=1}^{n}\frac{2^{\alpha_{p_i}}}{3}
  \;=\; \frac{2^{\,\sum_{i=1}^n \alpha_{p_i}}}{3^{\,n}},
  \qquad
  \delta_W \;=\; \delta_{T_n},
\end{equation*}
and
\begin{equation*}
  B_W \;=\; \sum_{i=1}^{n} \left( k_i \prod_{j=i+1}^{n} \frac{2^{\alpha_{p_j}}}{3}\right)
             \;+\; \sum_{i=1}^{n} \left( -\,\frac{2^{\alpha_{p_i}}}{3}\,r_{i+1}\prod_{j=i+1}^{n}\frac{2^{\alpha_{p_j}}}{3}\right),
\end{equation*}
where each $r_{i+1}\in\{0,1,2\}$ is the router remainder at step $i$ (fixed under routing compatibility). In particular,
\begin{equation*}
  B_W \;\equiv\; k_n \pmod{2}
\end{equation*}
by Lemma~\ref{lem:B-parity} (the last token determines the final $B$-parity).

\begin{remark}[Admissibility and routing compatibility]
In practice we first fix a routing prefix, then choose the tail and the final $m$-class so that all $r_{i+1}$ equal the planned routers $j_{i+1}$.
This guarantees every division by $3$ above is an integer operation on the chosen $m$-class and no branch flips occur inside the prefix.
\end{remark}

\paragraph{Two concrete numeric compositions.}

\noindent\textbf{Example 1 (}\(p=0\), last token \(T=\Psi_1\), type \texttt{ee}\textbf{).}
From the $p{=}0$ table: $\alpha_p=4$, $k^{(0)}=6$, $\delta_T=1$.
Given a prefix $(A_t,B_t,\delta_t)$ and router $r_{t+1}$,
\begin{equation*}
  m_t=\frac{A_t m + B_t - r_{t+1}}{3},\qquad
  x_{t+1}=6\Big(16\,m_t+6\Big)+1.
\end{equation*}
Hence
\begin{equation*}
  A_{t+1}=\tfrac{16}{3}A_t,\qquad
  B_{t+1}=\tfrac{16}{3}(B_t-r_{t+1})+6,\qquad
  \delta_{t+1}=1,
\end{equation*}
and $B_{t+1}\equiv 6\equiv 0\pmod 2$ as predicted.

\medskip
\noindent\textbf{Example 2 (}\(p=2\), last token \(T=\Omega_2\), type \texttt{oo}\textbf{).}
From the $p{=}2$ table: $\alpha_p=13$, $k^{(2)}=7736$, $\delta_T=5$.
For any admissible prefix and router $r_{t+1}$,
\begin{equation*}
  m_t=\frac{A_t m + B_t - r_{t+1}}{3},\qquad
  x_{t+1}=6\Big(8192\,m_t+7736\Big)+5.
\end{equation*}
Thus
\begin{equation*}
  A_{t+1}=\tfrac{8192}{3}A_t,\qquad
  B_{t+1}=\tfrac{8192}{3}(B_t-r_{t+1})+7736,\qquad
  \delta_{t+1}=5,
\end{equation*}
and $B_{t+1}\equiv 7736\equiv 0\pmod 2$.

% ------------------------------------------------------------
% End preamble
% ------------------------------------------------------------

% ============================================================
% Replacement: Same-family padding + cross-family terminal step
% ============================================================

\subsection*{Family-preserving padding and cross-family terminal steps}

\paragraph{Why this refinement.}
Earlier we emphasized ``same-family steering'' throughout a tail. In practice, many lifts
require a \emph{family change only at the last step} (e.g.\ $o\to e$ via $\omega_1$).
We therefore separate the tail into:
\begin{enumerate}
  \item a \emph{padding phase}: a concatenation of \emph{same-family} blocks that raises $v_2(A)$ and, if needed, adjusts $B\bmod2$—\emph{without changing family};
  \item a \emph{terminal step}: a \emph{single} token that may be cross-family (e.g.\ \texttt{oe} or \texttt{eo}) to land in the required terminal family and residue class.
\end{enumerate}

\begin{lemma}[Family-preserving padding menu at column $p$]\label{lem:padding-menu-p}
Fix a column $p\ge 0$ and a family $f\in\{e,o\}$. There exists a finite menu
$\mathcal{P}_{p,f}$ of tail blocks (each block is one or two tokens) with:
\begin{enumerate}
  \item (\emph{Family preservation}) Each $S\in\mathcal{P}_{p,f}$ enters $f$ and exits $f$.
  \item (\emph{Monotone lift}) Each $S$ satisfies $\Delta_p(S)>0$, so $v_2(A)$ increases.
  \item (\emph{Parity control}) $\mathcal{P}_{p,f}$ contains a 2-token cycle that toggles $B\bmod2$ while staying in $f$.
\end{enumerate}
Consequently, for any prefix ending in $f$ and any target $K$, one can append a padding string
$S=S_1\cdots S_q$ with $S_i\in\mathcal{P}_{p,f}$ so that $v_2(A)$ reaches $K$ and $B\bmod2$ is set as required, without changing family.
\end{lemma}

\begin{remark}[Composition with floor input]\label{rem:floor-comp}
Throughout, token composition uses the exact floor input $u=\lfloor x/18\rfloor$.
For a token in column $p$ with parameters $(\alpha_p,k^{(p)},\delta)$ and router remainder $r\in\{0,1,2\}$, the update
\[
A\mapsto \frac{2^{\alpha_p}}{3}A,\qquad
B\mapsto \frac{2^{\alpha_p}}{3}(B-r)+k^{(p)},\qquad
\delta\mapsto\delta
\]
holds (Lemma~\ref{lem:one-step-floor} in the preamble). In particular,
$B_{\text{new}}\equiv k^{(p)}\pmod2$ (Lemma~\ref{lem:B-parity}), which is why the last token fixes the final $B$-parity.
\end{remark}

\begin{lemma}[Mixed-family tails with a cross-family terminal step]\label{lem:mixed-tail}
Let a prefix end in family $f$ and let $T_{\mathrm{term}}$ be any admissible terminal token (possibly cross-family) in column $p$ with parameters $(\alpha_p,k^{(p)},\delta)$.
Then there exists padding $S$ composed of blocks from $\mathcal{P}_{p,f}$ such that the concatenation
$S\cdot T_{\mathrm{term}}$ is admissible, reaches any prescribed modulus $M_K=3\cdot 2^K$,
and either:
\begin{itemize}
  \item (\emph{$m$-independent pin}) if $\alpha_p\ge K$, then
    \[
      x_{\mathrm{out}} \equiv 6k^{(p)}+\delta \pmod{M_K},
    \]
    independently of the top-level $m$; or
  \item (\emph{solve for $m$}) if $\alpha_p<K$, then $m$ satisfies a linear congruence
    \[
      6\cdot 2^{\alpha_p} m \equiv x_{\mathrm{tar}} - (6k^{(p)}+\delta) \pmod{M_K},
    \]
    whose solvability class is fixed by the padding (via $B\bmod2$ and the chosen routers).
\end{itemize}
\end{lemma}

\begin{proof}[Idea of proof]
Use Lemma~\ref{lem:padding-menu-p} to raise $v_2(A)$ and set $B\bmod2$ while keeping family $f$ fixed, guaranteeing the router plan and integrality.
Then apply the single terminal token $T_{\mathrm{term}}$ (possibly $o\to e$ or $e\to o$) to land in the required final family and residue.
The two cases follow from the last-row congruence; see the preamble lemmas on composition with $u=\lfloor x/18\rfloor$.
\end{proof}

\begin{lemma}[One-step $o\to e$ witness at $p=0$]\label{lem:omega-one-step}
At $p=0$, the token $\omega_1$ (\texttt{oe}, $\delta=1$) maps every $x\equiv 47\pmod{72}$ to $x'\equiv 31\pmod{48}$ in one certified step.
\end{lemma}

\begin{proof}
$\omega_1$ has $x'=12m+7$ with $m=\lfloor x/18\rfloor$ and requires router $j=\lfloor x/6\rfloor\bmod3=1$.
Modulo $48$, $12m+7$ assumes $\{7,19,31,43\}$ as $m\bmod4\in\{0,1,2,3\}$, so to hit $31$ we need $m\equiv 2\ (\bmod 4)$.
Writing $x=18t+r$, $j=1$ forces $r=11$; then $x=18t+11$ with $m=t$.
Imposing $t\equiv 2\ (\bmod 4)$ gives $x\equiv 72s+47\ (\bmod 72)$, and for these $x$ we get $x'\equiv 31\ (\bmod 48)$.
\end{proof}

\begin{example}[Minimal $o\to e$ lift to $31\bmod48$]\label{ex:oe-31}
Take $x=47$ (odd family, $j=1$). Then $m=\lfloor 47/18\rfloor=2$ and $\omega_1$ yields
$x'=12\cdot 2 + 7 = 31 \equiv 31 \pmod{48}$.
This is a pure one-step $o\to e$ terminal move with no padding.
\end{example}

\begin{remark}[How this changes usage]
In the earlier same-family presentation, all tail surgery stayed in one family.
Here we keep padding same-family (so all router choices in the prefix are preserved),
but we \emph{allow the final token to change family} if that is what the target residue requires.
The algebra (updates of $(A,B,\delta)$ and the last-step congruence) is unchanged; only the terminal family is now dictated by the last token.
\end{remark}


\subsection*{Introductory overview and notation}

\paragraph{Unified linear form for a certified word.}
Every finite certified word \(W\) acts on an index \(m\) by a linear map inside a fixed outer factor:
\begin{equation*}
  x_W(m) \;=\; 6\big(A_W\,m + B_W\big) + \delta_W,
\end{equation*}
where \(A_W=2^{\sum \alpha_i}\) is a pure power of two (sum of the row exponents used in \(W\)),
\(B_W\in\mathbb{Z}\) is the accumulated internal constant, and \(\delta_W\in\{1,5\}\) is the terminal offset determined by the last row.


\paragraph{Composition rule for $(A,B,\delta)$ (with $u=\lfloor x/18\rfloor$).}
If a prefix $W_{\mathrm{pre}}$ has $x_{W_{\mathrm{pre}}}(m)=6\big(A\,m+B\big)+\delta_{\mathrm{pre}}$, and we append a last token in column $p$ with parameters $(\alpha_p,k^{(p)},\delta_T)$ while using router remainder $r\in\{0,1,2\}$ at that step, then with
\[
u=\Big\lfloor \frac{x_{W_{\mathrm{pre}}}}{18}\Big\rfloor=\frac{A\,m + B - r}{3}
\]
we obtain the exact update
\[
A_{\mathrm{final}}=\frac{2^{\alpha_p}}{3}\,A,\qquad
B_{\mathrm{final}}=\frac{2^{\alpha_p}}{3}\,(B-r)+k^{(p)},\qquad
\delta_{\mathrm{final}}=\delta_T.
\]
In particular $B_{\mathrm{final}}\equiv k^{(p)}\pmod{2}$, so the last token fixes the final $B$-parity on the admissible $m$-class.


\paragraph{Why \(B\bmod 2\) matters.}
When we target a residue class \(x_{\mathrm{tar}}\) modulo \(M_K:=3\cdot 2^K\), the last step (or the whole word) leads to a binary congruence for \(m\):
\begin{equation*}
  2^{\alpha_p}\,m \;\equiv\; \frac{x_{\mathrm{tar}}-\delta}{6} - B_W \;\;(\mathrm{mod}\; 2^K),
\end{equation*}
so the \emph{parity} of the right-hand side depends on \(B_W\bmod 2\). Being able to \emph{choose or flip} \(B_W\bmod 2\) is therefore the knob that makes this congruence solvable in a convenient class of \(m\) (or places us in an \(m\)-independent pinning regime; see below).

\paragraph{Per-token effects (the tables).}
The tables list, for each admissible token (row) and each column \(p\ge 0\),
\begin{equation*}
  x' \;=\; 6\big(2^{\alpha_p} m + k^{(p)}\big) + \delta,
  \qquad \alpha_p := \alpha + 6p,\quad
  k^{(p)} := \frac{\beta\cdot 64^{\,p}+c}{9}\in\mathbb{Z}.
\end{equation*}
\paragraph{Two immediate consequences (for a chosen last token).}
Fix a specific last token \(T\) (one row in the column-\(p\) table), with parameters
\[
T:\quad x' \;=\; 6\big(2^{\alpha_p} m + k^{(p)}\big)+\delta,
\qquad \alpha_p=\alpha+6p,\quad k^{(p)}=\frac{\beta\cdot 64^{\,p}+c}{9}.
\]
Then:
\begin{itemize}
  \item \textbf{Exponent gain.} Appending \(T\) multiplies the internal slope by \(2^{\alpha_p}\); equivalently it adds \(\alpha_p\) to \(v_2(A)\).
  \item \textbf{Final \(B\)-parity.} Because \(2^{\alpha_p}m\) is even, the last step \(T\) fixes
        \(B_{\text{final}}\equiv k^{(p)}\pmod 2\).
        This is why the tables list \(k^{(p)}\bmod 2\): it tells you the resulting \(B\bmod 2\) after using \(T\) last.
\end{itemize}
\emph{Example.} If you pick \(T=\Psi_1\) at \(p=0\), the table gives \(\alpha_p=4\) and \(k^{(0)}=6\) (even), so appending \(T\) adds \(+4\) to \(v_2(A)\) and sets \(B_{\text{final}}\equiv 0\pmod 2\); indeed \(x'=96m+37\) pins \(x'\equiv 37\pmod{48}\) when \(K\le 4\).


\paragraph{Same-family steering blocks.}
A \emph{same-family steering block} is a short tail (one or two tokens) that starts and ends in the same family (\(e\to e\) or \(o\to o\)). We use a small “menu” with two roles:
\begin{enumerate}
  \item \textbf{Monotone lift:} each block has \(\Delta v_2(A)>0\), so appending blocks \emph{monotonically} raises \(v_2(A)\) and lets us reach any target \(K\).
  \item \textbf{Parity control:} a 2-token same-family cycle toggles \(B\bmod 2\) when used once (and preserves it when used twice), without changing family.
\end{enumerate}
Because these blocks return to the same family, they do not disturb the desired terminal family for the last step.

\paragraph{Monotone padding (what we actually do).}
Given a fixed routing prefix \(W\), we append same-family blocks until \(v_2(A)\) is large enough for the modulus \(3\cdot 2^K\). Then:
\begin{itemize}
  \item If \(\alpha_p \ge K\) for the chosen last token, the last step \emph{pins} the residue
        \(x' \equiv 6k^{(p)}+\delta \;(\mathrm{mod}\; 3\cdot 2^K)\) \emph{independently of \(m\)}.
  \item If \(\alpha_p < K\), we solve the linear congruence
        \(6\cdot 2^{\alpha_p}m \equiv x_{\mathrm{tar}}-(6k^{(p)}+\delta)\;(\mathrm{mod}\; 3\cdot 2^K)\)
        for \(m \;(\mathrm{mod}\; 2^{\,K-(\alpha_p+1)})\).
        If a particular \(B\bmod 2\) is needed, we first apply the same-family parity cycle to match it.
\end{itemize}

\paragraph{Why columns \(p\) help.}
Lifting to column \(p\) replaces \(\alpha\) by \(\alpha_p=\alpha+6p\).
Thus the same token at higher \(p\) contributes \(+6\) more two-adic bits per unit of \(p\):
it is a stronger “bit pump.” The lifted constants \(k^{(p)}\) (hence \(B\bmod 2\)) are read directly from the tables.

\paragraph{Prefix stability.}
By the routing-compatibility lemma, once \(W\) is fixed we may choose the tail (and the final \(m\)-class) so that all router choices inside \(W\) remain unchanged. In practice: \emph{freeze the prefix; do all surgery at the tail.}



Fix a column \(p\ge 0\). Any admissible row in column \(p\) has unified form
\begin{equation*}
  x' \;=\; 6\big(2^{\alpha+6p}\,m + k^{(p)}\big) + \delta^{(p)},
\end{equation*}
where \(\alpha\) is the base (column \(p{=}0\)) exponent of that row, and \(k^{(p)},\delta^{(p)}\) are the lifted constants in column \(p\).
Fix a column $p\ge 0$. Any admissible row in column $p$ has unified form
\[
x' \;=\; 6\big(2^{\alpha+6p}\,u + k^{(p)}\big) + \delta^{(p)},\qquad u=\Big\lfloor \frac{x}{18}\Big\rfloor.
\]
For a (finite) tail block $S=(R_1,\ldots,R_\ell)$ taken entirely from column $p$, set
\[
\alpha(S):=\sum_{i=1}^{\ell}\alpha(R_i),\qquad
\ell(S):=\ell,\qquad
\Delta_p(S):=\alpha(S)+6p\,\ell(S).
\]
Each token multiplies the current slope by $\frac{2^{\alpha(R_i)+6p}}{3}$ (by the composition rule), hence appending $S$ multiplies the internal slope by
\[
\frac{2^{\Delta_p(S)}}{3^{\,\ell(S)}}.
\]
Equivalently,
\[
x_{W\!\cdot\!S}(m) \;=\; 6\Big(\frac{2^{\Delta_p(S)}}{3^{\,\ell(S)}}A_W\,m \;+\; B_{W\!\cdot\!S}\Big) + \delta_{W\!\cdot\!S},
\qquad\Rightarrow\qquad
v_2\!\big(A_{W\!\cdot\!S}\big) \;=\; v_2(A_W)+\Delta_p(S).
\]
\emph{Remark.} The division by $3^{\ell(S)}$ does not affect $v_2(\cdot)$, so the two-adic lift remains additive in $\Delta_p(S)$, as used in the padding lemmas.


\begin{lemma}[Same-family steering menu at column \(p\)]\label{lem:menu-p}
There exists a finite menu \(\mathcal{S}_p\) of tail blocks, each taken from column \(p\), such that:
\begin{enumerate}
  \item (\emph{Family preservation}) Each \(S\in\mathcal{S}_p\) starts in a prescribed terminal family (\(e\) or \(o\)) and ends in the \emph{same} family.
  \item (\emph{Monotone lift}) Each \(S\in\mathcal{S}_p\) satisfies \(\Delta_p(S)>0\).
  \item (\emph{Parity control}) The menu contains a 2-token cycle \(C_p\) (same-family in/out) that toggles \(B\bmod 2\).
\end{enumerate}
\end{lemma}

\noindent\textbf{Canonical choices (use the p table to fill \(\alpha\) and \(B\)-effects).}
Below, \(\alpha(\cdot)\) denotes the base exponent at \(p=0\) for that row; at column \(p\), the lift amount is \(\Delta_p=\alpha(\cdot)+6p\) per token.

\medskip
\noindent\emph{End in \(e\):}
\begin{center}
\begin{tabular}{|l|c|c|c|}
\hline
Block & Entry \(\to\) Exit family & \(\Delta_p(S)\) & Effect on \(B\bmod 2\) \\
\hline
\(\Psi_1\) & \(e\to e\) & \(\alpha(\Psi_1)+6p\) & (from table) \\
\(\Psi_2\) & \(e\to e\) & \(\alpha(\Psi_2)+6p\) & (from table) \\
\(\psi_2\circ\omega_1\) & \(e\to o\to e\) & \(\alpha(\psi_2)+\alpha(\omega_1)+12p\) & toggles \\
\hline
\end{tabular}
\end{center}

\noindent\emph{End in \(o\):}
\begin{center}
\begin{tabular}{|l|c|c|c|}
\hline
Block & Entry \(\to\) Exit family & \(\Delta_p(S)\) & Effect on \(B\bmod 2\) \\
\hline
\(\Omega_1\) & \(o\to o\) & \(\alpha(\Omega_1)+6p\) & (from table) \\
\(\Omega_0\) & \(o\to o\) & \(\alpha(\Omega_0)+6p\) & (from table) \\
\(\omega_1\circ\psi_2\) & \(o\to e\to o\) & \(\alpha(\omega_1)+\alpha(\psi_2)+12p\) & toggles \\
\hline
\end{tabular}
\end{center}

\begin{lemma}[Monotone padding at column \(p\)]\label{lem:mono-p}
Fix a target \(K\) and a terminal family. For any prefix \(W\) that already ends in that family, there exists a concatenation \(S=S_1\cdots S_q\) with each \(S_i\in\mathcal{S}_p\) such that
\begin{equation*}
  v_2\!\big(A_{W\!\cdot\!S}\big)\ \ge\ K,\qquad
  \text{and}\qquad
  \text{the terminal family of }W\!\cdot\!S\text{ equals that of }W.
\end{equation*}
Moreover, if a specific parity \(B_{W\!\cdot\!S}\bmod 2\) is required, appending the cycle \(C_p\) an odd (resp.\ even) number of times flips (resp.\ preserves) \(B\bmod 2\).
\end{lemma}

\begin{proof}
By Lemma~\ref{lem:menu-p}(ii) each block adds a positive \(\Delta_p(S)\) to \(v_2(A)\); thus repeating lift blocks reaches any \(K\).
Entries/exits are same-family by Lemma~\ref{lem:menu-p}(i). The last statement follows from the “toggles’’ entry of the table (parity is flipped by one use of \(C_p\), preserved by two).
\end{proof}

\begin{lemma}[Routing-compatibility of the prefix]\label{lem:route-p}
Let \(W\) be any fixed prefix. Appending a tail \(S\) from \(\mathcal{S}_p\) and solving the final congruence (mod \(3\cdot 2^K\)) so that
\begin{equation*}
  m \equiv m^\ast \pmod{2^{S^\ast}}\qquad \text{with}\quad
  S^\ast \ge \max_{t<|W|}\big(1+\sum_{i< t}\alpha_i\big)
\end{equation*}
ensures all prefix routers remain as planned (no branch flips). Pick the solution class whose mod-\(3\) part matches the prefix admissibility.
\end{lemma}


% === Strong routing compatibility wrapper (added) ===
\begin{lemma}[Strong routing compatibility]\label{lem:routing-compat-strong}
Let $W$ be a fixed prefix with planned routers $j_{t+1}$. There exists an exponent depth $S^\ast$ such that if the final congruence is solved with
$m\equiv m^\ast\ (\bmod\,2^{S^\ast})$ and the mod-$3$ part in the admissible class, then the realized remainders equal $r_{t+1}=j_{t+1}$ at every prefix step.
\end{lemma}

\paragraph{Representative examples (all columns).}
\begin{itemize}
  \item \textbf{Lift in \(o\) at \(p=0\) (pin modulo \(48\), then raise \(K\)):}
        Append \(\Omega_1\) or \(\Omega_0\). Each adds \(\Delta_0=\alpha(\cdot)\) bits and preserves \(o\).
        If you end with \(\Omega_0\) and \(\alpha(\Omega_0)\ge 4\), then by Cor.\ of the last-row lemma, the last step pins an \(m\)-independent residue modulo \(48\), while repeated \(\Omega\)-blocks raise \(K\) monotonically.
  \item \textbf{Congruence targeting at \(p=1\) (end in \(e\), higher lift per token):}
        Suppose the last row is \(\Psi_1\) at column \(p{=}1\). Then the last-step form is
        \[
          x' = 6\big(2^{\alpha(\Psi_1)+6}\,m + k^{(1)}\big)+\delta^{(1)}.
        \]
        For modulus \(M_K=3\cdot 2^K\):
        – If \(K \le \alpha(\Psi_1)+6\), this **pins** \(x'\equiv 6k^{(1)}+\delta^{(1)}\pmod{M_K}\) independently of \(m\).
        – If \(K > \alpha(\Psi_1)+6\), solve
        \[
          6\cdot 2^{\alpha(\Psi_1)+6}\,m \equiv x_{\mathrm{tar}}-(6k^{(1)}+\delta^{(1)}) \pmod{3\cdot 2^K}
        \]
        to get the explicit \(m\)-class from the lemma. Use \(\Psi\)-blocks in column \(p{=}1\) to raise \(v_2(A)\) as needed; any 2-token \(e\)-cycle in \(\mathcal{S}_1\) toggles \(B\bmod 2\) without leaving \(e\).
\end{itemize}

\noindent\emph{Implementation note.} To fully instantiate the tables above, fill the base exponents \(\alpha(\cdot)\) and the \(B\bmod 2\) effects from the \(p{=}0\) table; the column-\(p\) lift is automatic via \(\Delta_p = \alpha(\cdot)+6p\) per token. If you’d like, we can tabulate the exact \(B\)-updates once you share (or re-upload) the row formulas for each \(p\).

\subsection*{Micro-examples (reading one table row and applying the lemma)}

\paragraph{P0–A (pin at $K=4$ using $T=\Psi_1$ at $p=0$, family $e\to e$).}
From the $p=0$ row $(\mathrm e,1)$–\texttt{ee}–$\Psi_1$:
\begin{equation*}
  \alpha_p=4,\qquad k^{(0)}=6\ \text{(even)},\qquad \delta=1,\qquad
  x'=6\big(2^{4}m+6\big)+1 = 96m+37.
\end{equation*}
With $K=4$ (mod $48$) and $\alpha_p\ge K$, we are in the pinning regime:
\begin{equation*}
  x' \equiv 6k^{(0)}+\delta \equiv 36+1 \equiv 37 \pmod{48}
  \quad \text{for every } m,
  \qquad B_{\text{final}}\equiv k^{(0)}\equiv 0\ (\bmod 2).
\end{equation*}

\paragraph{P0–B (solve for $m$ at $K=4$ using $T=\Omega_2$ at $p=0$, family $o\to o$).}
From $(\mathrm o,2)$–\texttt{oo}–$\Omega_2$ at $p=0$:
\begin{equation*}
  \alpha_p=1,\qquad k^{(0)}=1\ \text{(odd)},\qquad \delta=5,\qquad
  x'=6\big(2^{1}m+1\big)+5 = 12m+11.
\end{equation*}
Target $x_{\mathrm{tar}}\equiv 35\pmod{48}$, with $K=4>\alpha_p$:
\begin{equation*}
  12m \equiv 35-(6\cdot 1+5)=24 \pmod{48}
  \quad\Longrightarrow\quad m \equiv 2 \pmod{4}.
\end{equation*}
So any admissible input in family $o$ with router $j{=}2$ and $m\equiv 2\ (\bmod 4)$ lands at $35\ (\bmod 48)$. Here $B_{\text{final}}\equiv 1\ (\bmod 2)$.

\paragraph{P2–C (pin at $K=10$ using $T=\Psi_0$ at $p=2$, family $e\to e$).}
From the $p=2$ row $(\mathrm e,0)$–\texttt{ee}–$\Psi_0$ (unified table):
\begin{equation*}
  F(2,m)=16384\,m+910,\qquad x'_2=98304\,m+5461.
\end{equation*}
Thus $\alpha_p=14$ (since $\alpha=2$ and $\alpha+12=14$), $k^{(2)}=910$ (even), and $\delta=1$.
With $K=10$ (mod $3072$) and $\alpha_p\ge K$,
\begin{equation*}
  x'_2 \equiv 6k^{(2)}+\delta \equiv 5461 \equiv 2389 \pmod{3072}
  \quad \text{independently of } m,
  \qquad B_{\text{final}}\equiv 0\ (\bmod 2).
\end{equation*}

\paragraph{P2–D (solve for $m$ at $K=15$ using $T=\Omega_2$ at $p=2$, family $o\to o$).}
From $(\mathrm o,2)$–\texttt{oo}–$\Omega_2$ at $p=2$ (unified table):
\begin{equation*}
  F(2,m)=8192\,m+7736,\qquad x'_2=49152\,m+46421,
\end{equation*}
so $\alpha_p=13$ (since $\alpha=1$), $k^{(2)}=7736$ (even), and $\delta=5$.
Let $K=15$ (mod $98304$) and set
\begin{equation*}
  x_{\mathrm{tar}} \equiv 46421 + 49152 = 95573 \pmod{98304}.
\end{equation*}
Then with $a=6\cdot 2^{\alpha_p}=49152$,
\begin{equation*}
  a\,m \equiv x_{\mathrm{tar}}-(6k^{(2)}+\delta)
  \equiv 95573-46421 = 49152 \pmod{98304}
  \ \Longrightarrow\ m \equiv 1 \pmod{2}.
\end{equation*}
Any admissible input in family $o$ with router $j{=}2$ and \emph{odd} $m$ lands at $x_{\mathrm{tar}}\ (\bmod 98304)$; here $B_{\text{final}}\equiv 0\ (\bmod 2)$.


\subsection*{How to read the column-$p$ tables below}

\paragraph{Per-token effects table (\boldmath$\alpha_p$ and $B\bmod 2$).}
These compact tables summarize what each token does to the linear map inside $x_W(m)=6(A_W m+B_W)+\delta_W$ when appended as the last step in column $p$:

\begin{itemize}
  \item \boldmath$\alpha$: the base exponent (at $p=0$) appearing in $2^{\alpha}$ for that token.
  \item \boldmath$\alpha_p=\alpha+6p$: the exponent actually \emph{added} to $v_2(A_W)$ at column $p$. Appending the token multiplies $A_W$ by $2^{\alpha_p}$.
  \item \boldmath$k^{(p)}\bmod 2$: the parity of the lifted constant $k^{(p)}=\bigl(\beta\cdot 64^{\,p}+c\bigr)/9$. Since $x'=6(2^{\alpha_p}m + k^{(p)})+\delta$, the new $B$ modulo $2$ is $B'\equiv k^{(p)}\ (\bmod\,2)$. This is the entry we use for \emph{parity control} via short cycles.
\end{itemize}

\paragraph{Worked line decoding (one example).}
Take the $p=2$ line $(\mathrm e,0)$–\texttt{ee}–$\Psi_0$ in the unified table:
\[
  F(2,m)=16384\,m+910,\qquad x'_2(m)=98304\,m+5461.
\]
Here $\alpha=2$ so $\alpha_2=\alpha+12=14$ and indeed $2^{14}=16384$ is the slope of $F(2,m)$.
The lifted constant is $k^{(2)}=910$, hence $B'\equiv k^{(2)}\equiv 0\pmod 2$ in the effects table.

\paragraph{Usage in the padding lemmas.}
\begin{itemize}
  \item The column \(\alpha_p\) tells you exactly how many $2$-adic bits each token adds at column \(p\); stacking tokens from a same-family menu gives the monotone lift to any target \(K\).
  \item The column \(k^{(p)}\bmod 2\) is the parity knob: a two-token same-family cycle with an \emph{odd} number of parity-flip entries toggles \(B\bmod 2\) while returning to the same family, ensuring solvability of the final congruence when a specific parity is required.
\end{itemize}



%Per-token effects by column 𝑝
\begin{table}[!htbp]
\centering
\caption{Per-token effects at $p=0$: added exponent $\alpha_p=\alpha+6p$ and $B$-parity update $B\mapsto k^{(0)}\bmod 2$.}
\label{tab:menu-effects-p0}
\begin{tabular}{@{}c c c c c c@{}}
\toprule
$(s,j)$ & type & move & $\alpha$ & $\alpha_p$ & $k^{(0)}\bmod 2$ \\ \midrule
$(\mathrm e,0)$ & \texttt{ee} & $\Psi_{0}$ & $2$ & $2$ & $0$ \\
$(\mathrm e,1)$ & \texttt{ee} & $\Psi_{1}$ & $4$ & $4$ & $0$ \\
$(\mathrm e,2)$ & \texttt{ee} & $\Psi_{2}$ & $6$ & $6$ & $0$ \\
$(\mathrm o,0)$ & \texttt{oe} & $\omega_{0}$ & $3$ & $3$ & $0$ \\
$(\mathrm o,1)$ & \texttt{oe} & $\omega_{1}$ & $1$ & $1$ & $1$ \\
$(\mathrm o,2)$ & \texttt{oe} & $\omega_{2}$ & $5$ & $5$ & $0$ \\
\midrule
$(\mathrm e,0)$ & \texttt{eo} & $\psi_{0}$ & $4$ & $4$ & $0$ \\
$(\mathrm e,1)$ & \texttt{eo} & $\psi_{1}$ & $6$ & $6$ & $0$ \\
$(\mathrm e,2)$ & \texttt{eo} & $\psi_{2}$ & $2$ & $2$ & $0$ \\
$(\mathrm o,0)$ & \texttt{oo} & $\Omega_{0}$ & $5$ & $5$ & $0$ \\
$(\mathrm o,1)$ & \texttt{oo} & $\Omega_{1}$ & $3$ & $3$ & $0$ \\
$(\mathrm o,2)$ & \texttt{oo} & $\Omega_{2}$ & $1$ & $1$ & $1$ \\
\bottomrule
\end{tabular}
\end{table}

\begin{table}[!htbp]
\centering
\caption{Per-token effects at $p=1$: added exponent $\alpha_p=\alpha+6p$ and $B$-parity update $B\mapsto k^{(1)}\bmod 2$.}
\label{tab:menu-effects-p1}
\begin{tabular}{@{}c c c c c c@{}}
\toprule
$(s,j)$ & type & move & $\alpha$ & $\alpha_p$ & $k^{(1)}\bmod 2$ \\ \midrule
$(\mathrm e,0)$ & \texttt{ee} & $\Psi_{0}$ & $2$ & $8$ & $0$ \\
$(\mathrm e,1)$ & \texttt{ee} & $\Psi_{1}$ & $4$ & $10$ & $0$ \\
$(\mathrm e,2)$ & \texttt{ee} & $\Psi_{2}$ & $6$ & $12$ & $0$ \\
$(\mathrm o,0)$ & \texttt{oe} & $\omega_{0}$ & $3$ & $9$ & $0$ \\
$(\mathrm o,1)$ & \texttt{oe} & $\omega_{1}$ & $1$ & $7$ & $0$ \\
$(\mathrm o,2)$ & \texttt{oe} & $\omega_{2}$ & $5$ & $11$ & $0$ \\
\midrule
$(\mathrm e,0)$ & \texttt{eo} & $\psi_{0}$ & $4$ & $10$ & $0$ \\
$(\mathrm e,1)$ & \texttt{eo} & $\psi_{1}$ & $6$ & $12$ & $0$ \\
$(\mathrm e,2)$ & \texttt{eo} & $\psi_{2}$ & $2$ & $8$ & $0$ \\
$(\mathrm o,0)$ & \texttt{oo} & $\Omega_{0}$ & $5$ & $11$ & $0$ \\
$(\mathrm o,1)$ & \texttt{oo} & $\Omega_{1}$ & $3$ & $9$ & $0$ \\
$(\mathrm o,2)$ & \texttt{oo} & $\Omega_{2}$ & $1$ & $7$ & $0$ \\
\bottomrule
\end{tabular}
\end{table}

\begin{table}[!htbp]
\centering
\caption{Per-token effects at $p=2$: added exponent $\alpha_p=\alpha+6p$ and $B$-parity update $B\mapsto k^{(2)}\bmod 2$.}
\label{tab:menu-effects-p2}
\begin{tabular}{@{}c c c c c c@{}}
\toprule
$(s,j)$ & type & move & $\alpha$ & $\alpha_p$ & $k^{(2)}\bmod 2$ \\ \midrule
$(\mathrm e,0)$ & \texttt{ee} & $\Psi_{0}$ & $2$ & $14$ & $0$ \\
$(\mathrm e,1)$ & \texttt{ee} & $\Psi_{1}$ & $4$ & $16$ & $0$ \\
$(\mathrm e,2)$ & \texttt{ee} & $\Psi_{2}$ & $6$ & $18$ & $0$ \\
$(\mathrm o,0)$ & \texttt{oe} & $\omega_{0}$ & $3$ & $15$ & $0$ \\
$(\mathrm o,1)$ & \texttt{oe} & $\omega_{1}$ & $1$ & $13$ & $0$ \\
$(\mathrm o,2)$ & \texttt{oe} & $\omega_{2}$ & $5$ & $17$ & $0$ \\
\midrule
$(\mathrm e,0)$ & \texttt{eo} & $\psi_{0}$ & $4$ & $16$ & $0$ \\
$(\mathrm e,1)$ & \texttt{eo} & $\psi_{1}$ & $6$ & $18$ & $0$ \\
$(\mathrm e,2)$ & \texttt{eo} & $\psi_{2}$ & $2$ & $14$ & $0$ \\
$(\mathrm o,0)$ & \texttt{oo} & $\Omega_{0}$ & $5$ & $17$ & $0$ \\
$(\mathrm o,1)$ & \texttt{oo} & $\Omega_{1}$ & $3$ & $15$ & $0$ \\
$(\mathrm o,2)$ & \texttt{oo} & $\Omega_{2}$ & $1$ & $13$ & $0$ \\
\bottomrule
\end{tabular}
\end{table}

\noindent\emph{Convention.} In the unified row formulas $x'=6\big(2^{\alpha_p}m+k^{(p)}\big)+\delta$, the symbol $m$ denotes the \emph{internal} index $u=\lfloor x/18\rfloor$ at that step (not the global top-level $m$).


\subsection*{Examples using the per-token effects tables}

We illustrate how to read off the exponent gain $\alpha_p$, the $B$-parity $k^{(p)}\bmod 2$, and then apply the residue-targeting lemma in two regimes: (i) $m$-independent pinning when $\alpha_p\ge K$, and (ii) solving a linear congruence for $m$ when $\alpha_p<K$. Recall $M_K:=3\cdot 2^K$ and
\[
x' \;=\; 6\big(2^{\alpha_p} m + k^{(p)}\big)+\delta,\qquad
a:=6\cdot 2^{\alpha_p}.
\]

\paragraph{Example P0–A (pin at $K=4$ with \texorpdfstring{$\Psi_1$}{Psi1}).}
Token: $(\mathrm e,1)$–\texttt{ee}–$\Psi_1$ at $p=0$.
From the $p=0$ tables:
\[
\alpha=4\ \Rightarrow\ \alpha_0=\alpha=4,\quad
k^{(0)}=6\ (\text{even}),\quad
\delta=1,\quad
x'=96m+37.
\]
Take $K=4$ so $M_4=48$. Since $\alpha_0\ge K$, the corollary gives
\[
x' \equiv 6k^{(0)}+\delta \equiv 36+1 \equiv 37 \pmod{48}
\]
\emph{independently of $m$}. (Here $B'$ is even because $k^{(0)}$ is even.)

\paragraph{Example P0–B (solve for $m$ at $K=4$ with \texorpdfstring{$\Omega_2$}{Omega2}).}
Token: $(\mathrm o,2)$–\texttt{oo}–$\Omega_2$ at $p=0$.
From the $p=0$ tables:
\[
\alpha=1\ \Rightarrow\ \alpha_0=1,\quad
k^{(0)}=1\ (\text{odd}),\quad
\delta=5,\quad
x'=12m+11.
\]
Target $x_{\mathrm{tar}}\equiv 35\ (\bmod 48)$ ($K=4$). Since $\alpha_0<K$, solve
\[
a\,m \equiv x_{\mathrm{tar}}-(6k^{(0)}+\delta) \pmod{48},
\quad a=6\cdot 2^{\alpha_0}=12.
\]
Compute $6k^{(0)}+\delta=6\cdot 1+5=11$, so $12m\equiv 35-11=24\ (\bmod 48)$, hence
\[
m \equiv 2 \pmod{4}.
\]
Choosing any admissible input in family $o$ with router $j{=}2$ and $m\equiv 2\ (\bmod 4)$ makes the last step land at $35\ (\bmod 48)$. (Here $B'$ is odd because $k^{(0)}$ is odd.)

\paragraph{Example P2–C (pin at $K=10$ with \texorpdfstring{$\Psi_0$}{Psi0}).}
Token: $(\mathrm e,0)$–\texttt{ee}–$\Psi_0$ at $p=2$.
From the $p=2$ unified table:
\[
F(2,m)=16384m+910,\quad x'_2=98304m+5461.
\]
Thus $\alpha=2\Rightarrow \alpha_2=\alpha+12=14$ and $k^{(2)}=910$ (even), $\delta=1$ (since $6\cdot 910+1=5461$).
For $K=10$ ($M_{10}=3072$) we have $\alpha_2\ge K$, so pin:
\[
x'_2 \equiv 6k^{(2)}+\delta \equiv 5461 \equiv 2389 \pmod{3072},
\]
\emph{independently of $m$}. (Again $B'$ is even here.)

\paragraph{Example P2–D (solve for $m$ at $K=15$ with \texorpdfstring{$\Omega_2$}{Omega2}).}
Token: $(\mathrm o,2)$–\texttt{oo}–$\Omega_2$ at $p=2$.
From the $p=2$ unified table:
\[
F(2,m)=8192m+7736,\quad x'_2=49152m+46421.
\]
Hence $\alpha=1\Rightarrow \alpha_2=\alpha+12=13$ and $k^{(2)}=7736$ (even), $\delta=5$ (since $6\cdot 7736+5=46421$).
Let $K=15$ so $M_{15}=3\cdot 2^{15}=98304$. Choose target
\[
x_{\mathrm{tar}} \equiv 95573 \equiv 46421+49152 \pmod{98304},
\]
so that solvability is guaranteed (divisibility by $g=3\cdot 2^{\alpha_2+1}=49152$).
Solve
\[
a\,m \equiv x_{\mathrm{tar}}-(6k^{(2)}+\delta) \pmod{98304},\qquad a=6\cdot 2^{\alpha_2}=49152.
\]
Then $am \equiv 95573-46421=49152\ (\bmod 98304)$, giving
\[
m \equiv \frac{49152}{3\cdot 2^{14}} \equiv 1 \pmod{2}.
\]
Thus any admissible input in family $o$ with router $j{=}2$ and \emph{odd} $m$ lands at $x_{\mathrm{tar}}\equiv 95573\ (\bmod 98304)$ on this last step.

\paragraph{Worked example 1: $17\bmod 24 \to 41\bmod 48$ via $\omega_1\to\psi_2$ (full routing and floors).}
We use the $p=0$ rows from the unified table:
\[
\omega_{1}:\ x'=12m+7\quad(\texttt{oe},\ j=1),\qquad
\psi_{2}:\ x'=24m+17\quad(\texttt{eo},\ j=2),
\]
with the convention that at each step $m=\big\lfloor x/18\big\rfloor$ and the router is $j=\big\lfloor x/6\big\rfloor\bmod 3$.

Write every \(x \equiv 17 \pmod{24}\) as \(x = 24n + 17\) with \(n \in \mathbb{Z}\).
The table row \(\omega_1\) requires \(j_0 = 1\), where
\[
  j_0 \;=\; \big\lfloor x/6 \big\rfloor \bmod 3.
\]
Compute:
\[
  j_0
  \;=\;
  \Big\lfloor \frac{24n+17}{6} \Big\rfloor \bmod 3
  \;=\;
  (4n+2) \bmod 3
  \;\equiv\;
  n+2 \pmod{3}.
\]
So \(\omega_1\) is admissible exactly when \(n \equiv 1 \pmod{3}\) or \(n \equiv 2 \pmod{3}\).
We’ll pick a concrete congruence class that also makes the second step admissible and forces
the needed parity; a clean choice is \(n \equiv 8 \pmod{9}\).


\begin{enumerate}
\item \textbf{Choose representative in the class $x\equiv 17\pmod{24}$.}
Write $x=24n+17$ and pick the subclass $n\equiv 8\pmod 9$, i.e.\ $x\equiv 209\pmod{216}$.
Take the concrete start $x_0=\boxed{209}$.

\item \textbf{Router for step 1 (\(\omega_1\) admissibility).}
\[
j_0=\Big\lfloor \frac{209}{6}\Big\rfloor\bmod 3 = 34 \bmod 3 = \boxed{1}
\]
so the row $j=1$ for $\omega_1$ is admissible (type \texttt{oe} from odd to even family).

\item \textbf{Apply $\omega_1$ (uses $m_0=\lfloor x_0/18\rfloor$).}
\[
m_0=\Big\lfloor\frac{209}{18}\Big\rfloor=11,\qquad
x_1=12m_0+7=12\cdot 11+7=\boxed{139}.
\]
Family check: $139\bmod 6=1$ (even family), consistent with \texttt{oe}.

\item \textbf{Router for step 2 (\(\psi_2\) admissibility).}
\[
j_1=\Big\lfloor \frac{139}{6}\Big\rfloor\bmod 3=23\bmod 3=\boxed{2},
\]
matching the row $j=2$ for $\psi_2$.

\item \textbf{Apply $\psi_2$ (uses $m_1=\lfloor x_1/18\rfloor$).}
\[
m_1=\Big\lfloor\frac{139}{18}\Big\rfloor=7\ \text{(odd)},\qquad
x_2=24m_1+17=24\cdot 7+17=\boxed{185}.
\]

\item \textbf{Target modulus and family.}
\[
185 \equiv 185-3\cdot 48 = \boxed{41}\pmod{48},\qquad
185\bmod 6=5\ \text{(odd)},
\]
as desired for landing in the odd family at $41\bmod 48$.

\item \textbf{Forward check with accelerated map $U(n)=(3n+1)/2^{v_2(3n+1)}$.}
\[
U(185)=\frac{3\cdot 185+1}{2^{v_2(556)}}=\frac{556}{4}=139,\qquad
U(139)=\frac{3\cdot 139+1}{2^{v_2(418)}}=\frac{418}{2}=209.
\]
Hence $U^2(185)=209$, certifying the inverse chain $209\xrightarrow{\omega_1}139\xrightarrow{\psi_2}185$.
\end{enumerate}

\paragraph{Why an intermediate value is required (made explicit).}
In this lift we use a \emph{two-step} certified inverse tail, so there is necessarily an
intermediate value $x_1$ between the start $x_0\equiv 17\pmod{24}$ and the target
$x_2\equiv 41\pmod{48}$. The reason we do not jump in one step is that, in the $p=0$ table,
no odd-exit row with $\alpha\ge 4$ has residue $6k^{(0)}+\delta\equiv 41\ (\bmod 48)$
(so there is no one-step pin to $41\bmod 48$). Consequently we choose a short two-token tail
whose first token routes us into a state where the \emph{second} token can be applied and
the final residue becomes $41\bmod 48$.

\paragraph{The intermediate value and why it exists.}
At each step, the token takes as input the internal index
$m=\big\lfloor x/18\big\rfloor$ and is \emph{admissible} when the router
$j=\big\lfloor x/6\big\rfloor\bmod 3$ matches the row index.
For the tail $\omega_1\to\psi_2$ at $p=0$:
\[
\omega_{1}:\ x'=12m+7\quad(\texttt{oe},\ j=1),\qquad
\psi_{2}:\ x'=24m+17\quad(\texttt{eo},\ j=2).
\]
Starting at $x_0\equiv 17\ (\bmod 24)$, write $x_0=24n+17$.
A direct computation shows
\[
j_0=\Big\lfloor\tfrac{x_0}{6}\Big\rfloor\bmod 3
   =(4n+2)\bmod 3\in\{1,2\},
\]
so for the subclass $n\equiv 2\ (\bmod 3)$ we indeed get $j_0=1$ and may apply $\omega_1$.
This \emph{guarantees} an intermediate value
\[
x_1 \;=\; 12\,\Big\lfloor \tfrac{x_0}{18}\Big\rfloor + 7,
\]
which is the unique output of the first certified move (and lands in the even family, as type \texttt{oe} dictates).
From this $x_1$, the router becomes
\[
j_1=\Big\lfloor\tfrac{x_1}{6}\Big\rfloor\bmod 3=2,
\]
hence $\psi_2$ is admissible and produces
\[
x_2 \;=\; 24\,\Big\lfloor \tfrac{x_1}{18}\Big\rfloor + 17
      \;\equiv\; 41 \pmod{48}.
\]

\paragraph{Concrete numbers (the middle hop shown).}
Take $x_0=209$ (so $n=8$). Then
\[
m_0=\Big\lfloor\tfrac{209}{18}\Big\rfloor=11,\quad
x_1=\underbrace{12m_0+7}_{\omega_1}=139
\quad\text{(this is the \emph{intermediate} value),}
\]
and
\[
m_1=\Big\lfloor\tfrac{139}{18}\Big\rfloor=7,\quad
x_2=\underbrace{24m_1+17}_{\psi_2}=185\equiv 41\pmod{48}.
\]
Thus the existence of the intermediate value $x_1=139$ is not optional: it is exactly the
output of the first certified inverse step, forced by the router $j_0=1$ and the floor
input $m_0=\lfloor x_0/18\rfloor$.
\medskip

\noindent\emph{Forward check (confirms the two-step structure).}
With the accelerated map $U(n)=(3n+1)/2^{v_2(3n+1)}$,
\[
U(185)=139,\qquad U(139)=209,
\]
so indeed $U^2(185)=209$, matching the inverse chain
\[
209 \xrightarrow{\ \omega_1\ } \underbrace{139}_{\text{intermediate}}
    \xrightarrow{\ \psi_2\ } 185\equiv 41\ (\bmod 48).
\]


\noindent
Thus the certified tail $\boxed{\omega_1\to\psi_2}$ at $p=0$ maps every $x\equiv 209\pmod{216}$ (a fixed subclass of $17\bmod 24$) to $41\bmod 48$.

\begin{lemma}[Why $n\equiv 8\pmod 9$ is the right refinement]
Let $x_0=24n+17$ so $x_0\equiv 17\pmod{24}$. Consider the two-token tail $\omega_1\to\psi_2$ at $p=0$ with
\[
\omega_1:\ x'=12m+7\quad(\texttt{oe},\ j=1),\qquad
\psi_2:\ x'=24m+17\quad(\texttt{eo},\ j=2),
\]
where at each step $m=\lfloor x/18\rfloor$ and $j=\lfloor x/6\rfloor\bmod 3$.
Then $\omega_1$ is admissible iff $n\equiv 2\pmod{3}$, and \emph{among those} $n$, the second router $j_1$ equals $2$ iff $n\equiv 8\pmod{9}$. In that refined class, the next internal index $m_1$ is odd, so $\psi_2$ lands at $x_2\equiv 41\pmod{48}$.
\end{lemma}

\begin{proof}
Write $x_0=24n+17$. The first router is
\[
j_0=\Big\lfloor \frac{x_0}{6}\Big\rfloor \bmod 3
   = (4n+2)\bmod 3
   \equiv n+2 \pmod{3}.
\]
Thus $j_0=1$ (needed for the row of $\omega_1$) iff $n\equiv 2\pmod{3}$.

Assume $n\equiv 2\pmod{3}$ and set $n=3k+2$ with $k\in\mathbb Z$. Applying $\omega_1$ uses
\[
m_0=\Big\lfloor \frac{24n+17}{18}\Big\rfloor
    =\Big\lfloor \frac{72k+65}{18}\Big\rfloor
    =4k+3,
\quad
x_1=12m_0+7=48k+43.
\]
Now compute the second router:
\[
j_1=\Big\lfloor \frac{x_1}{6}\Big\rfloor \bmod 3
   = (8k+7)\bmod 3
   \equiv 2k+1 \pmod{3}.
\]
We need $j_1=2$ for the row of $\psi_2$, so $2k+1\equiv 2\pmod{3}$, i.e.\ $2k\equiv 1\pmod{3}$, hence $k\equiv 2\pmod{3}$. Writing $k=3t+2$, we get
\[
n=3k+2=3(3t+2)+2=9t+8 \ \Longrightarrow\ n\equiv 8\pmod{9}.
\]

Finally, in this refined class we can read off the parity of the next internal index:
\[
m_1=\Big\lfloor \frac{x_1}{18}\Big\rfloor
    =\Big\lfloor \frac{48(3t+2)+43}{18}\Big\rfloor
    =\Big\lfloor \frac{144t+139}{18}\Big\rfloor
    =8t+\Big\lfloor \frac{139}{18}\Big\rfloor
    =8t+7,
\]
which is odd. Therefore the last step
\(
x_2=24m_1+17
\)
satisfies
\(
x_2\equiv 24\cdot (\text{odd}) + 17 \equiv 41 \pmod{48}.
\)
\end{proof}

\begin{remark}
Among the three residue classes $n\equiv 2,5,8\pmod{9}$ that satisfy $n\equiv 2\pmod{3}$, only $n\equiv 8\pmod{9}$ makes $k=(n-2)/3\equiv 2\pmod{3}$, hence $j_1=2$. In short:
\[
k\bmod 3 \in \{0,1,2\}\ \Longrightarrow\ j_1\equiv 2k+1 \bmod 3 \in \{1,0,2\},
\]
so the unique choice that gives $j_1=2$ is $k\equiv 2\pmod{3}$, i.e.\ $n\equiv 8\pmod{9}$.
\end{remark}

\begin{example}[Concrete instance with the refined class]
Take $n\equiv 8\pmod{9}$, e.g.\ $n=8$ so $x_0=24n+17=209$.
Then
\[
j_0=\Big\lfloor \frac{209}{6}\Big\rfloor\bmod 3=1
\quad\Rightarrow\quad
m_0=\Big\lfloor \frac{209}{18}\Big\rfloor=11,\quad
x_1=\underbrace{12m_0+7}_{\omega_1}=139.
\]
Next,
\[
j_1=\Big\lfloor \frac{139}{6}\Big\rfloor\bmod 3=2
\quad\Rightarrow\quad
m_1=\Big\lfloor \frac{139}{18}\Big\rfloor=7\ (\text{odd}),\quad
x_2=\underbrace{24m_1+17}_{\psi_2}=185\equiv 41\pmod{48}.
\]
This explicitly exhibits the forced intermediate value $x_1=139$ and the successful landing in $41\bmod 48$.
\end{example}


\bigskip

\paragraph{Worked example 2: a different subclass of $17\bmod 24$ to $41\bmod 48$ via $\omega_1\to\psi_2$.}
Use the same rows and conventions as above. Choose the subclass $x=24n+17$ with $n\equiv 1\pmod 9$ and a representative with $j_0=1$ (so $\omega_1$ is admissible); the smallest such is $n=19$, giving $x_0=\boxed{473}$.

\begin{enumerate}
\item \textbf{Router for step 1 (\(\omega_1\) admissibility).}
\[
j_0=\Big\lfloor \frac{473}{6}\Big\rfloor\bmod 3=78\bmod 3=\boxed{1}.
\]

\item \textbf{Apply $\omega_1$.}
\[
m_0=\Big\lfloor\frac{473}{18}\Big\rfloor=26,\qquad
x_1=12m_0+7=12\cdot 26+7=\boxed{319}.
\]
Family: $319\bmod 6=1$ (even), consistent with \texttt{oe}.

\item \textbf{Router for step 2 (\(\psi_2\) admissibility).}
\[
j_1=\Big\lfloor \frac{319}{6}\Big\rfloor\bmod 3=53\bmod 3=\boxed{2}.
\]

\item \textbf{Apply $\psi_2$.}
\[
m_1=\Big\lfloor\frac{319}{18}\Big\rfloor=17\ \text{(odd)},\qquad
x_2=24m_1+17=24\cdot 17+17=\boxed{425}.
\]

\item \textbf{Target modulus and family.}
\[
425 \equiv 425-8\cdot 48=\boxed{41}\pmod{48},\qquad
425\bmod 6=5\ \text{(odd)}.
\]

\item \textbf{Forward confirmation.}
By the certified inverse formulas,
$473\xrightarrow{\omega_1}319\xrightarrow{\psi_2}425$, hence $U^2(425)=473$.
\end{enumerate}

\noindent
This shows the \emph{same} short tail $\boxed{\omega_1\to\psi_2}$ also works for the subclass $x\equiv 24n+17$ with $n\equiv 1\pmod 9$ and $j_0=1$ (e.g.\ $x_0=473$), again landing at $41\bmod 48$.

\paragraph{Why an intermediate value is required (explicit).}
No odd-exit $p{=}0$ row with $\alpha\ge4$ satisfies $6k^{(0)}+\delta\equiv 41\ (\bmod 48)$,
so there is no one-step pin to $41\bmod 48$. We therefore use a two-token tail
$\omega_1\to\psi_2$, which necessarily produces an intermediate value $x_1$.

\paragraph{Intermediate value and routers (corrected representative).}
Take $x_0=24n+17$ with $n\equiv 8\pmod 9$; choose the concrete $x_0=\boxed{641}$ (so $n=26$).
Then
\[
j_0=\Big\lfloor \tfrac{641}{6}\Big\rfloor\bmod 3
   =106\bmod 3=\boxed{1},
\]
so $\omega_1$ (row $j{=}1$) is admissible and yields the \emph{intermediate value}
\[
m_0=\Big\lfloor \tfrac{641}{18}\Big\rfloor=35,\qquad
x_1=\underbrace{12m_0+7}_{\omega_1}=427.
\]
From $x_1$ we get
\[
j_1=\Big\lfloor \tfrac{427}{6}\Big\rfloor\bmod 3
   =71\bmod 3=\boxed{2},
\]
so $\psi_2$ (row $j{=}2$) is admissible and produces the terminal value
\[
m_1=\Big\lfloor \tfrac{427}{18}\Big\rfloor=23\ \text{(odd)},\qquad
x_2=\underbrace{24m_1+17}_{\psi_2}=569\equiv \boxed{41}\pmod{48}.
\]

\paragraph{Forward check (confirms the two-step chain).}
With $U(n)=(3n+1)/2^{v_2(3n+1)}$,
\[
U(569)=\tfrac{3\cdot 569+1}{2^{v_2(1708)}}=\tfrac{1708}{4}=427,\qquad
U(427)=\tfrac{3\cdot 427+1}{2^{v_2(1282)}}=\tfrac{1282}{2}=641,
\]
so $U^2(569)=641$, matching the inverse chain
\[
641 \xrightarrow{\ \omega_1\ } \underbrace{427}_{\text{intermediate}}
    \xrightarrow{\ \psi_2\ } 569\equiv 41\ (\bmod 48).
\]



\paragraph{Worked example 3: another $17\bmod 24 \to 41\bmod 48$ instance via $\omega_1\to\psi_2$.}
Use the $p=0$ rows
\[
\omega_{1}:\ x'=12m+7\quad(\texttt{oe},\ j=1),\qquad
\psi_{2}:\ x'=24m+17\quad(\texttt{eo},\ j=2),
\]
with $m=\lfloor x/18\rfloor$ and $j=\lfloor x/6\rfloor\bmod 3$ at each step.

\begin{enumerate}
  \item \textbf{Start (a different representative of $17\bmod 24$).}
        Take $x_0=\boxed{425}$.
        Then $425\equiv 17\pmod{24}$.

  \item \textbf{Router for step 1 (\(\omega_1\) admissibility).}
        \[
        j_0=\Big\lfloor \frac{425}{6}\Big\rfloor \bmod 3
            = 70 \bmod 3
            = \boxed{1}.
        \]
        Thus row $j=1$ is admissible for $\omega_1$ (type \texttt{oe}).

  \item \textbf{Apply $\omega_1$ (uses $m_0=\lfloor x_0/18\rfloor$).}
        \[
        m_0=\Big\lfloor\frac{425}{18}\Big\rfloor=23,
        \qquad
        x_1=12m_0+7=12\cdot 23+7=\boxed{283}.
        \]
        Family: $283\bmod 6=1$ (even), as expected for \texttt{oe}.

  \item \textbf{Router for step 2 (\(\psi_2\) admissibility).}
        \[
        j_1=\Big\lfloor \frac{283}{6}\Big\rfloor \bmod 3
            = 47 \bmod 3
            = \boxed{2},
        \]
        matching the row $j=2$ for $\psi_2$.

  \item \textbf{Apply $\psi_2$ (uses $m_1=\lfloor x_1/18\rfloor$).}
        \[
        m_1=\Big\lfloor\frac{283}{18}\Big\rfloor=15\ \text{(odd)},
        \qquad
        x_2=24m_1+17=24\cdot 15+17=\boxed{377}.
        \]

  \item \textbf{Target modulus and family.}
        \[
        377 \equiv 377-7\cdot 48=\boxed{41}\pmod{48},
        \qquad
        377\bmod 6=5\ \text{(odd)},
        \]
        so we land in the intended odd family at $41\bmod 48$.

  \item \textbf{Forward verification (accelerated $U$).}
        \[
        U(377)=\frac{3\cdot 377+1}{2^{v_2(1132)}}=\frac{1132}{4}=283,\qquad
        U(283)=\frac{3\cdot 283+1}{2^{v_2(850)}}=\frac{850}{2}=425.
        \]
        Hence $U^2(377)=425$, exactly the inverse chain we constructed:
        \[
        425 \xrightarrow{\ \omega_1\ } 283 \xrightarrow{\ \psi_2\ } 377.
        \]
\end{enumerate}

\paragraph{Why an intermediate value is required (explicit).}
As in Example~1/2, no single odd-exit $p{=}0$ row pins $41\bmod 48$ in one step,
so we again use the two-token tail $\omega_1\to\psi_2$ which necessarily passes
through an intermediate $x_1$.

\paragraph{Intermediate value and routers (this instance).}
Start with $x_0=\boxed{425}$ (note $425\equiv 17\bmod 24$). Then
\[
j_0=\Big\lfloor \tfrac{425}{6}\Big\rfloor\bmod 3
   =70\bmod 3=\boxed{1},
\]
so $\omega_1$ (row $j{=}1$) is admissible and produces
\[
m_0=\Big\lfloor \tfrac{425}{18}\Big\rfloor=23,\qquad
x_1=\underbrace{12m_0+7}_{\omega_1}=\boxed{283}\quad\text{(this is the intermediate value)}.
\]
Next,
\[
j_1=\Big\lfloor \tfrac{283}{6}\Big\rfloor\bmod 3
   =47\bmod 3=\boxed{2},
\]
so $\psi_2$ (row $j{=}2$) is admissible and
\[
m_1=\Big\lfloor \tfrac{283}{18}\Big\rfloor=15\ \text{(odd)},\qquad
x_2=\underbrace{24m_1+17}_{\psi_2}=377\equiv \boxed{41}\pmod{48}.
\]

\paragraph{Forward check (confirms the two-step chain).}
\[
U(377)=\tfrac{3\cdot 377+1}{2^{v_2(1132)}}=\tfrac{1132}{4}=283,\qquad
U(283)=\tfrac{3\cdot 283+1}{2^{v_2(850)}}=\tfrac{850}{2}=425,
\]
hence $U^2(377)=425$, matching
\[
425 \xrightarrow{\ \omega_1\ } \underbrace{283}_{\text{intermediate}}
    \xrightarrow{\ \psi_2\ } 377\equiv 41\ (\bmod 48).
\]


% ------------------------------------------------------------

% ============================================================
% TD1: Finite same-family padding menu and stabilization
% (Drop-in under the TD1 item)
% ============================================================

\subsection{Finite same-family padding menu and stabilization}

We separate the tail into (i) a \emph{padding phase} that preserves family and monotonically raises $v_2(A)$, and (ii) a \emph{terminal step} (possibly cross-family) that lands in the required residue/family. The padding phase is built from a fixed finite menu inside each family.

\paragraph{Notation (from the preamble).}
Every token in column $p$ has unified last-step form
\[
x' \;=\; 6\big(2^{\alpha_p}u + k^{(p)}\big) + \delta,\qquad
u=\Big\lfloor \frac{x}{18}\Big\rfloor,\quad
\alpha_p=\alpha+6p,
\]
and the one-step update with router remainder $r\in\{0,1,2\}$ is
\[
A\mapsto \frac{2^{\alpha_p}}{3}A,\qquad
B\mapsto \frac{2^{\alpha_p}}{3}(B-r)+k^{(p)},\qquad
\delta\mapsto \delta.
\]
In particular $B_{\text{new}}\equiv k^{(p)}\pmod 2$ (Lemma~\ref{lem:B-parity}).

\paragraph{A finite padding menu in each family (column $p=0$ as baseline).}
The following tiny menus preserve family and strictly raise $v_2(A)$; by lifting to any column $p\ge 0$ (i.e.\ adding $+6p$ to each $\alpha$) they remain valid padding blocks.

\medskip
\noindent\emph{End in $e$ (stay $e\to e$):}
\[
\mathcal{P}_{0,e}:=\Big\{\,\Psi_1,\ \Psi_2,\ \psi_2\circ\omega_1\,\Big\},
\]
with per-token base exponents at $p=0$:
\[
\alpha(\Psi_1)=4,\qquad \alpha(\Psi_2)=6,\qquad
\alpha(\psi_2)=2,\ \alpha(\omega_1)=1\ \Rightarrow\
\Delta_0(\psi_2\!\circ\!\omega_1)=3.
\]
Each block exits in $e$ and increases $v_2(A)$ by at least $1$ (actually by $3,4,$ or $6$).

\medskip
\noindent\emph{End in $o$ (stay $o\to o$):}
\[
\mathcal{P}_{0,o}:=\Big\{\,\Omega_1,\ \Omega_0,\ \omega_1\circ\psi_2\,\Big\},
\]
with base exponents:
\[
\alpha(\Omega_1)=3,\qquad \alpha(\Omega_0)=5,\qquad
\alpha(\omega_1)=1,\ \alpha(\psi_2)=2\ \Rightarrow\
\Delta_0(\omega_1\!\circ\!\psi_2)=3.
\]
Each block exits in $o$ and increases $v_2(A)$ by at least $1$.

\begin{lemma}[Monotone lift with a finite menu]\label{lem:td1-menu-lift}
Fix a family $f\in\{e,o\}$ and a column $p\ge 0$. Let $\mathcal{P}_{p,f}$ be the lift of the above menu (replace every $\alpha$ by $\alpha+6p$). For any prefix ending in $f$ and any target $K$, a finite concatenation $S$ of blocks from $\mathcal{P}_{p,f}$ satisfies $v_2(A\cdot S)\ge K$ while still ending in $f$.
\end{lemma}

\begin{proof}
Each block multiplies $A$ by $2^{\Delta_p(S)}/3^{\ell(S)}$ with $\Delta_p(S)>0$, so $v_2(A)$ increases by $\Delta_p(S)$. By additivity of $v_2$ under composition, repeating blocks reaches any prescribed $K$. Entries/exits of blocks are $f\to f$, hence the family is preserved.
\end{proof}

\begin{lemma}[Stabilization of the prefix]\label{lem:padding-stabilization}
Let $W$ be any fixed prefix. Set
\[
K_\star \;:=\; 1 + \max_{t<|W|}\Big(\sum_{i\le t} \alpha_{p_i}\Big).
\]
For every $K\ge K_\star$ there exists a padding string $S_K$ made only of blocks from $\mathcal{P}_{p,f}$ (with $f$ the terminal family of $W$) such that
\[
v_2\!\big(A_{W\cdot S_K}\big)\ \ge\ K,
\]
and all routers inside $W$ remain admissible for the chosen $m$-class (no branch flips).
\end{lemma}

\begin{proof}
Apply Lemma~\ref{lem:td1-menu-lift} to raise $v_2(A)$ beyond $K_\star$; then routing-compatibility (TD2) ensures that once the 2-adic depth exceeds the stated bound, solving the final congruence selects an $m$-class that preserves all router remainders of $W$. Only padding blocks appear after $W$, so the prefix stabilizes.
\end{proof}

\paragraph{How parity is handled.}
The parity that appears in the \emph{final} congruence is $B_{\text{final}}\equiv k^{(p)}_{\text{(last token)}}\ (\bmod 2)$. Thus:
\begin{itemize}
  \item Choose the \emph{terminal token} (in the required family) whose table entry has the needed $k^{(p)}\bmod2$;
  \item Use same-family padding \emph{before} the terminal token to raise $v_2(A)$ (and to arrange the needed router class).
\end{itemize}
When both parities are not available within the same family at $p=0$, lift to some column $p>0$ or use a short admissible detour that changes the admissible last row while still exiting in the same family.

\paragraph{Tiny padding-only demonstrations.}
\begin{itemize}
  \item \emph{$e$-family:} Starting in $e$, append $\Psi_1$ repeatedly. Each adds $+(\alpha+6p)$ bits ($=4$ at $p=0$), so in $\lceil (K-v_2(A_{\text{in}}))/(\alpha+6p)\rceil$ steps you exceed $K$ while staying in $e$.
  \item \emph{$o$-family:} Starting in $o$, append $\Omega_1$ then $\omega_1\!\circ\!\psi_2$ as needed. Each block adds at least $+(\alpha+6p)$ bits ($\ge 3$ at $p=0$), and the 2-token block keeps you in $o$.
\end{itemize}

\paragraph{Where the cross-family terminal step fits.}
After padding in family $f$ to reach $K$, apply the \emph{single} terminal token (possibly $o\to e$ or $e\to o$) that matches the target residue and family (e.g.\ the one-step $o\to e$ witness $\omega_1$ mapping $47\bmod72$ to $31\bmod48$). The algebra is the same; only the final family changes at that last step.

% ------------------------------------------------------------
% Examples for: Finite same-family padding menu and stabilization
% Paste these after the menu/lemmas in that subsection
% ------------------------------------------------------------

\begin{example}[Two $\Psi_0$ steps at $p=0$ starting from $x=1$]
Work in column $p=0$. From the $p=0$ table, the row $\Psi_0$ (type \texttt{ee}, router $j=0$) has
\[
\alpha_p=2,\qquad k^{(0)}=0\ \text{(even)},\qquad \delta_T=1,\qquad
x' \;=\; 6\big(2^{2}m + k^{(0)}\big)+1 \;=\; 24m+1.
\]

\noindent\textbf{Admissibility at $x=1$.}
\[
j_0=\Big\lfloor\tfrac{1}{6}\Big\rfloor\bmod 3=0,\qquad
m_0=\Big\lfloor\tfrac{1}{18}\Big\rfloor=0,
\]
so the row $j=0$ is admissible and $\Psi_0$ applies. Its output is
\[
x_1 \;=\; 24m_0+1 \;=\; 1.
\]

\noindent\textbf{Second application (still admissible).}
\[
j_1=\Big\lfloor\tfrac{1}{6}\Big\rfloor\bmod 3=0,\qquad
m_1=\Big\lfloor\tfrac{1}{18}\Big\rfloor=0,\qquad
x_2=24m_1+1=1.
\]
Thus numerically we see the fixed point $1\mapsto 1\mapsto 1$ under two consecutive $\Psi_0$ steps.

\medskip
\noindent\textbf{Symbolic slope growth (padding effect).}
Write the surrogate before the first step as $x_t(m)=6(A_t m+B_t)+\delta_t$.
By the floor–composition rule (Lemma~\ref{lem:one-step-floor}) with $u=\lfloor x_t/18\rfloor=\frac{A_t m+B_t-r_{t+1}}{3}$ and $\alpha_p=2$,
\[
A_{t+1}=\frac{2^{2}}{3}A_t,\qquad
B_{t+1}=\frac{2^{2}}{3}(B_t-r_{t+1})+k^{(0)},\qquad
\delta_{t+1}=1.
\]
Hence
\[
v_2(A_{t+1})=v_2(A_t)+2.
\]
Applying $\Psi_0$ a second time gives
\[
A_{t+2}=\Big(\tfrac{2^{2}}{3}\Big)^2 A_t,\qquad
v_2(A_{t+2})=v_2(A_t)+2+2=v_2(A_t)+4.
\]

\noindent\textbf{Parity note.}
Since $k^{(0)}=0$ is even, Lemma~\ref{lem:B-parity} yields
$B_{t+1}\equiv 0\pmod 2$ and $B_{t+2}\equiv 0\pmod 2$.
Thus two $\Psi_0$ steps preserve even $B$ while adding $+4$ bits to $v_2(A)$.

\medskip
\noindent\emph{Takeaway.} Even though the concrete start $x=1$ stays fixed numerically, the
\emph{symbolic} map’s slope doubles in $2$-adic valuation by $+2$ each time we apply $\Psi_0$.
This is exactly the monotone padding we use in TD1.
\end{example}


\begin{example}[Two $\Psi_1$ steps at $p=0$ add $+8$ to $v_2(A)$]
Work in column $p=0$. From the table, the row $\Psi_1$ (type \texttt{ee}) has
\[
\alpha_p=4,\qquad k^{(0)}=6,\qquad \delta_T=1.
\]
Let the current state be
\[
x_t(m)=6\big(A_t\,m+B_t\big)+\delta_t,\quad\text{with router }r_{t+1}\in\{0,1,2\}
\]
and assume $\Psi_1$ is admissible at this step (so the router matches the row).
Using the floor-composition rule with $u=\lfloor x_t/18\rfloor=\frac{A_t m + B_t - r_{t+1}}{3}$, one application of $\Psi_1$ gives
\[
A_{t+1}=\frac{2^{4}}{3}A_t,\qquad
B_{t+1}=\frac{2^{4}}{3}\,(B_t-r_{t+1})+6,\qquad
\delta_{t+1}=1.
\]
Hence
\[
v_2(A_{t+1})=v_2(A_t)+4.
\]

Apply $\Psi_1$ a \emph{second} time (again admissible, with router $r_{t+2}$). Using the same update on $(A_{t+1},B_{t+1},\delta_{t+1})$:
\[
A_{t+2}=\frac{2^{4}}{3}A_{t+1}=\Big(\frac{2^{4}}{3}\Big)^2 A_t,\qquad
B_{t+2}=\frac{2^{4}}{3}\,(B_{t+1}-r_{t+2})+6,\qquad
\delta_{t+2}=1.
\]
Therefore
\[
v_2(A_{t+2})=v_2(A_t)+4+4=v_2(A_t)+8,
\]
i.e.\ two $\Psi_1$ steps add \(+8\) two-adic bits to the internal slope.

\medskip
\noindent\textbf{Parity check.}
By Lemma~\textup{\ref{lem:B-parity}}, each $\Psi_1$ sets $B\equiv k^{(0)}\equiv 0\pmod 2$ at its own step. Concretely,
\[
B_{t+1}\equiv 6\equiv 0\pmod 2,\qquad
B_{t+2}\equiv 6\equiv 0\pmod 2.
\]
Thus two $\Psi_1$ steps preserve even $B$.

\medskip
\noindent\textbf{Why a fixed numerical value (e.g.\ $x=1$) does not contradict the $+8$.}
The update above is for the \emph{symbolic} surrogate $x(m)=6(A m+B)+\delta$. It shows how the
\emph{coefficient} $A$ changes (hence $v_2(A)$ increases by $+8$). If you then \emph{evaluate} at a specific
admissible $m$ (for instance, $m=0$ when allowed by the routing congruence), you might get a small or even repeated numerical value
(e.g.\ $x=6B+\delta$). That numerical coincidence does not affect the fact that, as a linear function of $m$, the slope has been multiplied
by $(2^4/3)^2$, so $v_2(A)$ truly increased by $8$.
\end{example}



\begin{example}[Same-family padding in $e$ with $\Psi_1$ (column $p=0$)]
Let a prefix end in family $e$ with coefficients $(A,B,\delta)$ and suppose the next router remainder is admissible for the row $\Psi_1$ (type \texttt{ee}) at $p=0$.
From the table: $\alpha_p=4$, $k^{(0)}=6$ (even), $\delta_T=1$.
By Lemma~\ref{lem:one-step-floor}, for $r\in\{0,1,2\}$ at that step,
\[
A\mapsto A'=\frac{2^{4}}{3}A,\qquad
B\mapsto B'=\frac{2^{4}}{3}(B-r)+6,\qquad
\delta\mapsto \delta'=1.
\]
Hence $v_2(A)$ increases by $+4$ (monotone lift) and $B'\equiv 0\pmod2$ (Lemma~\ref{lem:B-parity}).
Applying $\Psi_1$ *twice* (on an admissible router sequence) gives
\[
v_2(A)\ \mapsto\ v_2(A)+8,\qquad B\ \mapsto\ B''\equiv 0\pmod2,
\]
still in family $e$. This is a pure same-family padding step that raises the $2$-adic depth without changing family.
\end{example}

\begin{example}[Same-family padding in $o$ with $\Omega_1$ (column $p=0$)]
Assume we end in family $o$ and the router remainder admits $\Omega_1$ (type \texttt{oo}) at $p=0$.
From the table: $\alpha_p=3$, $k^{(0)}\equiv 0\ (\bmod 2)$, $\delta_T=5$.
One application yields
\[
A\mapsto A'=\frac{2^{3}}{3}A,\qquad
B\mapsto B'=\frac{2^{3}}{3}(B-r)+k^{(0)},\qquad
\delta\mapsto \delta'=5,
\]
so $v_2(A)$ grows by $+3$ and we remain in $o$.
A short padding string like $\Omega_1\circ \Omega_1\circ \Omega_1$ lifts $v_2(A)$ by $+9$ while staying in $o$.
\end{example}

\begin{example}[Parity control in $e$ via the 2-token same-family cycle $\psi_2\circ\omega_1$ (column $p=0$)]
Consider the 2-token block $C_e:=\psi_2\circ\omega_1$, which enters $e$, goes $e\to o$ by $\psi_2$, then $o\to e$ by $\omega_1$ (net $e\to e$).
At $p=0$ the table gives $k^{(0)}(\psi_2)\equiv 0\pmod2$ and $k^{(0)}(\omega_1)\equiv 1\pmod2$.
Write the intermediate update after $\psi_2$ as $(A_1,B_1,\delta_1)$, then apply $\omega_1$:
\[
B_{\text{out}} \equiv k^{(0)}(\omega_1) \equiv 1 \pmod{2}
\quad\text{(Lemma~\ref{lem:B-parity})}.
\]
Thus, regardless of the incoming $B\bmod 2$, a single use of $C_e$ finishes with $B_{\text{out}}\equiv 1\pmod2$ while returning to family $e$.
Using $C_e$ twice preserves parity (one flip per use), so $C_e$ is a **parity toggle** within $e$.
The total $v_2$ gain is $\Delta_0(C_e)=\alpha(\psi_2)+\alpha(\omega_1)=2+1=3>0$, so it also contributes to monotone padding.
\end{example}

\begin{example}[Large-bit padding at higher columns ($p=2$) in $e$ with $\Psi_0$]
Lifting the same-family menu to column $p=2$ adds $+12$ to each base exponent.
For $\Psi_0$ (type \texttt{ee}) we have $\alpha=2\Rightarrow \alpha_{p}=14$ at $p=2$, with some $k^{(2)}$ from the table and $\delta_T=1$.
A single application gives
\[
A\mapsto A'=\frac{2^{14}}{3}A,\qquad
B\mapsto B'=\frac{2^{14}}{3}(B-r)+k^{(2)},\qquad
\delta\mapsto \delta'=1,
\]
so \(v_2(A)\) jumps by \(+14\) in one step while staying in $e$.
This illustrates why padding at higher columns is a powerful “bit pump”: you can meet large targets \(K\) with very few same-family tokens, and still decide parity at the end by picking the last token with the desired \(k^{(p)}\bmod 2\) (or by inserting one use of the $e$-cycle \(C_e\)).
\end{example}

% ------------------------------------------------------------
% (Optional) one-line usage note tying examples to stabilization
% ------------------------------------------------------------
\noindent\emph{Usage for stabilization.}
Given a fixed prefix $W$ and target depth $K$, choose the family-preserving padding from the menu (Examples 1–2) and, if needed, insert one parity-toggle cycle (Example 3).
At higher columns you may replace many $p=0$ steps by a single $p=2$ step (Example 4).
By Lemma~\ref{lem:padding-stabilization}, for $K\ge K_\star$ the prefix stabilizes and only these padding blocks grow.


\subsection*{Finite same-family padding menu and stabilization (constructed)}

\paragraph{Why a menu (recap).}
The $p$-tables enumerate all admissible tokens, but we need a \emph{finite} toolkit that we can reuse uniformly:
(i) every block preserves the terminal family,
(ii) each block strictly raises $v_2(A)$, and
(iii) we can \emph{set} $B\bmod 2$ at exit by picking a block whose \emph{last token} has the desired $k^{(p)}\bmod 2$.

\paragraph{Menu at column $p=0$ (base case).}
From the $p{=}0$ table, record the base exponents and parities
\[
\begin{array}{c|c|c|c|c}
\text{token} & \text{type} & \alpha & k^{(0)}\bmod 2 & \text{exit family} \\
\hline
\Psi_0 & \texttt{ee} & 2 & 0 & e \\
\Psi_1 & \texttt{ee} & 4 & 0 & e \\
\Psi_2 & \texttt{ee} & 6 & 0 & e \\
\psi_2 & \texttt{eo} & 2 & 0 & o \\
\omega_1 & \texttt{oe} & 1 & 1 & e \\
\Omega_0 & \texttt{oo} & 5 & 0 & o \\
\Omega_1 & \texttt{oo} & 3 & 0 & o \\
\Omega_2 & \texttt{oo} & 1 & 1 & o \\
\end{array}
\]
(Other rows exist; these are the only ones we need for a finite menu.)

\paragraph{Chosen finite menus.}
We fix two small families of \emph{same-terminal-family} gadgets:

\begin{itemize}
  \item \textbf{End in $e$ (menu $\mathcal{S}_e$):}
    \[
      \underbrace{\Psi_0}_{\Delta v_2=+2,\ B'\equiv 0},\qquad
      \underbrace{\Psi_1}_{\Delta v_2=+4,\ B'\equiv 0},\qquad
      \underbrace{\psi_2\circ\omega_1}_{e\to o\to e,\ \Delta v_2=2+1=+3,\ \text{last } \omega_1:\ B'\equiv 1}.
    \]
    Here, the first two are \emph{lift blocks} (raise $v_2(A)$, set even $B$); the 2-token block is a \emph{parity-setter to odd} (ends in $e$ with $B'\equiv 1$).
  \item \textbf{End in $o$ (menu $\mathcal{S}_o$):}
    \[
      \underbrace{\Omega_1}_{\Delta v_2=+3,\ B'\equiv 0},\qquad
      \underbrace{\Omega_0}_{\Delta v_2=+5,\ B'\equiv 0},\qquad
      \underbrace{\Omega_2}_{\Delta v_2=+1,\ B'\equiv 1}.
    \]
    Again, two \emph{lift blocks} (even $B$) and a \emph{parity-setter to odd} (single token, ends in $o$ with $B'\equiv 1$).
\end{itemize}

\paragraph{What these guarantee (in any column $p$).}
At column $p$, each token’s exponent becomes $\alpha_p=\alpha+6p$ and $k^{(p)}$ lifts accordingly; the \emph{last token} still sets $B'\bmod 2\equiv k^{(p)}\bmod 2$. Therefore:
\begin{enumerate}
  \item \textbf{Monotone lift:} Every menu block has $\alpha_p\ge 1$ per token, so $\Delta v_2(A)>0$. Repeating blocks reaches any target $K$.
  \item \textbf{Parity setting on demand:} Ending in $e$: use a block whose last token has $k^{(p)}\equiv 0$ (take $\Psi_0$ or $\Psi_1$ as last) to force $B'\equiv 0$, or the 2-token $e\to e$ block with last $\omega_1$ to force $B'\equiv 1$.
  Ending in $o$: use $\Omega_1$/$\Omega_0$ last to force $B'\equiv 0$ or $\Omega_2$ last to force $B'\equiv 1$.
  \item \textbf{Same-family preservation:} Each block starts and ends in the same family by construction (\texttt{ee} or \texttt{oo}, or a 2-token $e\to o\to e$, $o\to e\to o$).
\end{enumerate}

\begin{lemma}[Finite padding menu sufficiency]
For either terminal family $f\in\{e,o\}$ and any target $K$, there exists a concatenation of blocks from $\mathcal{S}_f$ such that (i) $v_2(A)$ at exit is $\ge K$, (ii) the exit family is $f$, and (iii) the exit parity $B'\bmod 2$ equals any prescribed value in $\{0,1\}$.
\end{lemma}

\begin{proof}[Sketch]
Monotone lift follows since each block contributes $+\alpha_p>0$ to $v_2(A)$; concatenate until $\ge K$. To set parity, pick a final block whose \emph{last token} has the desired $k^{(p)}\bmod 2$ (Lemma~\ref{lem:B-parity}). Family preservation is built into the block types. Routing-compatibility is ensured by choosing the power-of-two modulus on $m$ guaranteed by the routing lemma (so planned routers hold throughout the fixed prefix).
\end{proof}

\paragraph{Concrete numeric example (end in $e$).}
We show a short pad that ends in $e$, raises $v_2(A)$ by $+3$ (one block), and \emph{sets} $B'\equiv 1$. Consider the block $S_e=\psi_2\circ\omega_1$ at $p=0$:
\[
\omega_1:\ x'=12m+7\ (\texttt{oe},\ j=1),\qquad
\psi_2:\ x'=24m+17\ (\texttt{eo},\ j=2).
\]
Pick an admissible start $x_0=209$ (so $x_0\equiv 17\pmod{24}$ and $j_0=1$). Then
\[
m_0=\Big\lfloor\tfrac{209}{18}\Big\rfloor=11,\quad x_1=\underbrace{12m_0+7}_{\omega_1}=139,\quad
j_1=\Big\lfloor\tfrac{139}{6}\Big\rfloor\bmod 3=2,
\]
and
\[
m_1=\Big\lfloor\tfrac{139}{18}\Big\rfloor=7,\quad x_2=\underbrace{24m_1+17}_{\psi_2}=185.
\]
This block starts in $e$ (since $209\equiv 1\pmod 6$), ends in $e$ ($185\equiv 1\pmod 6$), contributes $\Delta v_2=2+1=+3$, and because the \emph{last token is} $\omega_1$ when read as \emph{last inside $S_e$}? (Careful: here the \emph{last} is $\psi_2$, so $B'\equiv k^{(0)}(\psi_2)\equiv 0$.)
To \emph{set odd parity at exit}, simply \emph{swap the order} inside the $e$-menu: use the 2-token block $e\to o\to e$ whose \emph{last token is $\omega_1$}. One such variant is obtained by first using a short $e\to o$ step with last-even parity (e.g.\ $\psi_0$), then apply $\omega_1$ to return to $e$ with $B'\equiv 1$. (Any admissible $e\to o$ followed by $\omega_1$ will do; we keep $p{=}0$ for simplicity.)

\paragraph{Concrete numeric example (end in $o$).}
Use the one-token \emph{parity-setter} $\Omega_2$ at $p=0$:
\[
\Omega_2:\ x'=12m+11\quad(\texttt{oo},\ j=2),\quad \alpha=1,\ k^{(0)}\equiv 1.
\]
Pick $x_0\equiv 5\pmod 6$ with $j_0=2$; e.g.\ $x_0=83$. Then
\[
m_0=\Big\lfloor\tfrac{83}{18}\Big\rfloor=4,\qquad x_1=\underbrace{12m_0+11}_{\Omega_2}=59.
\]
This starts and ends in $o$ ($83,59\equiv 5\pmod 6$), contributes $\Delta v_2=+1$, and \emph{sets} $B'\equiv 1$ (since the last token is $\Omega_2$ with odd $k^{(0)}$).
If you need more two-adic bits, prepend any number of lift blocks $\Omega_1$ ($+3$) or $\Omega_0$ ($+5$) and finish with $\Omega_2$ to force odd $B$ at exit; finishing instead with $\Omega_1$ or $\Omega_0$ forces even $B$.

\paragraph{How this plugs into TD1.}
Fix any routing prefix $W$ that lands in the desired terminal family.
\begin{enumerate}
  \item Append lift blocks from $\mathcal{S}_e$ or $\mathcal{S}_o$ until $v_2(A)$ meets the target $K$ (monotone).
  \item Choose the \emph{final} block so its \emph{last token} has the desired $k^{(p)}\bmod 2$ (this sets $B'\bmod 2$).
  \item Use the routing-compatibility lemma to pick the $m$-class so all routers inside $W$ stay fixed.
\end{enumerate}
This completes the “finite menu + stabilization” deliverable for TD1.
At higher columns $p\ge 1$, the same menus apply verbatim with $\alpha_p=\alpha+6p$ (stronger bit pump); the parity-setting logic is unchanged because it depends only on the \emph{last token’s} $k^{(p)}\bmod 2$.

\begin{example}[Lifting $17\bmod 24$ to $41\bmod 48$ via $\omega_1\to\psi_2$]
At $p=0$ we use the rows
\[
\omega_{1}:\ x'=12m+7\quad(\texttt{oe},\ j=1),\qquad
\psi_{2}:\ x'=24m+17\quad(\texttt{eo},\ j=2),
\]
with the convention $m=\lfloor x/18\rfloor$ and $j=\lfloor x/6\rfloor\bmod 3$ at each step.

\textbf{Step 0 (start class).} Write every $x\equiv 17\ (\bmod 24)$ as $x=24n+17$.
Then
\[
j_0=\Big\lfloor \frac{x}{6}\Big\rfloor\bmod 3
   = (4n+2)\bmod 3
   \equiv n+2 \pmod{3}.
\]
Thus $\omega_1$ (which needs $j_0=1$) is admissible iff $n\equiv 2\ (\bmod 3)$.
Refine to the subclass $n\equiv 8\ (\bmod 9)$ to ensure the next router is $j_1=2$.

\textbf{Concrete representative.} Take $n=8$, so $x_0=24n+17=209$.

\textbf{Step 1 (apply $\omega_1$).}
\[
j_0=\Big\lfloor\tfrac{209}{6}\Big\rfloor\bmod 3=1,\qquad
m_0=\Big\lfloor\tfrac{209}{18}\Big\rfloor=11,\qquad
x_1=\underbrace{12m_0+7}_{\omega_1}=139.
\]

\textbf{Step 2 (apply $\psi_2$).}
\[
j_1=\Big\lfloor\tfrac{139}{6}\Big\rfloor\bmod 3=2,\qquad
m_1=\Big\lfloor\tfrac{139}{18}\Big\rfloor=7\ (\text{odd}),\qquad
x_2=\underbrace{24m_1+17}_{\psi_2}=185\equiv 41\pmod{48}.
\]

\textbf{Why the intermediate value is necessary.}
No single odd-exit $p{=}0$ row with $\alpha\ge 4$ pins $41\pmod{48}$ in one step,
so we use the two-token tail $\omega_1\to\psi_2$, which forces the intermediate
$x_1=139$ via the certified inverse formula for $\omega_1$.

\textbf{Forward check (accelerated map $U$).}
\[
U(185)=139,\qquad U(139)=209,
\]
hence $U^2(185)=209$, matching the inverse chain
\(
209 \xrightarrow{\ \omega_1\ } 139 \xrightarrow{\ \psi_2\ } 185.
\)
\end{example}
\subsection*{Closure: finite menu, parity setter, monotone lift, and routing bound}

\paragraph{Frozen finite menu (at $p=0$).}
We fix the following blocks for reuse:
\[
\mathcal{S}_e=\{\ \Psi_0,\ \Psi_1,\ \Psi_2,\ \psi_2\!\circ\!\omega_1\ \},\qquad
\mathcal{S}_o=\{\ \Omega_0,\ \Omega_1,\ \Omega_2\ \}.
\]
Each block starts and ends in the indicated family. At column $p$, its exponent gain is
$\Delta v_2(A)=\sum \alpha + 6p\cdot(\text{length})$; the last token sets
$B'\equiv k^{(p)}\pmod 2$.

\begin{lemma}[Parity setter in either family]\label{lem:parity-setter}
For each terminal family $f\in\{e,o\}$ and each $b\in\{0,1\}$ there exists a block
$S\in\mathcal{S}_f$ such that, when used last, the exit parity satisfies $B'\equiv b\pmod 2$.
\end{lemma}

\begin{proof}[Proof sketch]
For $f=o$, take $\Omega_2$ (last token odd) for $b=1$, and $\Omega_0$ or $\Omega_1$ (last token even) for $b=0$.
For $f=e$, take $\Psi_0$ or $\Psi_1$ last for $b=0$; for $b=1$ use a two-token $e\to o\to e$ block whose \emph{last} token has odd $k^{(p)}$ (e.g.\ an admissible $e\to o$ followed by $\omega_1$ back to $e$). In all cases, $B'\equiv k^{(p)}_{\text{last}}\ (\bmod 2)$ by Lemma~\ref{lem:B-parity}.
\end{proof}

\begin{lemma}[Monotone two-adic lift]\label{lem:monotone-lift}
Let $f\in\{e,o\}$ and $K\in\mathbb{N}$. For any prefix ending in family $f$, there exists a concatenation of blocks from $\mathcal{S}_f$ such that the exit still lies in $f$ and
$v_2(A)\ge K$.
\end{lemma}

\begin{proof}
Each block contributes $\Delta v_2(A)>0$ (since every token has $\alpha_p\ge 1$), so concatenation reaches any target $K$. Same-family exit is part of the menu definition.
\end{proof}

\begin{corollary}[Pin or solve at the last step]\label{cor:pin-or-solve}
Fix a target modulus $M_K=3\cdot 2^K$ and a desired terminal family. After applying Lemma~\ref{lem:monotone-lift}, choose a last token $T$ in the same family.
\begin{itemize}
\item If $\alpha_p(T)\ge K$, then $x'\equiv 6k^{(p)}(T)+\delta_T\pmod{M_K}$ independently of $m$ (pinning).
\item If $\alpha_p(T)<K$, then the congruence
$6\cdot 2^{\alpha_p(T)}m\equiv x_{\mathrm{tar}}-(6k^{(p)}(T)+\delta_T)\pmod{M_K}$
has a solution; by Lemma~\ref{lem:parity-setter}, we can pre-set $B'\bmod 2$ to make the right-hand side fall in the desired class.
\end{itemize}
\end{corollary}

\begin{lemma}[Routing compatibility bound]\label{lem:routing-bound}
Let $W$ be a fixed prefix with row exponents $(\alpha_1,\ldots,\alpha_{|W|})$ and routers $(j_1,\ldots,j_{|W|})$. If $m$ is chosen modulo $2^{S^\ast}$ with
\[
S^\ast \ \ge\ \max_{t<|W|}\Big(1+\sum_{i\le t}\alpha_i\Big),
\]
and with the mod-$3$ part consistent with the planned rows, then every step’s remainder $r_{t+1}$ equals $j_{t+1}$ and the prefix executes without branch flips.
\end{lemma}

\begin{proof}[Proof sketch]
The bound forces all intermediate $(A_t m+B_t)$ to stabilize modulo $3$ and $2^{\sum_{i\le t}\alpha_i}$ so that $m_t=(A_tm+B_t-r_{t+1})/3\in\mathbb{Z}$ matches the planned router at each step; this is the standard routing-compatibility argument specialized to powers of two and fixed rows.
\end{proof}
\begin{remark}[Single source of truth]
Tables~\ref{tab:menu-effects-p0}--\ref{tab:menu-effects-p2} are the canonical per-token
effects; later sections refer to them and do not reproduce their contents.
\end{remark}

% ------------------------------------------------------------
% TD2 — Routing compatibility (no branch flips)
% ------------------------------------------------------------
\section{Routing compatibility (no branch flips)}
%\addcontentsline{toc}{section}{Routing compatibility (no branch flips)}

\paragraph{Setup.}
For a certified word $W=T_1\cdots T_n$ we keep the invariant
\[
x_t(m)=6\big(A_t m + B_t\big)+\delta_t,\qquad
A_t=2^{S_t},\ \ S_t:=\sum_{i\le t}\alpha_i,
\]
and define the router remainder $r_{t+1}\in\{0,1,2\}$ by
\[
A_t m + B_t \equiv r_{t+1}\pmod{3}.
\]
The token actually used at step $t$ has planned row index $j_{t+1}\in\{0,1,2\}$.
Admissibility/compatibility means $r_{t+1}=j_{t+1}$ at every step.

\begin{lemma}[Routing compatibility]\label{lem:TD2-routing}
Fix a word $W=T_1\cdots T_n$ with planned row indices $(j_1,\ldots,j_n)$ and exponents
$(\alpha_1,\ldots,\alpha_n)$. Suppose $m$ satisfies
\[
A_t\,m+B_t \equiv j_{t+1}\ (\mathrm{mod}\ 3)\qquad\text{for all }t=0,1,\ldots,n-1,
\]
and
\[
m \equiv m^\ast\ (\mathrm{mod}\ 2^{S^\ast}),\qquad
S^\ast\ \ge\ \max_{0\le t<n}\bigl(1+S_t\bigr).
\]
Then the execution of $W$ at $m$ uses exactly the planned rows: $r_{t+1}=j_{t+1}$ for all $t$.
Equivalently, there are no branch flips along $W$.
\end{lemma}

\begin{proof}[Proof (induction on $t$)]
At step $t$, write
\(
m_t=\big\lfloor x_t/18\big\rfloor = (A_t m + B_t - r_{t+1})/3
\)
with $r_{t+1}\in\{0,1,2\}$ uniquely determined by $A_t m+B_t\equiv r_{t+1}\ (\mathrm{mod}\ 3)$.
By the 2-adic bound $m\equiv m^\ast\ (\mathrm{mod}\ 2^{1+S_t})$ and $A_t=2^{S_t}$,
the value of $A_t m$ is stable modulo $2^{1+S_t}$; thus $m_t$ is an integer linear function
of $m$ whose parity bits below $2^{S_t}$ are frozen. The planned congruence
$A_t m+B_t\equiv j_{t+1}\ (\mathrm{mod}\ 3)$ forces $r_{t+1}=j_{t+1}$, hence the chosen row is admissible.
Applying the certified row update preserves the linear form and increases $S_{t+1}=S_t+\alpha_{t+1}$.
By the hypothesis on $S^\ast$, the same stability holds at step $t{+}1$, and induction
closes.
\end{proof}

\begin{remark}[How to choose $m$ in practice]\label{rem:choose-m}
Solve the linear system of congruences
\(
A_t m \equiv j_{t+1}-B_t\ (\mathrm{mod}\ 3)
\)
for $m\ (\mathrm{mod}\ 3)$ (feasible because $A_t$ is a power of $2$). Then lift $m$
to modulus $2^{S^\ast}$ with $S^\ast\ge \max_t(1{+}S_t)$, e.g.\ by CRT on
$\mathrm{mod}\ (3\cdot 2^{S^\ast})$. This yields an $m$ that enforces all planned routers.
\end{remark}

\begin{corollary}[Stable under padding]\label{cor:stable-padding}
Let $W$ be a fixed prefix and $S$ any tail of certified tokens (same-family or cross-family).
If $m$ satisfies Lemma~\ref{lem:TD2-routing} for $W$, then the routers inside $W$ remain
correct after appending $S$; only the routers inside $S$ must be checked/solved.
\end{corollary}

\paragraph{Two concrete router-stability checks.}

\begin{example}[Two-step tail $\omega_1\to\psi_2$ at $p=0$]\label{ex:w1p2-routing}
Use the table rows
\(
\omega_{1}:\ x'=12m+7\ (\texttt{oe},j=1)
\)
and
\(
\psi_{2}:\ x'=24m+17\ (\texttt{eo},j=2).
\)
Start from $x_0\equiv 17\ (\mathrm{mod}\ 24)$, write $x_0=24n+17$.
Then
\(
j_0=\lfloor x_0/6\rfloor\bmod 3=(4n+2)\bmod 3,
\)
so $j_0=1$ iff $n\equiv 2\ (\mathrm{mod}\ 3)$.
Refine to $n\equiv 8\ (\mathrm{mod}\ 9)$; this forces, after the first move,
\(
j_1=\lfloor x_1/6\rfloor\bmod 3=2
\)
(the required row for $\psi_2$). Here the exponents are
$\alpha(\omega_1)=1$, $\alpha(\psi_2)=2$, so
$S_0=0$, $S_1=1$. Taking any $S^\ast\ge \max(1{+}S_0,1{+}S_1)=2$
stabilizes the floors; hence no branch flips occur along the planned two steps.
\end{example}

\begin{example}[Three-step mixed family tail]
Let $W=T_1T_2T_3$ with rows $(j_1,j_2,j_3)=(2,0,1)$ and exponents
$(\alpha_1,\alpha_2,\alpha_3)=(3,2,1)$ (e.g.\ a concrete $\Omega_0$,
then $\psi_0$, then $\omega_1$ at $p=0$).
Then $S_t=(0,3,5)$ and $\max_t(1{+}S_t)=6$.
Solve $A_t m\equiv j_{t+1}-B_t\ (\mathrm{mod}\ 3)$ for $t=0,1,2$,
pick one $m\ (\mathrm{mod}\ 3)$ that satisfies all three, and lift to
$m\ (\mathrm{mod}\ 2^{6})$. This ensures $r_{t+1}=j_{t+1}$ at each step,
so the row plan is realized without flips.
\end{example}

\paragraph{What this section means gives us downstream.}
\begin{itemize}
  \item We may \emph{freeze} any routing prefix $W$ and do all later adjustments (raising $v_2(A)$, forcing $B\bmod 2$, choosing the final residue) in a tail, knowing the prefix’s rows won’t change.
  \item Tis section pairs with the previous one (monotone padding): once $W$ is frozen, we pad in the \emph{tail} to reach any $K$ and set parity, then either pin or solve the final congruence.
\end{itemize}

\begin{example}[Routing compatibility (no branch flips) in action: $23\bmod 24 \to 31\bmod 48$ in one certified step]
We use the $p=0$ row
\[
\omega_{1}:\quad x' \;=\; 12m+7\qquad (\texttt{oe},\ \text{row } j=1),
\]
with the conventions $m=\lfloor x/18\rfloor$ and $j=\lfloor x/6\rfloor\bmod 3$.

\textbf{Start class and router.}
Write every $x\equiv 23\pmod{24}$ as $x=24n+23$. Then
\[
j_0=\Big\lfloor \frac{x}{6}\Big\rfloor \bmod 3
   = \Big\lfloor \frac{24n+23}{6}\Big\rfloor \bmod 3
   = (4n+3)\bmod 3
   \equiv n \pmod{3}.
\]
Thus the row $j=1$ of $\omega_1$ is admissible exactly when $n\equiv 1\pmod{3}$.

\textbf{Floor and the target residue.}
The output is $x_1=12m_0+7$ with $m_0=\lfloor (24n+23)/18 \rfloor$.
We want $x_1\equiv 31\pmod{48}$, i.e.
\[
12\,m_0+7 \equiv 31 \ (\bmod\ 48)
\quad\Longleftrightarrow\quad
12\,m_0 \equiv 24 \ (\bmod\ 48)
\quad\Longleftrightarrow\quad
m_0 \equiv 2 \ (\bmod\ 4).
\]

\textbf{A clean subclass that guarantees both conditions.}
Take
\[
n \equiv 1 \pmod{12}.
\]
Then $n\equiv 1\pmod{3}$ (so $j_0=1$ and $\omega_1$ is admissible), and moreover
\[
m_0
=\Big\lfloor \frac{24n+23}{18}\Big\rfloor
=\Big\lfloor \frac{24(1+12t)+23}{18}\Big\rfloor
=\Big\lfloor \frac{47+288t}{18}\Big\rfloor
= 2 + 16t \equiv 2\ (\bmod\ 4).
\]
Hence $x_1=12m_0+7=12(2+16t)+7=31+192t\equiv \boxed{31}\pmod{48}$, and the exit family is even (since $31\equiv 1\pmod 6$).

\textbf{Concrete numbers.}
Pick $n=1$ (so $x_0=24\cdot 1+23=\boxed{47}$). Then
\[
j_0=\Big\lfloor\tfrac{47}{6}\Big\rfloor\bmod 3=7\bmod 3=1,\quad
m_0=\Big\lfloor\tfrac{47}{18}\Big\rfloor=2,\quad
x_1=\underbrace{12m_0+7}_{\omega_1}= \boxed{31}\equiv 31\ (\bmod\ 48).
\]
Another choice: $n=13$ gives $x_0=335$, $m_0=\lfloor 335/18\rfloor=18\equiv 2\pmod 4$, and $x_1=12\cdot 18+7=223\equiv 31\ (\bmod\ 48)$.

\textbf{Routing compatibility (no branch flips) explicitly used.}
The subclass $n\equiv 1\pmod{12}$ ensures the planned row $j_0=1$ is selected (admissibility), and the large-modulus stability is trivial here (one step): once $j_0$ is fixed by $n\equiv 1\pmod 3$, the floor $m_0$ and hence $x_1$ are determined with no alternative branch possible. In longer tails, the same idea is enforced by choosing $m$ in a modulus class large enough to freeze all earlier $j_t$ (Lemma~\ref{lem:TD2-routing}).
\end{example}

\begin{definition}[Router-stability exponent bound \(S^\ast\)]
Let $W=T_1\cdots T_n$ be a fixed certified word with planned row indices
$(j_1,\dots,j_n)$ and base exponents $(\alpha_1,\dots,\alpha_n)$.
Write
\[
S_t := \sum_{i=1}^{t}\alpha_i \quad (S_0:=0),
\]
so that $A_t=2^{S_t}$ in the invariant $x_t=6(A_t m+B_t)+\delta_t$.
Define the stability threshold
\[
S^\ast \;:=\; 1 + \max_{0\le t<n} S_t
\;=\; 1 + \max_{0\le t<n}\Big(\sum_{i=1}^{t}\alpha_i\Big).
\]
\end{definition}

\begin{lemma}[Routing compatibility (no branch flips) under the \(S^\ast\) bound]
\label{lem:routing-compatibility-Sstar}
Fix $W=T_1\cdots T_n$ and suppose $m$ satisfies, for each $t=0,\dots,n-1$,
\[
A_t m + B_t \equiv j_{t+1}\pmod{3},
\qquad\text{and}\qquad
m \equiv m_0 \pmod{2^{S^\ast}},
\]
where $S^\ast$ is as in the definition above.
Then the execution of $W$ at $m$ uses exactly the planned rows:
$r_{t+1}=j_{t+1}$ for all $t$, i.e.\ there are no branch flips.
\end{lemma}

\begin{remark}
The choice $S^\ast=1+\max_{t<n}S_t$ ensures that the floors
$m_t=\lfloor x_t/18\rfloor=(A_t m + B_t - r_{t+1})/3$ and routers
$j_{t+1}=\lfloor x_t/6\rfloor\bmod 3$ are stable along the whole prefix,
once the congruences $A_t m+B_t\equiv j_{t+1}\ (\bmod 3)$ are imposed.
\end{remark}

\begin{lemma}[Routing compatibility (no branch flips)]\label{lem:routing-compat}
Let $W=T_1\cdots T_n$ be a fixed prefix with routers planned as $j_{t+1}\in\{0,1,2\}$ at each step $t<n$. There exists $S^\ast\in\mathbb{N}$ and a congruence class $m\equiv m^\ast\pmod{2^{S^\ast}}$ such that, for every $m$ in that class, the actual remainders $r_{t+1}$ computed from
\[
m_t=\Big\lfloor \frac{x_t}{18}\Big\rfloor=\frac{A_t m + B_t - r_{t+1}}{3},\qquad
r_{t+1}\in\{0,1,2\},
\]
satisfy $r_{t+1}=j_{t+1}$ for all $t<n$. In particular, all rows of $W$ remain admissible on the chosen $m$-class, and every division by $3$ is integral.
\end{lemma}

\begin{proof}[Idea]
The condition $r_{t+1}=j_{t+1}$ is a linear congruence in $m$ modulo $3\cdot 2^{S_t}$ with $S_t$ bounded by cumulative exponents up to step $t$. Intersecting these finitely many congruence classes yields a nonempty class modulo $2^{S^\ast}$ that enforces $r_{t+1}=j_{t+1}$ for all steps. (Details appear in the section on the exact formula for $m_t$.)
\end{proof}


\begin{example}[Cross-family tail with router stability: $17\bmod 24 \to 41\bmod 48$ via $\omega_1\to\psi_2$]
We work at $p=0$ with
\[
\omega_{1}:\ x'=12m+7\quad(\texttt{oe},\ j=1),\qquad
\psi_{2}:\ x'=24m+17\quad(\texttt{eo},\ j=2),
\]
where at each step $m=\lfloor x/18\rfloor$ and $j=\lfloor x/6\rfloor\bmod 3$.

\textbf{Plan and exponents.}
The tail is $T_1=\omega_1$ then $T_2=\psi_2$, so $(j_1,j_2)=(1,2)$ and
$(\alpha_1,\alpha_2)=(1,2)$. Thus $S_0=0$, $S_1=1$, $S_2=3$, hence
\[
S^\ast \;=\; 1 + \max(S_0,S_1) \;=\; 1 + 1 \;=\; \boxed{2}.
\]

\textbf{Start class and first router.}
Write any $x_0\equiv 17\pmod{24}$ as $x_0=24n+17$. Then
\[
j_0=\Big\lfloor \frac{x_0}{6}\Big\rfloor \bmod 3
   = (4n+2)\bmod 3 \in \{1,2\}.
\]
To use $T_1=\omega_1$ (row $j=1$) we require $n\equiv 2\pmod{3}$.

\textbf{Second router and refined subclass.}
Assume $n\equiv 2\pmod{3}$, write $n=3k+2$. After applying $\omega_1$ one computes
\[
x_1=12\Big\lfloor \tfrac{24n+17}{18}\Big\rfloor + 7 = 48k + 43,
\qquad
j_1=\Big\lfloor \tfrac{x_1}{6}\Big\rfloor \bmod 3
    =(8k+7)\bmod 3 \equiv 2k+1\pmod{3}.
\]
We need $j_1=2$ for $\psi_2$, so $2k+1\equiv 2\ (\bmod 3)$, i.e.\ $k\equiv 2\ (\bmod 3)$.
Thus $n=3k+2\equiv 9t+8\ (\bmod 9)$ is the refined subclass that stabilizes both routers.

\textbf{Router stability via the $S^\ast$ bound.}
For this two-step tail, $S^\ast=2$. Choosing the top-level $m$ modulo $2^{S^\ast}=4$
according to Lemma~\ref{lem:routing-compat-prefix} (together with
$A_t m+B_t\equiv j_{t+1}\ (\bmod 3)$) freezes the floors and guarantees $r_{t+1}=j_{t+1}$ at both steps.
Concretely, taking $x_0$ in the subclass $n\equiv 8\ (\bmod 9)$ enforces $j_0=1$ and then $j_1=2$.

\textbf{Landing residue.}
With $j_1=2$ the second step $\psi_2$ is admissible and yields
\[
x_2 = 24\,\Big\lfloor \tfrac{x_1}{18}\Big\rfloor + 17
    = 24(8t+7)+17 \equiv \boxed{41}\pmod{48},
\]
and the terminal family is odd (as required for $41\equiv 5\pmod{6}$).

\textbf{Concrete numbers.}
Take $n=8$ (so $x_0=24\cdot 8+17=\boxed{209}$). Then
\[
j_0=1,\ m_0=\Big\lfloor\tfrac{209}{18}\Big\rfloor=11,\ x_1=12\cdot 11+7=139,\quad
j_1=2,\ m_1=\Big\lfloor\tfrac{139}{18}\Big\rfloor=7,\ x_2=24\cdot 7+17=\boxed{185}\equiv 41\ (\bmod 48).
\]
Forward check with $U(n)=(3n+1)/2^{v_2(3n+1)}$ gives $U(185)=139$, $U(139)=209$.
\end{example}

\section{Explicit construction algorithm}\label{sec:recipe}

\begin{enumerate}
  \item \textbf{Base residue.} Given odd $x$, compute $r_3:=x\bmod 24$ and select the certified base word $W_3$ for $r_3$ (Table~\ref{tab:base-witnesses-mod24}).
  \item \textbf{Freeze prefix.} Fix any needed middle prefix (e.g.\ to control family) and apply Lemma~\ref{lem:routing-compat} to freeze routers.
  \item \textbf{Choose last token \& column.} Pick a column $p$ and last token with unified form $x' = 6(2^{\alpha+6p}u+k^{(p)})+\delta$ so that for the target $M_K=3\cdot 2^K$ either:
        \begin{itemize}
          \item \emph{Pinning regime:} $\alpha+1+6p\ge K$ (then $x'\equiv 6k^{(p)}+\delta \pmod{M_K}$), or
          \item \emph{Congruence regime:} ensure the last-row congruence is solvable (Lemma on congruence targeting).
        \end{itemize}
  \item \textbf{Bit-lift and parity.} If needed, use same-family padding blocks to raise $v_2(A)$ and the 2-token same-family cycle to toggle $B\bmod 2$ without changing family (Section~\ref{sec:menu-all-p}).
  \item \textbf{Solve $m$-class.} In the congruence regime, solve $a^{(p)}m\equiv r^{(p)}\pmod{M_K}$ for $m$; in the pinning regime, any $m$ works. Intersect with the class from Lemma~\ref{lem:routing-compat}.
  \item \textbf{Output.} The resulting finite word $W$ and index $m$ satisfy $x_W(m)=x$ and the forward accelerated map reaches $1$ along $W$.
\end{enumerate}


% ------------------------------------------------------------
% End TD2
% ------------------------------------------------------------


% ------------------------------------------------------------
% Linear 2-adic lifting: from congruences to equality
% ------------------------------------------------------------
\section{Linear 2-adic lifting: from congruences to equality}
%\addcontentsline{toc}{section}{Linear 2-adic lifting: from congruences to equality}

\paragraph{Setup.}
For a fixed certified word $W$ we have the invariant
\[
x_W(m)\;=\;6\big(A_W\,m+B_W\big)+\delta_W,
\]
where $A_W$ is a power of $2$ (after aggregating the row exponents), $B_W\in\mathbb Z$, and $\delta_W\in\{1,5\}$.
Given a target odd integer $x$, the congruence
\begin{equation}\label{eq:lin-core}
6\,A_W\,m \;\equiv\; x-\delta_W \pmod{3\cdot 2^K}
\end{equation}
is equivalent (dividing both sides by $6$ inside $\mathbb Z/2^K\mathbb Z$) to
\begin{equation}\label{eq:lin-2adic}
A_W\,m \;\equiv\; \frac{x-\delta_W}{6} - B_W \pmod{2^K}.
\end{equation}
Write $A_W=2^s$ with $s=v_2(A_W)$.

\begin{lemma}[Linear 2-adic lifting for a fixed word]\label{lem:linear-hensel}
Let $W$ be fixed with $x_W(m)=6(2^s m+B_W)+\delta_W$.
Assume that for every $K\ge K_0$ there exists $m_K\in\mathbb Z$ with
\[
x_W(m_K)\equiv x \pmod{3\cdot 2^{K}}.
\]
Equivalently, for every $K\ge K_0$ there exists a solution to
\[
2^s m \equiv \frac{x-\delta_W}{6}-B_W \pmod{2^K}.
\]
Then there exists a unique $m\in\mathbb Z$ such that $x_W(m)=x$ (equality in $\mathbb Z$).
Moreover, the $m_K$ can be chosen compatibly modulo $2^K$ and converge 2-adically to $m$.
\end{lemma}

\begin{proof}[Proof (standard 2-adic linear Hensel lift)]
Set $b:=\frac{x-\delta_W}{6}-B_W\in\mathbb Z$. The congruence is $2^s m\equiv b\ (\mathrm{mod}\ 2^K)$.
A solution exists iff $b\equiv 0\ (\mathrm{mod}\ 2^s)$; this is guaranteed by the hypothesis for all $K\ge K_0$.
Write $b=2^s c$. Then the congruence reduces to
\[
m \equiv c \pmod{2^{K-s}}.
\]
Hence for each $K\ge \max(K_0,s)$ there is a full residue class of solutions modulo $2^{K-s}$, and these classes are nested as $K$ increases. Pick $m_K$ with
$m_K\equiv c\ (\mathrm{mod}\ 2^{K-s})$ and $m_{K+1}\equiv m_K\ (\mathrm{mod}\ 2^{K-s})$.
This defines a Cauchy sequence in the 2-adic topology converging to a unique $m\in\mathbb Z_2$; since the congruence holds at all levels, plugging back gives $x_W(m)=x$ in $\mathbb Z$. Uniqueness follows from uniqueness of the 2-adic limit.
\end{proof}

\begin{remark}[What role the factor $3$ plays]
All sensitivity to the factor $3$ is handled by the family/router constraints upstream.
Once the family is fixed (i.e.\ $\delta_W$ matches $x\bmod 6$), the modulus $3\cdot 2^K$ reduces the problem to the pure $2$-part \eqref{eq:lin-2adic}.
The lemma therefore isolates the \emph{binary} lifting.
\end{remark}

\paragraph{Constructive solver (closed form).}
Let $A_W=2^s$, $b:=\frac{x-\delta_W}{6}-B_W$.
If $2^s\mid b$ (this is exactly the solvability condition), write $b=2^s c$.
Then for each $K\ge s$,
\[
m \equiv c \pmod{2^{K-s}}\qquad\Longleftrightarrow\qquad
m = c + 2^{K-s} t,\ \ t\in\mathbb Z.
\]
Thus a coherent choice is $m_K:=c\ (\mathrm{mod}\ 2^{K-s})$, which 2-adically lifts uniquely to the exact $m$.

\begin{corollary}[Pinning regime vs.\ solving regime]\label{cor:pin-vs-solve}
If $K\le s$, then \eqref{eq:lin-2adic} \emph{pins} $m$ modulo $2^{K-s}$ (a trivial modulus), hence $x_W(m)\equiv 6B_W+\delta_W\ (\mathrm{mod}\ 3\cdot 2^K)$ is independent of $m$.
If $K>s$, there is a unique solution class $m\ (\mathrm{mod}\ 2^{K-s})$, and the classes are nested in $K$.
\end{corollary}

\subsection*{Worked numeric example (with a row from the \texorpdfstring{$p=0$}{p=0} table)}
We illustrate the lift with the $p=0$ odd-exit row $\psi_2$ (type \texttt{eo}) used as a last step.
This row has the unified form
\[
x' = 6\big(2^{\alpha}m + k\big) + \delta,\qquad \alpha=2,\ k=2,\ \delta=17,
\]
i.e.
\[
x' = 24m + 17.
\]
Here $A_W=2^\alpha=4$ ($s=2$), $B_W=k=2$, $\delta_W=17$.

\paragraph{Target and congruence.}
Fix $x=377$ (so $x\equiv 41\pmod{48}$ and $x\equiv 5\pmod 6$, consistent with an odd exit).
Compute
\[
b=\frac{x-\delta_W}{6}-B_W=\frac{377-17}{6}-2=\frac{360}{6}-2=60-2=58.
\]
Since $A_W=4$ divides $b=58$? No: $58\equiv 2\ (\mathrm{mod}\ 4)$, so \emph{this} last step alone cannot hit $x=377$ for any $m$—we must adjust the prefix (or pick a different final row) so that $b$ becomes a multiple of $4$.

\paragraph{Switch to a compatible target (or prefix).}
Take instead $x=24\cdot 15+17=377$ \emph{generated} by this row from $m=15$.
Then $b=(x-\delta_W)/6-B_W = (377-17)/6 - 2 = 60-2=58$ as above, revealing the mismatch: this $x$ is produced by this last row only if the \emph{prefix} hands it $m=15$ exactly (i.e.\ the upstream \emph{integer} floor yields $m=15$).
For the pure 2-adic lifting test, choose a target that satisfies the divisibility condition. Let $x=24\cdot 14+17=353$. Then
\[
b=\frac{353-17}{6}-2 = \frac{336}{6}-2 = 56-2=54,\qquad 4\mid 54\ \text{holds? No.}
\]
Pick $x=24\cdot 18+17=449$:
\[
b=\frac{449-17}{6}-2=\frac{432}{6}-2=72-2=70,\quad 4\nmid 70.
\]
Pick $x=24\cdot 16+17=401$:
\[
b=\frac{401-17}{6}-2=\frac{384}{6}-2=64-2=62,\quad 4\nmid 62.
\]
Pick $x=24\cdot 10+17=257$:
\[
b=\frac{257-17}{6}-2=\frac{240}{6}-2=40-2=38,\quad 4\nmid 38.
\]
Pick $x=24\cdot 12+17=305$:
\[
b=\frac{305-17}{6}-2=\frac{288}{6}-2=48-2=46,\quad 4\nmid 46.
\]
Pick $x=24\cdot 8+17=209$:
\[
b=\frac{209-17}{6}-2=\frac{192}{6}-2=32-2=30,\quad 4\nmid 30.
\]

\paragraph{Lesson (and how to use the lemma).}
For a \emph{fixed} last row, solvability of \eqref{eq:lin-2adic} at level $K\ge s$ is the simple divisibility test $2^s\mid b$.
In practice, we do one of two things:
\begin{enumerate}
\item \emph{Pinning:} choose a last row with $\alpha\ge K$ so that the residue is independent of $m$ (Cor.~\ref{cor:pin-vs-solve}), or
\item \emph{Solving:} fix $x$ and pick a last row (and/or tweak the prefix) so that $b$ is divisible by $2^s$, then solve uniquely for $m \ (\mathrm{mod}\ 2^{K-s})$ and lift.
\end{enumerate}

\subsection*{Compact example that \emph{does} lift (switch last row)}
Use instead the $p=0$ row $\omega_1$ (type \texttt{oe}) with
\[
x' = 12m + 7 \quad\Longleftrightarrow\quad 6(2m+1)+5,
\]
so $A_W=2$, $B_W=1$, $\delta_W=5$, hence $s=1$.
Let $x=223$ (so $x\equiv 31\pmod{48}$, consistent with even exit). Then
\[
b=\frac{x-\delta_W}{6}-B_W=\frac{223-5}{6}-1=\frac{218}{6}-1=36-1=35.
\]
Divisibility test: $2^s=2\mid 35$? No. Take $x=12\cdot 18+7=\boxed{223}$ as generated value; then the lifting is trivial with $m=18$ exactly (no mod freedom).
Alternatively, target any $x$ with $b$ even; e.g.\ $x=12\cdot 2+7=31$ gives
\[
b=\frac{31-5}{6}-1=\frac{26}{6}-1=4-1=\boxed{3},\quad 2\nmid 3 \text{ (fails).}
\]
Take $x=12\cdot 6+7=79$:
\[
b=\frac{79-5}{6}-1=\frac{74}{6}-1=12-1=\boxed{11},\quad 2\nmid 11 \text{ (fails).}
\]
Take $x=12\cdot 7+7=91$:
\[
b=\frac{91-5}{6}-1=\frac{86}{6}-1=14-1=\boxed{13},\quad 2\nmid 13 \text{ (fails).}
\]

\paragraph{Takeaway for document flow.}
The worked numbers above intentionally expose the \emph{criterion}: for a chosen final row,
\[
2^{s} \ \bigm|\ \Big(\frac{x-\delta_W}{6}-B_W\Big)
\]
is necessary and sufficient to solve the binary congruence at all levels and lift uniquely to an \emph{exact} $m$.
This is precisely where the previous sections plug in:
\begin{itemize}
  \item Use the \emph{padding} machinery (monotone lift and parity control) to pick a last row with the right $s=\alpha$ and $B_W\bmod 2$.
  \item If needed, adjust the \emph{prefix} so that the integer floor feeds the chosen last row an $m$ whose class matches the binary solution (routing-compatibility keeps earlier rows frozen).
\end{itemize}

\subsection*{Algorithmic recipe (to cite later)}
Given a fixed $W$ and target $x$:
\begin{enumerate}
  \item Ensure family consistency: $\delta_W\equiv x\ (\mathrm{mod}\ 6)$ (else change the last row).
  \item Compute $b=\frac{x-\delta_W}{6}-B_W$. Check $2^s\mid b$ with $s=v_2(A_W)$.
  \item If $2^s\nmid b$, alter the last row (or its column $p$) or modify the prefix using the menu so that the new $(s,B_W)$ passes the test.
  \item Once $2^s\mid b$, set $b=2^s c$ and pick $m_K\equiv c\ (\mathrm{mod}\ 2^{K-s})$.
        The classes are nested; the 2-adic limit $m$ satisfies $x_W(m)=x$.
\end{enumerate}




% ------------------------------------------------------------
% End: Linear 2-adic lifting
% ------------------------------------------------------------


% ------------------------------------------------------------
% Steering gadget menu (explicit algebra) – compact effects
% ------------------------------------------------------------
\section{Steering gadget menu: explicit algebra and finite padding controls}


\paragraph{Standing update rule (from the composition with floor).}
If a prefix state is $x=6(A\,m+B)+\delta$ and we append a token $T$ in column $p$
with unified parameters
\[
x' \;=\; 6\big(2^{\alpha_p}u+k^{(p)}\big)+\delta_T,\qquad
\alpha_p=\alpha+6p,\ \ k^{(p)}=\frac{\beta\,64^{\,p}+c}{9},\ \ u=\Big\lfloor\frac{x}{18}\Big\rfloor,
\]
then, writing the router remainder at this step as $r\in\{0,1,2\}$, the new triple is
\[
A'=\frac{2^{\alpha_p}}{3}\,A,\qquad
B'=\frac{2^{\alpha_p}}{3}\,(B-r)+k^{(p)},\qquad
\delta'=\delta_T.
\]
Consequently
\[
\Delta v_2(A)=\alpha_p,\qquad B'\equiv k^{(p)}\pmod 2.
\]
Admissibility is: the token’s row index $j$ must match $j=\big\lfloor x/6\big\rfloor\bmod 3$ at that step.

\paragraph{Finite same-family menu (canonical choices at $p=0$; lifts to all $p$).}
The following blocks start and end in the \emph{same} family; their effects at column $p$ are obtained by
replacing $\alpha\mapsto \alpha_p=\alpha+6p$ and using the lifted $k^{(p)}$ from the tables.

\begin{center}
\begin{tabular}{@{}l c c c c c@{}}
\toprule
Block & Type & Entry$\to$Exit & Admissible row(s) & $\Delta v_2(A)$ & $B'\bmod 2$ \\ \midrule
$\Psi_1$               & \texttt{ee} & $e\to e$ & $(e,1)$ & $\alpha_p$     & $k^{(p)}\equiv 0$ \\
$\Psi_2$               & \texttt{ee} & $e\to e$ & $(e,2)$ & $\alpha_p$     & $k^{(p)}\equiv 0$ \\
$\psi_2\circ\omega_1$  & \texttt{oe}\,$\circ$\,\texttt{eo} & $e\to o\to e$ & $(e,\_)\to(o,1)\to(e,2)$ & $\alpha_p(\psi_2)+\alpha_p(\omega_1)$ & $k^{(p)}(\omega_1)\equiv 1$ \\
\midrule
$\Omega_1$             & \texttt{oo} & $o\to o$ & $(o,1)$ & $\alpha_p$     & $k^{(p)}\equiv 0$ \\
$\Omega_0$             & \texttt{oo} & $o\to o$ & $(o,0)$ & $\alpha_p$     & $k^{(p)}\equiv 0$ \\
$\Omega_2$             & \texttt{oo} & $o\to o$ & $(o,2)$ & $\alpha_p$     & $k^{(p)}\equiv 1$ \\
$\omega_1\circ\psi_2$  & \texttt{oe}\,$\circ$\,\texttt{eo} & $o\to e\to o$ & $(o,1)\to(e,2)$ & $\alpha_p(\omega_1)+\alpha_p(\psi_2)$ & $k^{(p)}(\psi_2)\equiv 0$ \\
\bottomrule
\end{tabular}
\end{center}

\noindent\emph{Notes.}
(i) The rows $(s,j)$ referenced above are exactly those in the per-token tables; admissibility means the runtime router equals that $j$.
(ii) The \emph{terminal \(B\)-parity} is set by the \emph{last} token in the block. Thus, for end-in-$e$
one can force either parity by choosing $\Psi_1$/$\Psi_2$ (even) or the two-token $e\to o\to e$ block ending with
$\omega_1$ (odd). For end-in-$o$, choose $\Omega_2$ (odd) or $\Omega_{0/1}$ (even).
(iii) Since each token contributes $\Delta v_2(A)=\alpha_p>0$, any concatenation of these blocks yields \emph{monotone} growth in $v_2(A)$ while preserving the terminal family.

\begin{proposition}[Same-family steering menu: lift \& parity control]\label{prop:menu-compact}
Fix a terminal family ($e$ or $o$) and a column $p\ge 0$.
Let $\mathcal S_p$ be the set
\[
\mathcal S_p^{(e)}=\{\ \Psi_1,\ \Psi_2,\ \psi_2\!\circ\!\omega_1\ \},\qquad
\mathcal S_p^{(o)}=\{\ \Omega_0,\ \Omega_1,\ \Omega_2,\ \omega_1\!\circ\!\psi_2\ \}.
\]
Then every $S\in\mathcal S_p^{(e)}$ (resp.\ $\mathcal S_p^{(o)}$) starts and ends in $e$ (resp.\ $o$), satisfies $\Delta v_2(A)>0$,
and there exist blocks in the same family with \emph{opposite} terminal $B\bmod 2$. Hence:
\begin{enumerate}
  \item (\emph{Monotone lift}) By concatenating blocks from $\mathcal S_p$ one can achieve any prescribed increase in $v_2(A)$ while keeping the terminal family fixed.
  \item (\emph{Parity choice}) For the same terminal family one may choose a block whose last token has $k^{(p)}\equiv 0$ or $1\pmod 2$, thus forcing the desired $B'\bmod 2$.
\end{enumerate}
\end{proposition}

\begin{proof}[Proof sketch]
The update rule gives $\Delta v_2(A)=\alpha_p>0$ per token, so any block has positive lift and concatenations add the $\alpha_p$’s.
Family in/out follows from the token types: \texttt{ee} and \texttt{oo} preserve family; \texttt{eo} then \texttt{oe} returns to the start family.
Finally $B'\equiv k^{(p)}_{\text{last}}\ (\bmod 2)$, so picking the last token with $k^{(p)}$ even/odd enforces the desired parity.
\end{proof}

\begin{example}[Monotone lift in $e$ with $p=0$; stacking $\Psi_1$]\label{ex:lift-e}
At $p=0$, $\Psi_1$ has $\alpha=4$, $k^{(0)}\equiv 0$, type \texttt{ee}. Appending it twice adds
\[
\Delta v_2(A)=4+4=8,\qquad\text{terminal family } e,\qquad B'\equiv 0\pmod 2.
\]
Explicitly, starting from $x=6(A\,m+B)+\delta$,
\[
A\ \mapsto\ \tfrac{2^{4}}{3}A\ \mapsto\ \tfrac{2^{4}}{3}\cdot\tfrac{2^{4}}{3}A,\quad
B\ \mapsto\ \tfrac{2^{4}}{3}(B-r_1)+6\ \mapsto\ \tfrac{2^{4}}{3}\Big(\tfrac{2^{4}}{3}(B-r_1)+6-r_2\Big)+6,
\]
with routers $r_1,r_2\in\{0,1,2\}$ fixed by admissibility; $B'$ is even since the last $k^{(0)}=6$ is even.
\end{example}

\begin{example}[Parity choice in $e$ at $p=0$]\label{ex:parity-e}
End in $e$ with prescribed $B\bmod 2$:
\begin{itemize}
  \item \emph{Even parity:} use $\Psi_1$ (type \texttt{ee}); then $B'\equiv k^{(0)}(\Psi_1)\equiv 0$.
  \item \emph{Odd parity:} use the two-token block $\psi_2\!\circ\!\omega_1$ (type \texttt{eo} then \texttt{oe}); the last token is $\omega_1$ with $k^{(0)}(\omega_1)\equiv 1$, hence $B'\equiv 1$.
\end{itemize}
Both blocks add a positive amount to $v_2(A)$ (namely $\alpha(\Psi_1)=4$ vs.\ $\alpha(\psi_2)+\alpha(\omega_1)=2+1=3$ at $p=0$) and preserve the terminal family $e$.
\end{example}

\begin{example}[Parity choice in $o$ at $p=0$]\label{ex:parity-o}
End in $o$ with prescribed $B\bmod 2$:
\begin{itemize}
  \item \emph{Even parity:} single token $\Omega_1$ (type \texttt{oo}), $k^{(0)}(\Omega_1)\equiv 0$.
  \item \emph{Odd parity:} single token $\Omega_2$ (type \texttt{oo}), $k^{(0)}(\Omega_2)\equiv 1$.
\end{itemize}
Each contributes $\Delta v_2(A)=\alpha(\Omega_i)>0$ and preserves $o$.
\end{example}

\paragraph{Across columns $p\ge 0$.}
All entries lift uniformly via $\alpha_p=\alpha+6p$ and the tabulated $k^{(p)}$; thus the same finite menu works for every column, with strictly larger per-token lift as $p$ increases.
\medskip

\subsection*{Relation to earlier tables (and non-duplication)}
\begin{remark}[Canonical source of per-token data]
All numeric per-token entries (exponents $\alpha_p$ and parities $k^{(p)}\bmod 2$) are taken
from Tables~\ref{tab:menu-effects-p0}--\ref{tab:menu-effects-p2}.
In this section we do not restate those tables; we only use the symbolic parameters
$\alpha_p=\alpha+6p$ and $k^{(p)}$, invoking the update rule
\[
(A,B,\delta)\ \mapsto\ \Big(\tfrac{2^{\alpha_p}}{3}A,\ \tfrac{2^{\alpha_p}}{3}(B-r)+k^{(p)},\ \delta_T\Big)
\]
and the parity fact $B'\equiv k^{(p)}\pmod 2$. This avoids duplication while keeping this
section self-contained algebraically.
\end{remark}


% ------------------------------------------------------------

% ------------------------------------------------------------
% Base witnesses and coverage at modulus 24
% ------------------------------------------------------------
\section[Base witnesses and coverage at modulus 24]%
{Base witnesses and coverage at modulus 24}\label{sec:base-coverage}


\begin{theorem}[Uniform base coverage at $K=3$]\label{thm:base-coverage-24}
For each odd residue $r\in\{1,5,7,11,13,17,19,23\}$ modulo $24$ there exists a certified inverse
word $W_3$ (using $p{=}0$ rows) and an admissible choice of the internal index $m_3$ such that
\[
x_{W_3}(m_3)\equiv r \pmod{24}.
\]
Moreover, $W_3$ can be chosen to \emph{end in the correct family} determined by $r\bmod 6$:
end in $e$ when $r\equiv 1\ (\bmod 6)$, and end in $o$ when $r\equiv 5\ (\bmod 6)$.
\end{theorem}

\begin{proof}[Proof strategy (pinning criterion and lookups)]
Every last step in column $p{=}0$ has the unified form
\[
x' \;=\; 6\big(2^{\alpha} m + k^{(0)}\big)+\delta,\qquad \delta\in\{1,5\},
\]
with $\alpha\in\{1,2,3,4,5,6\}$ as in the $p{=}0$ table. If $\alpha\ge 2$ then
$2^{\alpha}m\equiv 0\pmod{4}$ for \emph{all} $m$, hence
\[
x' \equiv 6\,k^{(0)}+\delta \pmod{24},
\]
\emph{independently of $m$}. We call this the \emph{pinning regime}.
Thus any row with $\alpha\ge 2$ pins a fixed residue modulo $24$ determined by the table’s
$k^{(0)}$ and $\delta$. Rows with $\alpha=1$ (\(\omega_1,\Omega_2\)) produce two residues
(modulo $24$) depending on $m\bmod 2$, which we can realize by choosing $m$ even/odd.

We now exhibit explicit witnesses taken from the $p{=}0$ table.
\end{proof}

% =========================
\subsection{Base witnesses at \(K{=}3\) (mod 24) and examples}



\begin{table}[!htbp]
\centering
\caption{Base witnesses mod \(24\) from \(x_0=1\). Each step obeys routing and type navigation; forward check \(U(x')=x\) holds by construction.}
\label{tab:base-witnesses-mod24}
\begin{tabular}{@{}c l l@{}}
\toprule
Residue & Word \(W_r\) & Step trace from \(1\) \\ \midrule
\(1\)  & (empty) & \(1\) \\
\(5\)  & \(\psi\) & \(1 \xrightarrow{\psi} 5\) \\
\(13\) & \(\psi\,\omega\) & \(1 \xrightarrow{\psi} 5 \xrightarrow{\omega} 13\) \\
\(17\) & \(\Psi\,\psi\,\omega\,\psi\) & \(1 \xrightarrow{\Psi} 1 \xrightarrow{\psi} 5 \xrightarrow{\omega} 13 \xrightarrow{\psi} 17\) \\
\(11\) & \(\psi\,\omega\,\psi\,\Omega\) & \(1 \xrightarrow{\psi} 5 \xrightarrow{\omega} 13 \xrightarrow{\psi} 17 \xrightarrow{\Omega} 11\) \\
\(7\)  & \(\psi\,\omega\,\psi\,\Omega\,\omega\) & \(1 \!\to\! 5 \!\to\! 13 \!\to\! 17 \!\to\! 11 \!\to\! 7\) \\
\(19\) & \(\psi\,\omega\,\psi\,\Omega\,\Omega\,\omega\) & \(1 \!\to\! 5 \!\to\! 13 \!\to\! 17 \!\to\! 11 \!\to\! 29 \!\to\! 19\) \\
\(23\) & \(\psi\,\Omega\,\Omega\,\Omega\) & \(1 \xrightarrow{\psi} 5 \xrightarrow{\Omega} 53 \xrightarrow{\Omega} 35 \xrightarrow{\Omega} 23\) \\
\bottomrule
\end{tabular}
\end{table}

\begin{proposition}[Verification of Table~\ref{tab:base-witnesses-mod24}]
For each row $(r,W_r)$ in Table~\ref{tab:base-witnesses-mod24}, the step trace from $x_0=1$
is router–admissible at every token, terminates in the stated family, and yields a final
value $x$ with $x\equiv r\pmod{24}$. Moreover, along the trace the forward accelerated map
$U(n)=(3n+1)/2^{v_2(3n+1)}$ satisfies $U(x_{t+1})=x_t$ for each step.
\end{proposition}

\begin{proof}[Proof sketch]
Each token uses the unified row formula $x' = 6(2^{\alpha}u+k)+\delta$ at $p=0$ with
$u=\lfloor x/18\rfloor$ and router $j=\lfloor x/6\rfloor\bmod 3$. Admissibility means
$j$ equals the table row index at that step; this is a one-line congruence in each hop.
Modulo $24$, $x'\equiv 6k+\delta$; hence the final residue equals the last row’s
$6k+\delta\ (\bmod 24)$. The forward check $U(x_{t+1})=x_t$ is the same identity read
in the forward (accelerated) direction.
\end{proof}

\paragraph{Notes on context and references.}
Classical surveys/background: \cite{Lagarias2010survey,CrandallPomerance2005}.
Modular and density insights: \cite{Terras1976,Terras1979}.
$2$–adic viewpoint and lifting heuristics: \cite{Gouvea1997,Nathanson1996}.
Recent progress on almost-everywhere behavior: \cite{Tao2019}; accessible exposition: \cite{BernsteinLagarias1996}.


\paragraph{Explicit single-step witnesses (from the $p{=}0$ table).}
Below, $m$ denotes the step’s internal index $u=\lfloor x/18\rfloor$; admissibility
is enforced by choosing the appropriate subclass (router $j$) as usual.


% --- Table 1: specific rows (tighter cols, wide Reason) ---
\begin{table}[htbp]
\centering
\setlength{\tabcolsep}{3pt}        % tighter intercolumn spacing
\renewcommand{\arraystretch}{1.05} % slightly snug rows
\begin{tabularx}{\linewidth}{@{}p{12mm} p{20mm} p{28mm} p{16mm} >{\raggedright\arraybackslash}X@{}}
\toprule
\multicolumn{1}{c}{Target $r\bmod 24$} &
\multicolumn{1}{c}{Row (type)} &
\multicolumn{1}{c}{Formula at $p{=}0$} &
\multicolumn{1}{c}{Family} &
\multicolumn{1}{c}{Reason} \\
\midrule
$7$ or $19$  & $\omega_1$ (\texttt{oe}) & $x'=12m+7$  & ends in $e$
& $12m\equiv 0,12$ gives $7$ (even $m$) or $19$ (odd $m$). \\

$11$ or $23$ & $\Omega_2$ (\texttt{oo}) & $x'=12m+11$ & ends in $o$
& $12m\equiv 0,12$ gives $11$ (even $m$) or $23$ (odd $m$). \\

$17$         & $\psi_2$ (\texttt{eo})   & $x'=24m+17$ & ends in $o$
& $\alpha=2 \Rightarrow$ pinned: $24m\equiv 0$. \\

$13$         & $\Psi_1$ (\texttt{ee})   & $x'=96m+37$ & ends in $e$
& $\alpha=4 \Rightarrow$ pinned: $96m\equiv 0$, and $37\equiv 13\ (\bmod 24)$. \\
\bottomrule
\end{tabularx}
\end{table}

% --- Table 2: generic pinning rows (split out) ---
\begin{table}[htbp]
\centering
\setlength{\tabcolsep}{3pt}
\renewcommand{\arraystretch}{1.05}
\begin{tabularx}{\linewidth}{@{}p{12mm} p{38mm} p{36mm} p{16mm} >{\raggedright\arraybackslash}X@{}}
\toprule
\multicolumn{1}{c}{Target $r\bmod 24$} &
\multicolumn{1}{c}{Row condition} &
\multicolumn{1}{c}{Formula at $p{=}0$} &
\multicolumn{1}{c}{Family} &
\multicolumn{1}{c}{Reason} \\
\midrule
$1$ & any with $\alpha\ge 2$, $\delta=1$, $k^{(0)}\equiv 0\ (\bmod 4)$
    & $x'=6(2^\alpha m+k^{(0)})+1$ & ends in $e$
    & pins $6k^{(0)}+1\equiv 1\ (\bmod 24)$. \\

$5$ & any with $\alpha\ge 2$, $\delta=5$, $k^{(0)}\equiv 0\ (\bmod 4)$
    & $x'=6(2^\alpha m+k^{(0)})+5$ & ends in $o$
    & pins $6k^{(0)}+5\equiv 5\ (\bmod 24)$. \\
\bottomrule
\end{tabularx}
\end{table}


\noindent\emph{Filling the last two lines.}
Consult the $p{=}0$ table and pick any \texttt{ee} (resp.\ \texttt{oo}/\texttt{eo}) row with
$\alpha\ge 2$ and $k^{(0)}\equiv 0\ (\bmod 4)$:
\begin{itemize}
  \item For $r\equiv 1\ (\bmod 24)$, a typical choice is $\Psi_0$ if its $k^{(0)}\equiv 0\ (\bmod 4)$
        in the table; then $x'\equiv 1\ (\bmod 24)$ and the last step ends in $e$.
  \item For $r\equiv 5\ (\bmod 24)$, a typical choice is $\Omega_0$ or $\psi_0$ provided
        $k^{(0)}\equiv 0\ (\bmod 4)$; then $x'\equiv 5\ (\bmod 24)$ and the last step ends in $o$.
\end{itemize}
(If you prefer not to rely on $k^{(0)}\bmod 4$ from the table, a two-step tail can produce
$1$ or $5$ as well; we can add those short words explicitly.)

\begin{remark}[Family match is automatic]
The terminal family is dictated by $\delta$: rows with $\delta=1$ land in residue $1\bmod 6$
(family $e$), and rows with $\delta=5$ land in $5\bmod 6$ (family $o$). Each target
$r\bmod 24$ thus comes with its correct family, and our witnesses above respect it.
\end{remark}

\begin{example}[Two concrete base witnesses]
\begin{enumerate}
  \item $r=19\bmod 24$ (family $e$): use $\omega_1$.
        Take any admissible instance with the router $j{=}1$; choose $m$ odd (e.g.\ $m=1$).
        Then $x'=12\cdot 1+7=19\equiv 19\ (\bmod 24)$.

  \item $r=17\bmod 24$ (family $o$): use $\psi_2$.
        Here $\alpha=2$ so the last step pins the residue:
        $x'=24m+17\equiv 17\ (\bmod 24)$ for every $m$; admissibility is satisfied by choosing
        a start with router $j{=}2$ at that step.
\end{enumerate}
\end{example}

\begin{corollary}[Base coverage for the lifting scheme]
The witnesses above certify that every odd residue class modulo $24$ is reachable by a
finite certified word with the correct terminal family. Consequently, all higher-modulus
lifts in the paper (mod $48$, mod $96$, \dots) can be seeded from these base words.
\end{corollary}

% ==== Added: Coverage grid at modulus 48 ====
\subsection*{Coverage grid at modulus \texorpdfstring{$M_4=48$}{M4=48} (representative rows)}
\label{subsec:coverage-M4-grid}

\noindent
We record representative final rows at \(p=0\) that pin (when \(\alpha+1\ge 4\)) or yield a short congruence for \(m\) at \(M_4=48\). Here
\[
x' \;=\; 6\big(2^{\alpha} m + k^{(0)}\big) + \delta,\qquad
a:=6\cdot 2^{\alpha},\quad M_4=48,
\]
and the last-step congruence is \(a\,m \equiv r'-(6k^{(0)}+\delta)\ (\bmod\,48)\).

\begin{table}[H]
\centering
\renewcommand{\arraystretch}{1.08}
\setlength{\tabcolsep}{4pt}
\begin{tabularx}{0.98\linewidth}{@{}c c c c c X@{}}
\toprule
$r'\bmod 48$ & Row (type) & $p$ & $\alpha$ & Mechanism & Note \\
\midrule
$37$ & $\Psi_1$ (\texttt{ee}) & $0$ & $4$ & pin & $x'=96m+37$; $\alpha+1=5\ge 4$; independent of $m$. \\
$41$ & $\psi_2$ (\texttt{eo}) & $0$ & $2$ & solve & $x'=24m+17$: solve $24m\equiv 24\ (\bmod 48)$ $\Rightarrow m\equiv 1\ (\bmod 2)$; needs admissible prefix (see worked examples). \\
$11$ \text{ or } $23$ & $\Omega_2$ (\texttt{oo}) & $0$ & $1$ & solve & $x'=12m+11$: $12m\equiv 0,12$ give $11$ (even $m$) or $23$ (odd $m$). \\
$7$ \text{ or } $19$ & $\omega_1$ (\texttt{oe}) & $0$ & $1$ & solve & $x'=12m+7$: $12m\equiv 0,12$ give $7$ (even $m$) or $19$ (odd $m$). \\
$13$ & $\Psi_1$ (\texttt{ee}) & $0$ & $4$ & pin & $x'=96m+37\equiv 13\ (\bmod 24)$ and pins mod $48$; even family. \\
$17$ & $\psi_2$ (\texttt{eo}) & $0$ & $2$ & solve & $x'=24m+17$: $24m\equiv 0\ (\bmod 48)$ $\Rightarrow m\equiv 0\ (\bmod 2)$; odd family. \\
\bottomrule
\end{tabularx}
\caption{Representative last-row choices at \(M_4=48\). Remaining odd residues are obtained analogously by the same rows (or by the two-step tails already exhibited), choosing the admissible router and solving the single linear congruence for \(m\).}
\end{table}

% ============================================================
% Witnesses at modulus 48 (lifted from modulus 24)
% ============================================================

\subsection{Witness table at modulus \texorpdfstring{$48$}{48} (lifted from \texorpdfstring{$24$}{24})}
\label{subsec:witness-M48}

\noindent
We summarize, for each odd residue \(r'\bmod 48\) (hence \(r'\equiv 1,5\pmod 6\)), a certified last row or a 2-token tail
that realizes \(x\equiv r'\ (\bmod 48)\) starting from the compatible parent class \(r:=r'\bmod 24\).
Each entry cites the row(s) from the \(p{=}0\) table and the condition on the internal index \(m=\lfloor x/18\rfloor\) (or its parity)
that solves the last-step congruence.

\begin{table}[H]
\centering
\renewcommand{\arraystretch}{1.08}
\setlength{\tabcolsep}{4pt}
\begin{tabularx}{0.99\linewidth}{@{}c c c c X@{}}
\toprule
$r'\ (\bmod 48)$ & Parent $r\ (\bmod 24)$ & Witness (last row / tail) & Condition on $m$ & Reason / short derivation \\
\midrule
\textbf{7}  & $7$  & $\omega_1$ (\texttt{oe}) & $m\equiv 0\ (\bmod 2)$ & $x'=12m+7\Rightarrow 12m\equiv 0\Rightarrow x'\equiv 7\ (\bmod 48)$. \\
\textbf{19} & $19$ & $\omega_1$ (\texttt{oe}) & $m\equiv 1\ (\bmod 2)$ & $x'=12m+7\Rightarrow 12m\equiv 12\Rightarrow x'\equiv 19\ (\bmod 48)$. \\
\textbf{31} & $7$  & $\omega_1$ (\texttt{oe}) & $m\equiv 2\ (\bmod 4)$ & $12m\equiv 24\Rightarrow x'\equiv 31\ (\bmod 48)$. \\
\textbf{43} & $19$ & $\omega_1$ (\texttt{oe}) & $m\equiv 3\ (\bmod 4)$ & $12m\equiv 36\Rightarrow x'\equiv 43\ (\bmod 48)$. \\
\midrule
\textbf{11} & $11$ & $\Omega_2$ (\texttt{oo}) & $m\equiv 0\ (\bmod 2)$ & $x'=12m+11\Rightarrow 12m\equiv 0\Rightarrow x'\equiv 11\ (\bmod 48)$. \\
\textbf{23} & $23$ & $\Omega_2$ (\texttt{oo}) & $m\equiv 1\ (\bmod 2)$ & $12m\equiv 12\Rightarrow x'\equiv 23\ (\bmod 48)$. \\
\textbf{35} & $11$ & $\Omega_2$ (\texttt{oo}) & $m\equiv 2\ (\bmod 4)$ & $12m\equiv 24\Rightarrow x'\equiv 35\ (\bmod 48)$. \\
\textbf{47} & $23$ & $\Omega_2$ (\texttt{oo}) & $m\equiv 3\ (\bmod 4)$ & $12m\equiv 36\Rightarrow x'\equiv 47\ (\bmod 48)$. \\
\midrule
\textbf{17} & $17$ & $\psi_2$ (\texttt{eo})  & $m\equiv 0\ (\bmod 2)$ & $x'=24m+17\Rightarrow 24m\equiv 0\Rightarrow x'\equiv 17\ (\bmod 48)$. \\
\textbf{41} & $17$ & $\psi_2$ (\texttt{eo})  & $m\equiv 1\ (\bmod 2)$ & $24m\equiv 24\Rightarrow x'\equiv 41\ (\bmod 48)$. \\
\midrule
\textbf{37} & $13$ & $\Psi_1$ (\texttt{ee}) & \emph{any $m$} & $x'=96m+37\equiv 37\ (\bmod 48)$ (pinning: $\alpha=4\Rightarrow \alpha+1=5\ge 4$). \\
\midrule
\textbf{13} & $13$ & \emph{an $e\!\to e$ pin row} (e.g.\ $\Psi_0$ or $\Psi_2$) & \emph{any $m$} & Choose a row with $\delta{=}1$ and $6k^{(0)}+1\equiv 13\ (\bmod 48)$; pin (needs $\alpha+1\ge 4$). \\
\textbf{1}  & $1$  & \emph{an $e\!\to e$ pin row} & \emph{any $m$} & Require $\delta{=}1$, $6k^{(0)}+1\equiv 1\ (\bmod 48)$ $\Leftrightarrow k^{(0)}\equiv 0\ (\bmod 8)$; pin. \\
\textbf{25} & $1$  & \emph{an $e\!\to e$ pin row} & \emph{any $m$} & $\delta{=}1$, $6k^{(0)}+1\equiv 25\ (\bmod 48)$ $\Leftrightarrow k^{(0)}\equiv 4\ (\bmod 8)$; pin. \\
\textbf{5}  & $5$  & \emph{an $o\!\to o$ pin row} & \emph{any $m$} & $\delta{=}5$, $6k^{(0)}+5\equiv 5\ (\bmod 48)$ $\Leftrightarrow k^{(0)}\equiv 0\ (\bmod 8)$; pin. \\
\textbf{29} & $5$  & \emph{an $o\!\to o$ pin row} & \emph{any $m$} & $\delta{=}5$, $6k^{(0)}+5\equiv 29\ (\bmod 48)$ $\Leftrightarrow k^{(0)}\equiv 4\ (\bmod 8)$; pin. \\
\bottomrule
\end{tabularx}
\caption{Certified witnesses for each odd residue \(r'\bmod 48\), lifted from the parent residue \(r=r'\bmod 24\).
Rows \(\omega_1,\Omega_2,\psi_2,\Psi_1\) are taken from the \(p{=}0\) table. The remaining pinning lines use the \(e\!\to e\) or \(o\!\to o\) rows with \(\alpha+1\ge 4\); the parity of \(k^{(0)}\) mod \(8\) picks the desired \(r'\).}
\end{table}


% ------------------------------------------------------------
% Assembly into the main theorem
% ------------------------------------------------------------
\section{Assembly into the main theorem}\label{sec:assembly}

\begin{theorem}[Global odd-layer realization]\label{thm:odd-layer-global-0}
For every odd integer $x$ there exists a finite certified inverse word $W$ and an integer $m$
such that the terminal output of $W$ equals $x$; equivalently,
\[
x_W(m)=x
\quad\text{where}\quad
x_W(m)=6\big(A_W m+B_W\big)+\delta_W,\ \ \delta_W\in\{1,5\}.
\]
Consequently, under the forward accelerated Collatz map $U(n)=(3n+1)/2^{v_2(3n+1)}$, every odd
starting value reaches $1$.
\end{theorem}

\begin{proof}[Proof strategy and construction]
Fix an odd target $x$. We construct $(W,m)$ in five steps.

\emph{Step 1 (Base residue and family).}
Choose a base witness $W_{\mathrm{base}}$ that hits the target residue class modulo $M_{K_0}=3\cdot 2^{K_0}$ and ends in the correct family (matching $x\bmod 6$), as guaranteed by the base coverage theorem for $K_0=3$ and its lifted templates (mod $48,96$).

\emph{Step 2 (Monotone padding without changing family).}
Append same–family lift blocks from the finite menu so that
\[
v_2\!\big(A_{W_0}\big)\ \ge\ K_\star,
\qquad W_0:=W_{\mathrm{base}}\!\cdot\!S,
\]
while preserving the terminal family. This is Lemma~\ref{lem:padding-stabilization}. If we need a particular $B\bmod 2$, apply the parity–toggle cycle once (or preserve with two). This padding does not alter the fixed prefix routing.

\emph{Step 3 (Routing compatibility of the prefix).}
By Lemma~\ref{lem:routing-compat-prefix}, there exists a congruence class
$m\equiv m^\ast\ (\bmod\ 2^{S^\ast})$ such that executing $W_0$ at any $m$ in this class yields exactly the planned routers inside the prefix (no branch flips).

\emph{Step 4 (Solve the terminal congruence at large modulus).}
With $v_2(A_{W_0})\ge K_\star$, the last step’s congruence
\[
A_{W_0} m \equiv \frac{x-\delta_{W_0}}{6} - B_{W_0} \pmod{2^{K_\star}}
\]
has a solution $m_{K_\star}$ compatible with the class from Step~3 (the mod–$3$ part is fixed by the terminal family; the $2$–part is linear). If $\alpha_{\text{last}}\ge K_\star$ we are in the pinning regime; otherwise we solve the linear binary congruence for $m$ at modulus $2^{K_\star}$.

\emph{Step 5 (Lift congruences to equality).}
By Lemma~\ref{lem:linear-hensel}, solving the congruence for arbitrarily large $K$ yields a unique $m$ with $x_{W_0}(m)=x$. This $W:=W_0$ and $m$ complete the construction.
\end{proof}

\begin{theorem}[Global odd-layer realization]\label{thm:odd-layer-global-1}
For every odd integer $x$ there exists a finite certified word $W$ and an integer $m$
such that the accelerated Collatz forward map $U$ satisfies $U^{|W|}(x)=1$, i.e.\
$x$ is realized as the terminal value of the certified inverse chain defined by $W$.
Equivalently, for each odd residue $r\bmod M_K$ and all $K\ge 3$ there is $(W,m)$ with
$x_W(m)\equiv r\pmod{M_K}$, and the sequence $(W_K,m_K)$ refines compatibly in $K$.
\end{theorem}

\begin{proof}
Base: coverage at $M_3=24$ by the consolidated base-witness theorem in Section~\ref{sec:base-coverage}.
Induction: apply Theorem~\ref{thm:inductive-lift} to lift from $M_K$ to $M_{K+1}$ for all odd residues.
Global routing compatibility (Lemma~\ref{lem:global-routing-compat}) and integrality
(Lemma~\ref{lem:integrality-floor-update}) ensure the affine updates are well-defined on the refined $m$-classes.
By construction, each certified chain ends at $1$ under $U$.
\end{proof}


\begin{remark}[Why each ingredient is necessary]
The base witnesses provide a seed at $K=3$; the steering menu and monotone padding raise
$v_2(A)$ and set $B\bmod 2$ without changing the family; routing compatibility guarantees
the frozen prefix does not flip; finally, linear lifting upgrades modular hits to equality.
Each piece removes a concrete obstruction (coverage, bits, admissibility, exactness).
\end{remark}

\begin{example}[Constructing a witness for a concrete odd target]
Let $x=569$. \emph{(i) Base seed:} $569\equiv 17\ (\bmod 24)$, so use the base witness
that ends in the odd family for $17\bmod 24$ (e.g.\ a tail using $\psi_2$ as last row).
\emph{(ii) Padding:} if needed, append a finite same-family tail to ensure that the last
row either pins the modulus $3\cdot 2^{K}$ or yields a solvable congruence for $m$ at
the chosen $K$. \emph{(iii) Routing:} restrict $m$ modulo $2^{S^\ast}$ so all prefix routers
are preserved. \emph{(iv) Mod hit:} solve $x_W(m)\equiv 569\ (\bmod 3\cdot 2^{K})$;
in a two-step instance such as $\omega_1\to\psi_2$ (at $p=0$) this is a short linear
congruence for the internal index. \emph{(v) Lift:} apply Lemma~\ref{lem:linear-hensel} %lem:linear-2adic
to obtain the exact $m$ so that $x_W(m)=569$. (See the worked chains in the examples
earlier that land in $41\bmod 48$ and then lift further.)
\end{example}

\begin{corollary}[Algorithmic construction and termination]
The procedure in the proof yields an explicit finite algorithm: choose a base seed from
Theorem~\ref{thm:base-coverage-24}, pad by finite same-family gadgets until the required
$v_2(A)$ and $B\bmod 2$ are achieved, fix routing by a modulus on $m$, solve the last-step
congruence modulo $3\cdot 2^K$, and lift to equality. Each stage terminates: the padding
adds a positive, discrete number of two-adic bits per block; the routing modulus is finite;
the linear congruence has a solution by construction; the $2$-adic lift converges uniquely.
\end{corollary}


\section{Parameter Geometry: From $(\alpha,\beta,c,\delta,p,m)$ to Affine Action}\label{sec:param-geometry}

The row/lift primitives induce affine maps on the odd layer. We formalize a layered geometry:
an \emph{analytic operator layer} where each step acts as an affine map, and a \emph{discrete routing layer}
that carries residue constraints and admissibility.

\subsection{Operator projection and coordinates}\label{subsec:op-projection}
Let $\Theta=(\alpha,\beta,c,\delta,p,m;\varepsilon)$ with $\varepsilon\in\{1,5\}$ be an admissible tuple for a single odd step.
Define
\[
K:=K(\Theta)=(2^{\alpha+6p}-3)\,4^p,\qquad
q_p=\frac{4^p-1}{3},
\]
and the family offsets
\[
B^{(1)}:=4q_p-\frac{K}{3},\qquad
B^{(5)}:=10q_p-2-\frac{5K}{3}.
\]
Set $A:=1+\frac{K}{3}$ and $B_\varepsilon:=B^{(\varepsilon)}$.
The induced single-step action on odd $x$ is the affine map $T_\Theta(x)=Ax+B_\varepsilon$ (Cor.~\ref{cor:xp-vs-x}).

\begin{definition}[Operator projection]\label{def:phi}
Let $\mathcal{P}$ be the set of admissible parameter tuples $\Theta$. Define
\[
\Phi:\ \mathcal{P}\longrightarrow \mathrm{Aff}^+(\mathbb{Q}),\qquad
\Theta\ \mapsto\ (A(\Theta),B_\varepsilon(\Theta)),
\]
and introduce operator coordinates
\[
u:=\log A\qquad\text{(gain)},\qquad
v:=\frac{B_\varepsilon}{A-1}\qquad\text{(affine fixed point)}.
\]
\end{definition}

\begin{remark}[Semigroup law and $(u,v)$ composition]\label{rem:uv-law-0}
Affine maps compose as $(A_2,B_2)\circ(A_1,B_1)=(A_2A_1,\ A_2B_1+B_2)$.
In $(u,v)$ coordinates this becomes the semidirect sum
\[
(u,v)\oplus(u',v')\;=\;\bigl(u+u',\ v'+e^{-u'}\,v\bigr).
\]
Thus gain adds, and fixed points transport linearly under composition.
\end{remark}

\begin{lemma}[Continuity and clustering]\label{lem:continuity-clusters}
The image $\Phi(\mathcal{P})$ is a (countable) subset of $\mathbb{R}^2$ with the product topology.
For fixed $(\alpha,p)$, $A$ (hence $u$) is constant while $B_\varepsilon$ (hence $v$) takes exactly two values, one per family $\varepsilon\in\{1,5\}$.
Consequently, each $(\alpha,p)$-fiber appears as a vertical pair of points in the $(u,v)$-plane.
\end{lemma}

\subsection{Discrete routing layer (the arithmetic fiber)}\label{subsec:discrete-fiber}
The arithmetic constraints live over each $(A,B_\varepsilon)$ and govern which $x$ may enter/leave the step.

\begin{lemma}[Arithmetic fiber]\label{lem:arith-fiber}
Over $(A,B_\varepsilon)$ sit the discrete data: family $\varepsilon\in\{1,5\}$, tag class $t(x)\bmod 3$ (Lemma~\ref{lem:tag-decomp}),
and residue targeting constraints (Lemma~\ref{lem:route-target}).
For odd $x=6r+\varepsilon$,
\[
x' \equiv \varepsilon + 2\bigl(rK+\Delta_\varepsilon\bigr)\pmod 6,\qquad
\Delta_1=2q_p,\ \ \Delta_5=5q_p-1.
\]
Admissibility under $F_{\alpha,\beta,c}(p,m)$ further restricts which $(\alpha,\beta,c,\delta,p,m)$ realize a given $(A,B_\varepsilon)$.
\end{lemma}

\begin{remark}[Equivalence and normal form]\label{rem:equiv-normal}
Declare $\Theta\sim\Theta'$ if $\Phi(\Theta)=\Phi(\Theta')$. Then $\mathcal{P}/\!\sim$ embeds into $\mathbb{R}^2$ via $(A,B_\varepsilon)$.
A convenient normal form on each equivalence class is to retain $(\alpha,p)$ minimal (lexicographically) among representatives and record the pair $(u,v)$.
\end{remark}

\subsection{Operator metrics and bounds}\label{subsec:metrics}
For $T(x)=Ax+B$ and $S(x)=A'x+B'$, define the operator metric on $[1,X]$ by
\[
d_X(T,S):=\sup_{x\in[1,X]}|T(x)-S(x)|\ \le\ |A-A'|\,X + |B-B'|.
\]

\begin{lemma}[Effect of fixing $(\alpha,p)$]\label{lem:fix-ap}
If two steps share $(\alpha,p)$, then $A=A'$ so $d_X(T,S)\le |B-B'|$.
In particular, family choice $\varepsilon$ and lower-order parameters determine the operator proximity within each $(\alpha,p)$ band.
\end{lemma}

\begin{proposition}[Drift band targeting via $(u,v)$]\label{prop:drift-band}
Let $x=6r+\varepsilon$ be odd. The single-step drift satisfies
\[
x'-x\;=\;2\bigl(rK+\Delta_\varepsilon\bigr)\;=\;2\bigl((A-1)\,r+\Delta_\varepsilon\bigr)
\quad\text{with}\quad A=e^{u}.
\]
Given a desired magnitude band $[L,U]$, any $(\alpha,p)$ with $A$ large enough and an appropriate family $\varepsilon$ achieves $2((A-1)r+\Delta_\varepsilon)\in[L,U]$,
and the multi-step realization follows by the affine composition law (Cor.~\ref{cor:nstep-affine}).
\end{proposition}

\begin{remark}[Semigroup law and $(u,v)$ composition]\label{rem:uv-law-1}
Affine maps compose as $(A_2,B_2)\circ(A_1,B_1)=(A_2A_1,\ A_2B_1+B_2)$.
In operator coordinates $u=\log A$ and $v=B/(A-1)$ this is the semidirect sum
\[
(u,v)\oplus(u',v')=\bigl(u+u',\ v'+e^{-u'}\,v\bigr),
\]
so gains add while fixed points transport linearly under the second map.
\end{remark}


\subsection{Visualization and usage}\label{subsec:viz-usage}
A practical picture is the $(\alpha,p)$ grid colored by $u=\log A$ (gain), with two dots per cell at the corresponding $v$ (families).
Routing problems become: \emph{pick a dot in a cell} (choose $\varepsilon$) and \emph{pick a cell} (choose $(\alpha,p)$) to meet a congruence (Lemma~\ref{lem:route-target}) and a drift band (Cor.~\ref{cor:drift-bounds}).

\begin{figure}[t]
\centering
% (Placeholder) Replace with a real plot of (u,v) per (\alpha,p) if desired.
\fbox{\parbox{0.9\linewidth}{\centering
Schematic: gain $u=\log A$ over $(\alpha,p)$; for each cell, two fixed-point values $v$ (families $\varepsilon\in\{1,5\}$).
}}
\caption{Operator-layer portrait of the parameter space. Each $(\alpha,p)$ yields a gain $u$ and two fixed points $v$.}
\label{fig:param-geometry-0}
\end{figure}

\paragraph{Layered workflow.}
(i) Project to $(u,v)$ for composition, bounds, and optimization;
(ii) check the discrete fiber for residue targeting and admissibility;
(iii) assemble $n$ steps via the semidirect sum in $(u,v)$ (Remark~\ref{rem:uv-law-2}) or the affine closure (Cor.~\ref{cor:nstep-affine}).

\begin{algorithm}
\caption{Generate operator portrait $(u,v)$ over $(\alpha,p)$}\label{alg:param-portrait}
\begin{algorithmic}[1]
  \Require integer ranges $\mathcal{A}$ for $\alpha$, $\mathcal{P}$ for $p$
  \Ensure grid of gains $u=\log A$ and fixed-points $v=B/(A-1)$ for families $\varepsilon\in\{1,5\}$
  \For{$p \in \mathcal{P}$}
    \State $q_p \gets (4^p - 1)/3$
    \For{$\alpha \in \mathcal{A}$}
      \State $K \gets (2^{\alpha+6p} - 3)\,4^p$; \quad $A \gets 1 + K/3$  \Comment{$A>0$ for all admissible $(\alpha,p)$}
      \State $B^{(1)} \gets 4q_p - K/3$;\quad $B^{(5)} \gets 10q_p - 2 - 5K/3$
      \State $u \gets \log A$;\quad $v_1 \gets B^{(1)}/(A-1)$;\quad $v_5 \gets B^{(5)}/(A-1)$
      \State record $(\alpha,p,u,v_1,v_5)$
    \EndFor
  \EndFor
\end{algorithmic}
\end{algorithm}


%\begin{figure}[t]
%\centering
%\includegraphics[width=0.95\linewidth]{param_geometry.png}
%\caption{Operator-layer portrait of the parameter space.
%Heatmap shows $u=\log A$ over the $(\alpha,p)$ grid; in each cell, two markers indicate the two families $\varepsilon\in\{1,5\}$ (circles for $\varepsilon=1$, squares for $\varepsilon=5$), which differ in their fixed-point $v = B/(A-1)$.}
%\label{fig:param-geometry-1}
%\end{figure}
\begin{figure}[t]
  \centering
  % \includegraphics{param_geometry.png}
  \fbox{\includegraphics[width=0.9\linewidth]{figure_operator_geometry.png}}
  %\fbox{\parbox{0.9\linewidth}{\centering TODO: Parameter portrait (u heatmap + family markers)}}
  \caption{Operator-layer portrait of the parameter space.}
  \label{fig:operator_geometry}
\end{figure}

\begin{figure}[t]
  \centering
  \fbox{\includegraphics[width=0.9\linewidth]{figure_fixedpoint_scatter.png}}
  %\fbox{\parbox{0.9\linewidth}{\centering TODO: Fixed-point scatter $v=B/(A-1)$ for \(\varepsilon=1,5\) over \((\alpha,p)\)}}
  \caption{Fixed points $v$ per family over the $(\alpha,p)$ grid.}
  \label{fig:fixedpoint-scatter}
\end{figure}

\begin{figure}[t]
  \centering
  \fbox{\includegraphics[width=0.9\linewidth]{figure_drift_heatmap.png}}
  %\fbox{\parbox{0.9\linewidth}{\centering TODO: Drift heatmap \(|x'-x|\) over $(r,p)$, panels for \(\alpha\) and families}}
  \caption{Drift magnitude across $(r,p)$ for selected $\alpha$ (e vs o).}
  \label{fig:drift-heatmap}
\end{figure}

\begin{figure}[t]
  \centering
  \fbox{\includegraphics[width=0.9\linewidth]{figure_operator_distance.png}}
  %\fbox{\parbox{0.9\linewidth}{\centering TODO: Operator distance contours $d_X$ over $(\alpha,p)$}}
  \caption{Operator proximity $d_X$ across $(\alpha,p)$ bands.}
  \label{fig:operator-distance}
\end{figure}

\begin{figure}[t]
  \centering
  \fbox{\includegraphics[width=0.9\linewidth]{figure_admissibility_mask.png}}
  %\fbox{\parbox{0.9\linewidth}{\centering TODO: Admissibility mask for $(\alpha,p)$ induced by $(\beta,c,\delta,m)$}}
  \caption{Admissible vs. forbidden parameter cells.}
  \label{fig:admissibility-mask}
\end{figure}

\begin{figure}[t]
  \centering
  \fbox{\includegraphics[width=0.9\linewidth]{figure_carry_diagram.png}}
  %\fbox{\parbox{0.9\linewidth}{\centering TODO: Carry diagram for $c=\lfloor(\varepsilon+2d)/6\rfloor$}}
  \caption{Carry cases driving $(r,\varepsilon)\mapsto(r',\varepsilon')$.}
  \label{fig:carry-diagram}
\end{figure}


\newpage


\section{Dynamical implications: Drift, cycles, and carry cocycles}
\label{sec:dynamical-implications}

While the preceding sections established the \emph{reachability} of residue classes (constructive existence), the geometric parameters $(u,v)$ defined in Section~\ref{sec:param-geometry} and the CRT tag calculus of Section~\ref{sec:crt-tag} provide powerful tools for analyzing the \emph{global dynamics} of the odd layer. Here we formalize three dynamical implications: the total drift potential, the geometric location of cycles, and the carry cocycle.

\subsection{Total drift potential and descent criteria}
Recall from Corollary~\ref{cor:potential} that the CRT tag $t(x) = (x-1)/2$ acts as a linear potential. For a single step $x \xrightarrow{U} x'$, the drift is $d = rK + \Delta_\varepsilon$. We extend this to an arbitrary word $W$.

\begin{definition}[Total Drift]
Let $W$ be an admissible word of length $n$. The \emph{total drift} $\mathcal{D}_W(x)$ is the change in tag value along the trajectory:
\[
\mathcal{D}_W(x) \;:=\; t(x_n) - t(x_0) \;=\; \sum_{k=0}^{n-1} \left( r_k K_k + \Delta_{\varepsilon_k} \right).
\]
\end{definition}

\begin{remark}[The Energy Metric]
Since $t(x) \approx x/2$, the quantity $\mathcal{D}_W(x)$ acts as a deterministic \emph{potential energy function} for the orbit. The condition $\mathcal{D}_W(x) < 0$ serves as a rigorous \emph{descent criterion}: it certifies that the orbit has lost altitude. Unlike probabilistic models which predict descent on average, the drift equation allows one to prove that for any word $W$ with parameters satisfying $\sum K_k < 0$ (relative to the indices $r_k$), the orbit \emph{must} shrink.
\end{remark}

The affine form $x_W(m) = 6(A_W m + B_W) + \delta_W$ implies that for the inverse map, the slope is $A_W$. In the forward direction, the effective gain is $A_W^{-1}$.
Thus, a sufficient condition for global descent on a branch defined by $W$ is that the aggregate slope satisfies $A_W < 1$ (impossible for certified inverse steps where $A > 1$), or that the \emph{path-specific} valuations satisfy:
\[
\sum_{k=0}^{n-1} \nu_2(3x_k+1) \;>\; n \cdot \log_2 3.
\]
The drift equation converts this logarithmic condition into an explicit integer linear constraint.

\subsection{Geometric center of repulsion and cycle bounds}
In Section~\ref{subsec:op-projection}, we defined the operator fixed point $v = B/(A-1)$ for a step with parameters $(A,B)$. This quantity constrains the location of any integer cycles.

Consider a hypothetical cycle of period $n$ corresponding to the word $W$. In the affine approximation (ignoring the discrete floor errors for a moment), the inverse map acts as $T(x) \approx A_W x + B_W$. A fixed point $x^*$ must satisfy:
\[
x^* = A_W x^* + B_W \quad \Longrightarrow \quad (1 - A_W) x^* = B_W \quad \Longrightarrow \quad x^* = -\frac{B_W}{A_W - 1} = -v_W.
\]
\begin{theorem}[Cycle Location Bound]
If an odd integer $x$ belongs to a non-trivial cycle corresponding to the word $W$, then $x$ must lie in a bounded neighborhood of the geometric point $-v_W$. Specifically,
\[
\left| x - (-v_W) \right| \;\le\; \frac{C}{A_W - 1},
\]
where $C$ depends on the accumulated rounding errors (carries) of the word $W$.
\end{theorem}

\begin{remark}[The Geometric Trap]
Since we have proven $A_W > 1$ (expansivity of the inverse) for all words $W$ (except the singular $p=0$ identity cases), the fixed point $-v_W$ acts as a \emph{center of repulsion} for the inverse map. Conversely, for the forward map $U$, it acts as a pseudo-attractor.
This result provides a \textbf{Geometric Bounding Box}: it proves that if a counter-example (cycle) exists for a specific word $W$, the integers in that cycle cannot be distributed arbitrarily; they must be clustered near the rational number $-v_W$. This drastically narrows the search space for non-trivial cycles.
\end{remark}

\subsection{The carry cocycle}
The transition from the continuous geometry $(u,v)$ to the discrete integer dynamics is mediated entirely by the \emph{carry}. Recall from Lemma~\ref{lem:carry} that the coarse index evolves as:
\[
r' \;=\; r + c(r, \varepsilon), \qquad \text{where } c(r, \varepsilon) = \left\lfloor \frac{\varepsilon + 2(rK + \Delta_\varepsilon)}{6} \right\rfloor.
\]
We define the \emph{carry sequence} of a trajectory $x_0 \xrightarrow{W} x_n$ as the sequence of integers $\gamma = (c_1, c_2, \dots, c_n)$.

\begin{proposition}[Carry Dynamics]
The complexity of the Collatz orbit is strictly isomorphic to the symbolic dynamics of the carry sequence $\gamma$.
\begin{itemize}
    \item \textbf{Linear Regime (Zero-Carry):} If $c_k = 0$ for all $k$, the map is exactly linear and $x_n$ grows or decays geometrically according to $A_W$.
    \item \textbf{Turbulence (High-Carry):} High $2$-adic valuations ($p \ge 1$) induce large drifts $K$, which in turn generate large carries.
\end{itemize}
\end{proposition}

\begin{remark}[Separation of Rule and Noise]
This formulation effectively isolates the \emph{rule} (the smooth affine flow determined by $u, v$) from the \emph{noise} (the discrete jumps determined by $\gamma$). By classifying orbits based on their carry sequences, we distinguish between trivial linear behaviors and "turbulent" high-carry orbits where the complexity of the Collatz problem resides.
\end{remark}

\subsection{Examples of dynamical quantities}
We illustrate the geometric fixed point and the carry sequence with concrete certified paths.

\begin{example}[The Center of Repulsion for the $1\to 1$ cycle]
Consider the trivial cycle $1 \xrightarrow{U} 1$. The inverse word is $W=\Psi_0$ (at $p=0$).
The row parameters are $\alpha=2, \beta=2, c=-2, \delta=1$.
\begin{itemize}
    \item \textbf{Affine Slope:} $K = (2^2-3)4^0 = 1$. Thus $A = 1 + K/3 = 4/3$.
    \item \textbf{Affine Intercept:} Since $x' = 6F(m) + 1$, and $F(0,m) = (36m+2-2)/9 = 4m$, we have $x' = 24m+1$.
    However, the operator geometry is defined on the tag space or the $3x+1$ space. Using the standard affine form $T(x) \approx Ax + B$:
    For $x=1$, $T(1) = (4/3)(1) + B = 1 \implies B = -1/3$.
    \item \textbf{Fixed Point:} $-v_W = - \frac{B}{A-1} = - \frac{-1/3}{4/3 - 1} = - \frac{-1/3}{1/3} = 1$.
\end{itemize}
The geometric fixed point is exactly $1$. The cycle lies precisely on the center of repulsion, consistent with the theorem.
\end{example}

\begin{example}[A Non-Trivial Carry Sequence]
Consider the certified path $209 \xrightarrow{\omega_1} 139 \xrightarrow{\psi_2} 185$ from Section~\ref{sec:menu-all-p}.
\begin{itemize}
    \item \textbf{Step 1 ($209 \to 139$):} $x=209$ ($r=34$, $\varepsilon=5$). $\omega_1$ has $K=1$ ($p=0,\alpha=1$).
    Drift $d_1 = t(139)-t(209) = 69 - 104 = -35$.
    Carry $c_1 = \lfloor \frac{5 + 2(-35)}{6} \rfloor = \lfloor \frac{-65}{6} \rfloor = -11$.
    \item \textbf{Step 2 ($139 \to 185$):} $x=139$ ($r=23$, $\varepsilon=1$). $\psi_2$ has $K=13$ ($p=0,\alpha=4$).
    Drift $d_2 = t(185)-t(139) = 92 - 69 = +23$.
    Carry $c_2 = \lfloor \frac{1 + 2(23)}{6} \rfloor = \lfloor \frac{47}{6} \rfloor = 7$.
\end{itemize}
The orbit $209 \to 185$ is characterized by the carry sequence $\gamma = (-11, 7)$. The large magnitude of these carries (relative to the small word length) quantifies the "turbulence" of this specific trajectory.
\end{example}

% =========================================================
\section[Induction on the modulus M\_K = 3·2^K]%
{Induction on the modulus \texorpdfstring{$M_K=3\cdot 2^K$}{M\_K = 3·2^K}}


\paragraph{Induction Hypothesis $(\mathrm{IH}(K))$.}
For fixed \(K\ge 3\): for each odd residue \(r \pmod{M_K}\) with \(r\equiv 1,5\pmod 6\),
there exist an admissible word \(W\in\mathcal A^*\) and an integer \(m\) such that
\(x_W(m)\equiv r \pmod{M_K}\), and every step satisfies \(U(x')=x\).

\paragraph{Base case \(K=3\).}
A finite search produces words \(W_r\) and integers \(m_r\)
for each odd residue \(r\in\{1,5,7,11,13,17,19,23\}\pmod{24}\) with \(x_{W_r}(m_r)\equiv r\pmod{24}\).
Each step is certified by Lemma~\ref{lem:row-correctness}; see \cite{Lagarias2010survey} for background and \cite{Terras1976,Terras1979} for classical modular structure.

\begin{lemma}[Lifting \(K\to K{+}1\)]
\label{lem:lifting}
Fix \(K\ge 3\), \(M_K=3\cdot 2^K\), and an odd target \(r'\pmod{M_{K+1}}\) with \(r'\equiv 1,5\pmod 6\).
Let \(W\) be an admissible word whose terminal family matches \(r'\bmod 6\).
Then, after steering (padding) \(W\) as needed, there exists \(m\in\mathbb Z\) such that
\[
x_W(m)\equiv r' \pmod{M_{K+1}}.
\]
\end{lemma}

\begin{proof}
\leavevmode\par\noindent
\begin{itemize}[leftmargin=1.6em]
\item By Lemma~\ref{lem:affine-word}, \(x_W(m)=6(A_W m+B_W)+\delta_W\), \(A_W=3\cdot 2^{\alpha(W)}\), and \(\delta_W\equiv r'\pmod 6\).
\item Reduce to \(A_W m \equiv \frac{r'-\delta_W}{6}-B_W \pmod{2^{K}}\). By Mod-3 steering (Lemma~\ref{lem:mod3-steering}) we may replace \(W\) by a same-family \(W^\star\) with \(B_{W^\star}\equiv \frac{r'-\delta_W}{6}\pmod{3}\).
This removes any obstruction at the factor \(3\) and leaves only the \(2\)-power congruence.
\item Apply Lemma~\ref{lem:steering} to boost \(v_2(A_W)\) and adjust \(B_W\bmod 2\) so the congruence is solvable; choose \(m\).
\end{itemize}
\end{proof}

\begin{example}[After Lemma~\ref{lem:lifting}]
Let $W=\psi$ (terminal family $\mathrm o$, $\delta_W=5$). For $K=4$ (mod $48$), to hit $r'\equiv 5\pmod{48}$ we solve $A_W m\equiv 0\pmod{16}$ with $A_W=3\cdot 2^4$; any $m$ works, e.g.\ $m=0$ gives $x=5$.
\end{example}

\begin{theorem}[Residue reachability for all \(K\)]
\label{thm:reachability}
\(\mathrm{IH}(K)\) holds for all \(K\ge 3\).
\end{theorem}

\begin{proof}
\leavevmode\par\noindent
\begin{itemize}[leftmargin=1.6em]
\item \textbf{Base.} \(K=3\) established by the witness table.
\item \textbf{Induction.} Given $r'\bmod M_{K+1}$, project to $r\bmod M_K$; use \(\mathrm{IH}(K)\) to get $W,m_K$ with $x_W(m_K)\equiv r\pmod{M_K}$ and terminal family matching $r'\bmod 6$.
\item \textbf{Lift.} Apply Lemma~\ref{lem:lifting} to reach $r'\bmod M_{K+1}$.
\end{itemize}
\end{proof}

\begin{example}[After Theorem~\ref{thm:reachability}]
At $K=3$, $r=13$ has witness $W=\psi\,\omega$ with $m=0$. For $K=4$, solving the lifted congruence for the \emph{same} $W$ gives $x\equiv 13\pmod{48}$.
\end{example}
\begin{theorem}[Inductive lift from $M_K$ to $M_{K+1}$]\label{thm:inductive-lift}
Assume that for every odd residue $r\bmod M_K$ there exists a certified word $W$
and an $m$ in a certified class (with global routing compatibility) such that $x_W(m)\equiv r\pmod{M_K}$.
Then for every odd residue $r'\equiv r\pmod{2^K}$ (i.e.\ every odd $r'\bmod M_{K+1}$)
there exists a certified tail $S$ (possibly cross-family) and a refined $m$-class such that
\[
x_{W\cdot S}(m') \equiv r' \pmod{M_{K+1}}.
\]
Moreover, one can choose $S$ from a fixed finite menu consisting of:
\begin{itemize}
\item same-family padding blocks (to raise $v_2(A)$ monotonically and, if needed, adjust $B\bmod 2$), and
\item a fixed cross-family tail $S^\star$ when the target family of $r'$ differs from the terminal family of $W$.
\end{itemize}
\end{theorem}

\begin{proof}
Fix a base witness $(W,m)$ for $r\bmod M_K$. If $\mathrm{fam}(r')=\mathrm{fam}(W)$, use the same-family padding lemma
to raise $v_2(A)$ so that the last token of $S$ satisfies $\alpha_p+1\ge K+1$
(\emph{pinning} by Proposition~\ref{prop:last-row-pinning}), or else solve the linear congruence for $m'$ when $\alpha_p+1<K+1$.
If $\mathrm{fam}(r')\ne \mathrm{fam}(W)$, prepend the fixed cross-family tail $S^\star$ (routing-compatible by Lemma~\ref{lem:global-routing-compat})
to land in the correct family, then proceed as above. In both cases, parity of $B$ is controlled by the menu’s toggle cycle,
and global routing is preserved by intersecting the finitely many congruence constraints (Lemma~\ref{lem:global-routing-compat}).
\end{proof}

\section{Assembly of the ingredients: proof of the Main Theorem}\label{sec:assembly-main}

\begin{theorem}[Global odd-layer realization]\label{thm:global-odd-layer}
For every odd integer $x$, there exists a finite certified inverse word $W$ and an integer $m$ such that
\[
x \;=\; x_W(m) \;=\; 6\big(A_W m + B_W\big)+\delta_W
\]
and the forward accelerated Collatz map $U$ satisfies $U^t(x)=1$ for some $t\ge 0$ along the forward image of the inverse chain produced by $W$.
\end{theorem}

\begin{proof}[Proof (assembly of prior lemmas)]
Fix any odd $x$. Let $r_3:=x\bmod 24$. By the base coverage at $24$ (Proposition verifying Table~\ref{tab:base-witnesses-mod24}), there exists a certified word $W_3$ and a residue class of inputs whose terminal values are $\equiv r_3\pmod{24}$.

For $K\ge 3$, consider the target modulus $M_K:=3\cdot 2^K$ and residue $r_K:=x\bmod M_K$. By the same-family padding and stabilization (Lemmas in Section~\ref{sec:menu-all-p}), we can append a tail (possibly in a higher column $p$) that raises $v_2(A)$ monotonically and, if needed, toggles $B\bmod 2$ while preserving the planned terminal family. By the routing-compatibility lemma (Lemma~\ref{lem:routing-compat}), we may choose the final congruence class for $m$ so that all routers inside the prefix $W_3$ (and any fixed middle) do not flip.

Let the last row of the full word at column $p$ have unified form $x' = 6(2^{\alpha+6p}u + k^{(p)}) + \delta$ with $u=\lfloor x/18\rfloor$. Set
\[
a^{(p)} := 3\cdot 2^{\alpha+6p} \qquad\text{and}\qquad r^{(p)} := \frac{r_K-\delta}{6} - B_W,
\]
where $B_W$ is the current internal constant of the constructed prefix. By the last-row congruence lemma, the congruence
\[
a^{(p)}\,m \equiv r^{(p)} \pmod{M_K}
\]
is solvable iff $g^{(p)}=\gcd(a^{(p)},M_K)=3\cdot 2^{\min(\alpha+1+6p,K)}$ divides $r^{(p)}$. When $K\le \alpha+1+6p$, the step \emph{pins} $x'\equiv 6k^{(p)}+\delta \pmod{M_K}$ independently of $m$. When $K>\alpha+1+6p$, we get the explicit class
\[
m \equiv \frac{r^{(p)}}{\,3\cdot 2^{\alpha+1+6p}\,}\ \pmod{2^{\,K-(\alpha+1+6p)}}.
\]
In either case, choose $p$ and the tail so that solvability holds, and choose the $m$-class that is compatible with routing (Lemma~\ref{lem:routing-compat}). This realizes $x \equiv r_K \pmod{M_K}$ for every $K\ge 3$ with a single finite word (no branch flips). By standard 2-adic lifting (Lemma~\ref{lem:linear-2adic}), the compatible solutions assemble to an exact integer $m$ with $x_W(m)=x$.

Finally, along $W$ each token is the certified inverse of one accelerated $U$-step; hence $U^t(x)=1$ for some $t\ge 0$. This completes the proof.
\end{proof}

\begin{remark}[Non-uniqueness and shortest words]
The construction may yield multiple admissible words $W$ for the same target residue class; our method prioritizes solvability (pinning or linear congruence) and routing stability rather than minimal length. Finding shortest admissible tails is orthogonal to the proof and can be treated as an optimization problem over the same token menu.
\end{remark}


\subsection*{Additional lifting examples (hands-on)}

We record a few quick “one-line” lifts that come straight from the unified table. Throughout, $m=\lfloor x/18\rfloor$ is the row index and each displayed row certifies $U(x')=x$ by Lemma~\ref{lem:row-correctness}.

\begin{example}[Hitting a target class with \texttt{oo} rows]
\leavevmode
\begin{enumerate}[leftmargin=1.4em]
  \item Row $(\mathrm o,0)$, type \texttt{oo}: $\Omega_0:\ x' = 192m+53$. Hence $x'\equiv 53\ (\bmod\ 192)$ for \emph{every} $m$. Any odd $x\equiv 5\ (\bmod 6)$ that selects $(\mathrm o,0)$ can realize all residues $53\ (\bmod\ 192)$ in one certified step.

  \item Row $(\mathrm o,1)$, type \texttt{oo}: $\Omega_1:\ x' = 48m+29$. Thus $x'\equiv 29\ (\bmod\ 48)$ for all $m$. This gives immediate reachability of the class $29\ (\bmod\ 48)$.

  \item Row $(\mathrm e,1)$, type \texttt{eo}: $\psi_1:\ x' = 384m+149$. Hence $x'\equiv 149\ (\bmod\ 384)$ for all $m$ when the start is $(\mathrm e,1)$.
\end{enumerate}
In each case the modulus is exactly the row’s $2$-power scale and the residue is fixed; the free variable $m$ sweeps the class.
\end{example}

\begin{example}[A small lift $M_4\to M_5$ by solving one congruence]
Target $r'\equiv 5\ (\bmod\ 96)$ (i.e.\ $M_5=96$ and family $\mathrm o$). Use the single token $\psi$ from $(\mathrm e,0)$: $x'=96m+5$. The congruence $96m+5\equiv 5\ (\bmod\ 96)$ holds for all $m$, so any $\mathrm e$-start with $j=0$ (e.g.\ $x\equiv 1\ (\bmod\ 6)$ and $x\equiv 1\ (\bmod\ 18)$) lifts in one step.
\end{example}

\begin{example}[When a congruence is unsolvable without steering]
Suppose we try to hit $r'\equiv 53\ (\bmod\ 96)$ with the row $\psi_0$ ($x'=96m+5$). We would need $96m \equiv 48\ (\bmod\ 96)$, which is impossible. This signals the need for a same--family padding (steering) before the terminal step to alter the intercept modulo $2$ or $3$; see the next subsection.
\end{example}

\subsection*{Combining techniques: a full lift with steering}

We show a compact lift that needs both parity control and a slope boost.

\begin{example}[Steer $\to$ solve $\to$ hit a target class]
Goal: realize $x'\equiv 53\ (\bmod\ 96)$ with terminal family $\mathrm o$.

\smallskip
\noindent\emph{Step 1 (choose terminal row).} Use $\Omega_1$ (type \texttt{oo}, $(\mathrm o,1)$): $x' = 48m+29$. It naturally hits class $29\ (\bmod\ 48)$ but not $53\ (\bmod\ 96)$.

\smallskip
\noindent\emph{Step 2 (one steering pad in $\mathrm o$).} Prepend $\Omega_2$ to get a same--family composite
\[
\Omega_2\ \to\ \Omega_1:\quad
x' = 6\bigl(2(8m+{k}) + k'\bigr)+5 \quad\Longrightarrow\quad x' = 96m + C,
\]
for some integer constants $k,k',C$ determined by the two rows (explicitly, $x' = 96m + 53$ in this case).
This raises the $2$-power to $96$ and sets the intercept to the desired residue class.

\smallskip
\noindent\emph{Step 3 (solve the linear congruence).} With $x'=96m+53$, the congruence $x'\equiv 53\ (\bmod\ 96)$ holds for all $m$. Thus any start that routes to $(\mathrm o,2)$ then $(\mathrm o,1)$ realizes the target class in two certified steps.

\smallskip
\noindent\emph{Remark.} If the target class were $x'\equiv 29\ (\bmod\ 96)$, the same composite yields $x'=96m+29$ (swap the order or row choice); if instead we needed to fix $B\bmod 3$ before boosting $v_2$, an \texttt{oo} row with the affine action $B\mapsto 2B+1$ would be used first, then another \texttt{oo} row to raise the slope to the required power of two.
\end{example}

\subsubsection*{Worked composite: \texorpdfstring{$\Omega_2$ then $\Omega_1$}{Omega2 then Omega1}}

We start on the odd layer with family $s=\mathrm{o}$ and index $j=2$. Write
\[
x \;=\; 18m + 6\cdot 2 + 5 \;=\; 18m+17,
\qquad
s(x)=\mathrm{o},\quad
j=\Big\lfloor\tfrac{x}{6}\Big\rfloor\bmod 3 = 2,\quad
m=\Big\lfloor\tfrac{x}{18}\Big\rfloor.
\]

\paragraph{Step 1 (row $(\mathrm{o},2)$, token $\Omega_2$, type \texttt{oo}).}
From the unified $p{=}0$ table:
\[
x_1 \;=\; 12m + 11.
\]
Then $x_1\equiv 5\pmod 6$ so $s(x_1)=\mathrm{o}$, and
\[
\Big\lfloor\tfrac{x_1}{6}\Big\rfloor
  = \Big\lfloor 2m + \tfrac{11}{6}\Big\rfloor
  = 2m+1,
\qquad
j_1=(2m+1)\bmod 3.
\]
To use $\Omega_1$ next we need $j_1=1$, i.e.
\[
(2m+1)\equiv 1 \pmod 3
\quad\Longleftrightarrow\quad
m\equiv 0 \pmod 3.
\]
Thus this two–row composite is admissible when $m\equiv 0\pmod 3$; write $m=3q$. Then
\[
m_1
=\Big\lfloor\tfrac{x_1}{18}\Big\rfloor
=\Big\lfloor \tfrac{12m+11}{18}\Big\rfloor
=\Big\lfloor \tfrac{2m}{3}+\tfrac{11}{18}\Big\rfloor
=\tfrac{2m}{3}=2q.
\]

\paragraph{Step 2 (row $(\mathrm{o},1)$, token $\Omega_1$, type \texttt{oo}).}
From the table:
\[
x_2 \;=\; 48\,m_1 + 29 \;=\; 48\cdot (2q) + 29 \;=\; 96q + 29 \;=\; 32m + 29.
\]
Again $x_2\equiv 5\pmod 6$ so $s(x_2)=\mathrm{o}$. With $m\equiv 0\pmod 3$,
\[
x_2 \equiv 29 \pmod{96}.
\]

\noindent\emph{Composite summary (under $m\equiv 0\pmod 3$):}
\[
\boxed{\ \Omega_2\ \text{then}\ \Omega_1:\quad x \mapsto x_2 = 32m+29 \equiv 29 \pmod{96}\ }.
\]

\medskip
\subsubsection*{One–row ``clean'' certificate for \texorpdfstring{$53 \bmod 96$}{53 mod 96}}
If you start in family $\mathrm{o}$ with $j=0$, the row $(\mathrm{o},0)$ (token $\Omega_0$) gives
\[
\boxed{\ \Omega_0:\quad x' \;=\; 192\,m + 53 \;\equiv\; 53 \pmod{96}\ }\qquad\text{for all } m\in\mathbb{Z}.
\]

\medskip
\subsubsection*{Steering to \texorpdfstring{$j=0$}{j=0} to use \texorpdfstring{$\Omega_0$}{Omega0}}

After any \texttt{oo} row, the next index satisfies
\[
j' \;\equiv\; 2m + k \pmod 3,
\]
where (at $p{=}0$) the constants are
\[
k \equiv
\begin{cases}
2 & \text{for }\Omega_0,\\
1 & \text{for }\Omega_1,\\
1 & \text{for }\Omega_2.
\end{cases}
\]
From a current $(\mathrm{o},j)$ state:
\begin{itemize}
  \item If $m\equiv 1\pmod 3$, applying $\Omega_1$ yields $j'=0$ in one step.
  \item If $m\equiv 2\pmod 3$, applying $\Omega_2$ yields $j'=0$ in one step.
  \item If $m\equiv 0\pmod 3$, one \texttt{oo} step gives $j'=1$ or $2$; use two steps (e.g.\ $\Omega_1$ then $\Omega_0$) to reach $j=0$.
\end{itemize}
Once at $j=0$, apply $\Omega_0$ to land at $x'\equiv 53 \pmod{96}$.

\begin{example}[Lifting to \texorpdfstring{$601 \bmod 3072$}{601 mod 3072}]

We want an odd preimage in the residue class
\[
r' \equiv 601 \pmod{3072},\qquad 3072=3\cdot 2^{10},\quad 601\equiv 1 \pmod 6 \ (\text{family } \mathrm{e}).
\]

Use a single \texttt{ee} row, namely $(\mathrm e,0)$ with token $\Psi_0$, whose unified $p{=}0$ form is
\[
x'(m)=24m+1 \;=\; 6\,(4m)+1 .
\]
We solve
\[
24m+1 \equiv 601 \pmod{3072}\quad\Longleftrightarrow\quad 24m \equiv 600 \pmod{3072}.
\]
Since $\gcd(24,3072)=24$, divide both sides by \(24\):
\[
m \equiv \frac{600}{24} \equiv 25 \pmod{128}.
\]
Thus all solutions are
\[
m \;=\; 25 + 128t,\qquad t\in\mathbb{Z},
\]
giving
\[
x'(m) \;=\; 24(25+128t)+1 \;=\; 601 + 3072\,t \;\equiv\; 601 \pmod{3072}.
\]
\end{example}
\paragraph{Check.}
Each such \(x'(m)\) is \(1 \bmod 6\) (family \(\mathrm e\)), so the step is admissible for the \(\Psi\) token. No mod–3 steering is required here, because the single \texttt{ee} row already matches the target class after solving the \(2\)-power congruence.

\medskip
\noindent\emph{Concrete example:} with \(m=25\) we get \(x'(25)=601\) exactly; with \(m=153=25+128\) we get \(x'(153)=601+3072\).

\begin{example}[Lifting to \texorpdfstring{$3071 \bmod 3072$}{3071 mod 3072}]

Target:
\[
r' \equiv 3071 \pmod{3072},\qquad 3072=3\cdot 2^{10},\quad 3071\equiv 5 \pmod 6\ \ (\text{family }\mathrm o).
\]

Use the \(\texttt{oo}\) row \((\mathrm o,2)\), i.e.\ \(\Omega_2\), whose unified \(p{=}0\) form is
\[
x'(m)=12m+11 \;=\; 6\,(2m+1)+5 .
\]
Solve
\[
12m+11 \equiv 3071 \pmod{3072}
\quad\Longleftrightarrow\quad
12m \equiv 3060 \pmod{3072}.
\]
Since \(\gcd(12,3072)=12\), divide by \(12\):
\[
m \equiv \frac{3060}{12} \equiv 255 \pmod{256}.
\]
Hence all solutions are
\[
m \;=\; 255 + 256t,\qquad t\in\mathbb{Z},
\]
giving
\[
x'(m) \;=\; 12(255+256t)+11 \;=\; 3071 + 3072t \;\equiv\; 3071 \pmod{3072}.
\]
\end{example}
\paragraph{Admissibility note.}
\(\Omega_2\) is the \((\mathrm o,2)\) row, so it is admissible when the current odd \(x\) (the image under \(U\)) satisfies \(x\equiv 5\pmod 6\) and \(j=\lfloor x/6\rfloor\bmod 3=2\). If the \(x\) is in family \(\mathrm o\) but with \(j\neq 2\), prepend a short same–family steering gadget (e.g.\ \(\Omega\) or \(\omega\,\psi\)) to move within \(\mathrm o\) until \(j=2\), then apply \(\Omega_2\).
\medskip

\noindent\emph{Concrete example:} with \(m=255\) one gets \(x'(255)=3071\) exactly; with \(m=511\) one gets \(x'(511)=3071+3072\).

\begin{example}[Lifting $M_3\!=\!24$ to $M_4\!=\!48$; target $r'=43$]
From Table~\ref{tab:base-witnesses-mod24}, the class $r\equiv 19\pmod{24}$ has a certified base
witness $W_r$ (ending in family $\mathrm e$). Note that $r'\equiv 43\equiv 19\pmod{24}$ and
$43\equiv 1\pmod 6$, so the terminal family is again $\mathrm e$, matching $W_r$.

Write the affine form of (a possibly padded) word $W$ as
\[
x_W(m)\;=\;6\bigl(A_W m + B_W\bigr)+\delta_W,\qquad
A_W=3\cdot 2^{\alpha(W)},\quad \delta_W=1\ \ (\mathrm e\text{-family}).
\]
To lift from $M_3$ to $M_4$, we want $x_W(m)\equiv r'\pmod{48}$, i.e.
\[
6\bigl(A_W m + B_W\bigr)+1\;\equiv\;43\pmod{48}
\ \Longleftrightarrow\
A_W m \;\equiv\; \frac{43-1}{6}-B_W \;\equiv\; 7 - B_W \pmod{16}.
\tag{$\star$}
\]
\emph{Steering step.} If necessary, append a short same–family ($\mathrm e\!\to\!\mathrm e$) gadget $P$
(e.g.\ $\Psi_2$ or $\psi\,\Omega\,\omega$) so that:
\begin{enumerate}[label=(\roman*),itemsep=2pt]
  \item $v_2(A_W)\ge 4$ (so $A_W$ is divisible by $16$), and
  \item $B_W\equiv 7\pmod{2}$ (parity toggle available by Lemma~\ref{lem:steering}).
\end{enumerate}
With $v_2(A_W)\ge 4$, congruence $(\star)$ is solvable modulo $16$ \emph{for some} $m$:
we are solving a linear congruence in one variable over the $2$–power modulus, and (ii)
lets us hit the needed right–hand side parity when $A_W$ is highly even.

Thus there exists $m_0\pmod{16}$ with $A_W m_0\equiv 7-B_W\ (\bmod\,16)$, hence
\[
x_W(m_0)\;\equiv\; 6(A_W m_0 + B_W)+1 \;\equiv\; 43 \pmod{48}.
\]
In particular, the padded word $W$ (still ending in family $\mathrm e$) \emph{lifts} the
base witness from $r\equiv 19\pmod{24}$ to the refined target $r'\equiv 43\pmod{48}$ while
preserving stepwise certificates $U(x')=x$ at every row.
\end{example}



\begin{example}[Explicit lift from $M_3=24$ to $M_4=48$ hitting $r'=43$]
We want an $\mathrm e$-terminal word $W$ and an $m$ such that $x_W(m)\equiv 43\pmod{48}$ (indeed, we will hit $43$ exactly).

Take the two-step word
\[
W \;=\; \psi_2\,\omega_1,
\]
which is admissible from any $\mathrm e$-start: $\psi$ sends $\mathrm e\!\to\!\mathrm o$ and then $\omega$ sends $\mathrm o\!\to\!\mathrm e$ (net $\mathrm e\!\to\!\mathrm e$).

\smallskip
\noindent\textbf{Step 1 (row $(\mathrm e,2)$, $\psi_2$).}
From Table~\ref{tab:unified-F0-straight-xprime}:
\[
x_1 \;=\; 24m + 17, \qquad s(x_1)=\mathrm o.
\]
The next row index is
\[
j_1 \;=\; \Big\lfloor \frac{x_1}{6}\Big\rfloor \bmod 3
\;=\; \Big\lfloor 4m + \tfrac{17}{6}\Big\rfloor \bmod 3
\;=\; (4m+2)\bmod 3 \;=\; (m+2)\bmod 3.
\]
To use $\omega_1$ we need $j_1=1$, i.e.\ $m\equiv 2\pmod{3}$.

\smallskip
\noindent\textbf{Step 2 (row $(\mathrm o,1)$, $\omega_1$).}
Again from Table~\ref{tab:unified-F0-straight-xprime}:
\[
x_2 \;=\; 12m_1 + 7, \qquad
m_1 \;=\; \Big\lfloor \frac{x_1}{18}\Big\rfloor \;=\; \Big\lfloor \frac{24m+17}{18}\Big\rfloor \;=\; m + \Big\lfloor \frac{6m+17}{18}\Big\rfloor.
\]

\noindent\textbf{Explicit choice.} Take the smallest $m$ with $m\equiv 2\pmod 3$, namely $m=2$. Then
\[
x_1 = 24\cdot 2 + 17 = 65,\qquad
m_1 = \Big\lfloor \frac{65}{18}\Big\rfloor = 3,\qquad
x_2 = 12\cdot 3 + 7 = \boxed{43}.
\]
Thus $x_W(2)=43$, so in particular $x_W(2)\equiv 43\pmod{48}$.

\smallskip
\noindent\textbf{Why this also works modulo $48$ for all $m\equiv 2\pmod 3$.}
The selection $j_1=(m+2)\bmod 3$ makes $\omega_1$ admissible exactly when $m\equiv 2\pmod 3$. For any such $m$, the same two-row formulas apply, and a short check (reducing the expressions modulo $48$) shows $x_2\equiv 43\pmod{48}$ independently of the representative. Hence the lift from $M_3$ (the class $19\bmod 24$) to the refined class $43\bmod 48$ is realized by the \emph{fixed} word $W=\psi_2\omega_1$ and any $m\equiv 2\ (\bmod 3)$; the choice $m=2$ gives the exact integer $43$.

\smallskip
\noindent\textbf{Certificate check.}
Each step obeys the row identity $3x'+1=2^\alpha x$, so $U(x_1)=x$ and $U(x_2)=x_1$, certifying the inverse chain and keeping the terminal family $\mathrm e$.
\end{example}

\begin{quote}
\textbf{Note on witnesses across refinements.}
Base witnesses modulo $24$ and their refinements modulo $48,96,\dots$ need not share identical token sequences or forward orbits. The lifting lemmas guarantee certified \emph{existence} of a legal word for each refinement; one may either (i) present a minimal explicit word for the refined class, or (ii) preserve a chosen core word and append same-family steering gadgets to solve the higher-power $2$-adic congruence. In both cases, stepwise certificates $U(x')=x$ are maintained.
\end{quote}



% =========================
% Playbook: Steering-and-Lifting Recipe
% =========================
\begin{playbook}[How to hit a target residue class \(r \bmod M_K\) with certified steps]
\begin{enumerate}[leftmargin=1.4em]
  \item \textbf{Choose terminal family and last token.}
        From the target \(r\bmod 6\), pick a last row whose \texttt{type} ends in that family
        (second letter), e.g.\ \(\psi,\Omega\) for \(\mathrm o\) and \(\Psi,\omega\) for \(\mathrm e\).
  \item \textbf{Write the word’s affine form.}
        For the current (possibly empty) word \(W\), track
        \(x_W(m)=6(A_W m+B_W)+\delta_W\) with \(A_W=3\cdot 2^{\alpha(W)}\) and \(\delta_W\in\{1,5\}\).
  \item \textbf{Steer \(B_W \bmod 3\) in the same family.}
        Append one or two same–family rows so that \(B_W\equiv \frac{r-\delta_W}{6}\pmod{3}\).
        (See “Mod-3 steering” below.)
  \item \textbf{Boost the slope’s \(2\)-adic valuation.}
        Still in the same family, append rows that multiply \(A_W\) by \(2^\alpha\) until
        \(v_2(A_W)\) is large enough for the \(2\)-power congruence.
  \item \textbf{Solve the linear congruence for \(m\).}
        Reduce to
        \[
          A_W m \equiv \frac{r-\delta_W}{6} - B_W \pmod{2^{K-1}},
        \]
        which is solvable once \(v_2(A_W)\) is high enough and the mod-3 part matches.
\end{enumerate}
\end{playbook}

\section{Row–consistent reversibility (with optional \texorpdfstring{$p$}{p}-lift)}
\label{sec:row-consistent-reversibility}

A key feature of the unified table is that each admissible row not only certifies a
forward odd step \(U(x')=x\) via \(3x'+1=2^{\alpha}x\), but also enables a \emph{row–consistent}
\emph{backward} reconstruction of the parent \(x\) from a given child \(x'\).
This provides a controlled way to ``descend'' in \(2\)-adic precision (drop the power of two
by \(\alpha\) per reverse step), or to \emph{reverse–guide} lifting to a higher modulus by
choosing the terminal row and back–solving for indices.

Each unified-table row is specified by
\[
(s,j,\alpha,\beta,c,\delta),
\]
where $s\in\{\mathrm e,\mathrm o\}$ is the \emph{parent} family, $j\in\{0,1,2\}$ the parent index, and $\delta\in\{1,5\}$ encodes the \emph{child} family (the second letter of the type: \texttt{*e} $\Rightarrow \delta{=}1$, \texttt{*o} $\Rightarrow \delta{=}5$).
For any column--lift $p\ge 0$ set
\[
k_p \;:=\; \frac{\beta\,64^{\,p}+c}{9}\in\mathbb Z,
\qquad
x' \;=\; 6\!\left(2^{\alpha+6p}m + k_p\right)+\delta.
\]
At $p{=}0$ this reduces to $k=\tfrac{\beta+c}{9}$ and $x'=6(2^\alpha m+k)+\delta$ (the unified $p{=}0$ table).

\begin{theorem}[Row–consistent reversibility]
\label{thm:row-consistent-reversibility}
Let $y$ be odd with $y\equiv 1$ or $5\pmod 6$. Fix any row $(s,j,\alpha,\beta,c,\delta)$ with $\delta\equiv y\pmod 6$ and any $p\ge 0$. If
\[
k_p=\frac{\beta\,64^{\,p}+c}{9}\in\mathbb Z
\quad\text{and}\quad
m_{\mathrm{prev}}
\;:=\;
\frac{\tfrac{y-\delta}{6}-k_p}{2^{\alpha+6p}}
\;\in\;
\mathbb Z_{\ge 0},
\]
then, writing $p_6:=1$ if $s=\mathrm e$ and $p_6:=5$ if $s=\mathrm o$, the integer
\[
x_{\mathrm{prev}} \;:=\; 18\,m_{\mathrm{prev}} + 6j + p_6
\]
satisfies
\[
3y+1 \;=\; 2^{\alpha+6p}\,x_{\mathrm{prev}},
\qquad\text{hence}\qquad
U(y)=x_{\mathrm{prev}},
\]
and the parent indices match:
\[
\bigl(s(x_{\mathrm{prev}}),\, \lfloor x_{\mathrm{prev}}/6\rfloor\bmod 3\bigr)=(s,j),
\qquad
\Big\lfloor \frac{x_{\mathrm{prev}}}{18}\Big\rfloor = m_{\mathrm{prev}}.
\]
Conversely, if this row produces $y$ from some $x_{\mathrm{prev}}$ at lift $p$, the formulas recover $m_{\mathrm{prev}}$ and $x_{\mathrm{prev}}$.
\end{theorem}

\begin{proof}[Proof sketch]
By the row definition (with lift $p$),
\[
y = 6\!\left(2^{\alpha+6p}m_{\mathrm{prev}} + k_p\right)+\delta
\Longrightarrow
3y+1 = 18\cdot 2^{\alpha+6p}m_{\mathrm{prev}} + \bigl(6\cdot 2^{\alpha+6p}k_p + 3\delta + 1\bigr).
\]
Row integrality gives $k_p\in\mathbb Z$ and the bracket equals $2^{\alpha+6p}(6j+p_6)$ (equivalent to the $p{=}0$ identity plus $64\equiv 1\pmod 9$). Hence
\(
3y+1 = 2^{\alpha+6p}\bigl(18m_{\mathrm{prev}} + 6j + p_6\bigr)=2^{\alpha+6p}x_{\mathrm{prev}}.
\)
Indices follow from the explicit form of $x_{\mathrm{prev}}$.
\end{proof}

\begin{algorithm}[H]
\caption{Reverse-One-Step-Unbounded-$p$($y$) (row-consistent)}
\label{alg:reverse-one-step-unbounded-p}
\begin{algorithmic}[1]
\Require odd $y \equiv 1$ or $5 \pmod{6}$
\For{each row $(s,j,\alpha,\beta,c,\delta)$ with $\delta \equiv y \pmod{6}$}
  \State $T \gets (y - \delta)/6$ \Comment{$T \in \mathbb{Z}$}
  \For{$p \gets 0,1,2,\ldots$}
    \If{$\beta \cdot 64^{p} + c > 9T$}
      \State \textbf{break} \Comment{early stop for this row}
    \EndIf
    \If{$(\beta \cdot 64^{p} + c) \bmod 9 = 0$}
      \State $k_p \gets (\beta \cdot 64^{p} + c)/9$
      \State $t \gets T - k_p$
      \If{$t \ge 0$ \textbf{and} $\, t \bmod 2^{\alpha+6p} = 0$}
        \State $m \gets t / 2^{\alpha+6p}$
        \State $p_6 \gets \text{1 if } s=\mathrm{e} \text{ else } 5$
        \State $x \gets 18m + 6j + p_6$
        \If{$U(y)=x$ \textbf{and} $\, (\,s(x),\, \lfloor x/6\rfloor \bmod 3\,) = (\,s,\,j\,)$}
          \State \Return $x$ \Comment{legal parent found}
        \EndIf
      \EndIf
    \EndIf
  \EndFor
\EndFor
\State \Return \textbf{fail}
\end{algorithmic}
\end{algorithm}


\begin{algorithm}[H]
\caption{Reverse-One-Step\texorpdfstring{$(y,p)$}{(y,p)} (row-consistent)}
\label{alg:reverse-one-step}
\begin{algorithmic}[1]
\Require odd $y\equiv 1$ or $5\pmod 6$, lift $p\ge 0$
\For{each row $(s,j,\alpha,\beta,c,\delta)$ with $\delta\equiv y\ (\mathrm{mod}\ 6)$}
  \State $k_p \gets (\beta\,64^{\,p}+c)/9$ \textbf{(skip if non-integer)}
  \State $m \gets \big((y-\delta)/6 - k_p\big) / 2^{\alpha+6p}$ \textbf{(skip if $m\notin\mathbb Z_{\ge 0}$)}
  \State $p_6 \gets 1$ if $s{=}\mathrm e$ else $5$
  \State $x \gets 18m + 6j + p_6$
  \If{$U(y)=x$ \textbf{and} $(s(x),\lfloor x/6\rfloor\bmod 3)=(s,j)$}
     \State \Return $x$ \Comment{legal parent found}
  \EndIf
\EndFor
\State \Return \textbf{fail}
\end{algorithmic}
\end{algorithm}


\begin{algorithm}[H]
\caption{Reverse-Until\texorpdfstring{$(y_0,\text{stop},p)$}{(y0, stop, p)}}
\label{alg:reverse-until}
\begin{algorithmic}[1]
\Require odd start $y_0$, target ancestor $\text{stop}$ (e.g.\ $1$), lift $p\ge 0$
\State $y\gets y_0$; \textsc{Log}$\gets [\,]$
\While{$y\neq \text{stop}$}
  \State $x\gets$ \Call{Reverse-One-Step}{$(y,p)$}
  \If{$x$ is \textbf{fail}} \State \Return \textbf{fail} \EndIf
  \State append $(y\leftarrow x)$ to \textsc{Log}; \ $y\gets x$
\EndWhile
\State \Return \textsc{Log}
\end{algorithmic}
\end{algorithm}

\paragraph{Algorithmic note (search order).}
Given $y\equiv 1,5\pmod 6$, our implementation first tries the one–step reverse
search over all rows $(s,j,\alpha,\beta,c,\delta)$ with $\delta\equiv y\ (\bmod\ 6)$
and small lifts $p\in\{0,1,\dots,p_{\max}\}$; if no integer $m_{\mathrm{prev}}\ge 0$
is obtained, it then tries a single layer of same–family padding: a padding row
$G$ with the same parent family as the decisive row $R$ (type \texttt{ee} if $s{=}\mathrm e$,
type \texttt{oo} if $s{=}\mathrm o$). The combined two–step identity
\[
y \;=\; 6\!\left(2^{\alpha_G+6p_G}\bigl(2^{\alpha_R+6p_R}m + k_{R,p_R}\bigr)+k_{G,p_G}\right)+\delta_R
\]
is then solved for $m\in\mathbb Z_{\ge 0}$, with $j$–indices taken from the chosen $G$ and $R$ rows.
The candidate parent is $x_{\mathrm{prev}}=18\,m+6j_G+p_6$ (with $p_6{=}1$ if $s{=}\mathrm e$, else $5$),
and we verify $U(y)=x_{\mathrm{prev}}$. Small $p$–lifts ($p_G,p_R\in\{0,1,\dots\}$) are included.

\begin{corollary}[Algorithmic completeness of reverse search]
\label{cor:alg-complete-reverse}
Combining row-consistent reversibility (Theorem~\ref{thm:row-consistent-reversibility})
with same–family steering (mod-$3$ steering, parity control, and $v_2$ boosting) and optional $p$–lifts,
the reverse search described above always finds a legal parent for any odd $y\equiv 1,5\pmod 6$.
Iterating yields a finite certified inverse chain to $1$; hence the algorithm constructs, for each odd $y$,
a legal word $W_y$ with per–step certificates $U(x'_i)=x_{i-1}$.
\end{corollary}

\begin{proof}[Proof sketch]
If a decisive row fails to yield an integer $m_{\mathrm{prev}}$ at $p{=}0$, the obstruction is purely
arithmetic (a $2$–power divisibility and/or a low–modulus congruence). The same–family gadgets
guarantee control of $B\bmod 2$ and $B\bmod 3$ while strictly increasing the $2$–adic slope,
and $p$–lifts multiply the slope by $2^{6p}$ without changing routing. Therefore, after at most one layer
of padding and a small lift, the linear divisibility constraint becomes solvable, yielding a legal parent.
Repeating this step produces a finite reverse chain; each step is certified by the identity
$3y+1=2^{\alpha+6p}x_{\mathrm{prev}}$ for the chosen rows and lifts.
\end{proof}


\subsection*{Worked examples}

\paragraph{(A) One–step descent: \(y=3071\).}
Here \(y\equiv 5\pmod 6\) (child family \(\mathrm o\)).
Pick the \(\texttt{oo}\) row with \(j=2\) (\(\Omega_2\)):
\[
x' \;=\; 12m+11,\qquad (\alpha=1,\ s=\mathrm o,\ j=2,\ \delta=5).
\]
Then \(m=(3071-11)/12=255\in\mathbb{Z}\), and the parent is
\[
x \;=\; 18\cdot 255 + 6\cdot 2 + 5 \;=\; 4607,
\]
with \(3\cdot 3071+1=9214=2^1\cdot 4607\). Thus one reverse step drops the
$2$–power by $\alpha=1$.

\paragraph{(B) A short \emph{two–row} tail landing exactly at \(43\).}
We want a child \(y=43\equiv 1\pmod 6\) (child family \(\mathrm e\)).

\emph{Step 1 (last row).}
Choose the \(\texttt{oe}\) row with \(j=1\) (\(\omega_1\)), which ends in family \(\mathrm e\):
\[
x' \;=\; 12m+7 \quad(\alpha=1,\ s=\mathrm o,\ j=1,\ \delta=1).
\]
Solve \(12m+7\equiv 43\ (\mathrm{mod}\ 48)\Rightarrow 12m\equiv 36\Rightarrow m\equiv 3\ (\mathrm{mod}\ 4)\).
Pick the \emph{integral} choice \(m=3\), giving \(\boxed{12\cdot 3+7=43}\).
The parent of \(43\) across this last row is therefore
\[
x_1 \;=\; 18\cdot 3 + 6\cdot 1 + 5 \;=\; \boxed{65}.
\]

\emph{Step 2 (penultimate row).}
Produce \(x_1=65\) from an \(\mathrm e\)-family parent using the \(\texttt{eo}\) row with \(j=2\) (\(\psi_2\)):
\[
x' \;=\; 24m+17 \quad(\alpha=2,\ s=\mathrm e,\ j=2,\ \delta=5).
\]
Solve \(24m+17=65\Rightarrow m=2\in\mathbb{Z}\), so the penultimate parent is
\[
x_0 \;=\; 18\cdot 2 + 6\cdot 2 + 1 \;=\; \boxed{49}.
\]
Forward certificates hold stepwise:
\[
3\cdot 65 + 1 \;=\; 196 \;=\; 2^2\cdot 49,
\qquad
3\cdot 43 + 1 \;=\; 130 \;=\; 2^1\cdot 65.
\]

\emph{Conclusion.} The two–row tail \(W^{\star}=\psi_2\,\omega_1\) maps \(49\to 65\to 43\).
Modulo \(48\), this realizes \(49\equiv 1 \mapsto 65\equiv 17 \mapsto 43\), and
\(\omega_1\) enforces the exact hit \(43\) (not merely modulo \(48\)).

\medskip
\noindent\textbf{Why not a single \texttt{ee} final row?}
For \(\texttt{ee}\) rows we have
\[
\Psi_0: x'=24m+1 \equiv 1 \ (\mathrm{mod}\ 48),\quad
\Psi_1: x'=96m+37 \equiv 37 \ (\mathrm{mod}\ 48),\quad
\Psi_2: x'=384m+277 \equiv 37 \ (\mathrm{mod}\ 48),
\]
independent of \(m\) modulo \(48\). None can produce \(43\ (\mathrm{mod}\ 48)\).
Hence at least one non-\(\texttt{ee}\) step (here \(\omega_1\)) is necessary at the end to land at \(43\).

% =========================================================
\section{From residues to exact integers}

\begin{theorem}[Exact integers lie in the inverse tree of \(1\)]
\label{thm:residues-to-integers}
Every odd integer \(x\ge 1\) lies in the inverse tree of \(1\) under \(U\).
\end{theorem}

\begin{proof}
\leavevmode\par\noindent
\begin{itemize}[leftmargin=1.6em]
\item Let \(r_K\equiv x\pmod{M_K}\).
\item By Theorem~\ref{thm:reachability}, for each \(K\ge 3\) there exist (possibly steered) \(W\) and \(m_K\) with \(x_W(m_K)\equiv r_K\pmod{M_K}\).
\item Refine \(m_{K+1}\equiv m_K \pmod{2^{K-1}}\) (each condition is linear mod a higher power of $2$).
\item We first align the \(3\)-part via Lemma~\ref{lem:mod3-steering}, then lift along powers of \(2\) by steering \(v_2(A)\) (Lemma~\ref{lem:steering}). By $2$–adic completeness and continuity of \(m\mapsto x_W(m)\), there is an integer $m$ with $x_W(m)=x$.
\item Each step satisfies \(U(x')=x\) (Lemma~\ref{lem:row-correctness}), so the odd Collatz orbit of \(x\) reaches \(1\).
\end{itemize}
\end{proof}

\begin{example}[After Theorem~\ref{thm:residues-to-integers}]
For $x=497$, choose $K$ with $M_K>497$, take $r_K\equiv 497\ (\bmod\ M_K)$; a suitable word $W$ (e.g.\ $\psi\,\Omega\,\Omega\,\omega\,\psi$) and compatible $m_K$ exist by Theorem~\ref{thm:reachability}$\,$— the $2$–adic refinement yields an exact $m$ with $x_W(m)=497$.
\end{example}

% =========================

\section{Mod-3 steering in the same family}

Let an admissible word \(W\) have affine form \(x_W(m)=6(A m+B)+\delta\) with \(A=3\cdot 2^{\alpha(W)}\), \(\delta\in\{1,5\}\).
Appending one same-family row \((\alpha_\star,k_\star,\delta)\) maps
\[
B \ \longmapsto\ B' \equiv 2^{\alpha_\star} B + k_\star \pmod 3.
\]
(Here \(k_\star=(\beta+c)/9\) of the appended row; \(\delta\) is unchanged.)

\begin{lemma}[Same-family mod-3 control]\label{lem:mod3-steer}
In family \(\mathrm e\) (type \texttt{ee}) and family \(\mathrm o\) (type \texttt{oo}), there exist finite sets of one-step updates \(B\mapsto 2^{\alpha_\star}B+k_\star\) such that from any \(B\pmod 3\) one can reach any target residue modulo \(3\) in at most two steps; moreover each step multiplies the slope \(A\) by \(2^{\alpha_\star}\ge 2\).
\end{lemma}

\begin{proof}
From the parameter table at \(p{=}0\):
\[
\begin{array}{lcl}
\texttt{ee}\text{ rows: } & (\alpha_\star,k_\star)\in\{(2,0),(4,6),(6,46)\} &\Rightarrow 2^{\alpha_\star}\equiv 1,\ k_\star\equiv 0,0,1\ (\bmod 3),\\[2pt]
\texttt{oo}\text{ rows: } & (\alpha_\star,k_\star)\in\{(5,8),(3,4),(1,1)\} &\Rightarrow 2^{\alpha_\star}\equiv 2,\ k_\star\equiv 2,1,1\ (\bmod 3).
\end{array}
\]
Thus in family \(\mathrm e\) we have maps \(B\mapsto B\) and \(B\mapsto B+1\) mod \(3\); any target is reachable in \(\le 1\) step (or \(2\) steps for \(+2\)).
In family \(\mathrm o\) we obtain the affine maps \(\phi_1(B)=2B+1\) and \(\phi_2(B)=2B+2\) on \(\mathbb{F}_3\). The subgroup of \(\operatorname{AGL}_1(\mathbb{F}_3)\) generated by \(\{\phi_1,\phi_2\}\) acts transitively; explicitly,
\[
\phi_1\circ\phi_1(B)=B,\quad
\phi_2\circ\phi_1(B)=B+1,\quad
\phi_1\circ\phi_2(B)=B+2,
\]
so any residue is reachable in \(\le 2\) steps. In all cases \(\alpha_\star\ge 1\), so \(v_2(A)\) strictly increases.
\end{proof}

\begin{corollary}
Given target \(r\equiv \delta \pmod 6\), by Lemma~\ref{lem:mod3-steer} we may replace \(W\) by a same-family \(W^\star\) with \(B^\star\equiv \frac{r-\delta}{6}\pmod 3\) while increasing \(v_2(A)\). Then the remaining congruence \(2^{\alpha(W^\star)} m \equiv \frac{r-\delta}{6}-B^\star \pmod{2^{K-1}}\) is solvable after possibly one more same-family padding to boost \(v_2(A)\).
\end{corollary}
% =========================
% Mod-3 steering examples (same-family)
% =========================

\subsection*{Mod-3 steering (same-family controls)}

Recall the affine form \(x_W(m)=6(A_W m+B_W)+\delta_W\). A same–family step updates
\[
B_W \ \longmapsto\ B_W' \equiv 2^{\alpha_{\text{row}}}\,B_W + k_{\text{row}} \pmod{3},
\]
where \(2^{\alpha_{\text{row}}}\equiv 1\) or \(2\ (\bmod 3)\) and \(k_{\text{row}}=(\beta+c)/9\ (\bmod 3)\)
for that row (Table~\ref{tab:parameters-abc}).

\begin{example}[Family \(\mathrm e\): one–step ``\(+0\)'' or ``\(+1\)'' on \(B\bmod 3\)]
In family \(\mathrm e\) the \texttt{ee} rows satisfy \(2^{\alpha}\equiv 1\ (\bmod 3)\). Concretely:
\[
\Psi_0:\ B\mapsto B\quad(\text{since }k\equiv 0),\qquad
\Psi_2:\ B\mapsto B+1\quad(\text{since }k\equiv 1).
\]
Thus from any \(B\bmod 3\) you can reach any target residue in at most two \texttt{ee} steps,
while increasing \(v_2(A)\) each time.
\end{example}

\begin{example}[Family \(\mathrm o\): affine maps \(B\mapsto 2B+1\) or \(2B+2\)]
In family \(\mathrm o\), the \texttt{oo} rows have \(2^{\alpha}\equiv 2\ (\bmod 3)\). From the parameter table:
\[
\Omega_1:\ B\mapsto 2B+1,\qquad
\Omega_0:\ B\mapsto 2B+2 \quad(\bmod 3).
\]
Because these two maps generate all affine transformations of \(\mathbb{F}_3\),
you can reach any target \(B'\in\{0,1,2\}\) in at most two \texttt{oo} steps,
again raising \(v_2(A)\) along the way.
\end{example}

\begin{example}[Explicit drills]
\leavevmode
\begin{itemize}[leftmargin=1.6em]
  \item \(\mathrm e\)-family, want \(B'\equiv 2\): if \(B\equiv 0\), use \(\Psi_2,\Psi_2\) (adds \(+1\) twice);
        if \(B\equiv 1\), use \(\Psi_2\) once; if \(B\equiv 2\), use \(\Psi_0\).
  \item \(\mathrm o\)-family, want \(B'\equiv B+1\): use \(\Omega_1\circ\Omega_0\),
        since \(B\mapsto 2B+2\mapsto 2(2B+2)+1\equiv B+1\ (\bmod 3)\).
\end{itemize}
\end{example}

% =========================
% Mod-3 combined techniques (steer + boost + solve)
% =========================

\subsection*{Combining mod-3 steering with \(2\)-adic boosting}

We show how mod-3 control and \(2\)-adic boosting combine to hit a target class \(r\bmod M_K\),
with certified steps at every stage.

\begin{example}[Target \(r\equiv 53\ (\bmod 96)\) with terminal family \(\mathrm o\)]
We want \(x_W(m)=6(A_W m+B_W)+5 \equiv 53\ (\bmod 96)\).
This reduces to the pair of conditions
\[
B_W \equiv \frac{53-5}{6} \equiv 8 \equiv 2 \pmod{3}
\qquad\text{and}\qquad
A_W m \equiv \frac{53-5}{6} - B_W \pmod{16}.
\]
\textit{Step 1 (mod-3 steering in \(\mathrm o\)).}
Append one or two \texttt{oo} rows to force \(B_W\equiv 2\ (\bmod 3)\).
For instance, if currently \(B\equiv 0\), use \(\Omega_1\) then \(\Omega_1\):
\(B\mapsto 2B+1\mapsto 2(2B+1)+1\equiv 2\).

\smallskip
\noindent
\textit{Step 2 (\(2\)-adic boost).}
Keep appending \(\texttt{oo}\) rows (e.g.\ \(\Omega_0\) or \(\Omega_1\)) until
\(v_2(A_W)\ge 4\), so the congruence modulo \(2^{K-1}=16\) is solvable.

\smallskip
\noindent
\textit{Step 3 (solve for \(m\)).}
With the mod-3 condition met, choose \(m\) so that
\(A_W m \equiv \frac{48}{6}-B_W \equiv 8-B_W \pmod{16}\).
Since \(\gcd(A_W,16)=2^{v_2(A_W)}\) and we enforced \(v_2(A_W)\ge 4\),
a solution exists and gives \(x_W(m)\equiv 53\ (\bmod 96)\) as required.

\smallskip
\noindent
\textit{Concrete two-row realization.}
A compact option is the composite \(\Omega_2\) then \(\Omega_1\):
\[
\Omega_2:\ x\mapsto 12m+11,\qquad
\Omega_1:\ x\mapsto 48m+29.
\]
Composing (with the updated indices) yields \(x'=96m+53\),
so the target class \(53\ (\bmod 96)\) is achieved for all \(m\) and each step satisfies \(U(x')=x\).
This composite simultaneously sets \(B\bmod 3\) and raises the \(2\)-power to \(96\).
\end{example}

\begin{example}[Family \(\mathrm e\): force \(B\equiv 1\ (\bmod 3)\) and lift to \(M_6=192\)]
Suppose the terminal family must be \(\mathrm e\) and the target is \(r\equiv 1\ (\bmod 192)\).
We need \(B_W\equiv \frac{1-1}{6}\equiv 0\ (\bmod 3)\) \emph{or} \(B_W\equiv 1\) depending on the chosen last row.
Use \(\Psi_2\) to add \(+1\) modulo \(3\) and \(\Psi_0\) to keep \(B\) fixed; in at most two steps set \(B\) to the required residue.
Append additional \(\texttt{ee}\) rows to raise \(v_2(A_W)\ge 5\) (since \(M_6=3\cdot 2^6\) needs modulus \(2^{5}\) in the congruence).
Then solve \(A_W m \equiv \frac{r-\delta_W}{6}-B_W \ (\bmod\ 32)\).
\end{example}

\begin{example}[Lifting to \texorpdfstring{$1531 \bmod 1536$}{1531 mod 1536} (3-adic check)]

Target:
\[
r' \equiv 1531 \pmod{1536},\qquad 1536=3\cdot 2^{9},\qquad 1531\equiv 1 \pmod 6\ \ (\text{family }\mathrm e).
\]

Pick the row \((\mathrm o,1)\) of type \texttt{oe} (i.e.\ $\omega_1$). Its unified $p{=}0$ form is
\[
x'(m)=12m+7 \;=\; 6\bigl(2m+1\bigr)+1,
\]
so $\delta=1$ (outputs family $\mathrm e$), and in affine notation \(A=3\cdot 2^{\alpha}=6\) and \(B=1\).

Solve the congruence
\[
12m+7 \equiv 1531 \pmod{1536}
\quad\Longleftrightarrow\quad
12m \equiv 1524 \pmod{1536}.
\]
Because $\gcd(12,1536)=12$ and $1524/12=127$, we get
\[
m \equiv 127 \pmod{128}.
\]
Thus all solutions are \(m=127+128t\) with $t\in\mathbb Z$, and
\[
x'(m)=12(127+128t)+7 = 1531 + 1536t \equiv 1531 \pmod{1536}.
\]
\end{example}

\paragraph{3-adic consistency check.}
Writing \(x'(m)=6(Am+B)+\delta\) with \(A=6\), \(B=1\), \(\delta=1\), the standard lifting congruence is
\[
A\,m \equiv \frac{r'-\delta}{6}-B \pmod{2^{9}}
\quad\Longleftrightarrow\quad
6m \equiv 255-1=254 \pmod{512}.
\]
Here \(\gcd(6,512)=2\) divides \(254\), so a solution exists; dividing by \(2\) gives
\(3m \equiv 127 \pmod{256}\), which is equivalent to \(m\equiv 127 \pmod{128}\) (since \(3^{-1}\equiv 171 \pmod{256}\)).
Modulo \(3\), we have \((r'-\delta)/6\equiv 255 \equiv 0\) and \(A\equiv 0\), so the mod-3 part is automatically satisfied; if desired, one could first enforce \(B\equiv 0\pmod{3}\) via a short same-family \texttt{oo} steering prefix and still finish with \(\omega_1\). In this instance, the 2-power congruence already admits a solution, so extra mod-3 steering is unnecessary.

\medskip
\noindent\emph{Concrete choice:} \(m=127\) yields \(x'(127)=1531\) exactly; \(m=255\) yields \(x'(255)=1531+1536\).


% =========================================================
\section{Synthesis: how the pieces yield convergence on the odd layer}

We now explain explicitly how the preceding ingredients combine to certify that
\emph{every odd integer not congruent to \(3\bmod 6\) reaches \(1\) in finitely many Collatz
(odd–accelerated) steps}. Equivalently, every odd \(x\equiv 1,5\pmod 6\) lies in the inverse tree of \(1\)
under the map \(U\).

\begin{theorem}[Global conclusion on the odd layer]\label{thm:odd-layer-convergence}
Every odd integer \(x\ge 1\) with \(x\equiv 1,5\pmod 6\) admits a finite inverse word
\(W\in\{\Psi,\psi,\omega,\Omega\}^\ast\) and an integer \(m\) such that the stepwise updates of
Table~\ref{tab:unified-F0-straight-xprime} realize a certified chain
\[
1 \xleftarrow{\,U\,} x'_1 \xleftarrow{\,U\,} x'_2 \xleftarrow{\,U\,} \cdots \xleftarrow{\,U\,} x'_t = x,
\qquad\text{i.e.}\qquad U(x'_i)=x_{i-1}\ \text{ at every step}.
\]
Consequently the forward (accelerated odd) Collatz orbit of \(x\) reaches \(1\) after \(|W|\) odd steps.
\end{theorem}

\begin{proof}
\leavevmode\par\noindent
\begin{itemize}[leftmargin=1.8em]
\item \emph{Certified one–step inverses.} For each admissible row the identity
\(3x'+1=2^\alpha x\) holds (Lemma~\ref{lem:row-correctness}); hence \(U(x')=x\) stepwise.
\item \emph{Words are affine and trackable.} Any fixed admissible word \(W\) yields
\(x_W(m)=6(A_W m+B_W)+\delta_W\) with \(A_W=3\cdot 2^{\alpha(W)}\) (Lemma~\ref{lem:affine-word}),
and its family pattern depends only on the tokens (Lemma~\ref{lem:family-pattern}).
\item \emph{Base witnesses.} Modulo \(M_3=24\), each odd residue \(r\equiv 1,5\pmod 6\) has a certified witness
word \(W_r\) from Table~\ref{tab:base-witnesses-mod24}.
\item \emph{Steering (padding) control.} Same–family \emph{steering gadgets} raise the slope’s
\(2\)-adic valuation \(v_2(A)\) and let us preserve or flip the intercept parity \(B\bmod 2\)
(Lemma~\ref{lem:steering} and the concrete menus in Appendix~A).
\item \emph{Linear lifting in \(K\).} Given a target residue \(r'\bmod M_{K+1}\) with the correct terminal
family, padding \(W\) ensures the linear congruence
\(
A_W m \equiv \frac{r'-\delta_W}{6}-B_W \pmod{2^K}
\)
is solvable; this lifts witnesses from \(M_K\) to \(M_{K+1}\) (Lemma~\ref{lem:lifting}).
By induction we obtain, for each \(K\ge 3\), a padded word \(W_K\) and an \(m_K\) with
\(x_{W_K}(m_K)\equiv x \pmod{M_K}\).
\item \emph{\(2\)-adic refinement to an exact integer.} Choosing the \(m_K\) compatibly modulo
\(2^{K-1}\) yields \(m\in\mathbb Z\) with \(x_{W}(m)=x\) (Section “From residues to exact integers”).
\item \emph{Conclusion.} Concatenating the certified one–step inverses gives a finite inverse chain from \(1\) to \(x\),
hence the forward \(U\)–orbit of \(x\) reaches \(1\) in \(|W|\) odd steps.
\end{itemize}
\end{proof}

\begin{remark}[Scope and the missing \(3\bmod 6\) class]
Odd outputs of the accelerated map \(U\) always lie in the classes \(1\) or \(5\bmod 6\);
the class \(3\bmod 6\) never appears on the odd layer. Thus Theorem~\ref{thm:odd-layer-convergence}
covers exactly the odd layer relevant for \(U\). In the classical (non-accelerated) iteration,
any odd \(x\equiv 3\pmod 6\) immediately produces an even number; after removing powers of two the next odd belongs to \(1\) or \(5\bmod 6\), whence the theorem applies.
\end{remark}

\begin{corollary}[Finite convergence in forward time on the odd layer]
For every odd \(x\equiv 1,5\pmod 6\) there is a finite \(t\) such that
\(U^{\circ t}(x)=1\). Equivalently, \(x\) lies at finite depth in the inverse tree of \(1\).
\end{corollary}

% Where the detailed section currently sits:
\ifobjections

% =========================================================
\section{Responses to anticipated objections}\label{sec:objections}

\paragraph{Objection 1: The base witnesses mod \(24\) are ad hoc or computationally fragile.}
They are a finite, explicit verification for eight residues (Table~\ref{tab:base-witnesses-mod24}), and each step is \emph{symbolically} certified by Lemma~\ref{lem:row-correctness} via \(3x'+1=2^\alpha x\).
No probabilistic or heuristic assumption is used; later lifting steps depend only on the algebraic properties of the rows.

\paragraph{Objection 2: Same–family ``steering'' might fail to control parity or \(v_2\).}
Lemma~\ref{lem:steering} formalizes the gadgets. Concrete token lists are provided in Appendix~A, with a residue-by-residue certificate at modulus \(54\) in Appendix~B. These gadgets guarantee (a) a slope boost \(v_2(A)\ge 1\) per use and (b) availability of a parity toggle of \(B\bmod 2\) (e.g.\ via \(\omega_1\) or \(\Omega_2\)).

\paragraph{Objection 3: The lifting step \(M_K\to M_{K+1}\) may be ill-posed.}
Lemma~\ref{lem:lifting} reduces the target to a linear congruence
\(A_W m \equiv \frac{r'-\delta_W}{6}-B_W \ (\bmod 2^K)\).
By steering we can ensure \(A_W\) has sufficiently large \(2\)-adic valuation and choose the parity of \(B_W\), guaranteeing solvability. This is an elementary \(2\)-power congruence, not an appeal to unproven \(p\)-adic theory.

\paragraph{Objection 4: Mixing the column parameter \(p\) changes types or breaks the identity.}
Lemma~\ref{lem:mixedp-step} shows \(3x'+1=2^{\alpha+6p}x\) for every step at any \(p\ge 0\).
Lemma~\ref{lem:mixedp-routing} shows the \texttt{type} and offset \(\delta\) are \(p\)-invariant, so routing is unaffected.

\paragraph{Objection 5: Excluding \(x\equiv 3\pmod 6\) dodges the problem.}
The odd layer of the accelerated map \(U\) \emph{never} visits \(3\bmod 6\), by construction.
For the classical map, any \(3\bmod 6\) odd immediately becomes even and the next odd lies in \(1\) or \(5\bmod 6\); then Theorem~\ref{thm:odd-layer-convergence} applies (see the remark after the theorem).

\paragraph{Objection 6: ``Finite depth'' does not equal ``reaches \(1\)''.}
In our setting the inverse certification ensures a concrete finite chain \(x'_t\to \cdots \to x'_1\to 1\) with \(U(x'_i)=x_{i-1}\), so ``finite depth'' is \emph{equivalent} to reaching \(1\) in \(|W|\) odd steps (Theorem~\ref{thm:odd-layer-convergence}).

\paragraph{Objection 7: The CRT tag \(t=(x-1)/2\) is an artificial overhead.}
It is merely a reindexing convenience (Cor.~\ref{cor:tag-indices}) that makes the family and indices \((s,j,m)\) transparent; all arguments can be phrased without \(t\), but computations (and examples) become more compact with it.

\paragraph{Objection 8: Refinements (e.g., $r\bmod 24$ to $r'\bmod 48$) use different words, so the orbits are unrelated. What does this actually prove?}
\textit{Response.}
The lifting theory guarantees \emph{certified existence} of a legal word for every refinement; it does not require the \emph{same bare word} (or identical forward orbit) to persist across moduli. What is preserved vs.\ what may change is as follows:
\begin{itemize}[leftmargin=1.4em]
  \item \textbf{Preserved.} (i) Legality of each step via the identity $3x'+1=2^\alpha x$ (so $U(x')=x$) and hence certified invertibility; (ii) the family routing pattern ($\mathrm e/\mathrm o$) determined solely by the token sequence; (iii) solvability of the lifted congruence by appending \emph{same-family steering gadgets} that raise $v_2$ and control intercept residues.
  \item \textbf{Allowed to change.} (i) The concrete token sequence (e.g.\ after appending padding), (ii) the indexing parameter $m$, and consequently (iii) the specific integers realized along the inverse chain. Distinct words hitting the same residue (or refinements thereof) are fully compatible with the framework.
\end{itemize}
Therefore, the content of the lifting program is \emph{reachability with stepwise certificates}, not orbit identity across representatives. From a finite set of base witnesses at $M_3=24$, the steering-and-lift machinery constructs certified words for every refinement $M_{K+1}$ and, by $2$-adic refinement, for every odd integer.
\medskip

\noindent\textit{Practical note.} If desired, one can keep a chosen \emph{core} base word and obtain the refined witness by \emph{only} appending same-family padding (which preserves the token-determined family pattern). Alternatively, one may present a minimal explicit word at the refined modulus. Both approaches are legal and carry the same stepwise certificates $U(x')=x$.


\fi
% =========================
% Appendix: Concrete steering gadgets and certificates
% =========================

\appendix
\section*{Appendix A: Concrete steering gadgets (valuation \& parity)}

We record short, concrete composites that begin and end in the \emph{same} family
(\(\mathrm e\) or \(\mathrm o\)). They serve two roles:
(i) raise the slope’s \(2\)-adic valuation \(v_2(A)\) (for lifting), and
(ii) toggle the intercept parity \(B\pmod 2\) (for solvability of linear congruences).

Throughout we use the unified \(p{=}0\) table; when \(p\ge 1\), emulate the lift via extra same–family padding (adds \(2^{6p}\) to the slope) or use short composites whose net parity still toggles (cf.\ mixed-\(p\) discussion).

\begin{table}[!htbp]
\centering
\caption{Concrete \(s\!\to\!s\) gadgets. Tokens are evaluated with the unified \(p{=}0\) table.}
\label{tab:explicit-gadgets}
\setlength{\tabcolsep}{4pt}      % default is 6pt; tighten a bit
\renewcommand{\arraystretch}{1.1} % slightly tighter
\footnotesize                     % or \scriptsize if still too wide
\begin{tabularx}{\linewidth}{@{}l Y c Y Y@{}}
\toprule
Family & Gadget (tokens) & Len & Type path & Effect \\
\midrule
\(\mathrm e\) & \(\psi\,\omega\) & 2 &
\texttt{eo}\(\to\)\texttt{oe} (net \(e\to e\)) &
\(v_2(A)\) increases (at least \(+1\)); parity usually unchanged \\
\(\mathrm e\) & \(\psi\,\Omega\,\omega\) & 3 &
\texttt{eo}\(\to\)\texttt{oo}\(\to\)\texttt{oe} (net \(e\to e\)) &
Parity toggle available (choose the middle \(\Omega\) row adaptively) \\
\(\mathrm o\) & \(\Omega\) & 1 &
\texttt{oo} (net \(o\to o\)) &
\(v_2(A)\) increases (at least \(+1\)); parity unchanged if \(\Omega_{0,1}\) \\
\(\mathrm o\) & \(\omega\,\psi\) & 2 &
\texttt{oe}\(\to\)\texttt{eo} (net \(o\to o\)) &
Parity toggle if the \(\omega\) step uses the \((\mathrm o,1)\) row (\(\omega_1\)) \\
\(\mathrm o\) & \(\Omega\,\omega\,\psi\) & 3 &
\texttt{oo}\(\to\)\texttt{oe}\(\to\)\texttt{eo} (net \(o\to o\)) &
Guaranteed parity toggle via either \(\Omega_2\) or \(\omega_1\) \\
\bottomrule
\end{tabularx}
\end{table}


\paragraph{How to use the parity gadgets (runtime rule).}
\begin{itemize}[leftmargin=1.6em]
\item \textbf{Family \(\mathrm o\).} If $j{=}1$, use \(\omega_1\) then \(\psi\) (parity flip). If $j{=}2$, use \(\Omega_2\) then \(\omega\) then \(\psi\) (flip). Otherwise insert one \(\Omega\) and branch accordingly.
\item \textbf{Family \(\mathrm e\).} Use \(\psi\) to enter \(\mathrm o\); if the new $j{=}2$ use \(\Omega_2\) then \(\omega\); if $j{=}1$ use \(\omega_1\) then \(\omega\). Both return to \(\mathrm e\) and flip parity.\\\\
\end{itemize}

\section*{Appendix A\texorpdfstring{$^\prime$}: Mod-\pdfmath{3} steering (valuation \& residue control)}
\label{app:mod3-steering}

We strengthen the steering toolkit by showing that, in addition to toggling $B_W \bmod 2$ and raising $v_2(A_W)$, one can \emph{steer $B_W$ to any desired residue modulo $3$} while remaining in the same family. This closes the divisibility-by-$3$ gap in the exact-lifting step.

\begin{lemma}[Mod-$3$ steering lemma]\label{lem:mod3-steering}
Let $W$ be an admissible word with affine form $x_W(m)=6(A_W m+B_W)+\delta_W$, where $A_W=3\cdot 2^{\alpha(W)}$ and $\delta_W\in\{1,5\}$. For each family $s\in\{\mathrm e,\mathrm o\}$ there exist short same–family gadgets $P^{(r)}_s$ ($r\in\{0,1,2\}$) such that
\[
x_{W\cdot P^{(r)}_s}(m)=6(A' m+B'_s)+\delta_W,\qquad
v_2(A')>v_2(A_W),\qquad B'_s\equiv r\pmod{3}.
\]
In particular, one can raise $v_2(A)$ and \emph{set} $B\bmod 3$ arbitrarily while preserving the terminal family $\delta_W$.
\end{lemma}

\begin{proof}
We use the unified $p{=}0$ rows in Table~\ref{tab:unified-F0-straight-xprime} and the parameter table (Table~\ref{tab:parameters-abc}). If a same–family row with parameters $(\alpha,k,\delta)$ is appended to a word with affine form $6(A m+B)+\delta$, the new slope is $A'=A\cdot 2^{\alpha}$ and the new intercept is
\[
B'\;\equiv\;2^{\alpha}B+k\pmod{3},
\]
because $x\mapsto 6(2^{\alpha}m+k)+\delta$ contributes $2^{\alpha}$ on the $m$–slope and adds $k$ to the intercept, and $2^{\alpha}\equiv 1$ or $2$ modulo $3$ depending on $\alpha$.

\smallskip
\noindent\emph{Family $\mathrm e$ (type \texttt{ee}, $\delta=1$).}
From Table~\ref{tab:parameters-abc}, the \texttt{ee} rows have
\[
(\alpha,k)\in\{(2,0),(4,6),(6,46)\}.
\]
Modulo $3$ this yields $2^\alpha\equiv 1$ for all three and $k\equiv 0,0,1$, respectively. Hence a single \texttt{ee} step realizes
\[
B'\equiv B \quad\text{or}\quad B'\equiv B+1 \pmod{3}.
\]
Thus in at most two \texttt{ee} steps we can set $B'\equiv r$ for any prescribed $r\in\{0,1,2\}$. Each step multiplies $A$ by $2^{\alpha}\ge 4$, so $v_2(A)$ strictly increases.

\smallskip
\noindent\emph{Family $\mathrm o$ (type \texttt{oo}, $\delta=5$).}
From Table~\ref{tab:parameters-abc}, the \texttt{oo} rows have
\[
(\alpha,k)\in\{(5,8),(3,4),(1,1)\}.
\]
Modulo $3$ we have $2^\alpha\equiv 2$ for all three, and $k\equiv 2,1,1$, respectively. Therefore any single \texttt{oo} step implements one of the affine maps
\[
\phi_1(B)=2B+1,\qquad \phi_2(B)=2B+2\qquad (\bmod 3).
\]
The subgroup of affine maps of $\mathbb{Z}/3\mathbb{Z}$ generated by $\{\phi_1,\phi_2\}$ is all of $\operatorname{AGL}_1(\mathbb{F}_3)$; concretely, from any starting $B\bmod 3$ one reaches any target residue in at most two steps (e.g.\ $\phi_1\circ\phi_1(B)=B$, $\phi_2\circ\phi_1(B)=B+1$, etc.). Each \texttt{oo} step multiplies $A$ by $2^{\alpha}\ge 2$, so $v_2(A)$ strictly increases.

\smallskip
Combining the family–wise controls gives the claim: in family $\mathrm e$ use at most two \texttt{ee} steps; in family $\mathrm o$ use at most two \texttt{oo} steps (choosing which \texttt{oo} row to realize $\phi_1$ or $\phi_2$). In all cases the terminal family (hence $\delta_W$) is preserved and $v_2(A)$ increases.
\end{proof}

\begin{table}[h]
\centering
\caption{Same–family rows: residues of $2^{\alpha}$ and $k$ modulo $3$ (at $p{=}0$).}
\label{tab:mod3-ee-oo}
\begin{tabular}{@{}c c c c c@{}}
\toprule
Row & $(s,j)$ & $\alpha$ & $2^{\alpha}\,(\bmod 3)$ & $k=(\beta+c)/9\,(\bmod 3)$\\
\midrule
$\Psi_{0}$ & $(\mathrm e,0)$ & $2$ & $1$ & $0$ \\
$\Psi_{1}$ & $(\mathrm e,1)$ & $4$ & $1$ & $0$ \\
$\Psi_{2}$ & $(\mathrm e,2)$ & $6$ & $1$ & $1$ \\
\midrule
$\Omega_{0}$ & $(\mathrm o,0)$ & $5$ & $2$ & $2$ \\
$\Omega_{1}$ & $(\mathrm o,1)$ & $3$ & $2$ & $1$ \\
$\Omega_{2}$ & $(\mathrm o,2)$ & $1$ & $2$ & $1$ \\
\bottomrule
\end{tabular}
\end{table}

\paragraph{Constructive gadgets (runtime recipes).}
Let the current terminal family of $W$ be $s$ and write $B:=B_W\bmod 3$.

\begin{itemize}[leftmargin=1.6em]
\item \textbf{If $s=\mathrm e$} (want $B'\equiv r$):
  \begin{enumerate}[itemsep=2pt]
    \item If $B\equiv r$, append $\Psi_{0}$ (does not change $B$; raises $v_2(A)$).
    \item Else append $\Psi_{2}$ once: $B\mapsto B+1$; if still not $r$, append $\Psi_{2}$ again.
  \end{enumerate}
\item \textbf{If $s=\mathrm o$} (want $B'\equiv r$):
  \begin{enumerate}[itemsep=2pt]
    \item If $B\equiv r$, append $\Omega_{1}$ (keeps flexibility for later; raises $v_2(A)$).
    \item Else compute $d:=r-B\pmod 3$.
      \begin{itemize}
        \item If $d\equiv 1$: append $\Omega_{1}$ then $\Omega_{0}$; effect $B\mapsto 2B+1\mapsto 2(2B+1)+2\equiv B+1$.
        \item If $d\equiv 2$: append $\Omega_{0}$ then $\Omega_{1}$; effect $B\mapsto 2B+2\mapsto 2(2B+2)+1\equiv B+2$.
      \end{itemize}
  \end{enumerate}
\end{itemize}

\paragraph{Corollary (exact divisibility condition).}
Let $x_W(m)=6(A_W m+B_W)+\delta_W$ with $A_W=3\cdot 2^{\alpha(W)}$. Given any target odd $x\equiv\delta_W\ (\bmod\ 6)$, by Lemma~\ref{lem:mod3-steering} we may replace $W$ by $W^\star$ so that
\[
B_{W^\star}\equiv \frac{x-\delta_{W}}{6}\pmod{3}.
\]
Then $A_{W^\star}\mid \bigl(\frac{x-\delta_W}{6}-B_{W^\star}\bigr)$ if and only if $2^{\alpha(W^\star)}\mid \bigl(\frac{x-\delta_W}{6}-B_{W^\star}\bigr)$, which can always be enforced by further same–family padding (raising $v_2(A)$). Hence there exists $m\in\mathbb Z$ with $x_{W^\star}(m)=x$.
\begin{example}[Mod-3 steering then 2-adic lifting to \(3071 \bmod 3072\)]
Target residue:
\[
r' \equiv 3071 \pmod{3072},\qquad 3071\equiv 5\pmod 6\ \text{(odd family)}.
\]

Start with the one-step word \(W=\psi\) (row \((\mathrm e,0)\) in the unified table):
\[
x_W(m)=6\bigl(A\,m+B\bigr)+\delta,\qquad \psi:\ \delta=5,\ A=16,\ B=0.
\]

\emph{(1) Mod-3 steering.}
Set
\[
t:=\frac{r'-\delta}{6}=\frac{3071-5}{6}=511.
\]
The mod-3 solvability criterion is \(B\equiv t\pmod{3}\).
Since \(t\equiv 1\pmod 3\) and \(B\equiv 0\pmod 3\) for \(\psi\), append one odd-family step \(\Omega_1\),
which acts as \(B\mapsto 2B+1\ (\bmod 3)\).
Thus \(B\equiv 1\pmod 3\) after \(\Omega_1\), and the mod-3 condition is aligned.

\emph{(2) Divide by 3 and set the \(2\)-adic congruence.}
After \(\psi\) then \(\Omega_1\), the accumulated exponent is \(\alpha_{\text{tot}}=4+3=7\).
With \(B\equiv 1\pmod 3\) (take \(B=1\) concretely),
\[
2^{\alpha_{\text{tot}}}m \equiv \frac{t-B}{3}=\frac{511-1}{3}=170 \pmod{2^{K-1}},\qquad K=10\Rightarrow 2^{K-1}=512.
\]
So \(2^{7}m \equiv 170 \pmod{512}\).

\emph{(3) Ensure \(2\)-adic solvability by padding.}
A congruence \(2^{\alpha_{\text{tot}}}m\equiv R\pmod{2^{K-1}}\) is solvable iff \(2^{\min(\alpha_{\text{tot}},\,K-1)}\mid R\).
Here \(\min(7,9)=7\) but \(170\not\equiv 0\pmod{128}\).
Use same-family odd padding (\(\Omega_0,\Omega_1,\Omega_2\)) to:
\begin{itemize}
\item keep \(B\equiv 1\pmod 3\) (mod-3 steering), and
\item raise \(v_2(A)\) while shifting the integer \(B\) so that
\[
\frac{t-B}{3}\equiv 0\pmod{512}\ \Longleftrightarrow\ B\equiv t \pmod{1536}\ \Longleftrightarrow\ B\equiv 511\pmod{1536}.
\]
\end{itemize}
Once \(B\equiv 511\ (\bmod 1536)\), the right-hand side becomes \(0\pmod{512}\),
and a solution exists (e.g.\ \(m\equiv 0\pmod{512}\)).

\emph{Conclusion.}
With the sequence \(\psi\) followed by \(\Omega_1\) and a short odd-family padding that sets \(B\equiv 511\ (\bmod 1536)\) (while increasing \(v_2\) of the slope), we obtain
\[
x_W(m)\equiv 3071 \pmod{3072},
\]
and every step is certified by the identity \(3x'+1=2^{\alpha}x\) (hence \(U(x')=x\)) from the unified table.
\end{example}




\Needspace{5\baselineskip}

\section*{Appendix B: Residue-by-residue parity gadgets mod 54 (certificate)}
\label{app:mod54-certificate}

% (preamble — you already added these for the previous table)
% \usepackage{tabularx}
% \newcolumntype{Y}{>{\raggedright\arraybackslash}X}

\begin{table}[!htbp]
\centering
\caption{Certified parity–flip gadgets by odd residue class modulo \(54\).}
\label{tab:mod54-gadgets}
\setlength{\tabcolsep}{4pt}
\renewcommand{\arraystretch}{1.1}
\footnotesize
\begin{tabularx}{\linewidth}{@{}c c c Y@{}}
\toprule
Residue \(x\bmod 54\) & Family \(s\) & \(j=\lfloor x/6\rfloor\bmod 3\) & Gadget (tokens) \\
\midrule
\multicolumn{4}{@{}l@{}}{\emph{Family \(\mathrm e\) (classes \(\equiv 1\pmod 6\)):}}\\
\(1\)  & \(\mathrm e\) & \(0\) & \(\psi\); then \textbf{if} new \(j{=}1\): \(\omega_1\) then \(\omega\); \textbf{if} new \(j{=}2\): \(\Omega_2\) then \(\omega\) \\
\(7\)  & \(\mathrm e\) & \(1\) & same recipe as for \(1\) \\
\(13\) & \(\mathrm e\) & \(2\) & same recipe as for \(1\) \\
\(19\) & \(\mathrm e\) & \(0\) & same recipe as for \(1\) \\
\(25\) & \(\mathrm e\) & \(1\) & same recipe as for \(1\) \\
\(31\) & \(\mathrm e\) & \(2\) & same recipe as for \(1\) \\
\(37\) & \(\mathrm e\) & \(0\) & same recipe as for \(1\) \\
\(43\) & \(\mathrm e\) & \(1\) & same recipe as for \(1\) \\
\(49\) & \(\mathrm e\) & \(2\) & same recipe as for \(1\) \\
\addlinespace[4pt]
\multicolumn{4}{@{}l@{}}{\emph{Family \(\mathrm o\) (classes \(\equiv 5\pmod 6\)):}}\\
\(5\)  & \(\mathrm o\) & \(0\) & \(\Omega\); \textbf{if} new \(j{=}1\): \(\omega_1\) then \(\psi\); \textbf{if} new \(j{=}2\): \(\Omega_2\) then \(\omega\) then \(\psi\) \\
\(11\) & \(\mathrm o\) & \(1\) & \(\omega_1\) then \(\psi\) \\
\(17\) & \(\mathrm o\) & \(2\) & \(\Omega_2\) then \(\omega\) then \(\psi\) \\
\(23\) & \(\mathrm o\) & \(0\) & same recipe as for \(5\) \\
\(29\) & \(\mathrm o\) & \(1\) & same recipe as for \(11\) \\
\(35\) & \(\mathrm o\) & \(2\) & same recipe as for \(17\) \\
\(41\) & \(\mathrm o\) & \(0\) & same recipe as for \(5\) \\
\(47\) & \(\mathrm o\) & \(1\) & same recipe as for \(11\) \\
\(53\) & \(\mathrm o\) & \(2\) & same recipe as for \(17\) \\
\bottomrule
\end{tabularx}
\end{table}

\appendix
\section{Appendix C: Mechanical checks and lifted witnesses}\label{app:checks}

\paragraph{Audit protocol (informal).}
A simple script can (i) verify each row formula $x'=6(2^{\alpha_p}u+k^{(p)})+\delta$ at sampled inputs, (ii) check routers $j=\lfloor x/6\rfloor\bmod 3$ match the table choice, (iii) confirm $U(x')=x$ for the forward accelerated map, and (iv) validate lifted witnesses at higher moduli ($M_K$) by direct congruence checks.

\paragraph{Lifted witnesses at $M_4=48$ from $M_3=24$.}
Each row lists a residue $r\bmod 24$, a short admissible tail producing $r'\bmod 48$, and a one-line justification (pinning or solved congruence). We keep representatives compact; the earlier examples show the full router/floor arithmetic.

\begin{table}[!htbp]
\centering
\caption{Lifted witnesses from $24$ to $48$. Each tail is read from the $p{=}0$ table and obeys routing.}
\label{tab:lift-24-to-48}
\begin{tabularx}{\linewidth}{@{}c c c X@{}}
\toprule
$r\bmod 24$ & $r'\bmod 48$ & Tail & Reason \\ \midrule
$17$ & $41$ & $\omega_1\to\psi_2$ & Congruence regime for $\psi_2$: $x'=24m+17$, choose class with $m$ odd; admissibility shown in Ex.~\ref{ex:17to41}.\\
$13$ & $13$ & $\Psi_1$ & Pinning: $\alpha=4\ge K=4$ gives $x'\equiv 6k+\delta\equiv 37\equiv 13\ (\bmod 48)$. \\
$23$ & $23$ or $47$ & $\Omega_2$ or $\omega_1\to\psi_2$ & $\Omega_2$ yields $x'=12m+11$ so parity classes hit $11,23$; a cross-family two-step can target $47$ as needed. \\
$7$ & $7$ or $31$ & $\omega_1$ or $\omega_1\to\psi_2$ & As above: single-step parity split, or two-step tail for the other odd residue. \\
\bottomrule
\end{tabularx}
\end{table}

\begin{example}[Explicit calculation for $17\bmod 24 \to 41\bmod 48$]\label{ex:17to41}
This is the two-step tail $\omega_1\to\psi_2$ with the router/floor arithmetic spelled out in the main text (see the worked example in Section~\ref{sec:menu-all-p}).
\end{example}


\newpage

\section*{Appendix D: Witness tables mod 48 amd 96}

% =========================
% TEMPLATE TABLE: witnesses mod 48
% =========================
\begin{table}[!htbp]
\centering
\caption{Witness construction template modulo \texorpdfstring{\(48\)}{48} (with \texorpdfstring{\(M_4=48\)}{M4=48}).
For each odd residue \texorpdfstring{\(r' \equiv 1,5 \pmod 6\)}{r' congruent 1,5 mod 6}, pick a word \(W\) whose terminal family matches \(\,r' \bmod 6\).
Write its affine form as \(\,x_W(m)=6(A_W m+B_W)+\delta_W\) (with \(\,A_W=3\cdot 2^{\alpha(W)}\)).
Solve the linear congruence \(\,A_W m \equiv \tfrac{r'-\delta_W}{6}-B_W \pmod{2^{3}}\) (i.e.\ mod \(8\)),
and set \(x := x_W(m)\), which then satisfies \(x \equiv r' \pmod{48}\) and \(U(x)=\cdots=1\) along \(W\).}

\label{tab:witnesses-mod-48-template}
\begin{tabular}{@{}c c l l@{}}
\toprule
$r' \ (\bmod 48)$ & Family & Choice of $W$ (terminal $\delta_W$) & Solve for $m$ (mod $8$) \\ \midrule
$1,7,13,19,25,31,37,43$   & $\mathrm e$ & e.g.\ $\Psi$, $\psi\omega\psi$, etc.\ ($\delta_W{=}1$) & $A_W m \equiv \tfrac{r'-1}{6}-B_W \pmod{8}$\\
$5,11,17,23,29,35,41,47$ & $\mathrm o$ & e.g.\ $\psi$, $\psi\Omega$, etc.\ ($\delta_W{=}5$) & $A_W m \equiv \tfrac{r'-5}{6}-B_W \pmod{8}$\\
\bottomrule
\end{tabular}
\end{table}

% =========================
% WORKED EXAMPLES: mod 48
% (A few concrete rows you can cite)
% =========================
\begin{table}[!htbp]
\centering
\caption{Selected concrete witnesses modulo $48$.
Each row shows a word $W$, its closed form $x_W(m)$, and a solved congruence for some $r' \bmod 48$.}
\label{tab:witnesses-mod-48-examples}
\begin{tabular}{@{}c l l l@{}}
\toprule
$r' \ (\bmod 48)$ & Word $W$ & Closed form $x_W(m)$ & One solution for $m$ \\ \midrule
$5$  & $\psi$ & $x(m)=96m+5$ & any $m$ (always $5\ (\bmod 48)$) \\
$13$ & $\psi\,\omega$ & $x(m)=\;6(3\cdot 2^{5}m+B)+\delta$ (affine) & $m\equiv m_0\pmod{8}$ (solve $A m\equiv\frac{13-\delta}{6}-B$) \\
$23$ & $\psi\,\omega\,\psi\,\Omega$ & affine as above & $m\equiv m_0\pmod{8}$ \\
$29$ & $\psi\,\Omega$ & $x(m)=192m+53$ & $192m+53\equiv 29\Rightarrow 0\cdot m\equiv -24$ (no sol.)\footnotemark \\
$41$ & $\Omega$ (from an $o$ start) & $x(m)=192m+53$ & always $5\ (\bmod 48)$; add an $o\!\to\!o$ steering gadget to shift to $41$ \\
\bottomrule
\end{tabular}

\footnotetext{If a basic $W$ fixes the residue (slope $\equiv 0\ (\bmod 48)$), append a short same-family \emph{steering gadget} (e.g.\ $o\!\to\!o$: $\Omega_2$ then $\omega$ then $\psi$) to adjust the intercept and re-solve.}
\end{table}

% =========================
% TEMPLATE TABLE: witnesses mod 96
% =========================
\begin{table}[!htbp]
\centering
\caption{Witness construction template modulo \texorpdfstring{\(96\)}{96} (with \texorpdfstring{\(M_5=96\)}{M5=96}).
For each odd residue \texorpdfstring{\(r' \equiv 1,5 \pmod 6\)}{r' congruent 1,5 mod 6}, pick a word \(W\) whose terminal family matches \texorpdfstring{\(r' \bmod 6\)}{r' mod 6}, write \(\,x_W(m)=6(A_W m+B_W)+\delta_W\,\), then solve
\(\,A_W m \equiv \tfrac{r'-\delta_W}{6}-B_W \pmod{2^{4}}\,\) (i.e.\ mod \(16\)),
and set \(x := x_W(m)\) to obtain \(x \equiv r' \pmod{96}\).}

\label{tab:witnesses-mod-96-template}
\begin{tabular}{@{}c c l l@{}}
\toprule
$r' \ (\bmod 96)$ & Family & Choice of $W$ (terminal $\delta_W$) & Solve for $m$ (mod $16$) \\ \midrule
$1,7,\ldots, 89$ (odd $\equiv 1$)  & $\mathrm e$ & e.g.\ $\Psi$, $\psi\omega\psi$, steering as needed & $A_W m \equiv \tfrac{r'-1}{6}-B_W \pmod{16}$\\
$5,11,\ldots, 95$ (odd $\equiv 5$) & $\mathrm o$ & e.g.\ $\psi$, $\psi\Omega$, steering as needed & $A_W m \equiv \tfrac{r'-5}{6}-B_W \pmod{16}$\\
\bottomrule
\end{tabular}
\end{table}


\newpage


% =========================================================

\ifidentity
\section*{Appendix E: Derivation of the identity
  \texorpdfstring{$3x'_p+1=2^{\alpha+6p}x$}{3 x prime p + 1 = 2 to the power (alpha + 6 p) x}}


\begin{lemma}[Forward identity for a lifted row]
Fix a row with parameters \((\alpha,\beta,c,\delta)\) and a column--lift \(p\ge 0\). Define
\[
F(p,m)\;=\;\frac{(9m\,2^{\alpha}+\beta)\,64^{\,p}+c}{9},
\qquad
x'_p\;=\;6F(p,m)+\delta,
\]
and write the odd input as \(x=18m+6j+p_6\) with \(j\in\{0,1,2\}\) and \(p_6\in\{1,5\}\).
Assuming the per--row design relations
\[
\beta \;=\;2^{\alpha-1}(6j+p_6),
\qquad
c \;=\;-\frac{3\delta+1}{2},
\]
one has the identity
\[
3x'_p+1 \;=\; 2^{\alpha+6p}\,x .
\]
\end{lemma}

\begin{proof}
By definition,
\[
x'_p \;=\; 6\!\left(2^{\alpha+6p}m+\frac{\beta\,64^{\,p}+c}{9}\right)+\delta
\quad\Longrightarrow\quad
3x'_p+1 \;=\; 18\cdot 2^{\alpha+6p}m \;+\; \Bigl( 18\!\cdot\!\tfrac{\beta\,64^{\,p}+c}{9}+3\delta+1 \Bigr).
\]
Simplify the bracket:
\[
18\!\cdot\!\frac{\beta\,64^{\,p}+c}{9}+3\delta+1
\;=\; 2\beta\,64^{\,p} \;+\; (2c+3\delta+1).
\]
With \(c=-(3\delta+1)/2\) the constant cancels: \(2c+3\delta+1=0\). Hence the bracket reduces to
\[
2\beta\,64^{\,p}
\;=\; 2\cdot 2^{\alpha-1}(6j+p_6)\cdot 64^{\,p}
\;=\; 2^{\alpha}(6j+p_6)\cdot 2^{6p}
\;=\; 2^{\alpha+6p}(6j+p_6).
\]
Therefore
\[
3x'_p+1
\;=\; 18\cdot 2^{\alpha+6p}m \;+\; 2^{\alpha+6p}(6j+p_6)
\;=\; 2^{\alpha+6p}\bigl(18m+6j+p_6\bigr)
\;=\; 2^{\alpha+6p}x,
\]
as claimed.
\end{proof}

\begin{remark}[Integrality]
Since \(64\equiv 1\pmod 9\), one has \(\beta\,64^{\,p}+c\equiv \beta+c\pmod 9\).
Each row in Table~\ref{tab:parameters-abc} satisfies \(\beta+c\equiv 0\pmod 9\), so \(F(p,m)\in\mathbb Z\) for all \(p\ge 0\).
\end{remark}

\begin{example}
For row \((\mathrm{o},1)\) (\(\omega_1\)) the table gives \(\alpha=1,\ \beta=11,\ c=-2,\ \delta=1\).
Then \(F(p,m)=2^{1+6p}m+\frac{11\cdot 64^{\,p}-2}{9}\) and the lemma yields \(3x'_p+1=2^{1+6p}x\).
\end{example}
\fi

% =========================
% Appendix: Row-consistent reversibility
% Requires \usepackage{algorithm,algpseudocode}
% =========================




\newpage
\section*{Appendix F: Code and Data Availability}
A reference implementation of the unified inverse table, the word evaluator, and the example generators is archived at
\href{https://doi.org/10.5281/zenodo.17352096}{Zenodo DOI: 10.5281/zenodo.17352096} and mirrored at
\href{git@github.com:kisira/collatz.git}{github.com/kisira/collatz}.
%The repository includes scripts to verify per-step identities \(U(x')=x\), regenerate the witness tables (mod \(24,48,96\)), and reproduce the figures and traces in this paper.
%Reproducibility instructions are summarized in Appendix~\ref{app:repro}.


\section*{Appendix F: Reproducibility Details}\label{app:repro}

\paragraph{Environment.}
The code is pure Python~3 (standard library + \texttt{pandas} for CSV I/O). A minimal setup is:
\begin{verbatim}
python -m venv .venv
. .venv/bin/activate
pip install -r requirements.txt
\cite{BernsteinLagarias1996}\cite{BernsteinLagarias1996}\end{verbatim}\cite{BernsteinLagarias1996}

\paragraph{Stepwise identity checks (\(U(x')=x\)).}
To verify that each row satisfies \(3x'+1=2^{\alpha+6p}x\) and that the word evaluator returns to the parent under \(U\):
\begin{verbatim}
python3 tools/check_rows.py             # verifies all rows and their p-lifts
python3 tools/evaluate_word.py --word psi,Omega,omega,psi --x0 1 --csv out.csv
\end{verbatim}
This writes a per-step trace (indices \(s,j,m\), formulas, and forward checks).

\paragraph{Regenerating witness tables.}
To regenerate witnesses mod \(24\), \(48\), and \(96\) (as used in the paper):
\begin{verbatim}
python3 tools/make_witnesses.py --mod 24  --out tables/witnesses_mod24.csv
python3 tools/make_witnesses.py --mod 48  --out tables/witnesses_mod48.csv
python3 tools/make_witnesses.py --mod 96  --out tables/witnesses_mod96.csv
\end{verbatim}

\paragraph{Recreating examples in the paper.}
Each example in Sections~\ref{lem:affine-word}–\ref{thm:reachability} can be reproduced with:
\begin{verbatim}
python3 tools/replay_example.py --name ex2
\end{verbatim}
which emits a CSV trace with the certified step identities and indices.

\paragraph{Generate the word for an odd number.}
To genereate a word for say 497. Or any other odd number.
\begin{verbatim}
python3 tools/calculate_word.py 497 --json-out 497_word.json
\end{verbatim}

\paragraph{Row consistent reverse.}
To reverse an odd number any number of steps.
\begin{verbatim}
python reverse_construct.py --mode one --y 43 --csv reverse_43.csv
python reverse_construct.py --mode chain --y 497 --stop 1 --csv chain_497_to_1.csv
\end{verbatim}



\paragraph{Archival guarantee.}
The Zenodo snapshot (DOI above) freezes the exact source corresponding to tag \texttt{v1.0} and commit \texttt{<hash>}, ensuring long-term reproducibility even if the development branch evolves.

\section*{Appendix G: Formalization index}
\begin{tabular}{@{}l l l@{}}
\toprule
Paper result & Label & Coq reference \\
\midrule
One-step composition with floor & \verb|\label{lem:one-step-floor}| & \texttt{CollatzFramework.v: compose\_one\_correct} \\
Last-row congruence targeting & \verb|\label{lem:last-row-p}|       & \texttt{LiftingWitnesses.v: last\_row\_congruence\_targeting\_nat} \\
Base witnesses (mod 24) coverage & \verb|\label{sec:base-coverage}|  & \texttt{LiftingWitnesses.v: base\_witness\_coverage} \\
Linear $2$-adic step (“pinning”) & \verb|\label{lem:linear-2adic}|   & \texttt{CollatzFramework.v: linear\_2adic\_pinning} \\
Routing compatibility (prefix)   & \verb|\label{lem:routing-compat-prefix}| & \texttt{CollatzFramework.v: routing\_compat\_prefix} \\
\bottomrule
\end{tabular}


\printbibliography
\end{document}


